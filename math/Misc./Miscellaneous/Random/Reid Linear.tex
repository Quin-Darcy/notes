\documentclass[12pt, a4paper]{article}
\usepackage[margin=1in]{geometry}
\usepackage[latin1]{inputenc}
\usepackage{titlesec}
\usepackage{amsmath}
\usepackage{amsthm}
\usepackage{amsfonts}
\usepackage{amssymb}
\usepackage{array}
\usepackage{booktabs}
\usepackage{ragged2e}
\usepackage{enumerate}
\usepackage{enumitem}
\usepackage{commath}
\usepackage{colonequals}
\renewcommand{\baselinestretch}{1.1}
\usepackage[mathscr]{euscript}
\let\euscr\mathscr \let\mathscr\relax
\usepackage[scr]{rsfso}
\newcommand{\powerset}{\raisebox{.15\baselineskip}{\Large\ensuremath{\wp}}}
\usepackage{longtable}
\usepackage{multirow}
\newcolumntype{C}{>$c<$}
\usepackage{fancyhdr}
\pagestyle{fancy}
\fancyhf{}
\renewcommand{\headrulewidth}{0pt}
\fancyhead[R]{\thepage}


\newcommand*{\logeq}{\ratio\Leftrightarrow}

\setlist[description]{leftmargin=10mm,labelindent=10mm}




\begin{document}




\justifying

\textbf{Problem 5}\par

\vspace{4mm}

    Let $T_A\colon\mathbf{R}^n\rightarrow\mathbf{R}^n$ be matrix multiplication by an invertible matrix $A$, and let $\{\Vec{u_1}, ..., \Vec{u_n}\}$ be a basis for $\mathbf{R}^n$. Prove that $\{T_A(\Vec{u_1}), ..., T_A(\Vec{u_n})\}$ is also a basis for $\mathbf{R}^n$.
    
\vspace{4mm}

\textbf{Initial thoughts}\par

\vspace{4mm}

    We need to show\par
    
\vspace{4mm}
    
    \centerline{$c_1T_A(\Vec{u_1})+\dots+c_nT_A(\Vec{u_n})=\Vec{0}$}
    
\vspace{4mm}

has only the trivial solution, or rather when $c_1=\dots =c_n=0$. We also need to show\par

\vspace{4mm}

\centerline{span$(\{T_A(\Vec{u_1}),\dots T_A(\Vec{u_n})\})=\mathbf{R}^n$.}

\vspace{4mm}

The following only has trivial solution, by definition\par
    
\vspace{4mm}
    
    \centerline{$c_1\Vec{u_1}+\ldots c_n\Vec{u_n}=\Vec{0}$.}
    
\vspace{4mm}

Multiplying both sides by $A$ is the same as applying $T_A$ to each vector. Thus, \par

\vspace{4mm}

\centerline{$A(c_1\Vec{u_1}+\dots+c_n\Vec{u_n})=\Vec{0}$.}

\vspace{4mm}

Thus

\vspace{4mm}

\centerline{$c_1T_A(\Vec{u_1})+\dots+c_nT_A(\Vec{u_n})=\Vec{0}$.}

\vspace{4mm}

Note that\par

\vspace{4mm}

\centerline{$A(c_1\Vec{u_1}+\dots+c_n\Vec{u_n})=\Vec{0}\Rightarrow (A=\Vec{0})\vee(c_1\Vec{u_1}+\ldots c_n\Vec{u_n}=\Vec{0})$.}

\vspace{4mm}

However, $A\neq\Vec{0}$ since $A$ is invertible. Thus, $c_1\Vec{u_1}+\ldots c_n\Vec{u_n}=\Vec{0}$ and this only has the trivial solution. Thus,\par

\vspace{4mm}

    \centerline{$c_1T_A(\Vec{u_1})+\dots+c_nT_A(\Vec{u_n})=\Vec{0}\Leftrightarrow c_1=\dots=c_n=0$.}
    
\vspace{4mm}

    Therefore, $\{T_A(\Vec{u_1}), ..., T_A(\Vec{u_n})\}$ is linearly independent. Now consider some $T_A(\Vec{u_i})$ from the span of the transformed basis set. By definition, $T_A\colon\mathbf{R}^n\rightarrow\mathbf{R}^n$. Thus\par
    
\vspace{4mm}

    \centerline{$T_A(\Vec{u_i})\in$span$(\{T_A(\Vec{u_1}), ..., T_A(\Vec{u_n})\})\Rightarrow T_A(\Vec{u_i})\in\mathbf{R}^n$}
    
\vspace{4mm}

    Therefore, span$(\{T_A(\Vec{u_1}), ..., T_A(\Vec{u_n})\})\subseteq\mathbf{R}^n$. Now let $\Vec{v},\Vec{w}\in\mathbf{R}^n$,such that $\langle\Vec{v},\Vec{w}\rangle=0$. Since, $\{\Vec{u_1}, ..., \Vec{u_n}\}$ is a basis for $\mathbf{R}^n$ then there exists $c_1,\ldots, c_n\in\mathbf{R}$ such that\par
    
\vspace{4mm}

\centerline{$c_1\Vec{u_1}+\dots+c_n\Vec{u_n}=\vec{v}$.}

\vspace{4mm}

\centerline{$A(c_1\Vec{u_1}+\dots+c_n\Vec{u_n})=A\vec{v}$}

    


\end{document}