\documentclass[12pt, a4paper]{article}
\usepackage[margin=1in]{geometry}
\usepackage[latin1]{inputenc}
\usepackage{titlesec}
\usepackage{amsmath}
\usepackage{amsthm}
\usepackage{amsfonts}
\usepackage{amssymb}
\usepackage{array}
\usepackage{booktabs}
\usepackage{ragged2e}
\usepackage{enumerate}
\usepackage{enumitem}
\usepackage{commath}
\usepackage{lipsum}
\usepackage{colonequals}
\renewcommand{\baselinestretch}{1.1}
\usepackage[mathscr]{euscript}
\let\euscr\mathscr \let\mathscr\relax
\usepackage[scr]{rsfso}
\newcommand{\powerset}{\raisebox{.15\baselineskip}{\Large\ensuremath{\wp}}}
\usepackage{longtable}
\usepackage{multirow}
\usepackage{multicol}
\newcounter{proofc}
\renewcommand\theproofc{(\arabic{proofc})}
\DeclareRobustCommand\stepproofc{\refstepcounter{proofc}\theproofc}
\newenvironment{twoproof}{\tabular{@{\stepproofc}c|l}}{\endtabular}
\newcolumntype{C}{>$c<$}
\usepackage{fancyhdr}
\pagestyle{fancy}
\fancyhf{}
\renewcommand{\headrulewidth}{0pt}
\fancyhead[R]{\thepage}
\usepackage{enumitem}
\usepackage{tikz}
\usepackage{commath}
\usepackage{colonequals}
\usepackage{bm}
\usepackage{tikz-cd}
\renewcommand{\baselinestretch}{1.1}
\usepackage[mathscr]{euscript}
\let\euscr\mathscr \let\mathscr\relax
\usepackage[scr]{rsfso}



\newcommand*{\logeq}{\ratio\Leftrightarrow}

\setlist[description]{leftmargin=10mm,labelindent=10mm}



\begin{document}


\section*{\centerline{ZFC Axioms}}

\justifying

\begin{flushleft}

\underline{The $\in$-Relation}

\vspace{4mm}

Set theory is built on the postulate that there is a fundamental relation $\in$. Where a relation is a short term for a predicate of two variables. Aside from a two variable predicate, there will be no definition of $\in$, and furthermore, there will be no definition of what a set is. Instead, we will present 9 axioms that speak of $\in$ and sets. 

\end{flushleft}



\begin{flushleft}

\underline{Overview of axioms}

\end{flushleft}

\vspace{4mm}

Basis Existence Axioms\hspace{4mm} $\left\{ \begin{array}{rcl} E \\ E \end{array}\right$

\vspace{2mm}

Construction Axioms\hspace{8mm} $\left\{ \begin{array}{rcl} P \\ U \\ R \\ P \end{array}\right$

\vspace{2mm}

Further Existence Axioms $\left\{ \begin{array}{rcl} I \\ C \end{array}\right$

\vspace{2mm}

Non-existence Axiom\hspace{9.5mm} $\left\{ \begin{array}{rcl} F \end{array}\right$

\vspace{12mm}

\begin{flushleft}

\underline{The $\in$-Relation (cont.)}

\vspace{4mm}

Using the $\in$-relation, we can immediately define\par


\begin{description}

    \item $  x\notin y \logeq\neg(x\in y)$.
    
    \item $ x\subseteq y\logeq \forall a(a\in x\Rightarrow a\in y)$.
    
    \item $ x=y\logeq x\subseteq y\wedge y\subseteq x$.

\end{description}

\end{flushleft}


\begin{flushleft}

\underline{Zermelo-Fraenkel Axioms of Set Theory}

\end{flushleft}

\textbf{(E) Axiom on $\in$-Relation} : $x\in y$ is a proposition if and only if $x$ and $y$ are both sets. To be precise:\par

\vspace{4mm}

\centerline{$\forall x\forall y[(x\in y)\veebar\neg(x\in y)]$.}

\newpage

\begin{flushleft}

\underline{Counter Example}

\vspace{4mm}

Assume there is some $u$ that contains all sets that do not contain themselves as an element. To be precise\par

\vspace{4mm}

\centerline{$\exists u\forall z(z\in u\Leftrightarrow z\notin z)$.}

\vspace{4mm}

\underline{Question}: is $u$ a set?\par

\vspace{4mm}

If $u$ was a set, then one must be able to decide if $u\in u$ is true or false, since if $u$ was a set then $u\in u$ is a proposition and is thereby falsifiable.\par

\vspace{4mm}

Assume $u\in u$ is true. Then it is some particular $z$ such that $z\notin z$. Thus, $u\in u\Rightarrow u\notin u$. This is a contradiction.\par

\vspace{4mm}

Assume $u\in u$ is false. This is equivalent to $u\notin u$. If $u\notin u$, then $u\in u$ since $u$ contains all the sets that do not contain themselves. Thus, $u\notin u\Rightarrow u\in u$. This is a contradiction.\par

\vspace{4mm}

Thus, $u\in u$ is neither true nor false and hence, is not a proposition. Therefore, $u$ is not a set.\par

\end{flushleft}

\textbf{(E) Axiom on existence of an empty set: }There exists a set that contains no elements. To be precise:\par

\vspace{4mm}

\centerline{$\exists x\forall y(y\notin x)$.}


\vspace{6mm}


\textsc{Theorem: }There is only one empty set, call it $\varnothing$.\par

\vspace{4mm}

\textsc{Proof} (textbook): Assume $x$ and $x'$ are both empty sets. But then\par

\vspace{4mm}

\centerline{$\forall y[(y\in x)\Rightarrow (y\in x')]$ (ex falso quadlibert)}

\vspace{4mm}

Thus, by definition\par

\vspace{4mm}

\centerline{$x\subseteq x'$.}\par

\vspace{4mm}

Conversely,\par

\vspace{4mm}

\centerline{$\forall y[(y\in x')\Rightarrow (y\in x)]$.}

\vspace{4mm}

Hence,\par

\vspace{4mm}

\centerline{$x'\subseteq x$.}

\vspace{4mm}

Thus, $x\subseteq x'\wedge x'\subseteq x$. Therefore, $x=x'$\par

\vspace{4mm}

\newpage

\textsc{proof} (formal):\par

\vspace{4mm}


\begin{center}
\begin{tabular}{ c c }
\hline
 1. $\forall y(y\notin x)$\hspace{41mm} &\hspace{10mm} Assumption \\ 
 2. $\forall y(y\notin x')$\hspace{40mm} &\hspace{10mm} Assumption \\  
 3. $\forall y(y\notin x)\Rightarrow\forall y(y\in x\Rightarrow y\in x')$ &\hspace{7mm} Tautology \\
 4. $\forall y(y\notin x)$\hspace{40mm} &\hspace{1mm} Given
\end{tabular}
\end{center}



















\end{document}