\documentclass[12pt, a4paper]{article}
\usepackage[margin=1in]{geometry}
\usepackage[latin1]{inputenc}
\usepackage{titlesec}
\usepackage{amsmath}
\usepackage{amsthm}
\usepackage{amsfonts}
\usepackage{amssymb}
\usepackage{array}
\usepackage{booktabs}
\usepackage{ragged2e}
\usepackage{enumerate}
\usepackage{enumitem}
\usepackage{commath}
\usepackage{colonequals}
\renewcommand{\baselinestretch}{1.1}
\usepackage[mathscr]{euscript}
\let\euscr\mathscr \let\mathscr\relax
\usepackage[scr]{rsfso}
\newcommand{\powerset}{\raisebox{.15\baselineskip}{\Large\ensuremath{\wp}}}
\usepackage{longtable}
\usepackage{multirow}
\newcolumntype{C}{>$c<$}

\usepackage{enumitem}
\usepackage{tikz}
\usepackage{commath}
\usepackage{colonequals}
\usepackage{bm}
\usepackage{tikz-cd}
\renewcommand{\baselinestretch}{1.1}
\usepackage[mathscr]{euscript}
\let\euscr\mathscr \let\mathscr\relax
\usepackage[scr]{rsfso}



\newcommand*{\logeq}{\ratio\Leftrightarrow}

\setlist[description]{leftmargin=0mm,labelindent=10mm}



\begin{document}


\justifying

\begin{flushleft}

\blacksquare \textbf{ AXIOMS [Frege-Lukasiewicz]}

\vspace{4mm}

\hspace{4mm} Let $p,q,$ and $r$ be propositional forms.

\begin{description}

    \item \bullet \textbf{ [FL1] } $p \to(q\to p)$
    \item \bullet \textbf{ [FL2] } $p \to(q\to r)\to(p\to q\to[p\to r])$
    \item \bullet \textbf{ [FL3] } $\neg p\to\neg q\to(q\to p)$.

\end{description}

\vspace{8mm}

\blacksquare \textbf{ INFERENCE RULES }

\vspace{4mm}

\hspace{4mm} Let $p,q,r,$ and $s$ be propositional forms.

\begin{description}

    \item \bullet \textbf{ Modus Ponens [MP]}\par
    \hspace{15mm}$p\to q, p\Rightarrow q$\par
    
    \item \bullet \textbf{ Modus Tolens [MT]}\par
    \hspace{15mm}$p\to q, \neg q\Rightarrow\neg p$\par
    
    \item \bullet \textbf{ Constructive Dilemma [CD]}\par\hspace{15mm}
    $(p\to q)\wedge(r\to s), p\vee r\Rightarrow q\vee s$\par
    
    \item \bullet \textbf{ Destructive Dilemma [DD]}\par\hspace{15mm}
    $(p\to q)\wedge(r\to s),\neg q\vee\neg s\Rightarrow \neg p\vee\neg r$
    
    \item \bullet \textbf{ Disjunctive Syllogism [DS]}\par\hspace{15mm}
    $p\vee q,\neg p\Rightarrow q$
    
    \item \bullet \textbf{ Hypothetical Syllogism [HS]}\par\hspace{15mm}
    $p\to q,q\to r, r\Rightarrow p\to r$
    
    \item \bullet \textbf{ Conjunction [Conj]}\par\hspace{15mm}
    $p,q\Rightarrow p\wedge q$
    
    \item \bullet \textbf{ Simplification [Simp]}\par\hspace{15mm}
    $p\wedge q\Rightarrow p$
    
    \item \bullet \textbf{ Addition [Add]}\par\hspace{15mm}
    $p\Rightarrow p\vee q$.


\end{description}

\end{flushleft}

\newpage

\begin{flushleft}

\blacksquare \textbf{ REPLACEMENT RULES }

\vspace{4mm}

\hspace{4mm} Let $p,q,$ and $r$ be propositional forms. 

\begin{description}

    \item \bullet \textbf{ Associative Laws [Assoc] }\par\hspace{15mm}
    $p\wedge q\wedge r\Leftrightarrow p\wedge(q\wedge r)$\par\hspace{15mm}
    $p\vee q\vee r\Leftrightarrow p\vee(q\vee r)$
    
    \item \bullet \textbf{ Commutative Laws [Com] }\par\hspace{15mm}
    $p\wedge q\Leftrightarrow q\wedge p$\par\hspace{15mm}
    $p\vee q\Leftrightarrow q\vee p$
    
    \item \bullet \textbf{ Distributive Laws [Distr] }\par\hspace{15mm}
    $p\wedge(q\vee r)\Leftrightarrow p\wedge q\vee p\wedge r$\par\hspace{15mm}
    $p\vee q\wedge r\Leftrightarrow (p\vee q)\wedge(p\vee r)$
    
    \item \bullet \textbf{ Contrapositive Law [Contra] }\par\hspace{15mm}
    $p\to q\Leftrightarrow\neg q\to\neg p$\par\hspace{15mm}
    
    \item \bullet \textbf{ Double Negation [DN] }\par\hspace{15mm}
    $p\Leftrightarrow\neg\neg p$\par\hspace{15mm}
    
    \item \bullet \textbf{ De Morgan's Laws [DeM] }\par\hspace{15mm}
    $\neg(p\wedge q)\Leftrightarrow\neg p\vee\neg q$\par\hspace{15mm}
    $\neg(p\vee q)\Leftrightarrow\neg p\wedge\neg q$
    
    \item \bullet \textbf{ Idempotentcy [Idem] }\par\hspace{15mm}
    $p\wedge p\Leftrightarrow p$\par\hspace{15mm}
    $p\vee p\Leftrightarrow p$
    
    \item \bullet \textbf{ Material Equivalence [Equiv] }\par\hspace{15mm}
    $p\leftrightarrow q\Leftrightarrow(p\to q)\wedge(q\to p)$\par\hspace{15mm}
    $p\leftrightarrow q\Leftrightarrow p\wedge q\vee\neg p\wedge\neg q$
    
    \item \bullet \textbf{ Material Implication [Impl] }\par\hspace{15mm}
    $p\to q\Leftrightarrow\neg p\vee q$\par\hspace{15mm}
    
    \item \bullet \textbf{ Exportation [Exp] }\par\hspace{15mm}
    $p\wedge q\to r\Leftrightarrow p\to(q\to r)$

\end{description}

\end{flushleft}

\newpage

\begin{flushleft}

\blacksquare \textbf{ DEFINITION }

\vspace{4mm}

\hspace{4mm} A \textbf{predicate} is an expression that ascribes a property to the objects identified by the variables of the sentence.\par

\end{flushleft}

\begin{flushleft}

\blacksquare \textbf{ REPLACEMENT RULES [Quantifier Negation (QN)] }

\end{flushleft}

Let S be theory symbols and $p$ be an S-formula.\par

\begin{description}

    \item \bullet $ \neg\forall xp\Leftrightarrow \exists x\neg p$
    
    \item \bullet $ \neg\exists xp\Leftrightarrow \forall x\neg p$

\end{description}

\end{document}