\documentclass[12pt, a4paper]{article}
\usepackage[margin=1in]{geometry}
\usepackage[latin1]{inputenc}
\usepackage{titlesec}
\usepackage{amsmath}
\usepackage{amsthm}
\usepackage{amsfonts}
\usepackage{amssymb}
\usepackage{array}
\usepackage{booktabs}
\usepackage{ragged2e}
\usepackage{enumerate}
\usepackage{enumitem}
\usepackage{cleveref}
\usepackage{slashed}
\usepackage{commath}
\usepackage{lipsum}
\usepackage{colonequals}
\usepackage{addfont}
\addfont{OT1}{rsfs10}{\rsfs}
\renewcommand{\baselinestretch}{1.1}
\usepackage[mathscr]{euscript}
\let\euscr\mathscr \let\mathscr\relax
\usepackage[scr]{rsfso}
\newcommand{\powerset}{\raisebox{.15\baselineskip}{\Large\ensuremath{\wp}}}
\usepackage{longtable}
\usepackage{multirow}
\usepackage{multicol}
\usepackage{calligra}
\usepackage[T1]{fontenc}
\newcounter{proofc}
\renewcommand\theproofc{(\arabic{proofc})}
\DeclareRobustCommand\stepproofc{\refstepcounter{proofc}\theproofc}
\newenvironment{twoproof}{\tabular{@{\stepproofc}c|l}}{\endtabular}
\newcolumntype{C}{>$c<$}
\usepackage{fancyhdr}
\pagestyle{fancy}
\fancyhf{}
\renewcommand{\headrulewidth}{0pt}
\fancyhead[R]{\thepage}
\usepackage{enumitem}
\usepackage{tikz}
\usepackage{commath}
\usepackage{colonequals}
\usepackage{bm}
\usepackage{tikz-cd}
\renewcommand{\baselinestretch}{1.1}
\usepackage[mathscr]{euscript}
\let\euscr\mathscr \let\mathscr\relax
\usepackage[scr]{rsfso}



\newcommand*{\logeq}{\ratio\Leftrightarrow}

\setlist[description]{leftmargin=8mm,labelindent=8mm}




\newtheorem{theorem}{Theorem}
 
\begin{document}

\section{Set-up}
\hline

\vspace{4mm}

Consider a two dimensional Cartesian coordinate system with vertical axis labeled $y$ and horizontal axis labeled $x$. From the origin, picture a point $(0,n)$ on the $y$-axis and a point $(m, 0)$ on the $x$-axis. Now with these two points we can form an $m\times n$ rectangular region, call it $R$, by connecting the point in which the lines $y=n$ and $x=n$ intersect, and connecting this common point to the two points on the $x$ and $y$-axis. The point of intersection would be the coordinate $(m,n)$, and thus the set of all points lying within the region can be described as

\vspace{4mm}

\centerline{$R=\{(x,y)\in\mathbb{R}^2\mid (0\leq x\leq m)\wedge(0\leq y\leq n)\}$.}

\vspace{4mm}\par

Now suppose we are given some point $(a,b)\in R$. What we will first do is find the equation of the line that passes through the origin $(0,0)$ and the point $(a,b)$. It is easily verified that the equation is 

    \begin{equation}
        \begin{split}
            y_1 &= \frac{b}{a}x. \\
        \end{split}
    \end{equation}
    
\vspace{4mm}

\noindent Another equation we will need is the one for the line that is perpendicular to $(1)$ and which passes through the origin. The equation for this line is simply 

    \begin{equation}
        \begin{split}
            y_2 &= -\frac{a}{b}x. \\
        \end{split}
    \end{equation}
    
\vspace{4mm}

\noindent What we now need to find is the equation for the line which passes through $(a,b)$ and intersects with $(2)$. Thus, we need to solve for some $y_3=cx+d$ which satisfies the following conditions

    \begin{equation}
        \begin{split}
            b &= c(a)+d \\ 
        \end{split}
    \end{equation}
    
    \begin{equation}
        \begin{split}
            -\frac{a}{b}(x_0) &= c(x_0)+d. \\ 
        \end{split}
    \end{equation}
    
\vspace{4mm}

\noindent From $(3)$ we get that $d=b-c(a)$, and plugging this into $(4)$ we get that

\begin{equation}
        \begin{split}
            -\frac{a}{b}(x_0) &= c(x_0)+b-c(x_0) \\
            &= b \\
            &\Rightarrow x_0 = -\frac{b^2}{a}
        \end{split}
    \end{equation}
    
\newpage

We now consider another relationship which will help us solve for the coefficients $c$ and $d$. We know the length of the line segment between $(0,0)$ and $(a,b)$ is $\sqrt{a^2+b^2}$, if we treat this as the adjacent side to a right triangle whose main vertex is at the point $(a,b)$, second vertex at $(0,0)$ and third vertex is at the point of intersection with $y_2$, then we can solve for the coordinates of the point of intersection. The angle formed clock-wise from the $y-axis$ to our line $y_1$, we will call $\beta$. We can determine this angle by noting that $\tan(\beta)=a/b$, and thus $\beta=\tan^{-1}(a/b)$.\par
It can be shown through basic trigonometry that the angle at the main vertex between the adjacent side and the hypotenuse is equal to $\beta$. This means we have a side length and angle available. Thus, the length of the hypotenuse, call it $l$ can be expressed as

    \begin{equation}
        \begin{split}
            l &= \frac{\sqrt{a^2+b^2}}{\cos(\beta)}. \\ 
        \end{split}
    \end{equation}
    
\vspace{4mm}

\noindent Additionally, we know that the length of the line between two points $(x_1,y_1),(x_2,y_2)$ is equal to $\sqrt{(x_2-x_1)^2+(y_2-y_1)^2}$. So letting assuming $y_3$ intersects with $y_2$, then $y_0=-a/b(x_0)$ is a solution which satisfies both equations. Thus,

    \begin{equation*}
        \begin{split}
            l &= \sqrt{(a-x_0)^2+(b-y_0)^2} \\
            &= \sqrt{(a-x_0)^2+(b+\frac{a}{b}x_0)^2} \\
            &= \sqrt{(a^2-2ax_0+x_0^2)+(b^2+2ax_0+\frac{a^2}{b^2}x_0^2)} \\
            &= \sqrt{a^2+b^2+x_0^2(1+\frac{a^2}{b^2}} \\
            &= \sqrt{a^2+b^2+x_0^2\sec^2(\beta)}.
        \end{split}
    \end{equation*}
    
\vspace{4mm}

\noindent Thus,

    \begin{equation*}
        \begin{split}
            a^2+b^2+x_0^2\sec^2(\beta) &= (a^2+b^2)\sec^2(\beta) \\
            \frac{a^2+b^2}{\sec^2(\beta)}+x_0^2 &= a^2+b^2 \\
            x_0^2 &= a^2+b^2-(a^2+b^2)\cos^2(\beta) \\
            &= (a^2+b^2)(1-\cos^2(\beta)) \\
            &= (a^2+b^2)\sin^2(\beta) \\
            x_0 &= l\sin(\beta).
        \end{split}
    \end{equation*}

\vspace{4mm}

\noindent Thus, using $(4)$, we get that $y_0=-\frac{a}{b}l\sin(\beta)$. Now we will use the point slope formula.

\newpage

    \begin{equation*}
        \begin{split}
            m &= \frac{b+\frac{a}{b}l\sin(\beta)}{a-l\sin(\beta)} \\
            &= \frac{b^2+al\sin(\beta)}{ab-bl\sin(\beta)}
        \end{split}
    \end{equation*}




\end{document}