\documentclass[12pt, a4paper]{article}
\usepackage[margin=1in]{geometry}
\usepackage[latin1]{inputenc}
\usepackage{titlesec}
\usepackage{amsmath}
\usepackage{amsthm}
\usepackage{amsfonts}
\usepackage{amssymb}
\usepackage{array}
\usepackage{booktabs}
\usepackage{ragged2e}
\usepackage{enumerate}
\usepackage{enumitem}
\usepackage{cleveref}
\usepackage{slashed}
\usepackage{commath}
\usepackage{lipsum}
\usepackage{colonequals}
\usepackage{addfont}
\addfont{OT1}{rsfs10}{\rsfs}
\renewcommand{\baselinestretch}{1.1}
\usepackage[mathscr]{euscript}
\let\euscr\mathscr \let\mathscr\relax
\usepackage[scr]{rsfso}
\newcommand{\powerset}{\raisebox{.15\baselineskip}{\Large\ensuremath{\wp}}}
\usepackage{longtable}
\usepackage{multirow}
\usepackage{multicol}
\usepackage{calligra}
\usepackage[T1]{fontenc}
\newcounter{proofc}
\renewcommand\theproofc{(\arabic{proofc})}
\DeclareRobustCommand\stepproofc{\refstepcounter{proofc}\theproofc}
\usepackage{fancyhdr}
\pagestyle{fancy}

\renewcommand{\headrulewidth}{0pt}
\fancyhead[R]{}
\usepackage{enumitem}
\usepackage{tikz}
\usepackage{commath}
\usepackage{colonequals}
\usepackage{bm}
\usepackage{tikz-cd}
\renewcommand{\baselinestretch}{1.1}
\usepackage[mathscr]{euscript}
\let\euscr\mathscr \let\mathscr\relax
\usepackage[scr]{rsfso}
\usepackage{titlesec}
\usepackage{scrextend}
\usepackage{lscape}
\usepackage{relsize}

\usepackage[english]{babel}
\usepackage{blindtext}
\usepackage{polynom}



\begin{document}

\noindent\textbf{Definition 1.} Let $(G,+)$ be a group and let $H\subseteq G$. The set $H$ is a \textbf{subgroup of G} if\par

\vspace{2mm}

\begin{enumerate}
    \item $\forall a\forall b(a\in H\wedge b\in H\rightarrow[a-b\in H])$.
\end{enumerate}

\noindent\textbf{Definition 2.} Let $(R,+,\cdot)$ be a ring and let $I\subseteq R$. The set $I$ is an \textbf{ideal of R} if\par

\vspace{2mm}

\begin{enumerate}
    \item $(I, +)$ is a subgroup of $(R,+)$.
    \item $\forall r\forall i(r\in R\wedge i\in I\rightarrow [r\cdot i\in I\wedge i\cdot r\in I])$
\end{enumerate}

\noindent\textbf{Definition 3.} Let $(R,+,\cdot)$ be a ring and let $I\subseteq R$ be an ideal of $R$. Then $I$ is \textbf{maximal} if

\begin{enumerate}
    \item $\forall J(J\subseteq R\wedge I\subseteq J\rightarrow[J=I\vee J=R])$.
\end{enumerate}

\noindent\textbf{Proposition:} Let $\mathbb{Z}_3[X]$ be the ring of polynomials with coefficients from $\mathbb{Z}_3$. Then the ideal $J=(X^2+3)_i$ is maximal.

\vspace{4mm}

\noindent\textbf{\textit{Proof.}} Assume $I\subseteq\mathbb{Z}_3[X]$ is an ideal and that $J\subseteq I$. Since $I$ is an ideal, then there is some $p(X)\in\mathbb{Z}_3[X]$ such that

\vspace{4mm}

\centerline{$I=\{h(X)\cdot p(X)\mid h(X)\in\mathbb{Z}_3[X]\}$.}

\vspace{4mm}

\noindent Consequently, because $p(X)$ is any arbitrary polynomial, then either

\vspace{4mm}

\centerline{$(p(X)=1)\vee (p(X)\neq 1)$.}

\vspace{8mm}\par

\underline{Case 1}: Assume $p(X)=1$. Then

\begin{equation*}
        \begin{split}
            I &= \{h(X)\cdot p(X)\mid h(X)\in\mathbb{Z}_3[X]\} \\
            &= \{h(X)\cdot(1)\mid h(X)\in\mathbb{Z}_3[X]\} \\
            &=\mathbb{Z}_3[X],
        \end{split}
    \end{equation*}
    
\vspace{4mm}

\par and we are done.

\vspace{8mm}\par

\underline{Case 2}: Assume $p(X)\neq 1$. Recall that $J=\{g(X)\cdot(X^2+3)\mid g(X)\in\mathbb{Z}_3[X]\}$ and then suppose $f(X)\in J$. Then there exists some $g(X)\in\mathbb{Z}_3[X]$ such that

\vspace{4mm}

\centerline{$f(X)=g(X)\cdot(X^2+3)$.}

\vspace{4mm}

\noindent However, since $J\subseteq I$, then there exists some $h(X)\in\mathbb{Z}_3[X]$ such that 

\vspace{4mm}

\centerline{$f(X)=h(X)\cdot p(X)$.}

\vspace{4mm}

\noindent Thus, $g(X)\cdot(X^2+3)=h(X)\cdot p(X)$. Now suppose $g(X)=1$. 

\newpage

\noindent Then $(X^2+3)=h(X)\cdot p(X)$. Thus, $p(X)\mid(X^2+3)$. However, since $(X^2+3)$ is irreducible, then $p(X)=1$ or $p(X)=X^2+3$. If $p(X)=1$, then refer to Case 1. If $p(X)=X^2+3$, then 

\vspace{4mm}

\begin{equation*}
        \begin{split}
            I &= \{h(X)\cdot p(X)\mid h(X)\in\mathbb{Z}_3[X]\} \\
            &= \{h(X)\cdot(X^2+3)\mid h(X)\in\mathbb{Z}_3[X]\} \\
            &= J.
        \end{split}
    \end{equation*}
    
\vspace{4mm}





\end{document}