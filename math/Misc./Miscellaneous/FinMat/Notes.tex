\documentclass{article}
\usepackage{graphicx}
\usepackage{amsmath}
\usepackage{authblk}
\usepackage{titlesec}
\usepackage{amsthm}
\usepackage{amsfonts}
\usepackage{amssymb}
\usepackage{array}
\usepackage{booktabs}
\usepackage{ragged2e}
\usepackage{enumerate}
\usepackage{enumitem}
\usepackage{cleveref}
\usepackage{slashed}
\usepackage{commath}
\usepackage{lipsum}
\usepackage{colonequals}
\usepackage{addfont}
\usepackage{enumitem}
\usepackage{sectsty}
\usepackage{lastpage}
\usepackage{fancyhdr}
\usepackage{accents}
\usepackage[inline]{enumitem}
\pagestyle{fancy}
\setlength{\headheight}{10pt}

\subsectionfont{\itshape}

\newtheorem{theorem}{Theorem}[section]
\newtheorem{corollary}{Corollary}[theorem]
\newtheorem{lemma}[theorem]{Lemma}
\theoremstyle{definition}
\newtheorem{definition}{Definition}[section]
\theoremstyle{remark}
\newtheorem*{remark}{Remark}
 
\makeatletter
\renewenvironment{proof}[1][\proofname]{\par
  \pushQED{\qed}%
  \normalfont \topsep6\p@\@plus6\p@\relax
  \list{}{\leftmargin=0mm
          \rightmargin=0mm
          \settowidth{\itemindent}{\itshape#1}%
          \labelwidth=\itemindent
          \parsep=0pt \listparindent=\parindent 
  }
  \item[\hskip\labelsep
        \itshape
    #1\@addpunct{.}]\ignorespaces
}{%
  \popQED\endlist\@endpefalse
}

\newenvironment{solution}[1][\bf{\textit{Solution}}]{\par
  
  \normalfont \topsep6\p@\@plus6\p@\relax
  \list{}{\leftmargin=0mm
          \rightmargin=0mm
          \settowidth{\itemindent}{\itshape#1}%
          \labelwidth=\itemindent
          \parsep=0pt \listparindent=\parindent 
  }
  \item[\hskip\labelsep
        \itshape
    #1\@addpunct{.}]\ignorespaces
}{%
  \popQED\endlist\@endpefalse
}

\let\oldproofname=\proofname
\renewcommand{\proofname}{\bf{\textit{\oldproofname}}}


\newlist{mylist}{enumerate*}{1}
\setlist[mylist]{label=(\alph*)}

\begin{document}

\title{Notes}
\author{Quin Darcy}
\date{16 March 2019}
\affil{\small{California State University Sacramento}}
\maketitle

\section{Exercises}

\begin{enumerate}[leftmargin=*]
    \item Suppose that \$3659 is deposited in a savings account which earns 6.5\% simple interest. What is the amount due after 5 years?
    
    \begin{solution} We see that the principle amount $P=3659$, the interest rate $r=0.065$, and since this is simple interest, it follows that after $t=5$ years the amount due is
    
        \begin{equation*}
            A=P\big(1+rt)=(3659)\big[1+(0.065)(5)\big]=\$4848.18.
        \end{equation*}
    \end{solution}
    
    \item Suppose that \$3993 is deposited in an account which earns 4.3\% interest. What is the compound amount after two years if the interest is compounded
    
    \begin{enumerate}[label=(\alph*)]
        \item monthly?
        \item weekly?
        \item daily?
        \item continuously?
    \end{enumerate}
    
    \begin{solution}\hfill\par
        \begin{enumerate}[label=(\alph*)]
            \item 
        \end{enumerate}
    
    \end{solution}
\end{enumerate}

\end{document}