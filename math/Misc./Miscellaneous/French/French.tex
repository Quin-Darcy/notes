\documentclass{article}
\usepackage[margin=1in]{geometry} 
\usepackage{graphicx}
\usepackage{tikz}
\usepackage{amsmath}
\usepackage{authblk}
\usepackage{titlesec}
\usepackage{amsthm}
\usepackage{amsfonts}
\usepackage{amssymb}
\usepackage{array}
\usepackage{booktabs}
\usepackage{ragged2e}
\usepackage{enumerate}
\usepackage{enumitem}
\usepackage{cleveref}
\usepackage{slashed}
\usepackage{commath}
\usepackage{lipsum}
\usepackage{colonequals}
\usepackage{addfont}
\usepackage{enumitem}
\usepackage{sectsty}
\usepackage{mathtools}
\usetikzlibrary{decorations.pathreplacing}
\usetikzlibrary{arrows.meta}

\subsectionfont{\itshape}

\newtheorem{theorem}{Theorem}[section]
\newtheorem{corollary}{Corollary}[theorem]
\newtheorem{lemma}[theorem]{Lemma}
\theoremstyle{definition}
\newtheorem{prop}{Proposition}[section]
\newtheorem{definition}{Definition}[section]
\theoremstyle{remark}
\newtheorem*{remark}{Remark}


\let\oldproofname=\proofname
\renewcommand{\proofname}{\bf{\textit{\oldproofname}}}

\newcommand{\closure}[2][3]{%
  {}\mkern#1mu\overline{\mkern-#1mu#2}}

\theoremstyle{definition}
\newtheorem{example}{Example}[section]

\newtheorem*{discussion}{Discussion}





\begin{document}

\title{FREN 1B Dialogue}
\date{May 5, 2019}
\affil{\small{California State University, Sacramento}}
\maketitle

\section{Introduction}
\hline

\vspace{4mm}

\begin{enumerate}[leftmargin=*]
    \item\textbf{Q:} Comment \c ca va? [``How's it going?'']\par 
        \noindent\textbf{A:} \c Ca va bien. [``It's going well.'']\par \vspace{3mm}
        \hline
        
    \item\textbf{Q:} Comment t'appelles-tu? [``What are you called?'']\par 
        \noindent\textbf{A:} Je m'appelle (\textit{name}). [``My name is (\textit{name}).'']\par \vspace{3mm}
        \hline
    
    \item\textbf{Q:} Quel \^age as-tu? [``How old are you?'']\par
        \noindent\textbf{A:} J'ai (\textit{number}) ans. [``I am (\textit{number}) years old.'']\par \vspace{3mm}
        \hline
    
    \item\textbf{Q:} O\`u habites-tu? [``Where do you live?'']\par 
        \noindent\textbf{A:} J'habite \`a Sacramento. [``I live in Sacramento.'']\par \vspace{3mm}
        \hline 
        
    \item\textbf{Q:} Est-ce que tu habites dans un appartement ou une maison?\par\hspace{5.5mm}[``Do you live in an appartement or a house?'']\par 
        \noindent\textbf{A:} J'habite un(e) (appartement/maison). [``I live in a (appartement/house).'']\par \vspace{3mm}
        
        
\end{enumerate}

\section{Personal Information}
\hline

\vspace{4mm}

\begin{enumerate}[leftmargin=*]
    \item[6.]\textbf{Q:} Est-ce que vous avez des fr\`eres et des s\oe urs? [``Do you have brothers and sisters?'']\par 
        \noindent\textbf{A1:} J'ai (\textit{number}) fr\`eres et (\textit{number}) s\oe urs. [``I have (\textit{number}) brothers and (\textit{number}) sisters.'']\par 
        \noindent\textbf{A2:} Je n'ai pas de fr\`eres et s\oe urs. [``I do not have brothers and sisters.'']\par \vspace{3mm}
        \hline
        
    \item[7.]\textbf{Q:} Comment ils s'appellent? [``What are their names?'']\par 
        \noindent\textbf{A:} Ils s'appellent (\textit{name}), (\textit{name}), $\cdots$, et (\textit{name}). [``Their names are (\textit{name}), (\textit{name}), $\cdots$, and (\textit{name}).'']
        
    \item[8.]\textbf{Q:} Qu'est-ce que tu \'etudies? [``What do you study?''] \par 
        \noindent\textbf{A:} J'\'etudie (\textit{subject}). [``I study (\textit{subject}).'']\par \vspace{3mm}
        \hline
        
    \item[9.]\textbf{Q:} Quel cours as-tu ce semestre? [``What classes do you have this semester?'']\par 
        \noindent\textbf{A:} J'ai (\textit{class}), $\cdots$, et (\textit{class}) ce semestre. [``I have (\textit{class}), $\cdots$, and (\textit{class}) this semester.'']
        
    \end{enumerate}
        
    \newpage
    
    \section{Recreation/Misc.}
    \hline 
    
    \vspace{4mm}
    
    \begin{enumerate}[leftmargin=*]
        \item[10.]\textbf{Q:} Qu'est-ce que tu as fait ce week-end? [``What did you do this weekend?'']\par 
            \noindent\textbf{A:} Ce week-end, j(e/'ai) ('ai dormi/e suis all\'e \`a/'ai regard\'e) (\textit{article}) (\textit{noun}). \par\hspace{5.5mm}[``This weekend, I (slept/went to/watched) (\textit{article}) (\textit{noun}).'']\par \vspace{3mm}
            \hline
            
        \item[11.]\textbf{Q1:} O\`u \^es-tu all\'e? [``Where did you go?'']\par 
            \noindent\textbf{A1:} Je suis all\'e au (\textit{noun}). [``I went to the (\textit{noun}).'']\par 
            \noindent\textbf{Q2:} Qu'est-ce que tu as regard\'e? [``What did you watch?'']\par 
            \noindent\textbf{A2:} J'ai regard\'e (\textit{title}). [``I watched (\textit{title}).'']\par \vspace{3mm}
            \hline
            
        \item[12.]\textbf{Q:} Qu'est-ce que tu as comme passe-temps? [``What do you have as a hobby?'']\par 
            \noindent\textbf{A:} Comme passe-temps, j(e/'ai) (\textit{verb$_{\text{inf}}$}) (\textit{noun})$^*$. [``As a hobby, I (\textit{verb$_{\text{inf}}$}) (\textit{noun})$^*$.'']\par \vspace{3mm}
            \hline
            
        \item[13.]\textbf{Q:} Qu'est-ce tu aimes comme musique? [``What kind of music do you like?'']\par 
            \noindent\textbf{A:} La musique que j'aime c'est Le (\textit{genre}), Le (\textit{genre}), $\dots$, et Le (\textit{genre}).\par\hspace{4.5mm}
            [``The music I like is (\textit{genre}), (\textit{genre}), $\dots$, and (\textit{genre}).'']\par \vspace{3mm}
            \hline
            
        \item[14.]\textbf{Q:} Tu vas souvent au cin\'ema? [``You often go to the cinema?'']\par 
            \noindent\textbf{A1:} Oui, je vais souvent au cin\'ema. [``Yes, I often go to the cinema.'']\par 
            \noindent\textbf{A2:} Non, je ne vais pas souvent au cin\'ema. [``No, I do not go to the cinema often.'']\par \vspace{3mm}
            \hline
            
        \item[15.]\textbf{Q:} Quel est le dernier film que tu as vu? [``What is the last film you saw?'']\par 
            \noindent\textbf{A:} Le dernier film que j'ai vu \'etait (\textit{title}). [``The last film I saw was (\textit{title}).'']\par \vspace{3mm}
            \hline
            
        \item[16.]\textbf{Q:} Tu aimes lire? [``You like to read?'']\par 
            \noindent\textbf{A1:} Oui, j'aime lire. [``Yes, I like to read.'']\par 
            \noindent\textbf{A2:} Non, je n'aime pas lire. [``No, I do not like to read.'']\par \vspace{3mm}
            \hline
            
        \item[17.]\textbf{Q:} Quel est votre livre pr\'ef\'ere? [``What is your favorite book?'']\par 
            \noindent\textbf{A:} Mon livre pr\'ef\'ere est (\textit{title}) de (\textit{author}). [``My favorite book is (\textit{title}) by (\textit{author}).'']\par \vspace{3mm}
            \hline
            
        \item[18.]\textbf{Q:} Est-ce que tu as voyag\'e? [``Have you travelled a lot?'']\par 
            \noindent\textbf{A1:} Oui, j'ai beaucoup voyag\'e. [``Yes, I have travelled a lot.'']\par 
            \noindent\textbf{A2:} Non, je n'ai pas beaucoup voyag\'e. [``No, I have not travelled a lot.'']\par \vspace{3mm}
            \hline
            
        \item[19.]\textbf{Q:} Qu'est-ce que tu regardes \`a la t\'el\'e? [``What do you watch on T.V.?'']\par 
            \noindent\textbf{A:} Sur(?) la t\'el\'e, je regarde (\textit{title}). [``On T.V., I watch (\textit{title}).'']\par \vspace{3mm}
            \hline
            
        \item[20.]\textbf{Q:} Est-ce que tu aimes le fran\c cais? [``Do you like french?'']\par 
            \noindent\textbf{A:} Oui, j'aime le fran\c cais. [``Yes, I like french.'']
    \end{enumerate}

\end{document}