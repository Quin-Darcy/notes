\documentclass[12pt]{article}
\usepackage[margin=1in]{geometry} 
\usepackage{graphicx}
\usepackage{amsmath}
\usepackage{authblk}
\usepackage{titlesec}
\usepackage{amsthm}
\usepackage{amsfonts}
\usepackage{amssymb}
\usepackage{array}
\usepackage{booktabs}
\usepackage{ragged2e}
\usepackage{enumerate}
\usepackage{enumitem}
\usepackage{cleveref}
\usepackage{slashed}
\usepackage{commath}
\usepackage{lipsum}
\usepackage{colonequals}
\usepackage{addfont}
\usepackage{enumitem}
\usepackage{sectsty}
\usepackage{lastpage}
\usepackage{fancyhdr}
\usepackage{accents}
\usepackage{xcolor}
\usepackage[inline]{enumitem}
\pagestyle{fancy}
\setlength{\headheight}{10pt}

\subsectionfont{\itshape}

\newtheorem{theorem}{Theorem}[section]
\newtheorem{corollary}{Corollary}[theorem]
\newtheorem{prop}{Proposition}[section]
\newtheorem{lemma}[theorem]{Lemma}
\theoremstyle{definition}
\newtheorem{definition}{Definition}[section]
\theoremstyle{remark}
\newtheorem*{remark}{Remark}
 
\makeatletter
\renewenvironment{proof}[1][\proofname]{\par
  \pushQED{\qed}%
  \normalfont \topsep6\p@\@plus6\p@\relax
  \list{}{\leftmargin=0mm
          \rightmargin=4mm
          \settowidth{\itemindent}{\itshape#1}%
          \labelwidth=\itemindent
          \parsep=0pt \listparindent=\parindent 
  }
  \item[\hskip\labelsep
        \itshape
    #1\@addpunct{.}]\ignorespaces
}{%
  \popQED\endlist\@endpefalse
}

\newenvironment{solution}[1][\bf{\textit{Solution}}]{\par
  
  \normalfont \topsep6\p@\@plus6\p@\relax
  \list{}{\leftmargin=0mm
          \rightmargin=0mm
          \settowidth{\itemindent}{\itshape#1}%
          \labelwidth=\itemindent
          \parsep=0pt \listparindent=\parindent 
  }
  \item[\hskip\labelsep
        \itshape
    #1\@addpunct{.}]\ignorespaces
}{%
  \popQED\endlist\@endpefalse
}

\let\oldproofname=\proofname
\renewcommand{\proofname}{\bf{\textit{\oldproofname}}}


\newlist{mylist}{enumerate*}{1}
\setlist[mylist]{label=(\alph*)}

\begin{document}

\justifying

% Document begins here



\begin{prop}
    Let $G$ be a group, $a\in G$, and $o(a)=t$. For all $k\in\mathbb{Z}^{+}$, if $a^k=e$, then $t\mid k$.
\end{prop}
    \begin{proof}
        \hfill\par\hfill\par\hfill\par\hfill\par\hfill\par\hfill\par\hfill\par\hfill\par\hfill\par\hfill\par\hfill\par\hfill\par\hfill\par\hfill
    \end{proof}
    
\vspace{10mm}

\begin{prop}
    Let $G$ be a group, and let $a,b\in G$. Assume $ab=ba$. Then for all $n\in\mathbb{Z}^{+}$, $(ab)^n=a^nb^n$.
\end{prop}
    \begin{proof}
        \hfill\par\hfill\par\hfill\par\hfill\par\hfill\par\hfill\par\hfill\par\hfill\par\hfill\par\hfill\par\hfill\par\hfill\par\hfill\par\hfill
    \end{proof}

\newpage
    
\begin{prop}
    Let $G$ be a group, $a,b\in G$, $o(a)=m$, $o(b)=n$, $(m,n)=1$, and $ab=ba$. Then $o(ab)=o(a)o(b)$.
\end{prop}
    \begin{proof}
        \hfill\par\hfill\par\hfill\par\hfill\par\hfill\par\hfill\par\hfill\par\hfill\par\hfill\par\hfill\par\hfill\par\hfill\par\hfill\par\hfill
    \end{proof}
    
\begin{prop}
    Let $G$ be a group, $a,b\in G$, $o(a)=m$, $o(b)=n$, $(m,n)=1$. For all $k,t\in\mathbb{Z}$, if $a^k=b^t$, then $a^k=b^t=e$.
\end{prop}
    \begin{proof}
        \hfill\par\hfill\par\hfill\par\hfill\par\hfill\par\hfill\par\hfill\par\hfill\par\hfill\par\hfill\par\hfill\par\hfill\par\hfill\par\hfill
    \end{proof}



\end{document}