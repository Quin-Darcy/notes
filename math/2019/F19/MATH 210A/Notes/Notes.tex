\documentclass{article}
\usepackage{graphicx}
\usepackage{tikz}
\usepackage{amsmath}
\usepackage{authblk}
\usepackage{titlesec}
\usepackage{amsthm}
\usepackage{amsfonts}
\usepackage{amssymb}
\usepackage{array}
\usepackage{booktabs}
\usepackage{ragged2e}
\usepackage{enumerate}
\usepackage{enumitem}
\usepackage{cleveref}
\usepackage{slashed}
\usepackage{commath}
\usepackage{lipsum}
\usepackage{colonequals}
\usepackage{addfont}
\usepackage{enumitem}
\usepackage{sectsty}
\usepackage{mathtools}
\usepackage{mathrsfs}

\usepackage{hyperref}
\hypersetup{
    colorlinks=true,
    linkcolor=blue,
    filecolor=magenta,      
    urlcolor=cyan,
}

\usetikzlibrary{decorations.pathreplacing}
\usetikzlibrary{arrows.meta}


%\subsectionfont{\itshape}

\newtheorem{theorem}{Theorem}[section]
\newtheorem{corollary}{Corollary}[theorem]
\newtheorem{lemma}{Lemma}[theorem]
\theoremstyle{definition}
\newtheorem{prop}{Proposition}[section]
\newtheorem{definition}{Definition}[section]
\theoremstyle{remark}
\newtheorem*{remark}{Remark}

\let\oldproofname=\proofname
\renewcommand{\proofname}{\bf{\textit{\oldproofname}}}

\newcommand{\closure}[2][3]{%
  {}\mkern#1mu\overline{\mkern-#1mu#2}}

\theoremstyle{definition}
\newtheorem{example}{Example}[section]

\newtheorem*{discussion}{Discussion}

\makeatletter
\renewenvironment{proof}[1][\proofname]{\par
  \pushQED{\qed}%
  \normalfont \topsep6\p@\@plus6\p@\relax
  \list{}{\leftmargin=0mm
          \rightmargin=0mm
          \settowidth{\itemindent}{\itshape#1}%
          \labelwidth=4mm
          \parsep=0pt \listparindent=0mm%\parindent 
  }
  \item[\hskip\labelsep
        \itshape
    #1\@addpunct{.}]\ignorespaces
}{%
  \popQED\endlist\@endpefalse
}

\makeatletter
\renewenvironment{proof*}[1][\proofname]{\par
  \pushQED{\qed}%
  \normalfont \topsep6\p@\@plus6\p@\relax
  \list{}{\leftmargin=0mm
          \rightmargin=0mm
          \settowidth{\itemindent}{\itshape#1}%
          \labelwidth=\itemindent
          \parsep=0pt \listparindent=0mm%\parindent 
  }
  \item[\hskip\labelsep
        \itshape
    #1\@addpunct{.}]\ignorespaces
}{%
  \popQED\endlist\@endpefalse
}

\newenvironment{solution}[1][\bf{\textit{Solution}}]{\par
  
  \normalfont \topsep6\p@\@plus6\p@\relax
  \list{}{\leftmargin=0mm
          \rightmargin=0mm
          \settowidth{\itemindent}{\itshape#1}%
          \labelwidth=\itemindent
          \parsep=0pt \listparindent=\parindent 
  }
  \item[\hskip\labelsep
        \itshape
    #1\@addpunct{.}]\ignorespaces
}{%
  \popQED\endlist\@endpefalse
}


\begin{document}

\title{Notes for MATH 210A}
\author{Quin Darcy}
\date{Sep 28, 2019}
\affil{\small{California State University, Sacramento}}
\maketitle

\section{Exam 1}
    \begin{prop}\label{prop:1.1}
        If $(a,b)=g$, $d\mid a$, and $d\mid b$, then $d\mid g$.
    \end{prop}
        \begin{proof}
            Assume that $(a,b)=g$, $d\mid a$, and $d\mid b$. Since $(a,b)=g$, then there exists $x,y\in\mathbb{Z}$ such that $g=ax+by$. Moreover, since $d\mid a$ and $d\mid b$, then there exists $k_1,k_2\in\mathbb{Z}$ such that $a=k_1d$ and $b=k_2d$. Substituting in for the first equation, we get that $g=(k_1d)x+(k_2d)y=d(k_1x+k_2y)$. Hence, there exists $m\in\mathbb{Z}$ such that $g=dm$, namely $m=(k_1x+k_2y)$. Thus, $d\mid g$.
        \end{proof}
    \begin{prop}\label{prop:1.2}
        If $(a,c)=1$ and $(b,c)=1$, then $(ab,c)=1$.
    \end{prop}
        \begin{proof}
            Assume that $(a,c)=1$ and that $(b,c)=1$. Since $(a,c)=1$ then there exists $x,y\in\mathbb{Z}$ such that $1=ax+cy$ and since $(b,c)=1$, then there exists $v,w\in\mathbb{Z}$ such that $1=bv+cw$. Thus,
        
            \begin{equation*}
                \begin{split}
                    (ax+cy)(bv+cw) &= ab(xv)+ac(xw)+bc(vy)+c^2(yw) \\
                    &=ab(xv)+c(axw+bvy+cyw) \\
                    &= 1.
                \end{split}
            \end{equation*}
        
            Since $xv\in\mathbb{Z}$ and $axw+bvy+cyw\in\mathbb{Z}$, then $(ab,c)=1$.
        \end{proof}
    \begin{prop}\label{prop:1.3}
        If $(a,n)=1$, then there exists $x\in\mathbb{Z}$ such that   $ax\equiv 1(\text{mod }n)$, and $(x,n)=1$.
    \end{prop}
        \begin{proof}
            Assume that $(a,n)=1$. Then it follows that there exists $x,y\in\mathbb{Z}$ such that $ax+ny=1$. Thus, $ax-1=ny$. Thus, $n\mid(ax-1)$. Hence, $ax\equiv 1(\text{mod }n)$. Additionally, since $a\in\mathbb{Z}$ and $ax+ny=1$, then $(x,n)=1$.
        \end{proof}

\newpage
    \begin{prop}\label{prop:1.4}
        Assume $(G,\star)$ is a finite group, and $\abs{G}=n$. If $g\in G$, then $o(g)$ is finite and if $g^t=e$, then $o(g)\mid t$. 
    \end{prop}
        \begin{proof}
             Let $G$ be a finite group with order $n$. Take $g\in G$ and assume $o(g)$ is not finite. Then there does not exist $k\in\mathbb{Z}$ with $k>0$ such that $g^k=e$. Now let $k_1,k_2\in\mathbb{Z}$ where $k_1,k_2>0$ and $k_1\neq k_2$. Assume $g^{k_1}=g^{k_2}$. Then $g^{k_2-k_1}=e$. Since $k_2-k_1\in\mathbb{Z}$, and $o(g)\leq k_2-k_1$, then $o(g)$ is finite. Since this is a contradiction, then it is the case that for all $k_1,k_2\in\mathbb{Z}$ where $k_1,k_2>0$ and $k_1\neq k_2$, then $g^{k_1}\neq g^{k_2}$. Thus, there exists a one-to-one and onto correspondence between $\langle g\rangle$ and $\mathbb{Z}$. However, since $\langle g\rangle\subseteq G$ and $\abs{G}=n$, then we a contradiction and $o(g)$ must be finite.\par\hspace{4mm} Let $t\in\mathbb{Z}$ such that $g^t=e$. Then $g^t=g^{o(g)}$ which implies $g^{t-o(g)}=e$. We now have that either $t=o(g)$, in which case $o(g)\mid t$, or $t\neq o(g)$. If $t\neq o(g)$, then either $o(g)\nmid t$ or $o(g)\mid t$. If $o(g)\nmid t$, then by the division algorithm, there exists $q,r\in\mathbb{Z}$ with $0\leq r<o(g)$, such that $t=q(o(g))+r$. Thus,
    
            \begin{equation*}
                \begin{split}
                    g^t &= g^{q(o(g))+r} \\
                    &= g^{q(o(g))}\star g^r \\
                    &= \big(g^{o(g)}\big)^q\star g^r \\
                    &= e^q\star g^r \\
                    &= g^r.
                \end{split}
            \end{equation*}
    
            However, since $g^t=e$ and $g^r=g^t$, then $g^r=e$. This is a contradiction since $r<o(g)$. Therefore, $o(g)\mid t$.
        \end{proof}
    \begin{prop}\label{prop:1.5}
        If $(G,\star)$ is a group, $a\in G$, $o(a)=n$, $m\in\mathbb{Z}^{+}$, and $d=(m,n)$, then $o(a^m)=o(a^d)$.
    \end{prop}
        \begin{proof}
            Assume $(G,\star)$ is a group, $a\in G$, $o(a)=n$, $m\in\mathbb{Z}^{+}$, and that $d=(m,n)$. Let $t=o(a^d)$ and $y=o(a^m)$. We want to show that $t\mid y$ and $y\mid t$. Since $d=(m,n)$, then $d\mid m$. Thus, there exists $k\in\mathbb{Z}$ such that $m=dk$. Thus, $(a^m)^t=(a^{dk})^t=a^{dkt}=e$. Thus, by Proposition \ref{prop:1.4}, $y\mid t$.\par\hspace{4mm} Since $d=(m,n)$, then there exists $u,v\in\mathbb{Z}$ such that $d=um+vn$. Thus, $(a^d)^y=a^{dy}=a^{(um+vn)y}=a^{umy}\star a^{vny}=(a^m)^{uy}\star(a^n)^{vy}=e^u\star e^{vy}=e$. Thus, by Proposition \ref{prop:1.4}, $t\mid y$. Thus, $t=y$. Therefore, $o(a^m)=o(a^d)$.
        \end{proof}
        
\newpage
    \begin{definition}\label{df:1.1}
        A group $(G,\star)$ \textbf{\textit{acts on a set}} $S$ if and only if there exists $\varphi\colon G\times S\rightarrow S$ (called a \textbf{\textit{group action}}) such that for all $g,h\in G$ and $s\in S$
        
        \begin{enumerate}[label=(\roman*)]
            \item $\varphi((g\star h,s))=\varphi((g,\varphi(h,s)))$
            \item $\varphi((e,s))=s$
        \end{enumerate}
    \end{definition}
    \begin{definition}\label{df:1.2}
        Given a group $(G,\star)$, a set $S$, an action $\varphi$, and $s\in S$, the set $G_s=\{g\in G\colon\varphi((g,s))=s\}$ is called the \textbf{\textit{stabilizer of $s$ in $G$}}, or the \textbf{\textit{isotropy group of $s$}}.
    \end{definition}
    \begin{definition}\label{df:1.3}
        Given $s\in S$, the set $G(s)=\{\varphi((g,s))\colon g\in G\}$ is called the \textbf{\textit{orbit of $s$ in $G$}}. We will also sometimes denote the orbit of $s\in S$ as orb$_{\varphi}(s)$.
    \end{definition}
    \begin{definition}\label{df:1.4}
        Given a group $(G,\star)$ and an element $s\in G$, the set $N(s)=\{g\in G\colon g\star s=s\star g\}$ is called the \textbf{\textit{centralizer}} or \textbf{\textit{normalizer}} of $s$.
    \end{definition}
    
    \begin{prop}\label{prop:1.6}
        Let $(G,\star)$ be a group, $S$ be a set, $\varphi$ be a action of $G$ on $S$, and let $s\in S$, then $G_s\subseteq_g G$.
    \end{prop}
        \begin{proof}
            First we must show that $G_s\neq\varnothing$. Recall, that by Definition \ref{df:1.2}, $G_s=\{g\in G\colon\varphi((g,s))=s\}$. Since $\varphi$ is an action of $G$ on $S$, then by (ii) of Definition \ref{df:1.1}, $\varphi((e,s))=s$, for all $s\in S$. Thus, $e\in G_s$ and therefore $G_s\neq\varnothing$. Next, we want to show that $a\star b^{-1}\in G_s$ whenever $a,b\in G_s$. So let $a,b\in G_s$. It follows that $\varphi((a,s))=s$ and that $\varphi((b,s))=s$. Additionally, since $\varphi((e,s))=s$, then it follows that
            
            \begin{equation*}
                \begin{split}
                    \varphi((a,s)) &= \varphi((a,\varphi((e,s)))) \\
                    &= \varphi((a,\varphi((b^{-1}\star b,s)))) \\
                    &= \varphi((a\star b^{-1}\star b,s)) \\ 
                    &= \varphi(((a\star b^{-1})\star b,s)) \\
                    &= \varphi((a\star b^{-1},\varphi((b,s)))) \\
                    &= \varphi((a\star b^{-1},s)) \\
                    &= s.
                \end{split}
            \end{equation*}
            
            Thus, by the last equality, we have that $\varphi((a\star b^{-1},s))=s$. Hence, $a\star b^{-1}\in G_s$. Therefore, $G_s\subseteq_g G$.
        \end{proof}
        
    \newpage
        
    \begin{definition}\label{df:1.5}
        If $\theta\colon G\rightarrow G$ is an isomorphism, then $\theta$ is called an \textbf{\textit{automorphism}} of $G$. 
    \end{definition}
    
    \begin{prop}\label{prop:1.7}
        The set, $\text{Aut}(G)$, of automorphisms of $G$ equipped with the operation $\circ$ of composition is a group.  
    \end{prop}
        \begin{proof}
            Since the identity map on $G$ is an automorphism, then $i_G\in\text{Aut}(G)$ and thus, $\text{Aut}(G)\neq\varnothing$. Composition of maps is associative, the composition of two bijective maps is a bijective map, and since each map is bijective, it has an inverse which is also in $\text{Aut}(G)$. Thus, $(\text{Aut}(G),\circ)$ is a group.
        \end{proof}
        
    \begin{prop}\label{prop:1.8}
        Assume that $\varphi\colon G\times S\rightarrow S$ is an action of $G$ on $S$. For each $g\in G$, define $\theta_g\colon S\rightarrow S$ by $\theta_g(t)=\varphi((g,t))$. Then $\sigma_g$ is a permutation of $S$.
    \end{prop}
        \begin{proof}
            We want to show that $\sigma_g\colon S\rightarrow S$ is bijective (i.e., invertible). Thus, we need to show that for all $g\in G$, $\sigma_g$ has an inverse, namely $\sigma_{g^{-1}}$. We want $(\sigma_g\circ\sigma_{g^{-1}})(x)=x$, for all $x\in S$. Thus, we let $x\in S$, then 
            
            \begin{equation*}
                \begin{split}
                    (\sigma_g\circ\sigma_{g^{-1}})(x) &= \sigma_g(\sigma_{g^{-1}}(x)) \\
                    &= \sigma_g(\varphi(g^{-1},x)) \\
                    &= \varphi(g,\varphi(g^{-1},x)) \\
                    &= \varphi(gg^{-1},x) \\
                    &= \varphi(e,x) \\
                    &= x.
                \end{split}
            \end{equation*}
            
            Similarly, $(\sigma_{g^{-1}}\circ\sigma_g)(x)=x$. Thus, $\sigma_g$ is invertible and therefore is a bijection. Thus, $\sigma_g$ is a permutation of $S$.
        \end{proof}
        
    \begin{prop}\label{prop:1.9}
        Recall that $A(S)$ denotes the group of all permutations is $S$. Define $\theta\colon G\rightarrow A(S)$ by $\theta(g)=\sigma_g$. Then $\theta$ is a homomorphism.
    \end{prop}
        \begin{proof}
            $\theta$ is a function by definition since, for each $g\in G$. $\theta(g)=\sigma_g\in A(S)$. To show that $\theta$ is well defined, let $g,h\in G$ and assume that $g=h$. Then $\theta(g)=\sigma_g$ and $\theta(h)=\sigma_h$. Let $t\in S$, then $\sigma_g(t)=\varphi(g,t)=\varphi(h,t)=\sigma_h(t)$. Thus, $\sigma_g=\sigma_h$. Thus, $\theta(g)=\theta(h)$ and $\theta$ is therefore well defined.\par\hspace{4mm} Now we will show that $\theta$ is a homomorphism. Let $g,h\in G$. Then $\theta(gh)=\sigma_{gh}$. Thus, for any $x\in S$, we have that 
            
            \begin{equation*}
                \begin{split}
                    \sigma_{gh}(x) &= \varphi(gh,x) \\
                    &= \varphi(g,\varphi(h,x)) \\
                    &= \varphi(g,\sigma_h(x)) \\
                    &= \sigma_g(\sigma_h(x)) \\
                    &= (\sigma_g\circ\sigma_h)(x).
                \end{split}
            \end{equation*}
            
            Thus, $\theta(gh)=\theta(g)\circ\theta(h)$. Therefore, $\theta$ is a homomorphism. The converse of the above is also true. Assume that $\alpha\colon G\rightarrow A(S)$ is a homomorphism, and define $\varphi\colon G\times S\rightarrow S$ by $\varphi((g,s))=(\alpha(g))(s)$. Then $\varphi$ is a group action. \textit{Therefore actions of a group $G$ on a set $S$ and homomorphisms from $G$ into $A(S)$ are essentially the same.}\par\hspace{4mm} Note that
        \end{proof}
        
    \begin{prop}\label{prop:1.10}
        Assume that $(G.\star)$ is a group, $N\triangleleft G$ and that $\varphi\colon G\times N\rightarrow N$ by $\varphi((g,n))=g\star n\star g^{-1}$. Then $\varphi$ is an action on $N$.
    \end{prop}
        \begin{proof}
            Since $N\triangleleft G$, then for all $g\in G$ and $n\in N$, $\varphi(g,n)=g\star n\star g^{-1}\in N$. Thus, $\varphi\colon G\time N\rightarrow N$. Now let $(g,n),(h,m)\in G\times N$ and assume $(g,n)=(h,m)$. Then since this implies that $g=h$ and $n=m$, then $\varphi(g,n)=g\star n\star g^{-1}=h\star m\star h^{-1}=\varphi(h,m)$. Thus, $\varphi$ is well defined.\par\hspace{4mm} We want to show that for all $h,g\in G$ and all $n\in N$ that 
                        
                        \begin{equation*}
                            \varphi(h\star g,n)=\varphi(h,\varphi(g,n))\quad\text{and}\quad\varphi(e,n)=n.
                        \end{equation*}
                        
                        Let $h,g\in G$ and let $n\in N$. Then
                        
                        \begin{equation*}
                            \begin{split}
                                \varphi(h\star g,n) &= (h\star g)\star n\star(h\star g)^{-1} \\
                                &= (h\star g)\star n\star(g^{-1}\star h^{-1}) \\
                                &= h\star (g\star n\star g^{-1})\star h^{-1} \\
                                &= \varphi(h,g\star n\star g^{-1}) \\
                                &= \varphi(h,\varphi(g,n)).
                            \end{split}
                        \end{equation*}
                        
                        Note that the fourth equality holds since $N$ is normal which means that for all $g\in G$ and $n\in N$, $g\star n\star g^{-1}\in N$ and so $(h,g\star n\star g^{-1})\in G\times N$. We have satisfied the first of the two equalities. Now consider
                        
                        \begin{equation*}
                            \varphi(e,n)=e\star n\star e^{-1}=e\star n\star e= n.
                        \end{equation*}
                        
                        Therefore, $\varphi$ is an action of $G$ on $N$.
        \end{proof}
        
    \begin{prop}\label{prop:1.11}
        Let $G$ be a group, $N\triangleleft G$, and $\varphi\colon G\times N\rightarrow N$. By Proposition \ref{prop:1.10}, we know $\varphi$ is an action. Then let $\theta$ be the permutation representation associated with the action $\varphi$. Then $\theta$ is a homomorphism from $G$ to Aut$(N)$. 
    \end{prop}
        \begin{proof}
            By Proposition \ref{prop:1.9}, we know that $\theta\colon G\rightarrow A(N)$ is a homomorphism. We want to show that for each $g\in G$, $\theta(g)=\sigma_g$ is an isomorphism. By definition, we already have that $\sigma_g$ is bijective, and so we need only that $\sigma_g$ is a homomorphism. Thus, we let $m,n\in N$. Then $\sigma_g(mn)=\varphi(g,mn)=gmng^{-1}=(gmg^{-1})(gng^{-1})=\varphi(g,m)\varphi(g,n)$. Thus, $\sigma_g$ is a homomorphsim and thus $\sigma_g$ is an isomorhism from $N$ to $N$. Hence, for all $g\in G$,  $\theta(g)=\sigma_g\in\text{Aut}(N)$. Therefore, $\theta\colon G\rightarrow\text{Aut}(N)$ is a homomorphism.
        \end{proof}
        
    Note that 
    
    \begin{equation*}
        \begin{split}
            \ker\theta &= \{g\in G\mid \theta(g)= i_G\} \\
            &=\{g\in G\mid \sigma_g=i_G\} \\
            &=\{g\in G\mid \forall n\in N\colon \sigma_g(n)=n\} \\
            &= \{g\in G\mid \forall n\in N\colon \varphi(g,n)=n\} \\
            &= \{g\in G\mid\forall n\in N\colon gng^{-1}=n\} \\
            &= \{g\in G\mid \forall n\in N\colon gn=ng\} \\
            &= C_G(N).
        \end{split}
    \end{equation*}
        
    \begin{prop}\label{prop:1.12}
        Assume that $G$ is a group, that $H\triangleleft G$, $N\triangleleft G$, and that $N\cap H=\{e\}$.Then 
        \begin{enumerate}[label=(\roman*)]
            \item For all $n\in N$ and for all $h\in H$, $nh=hn$.
            \item $N\times H\cong NH$.
        \end{enumerate}
    \end{prop}
        \begin{proof}\hfill\par
            \begin{enumerate}[label=(\roman*)]
                \item Let $n\in N$ and let $h\in H$. Then since $N$ is normal in $G$, then $hnh^{-1}\in N$. By closure of $N$, it follows that $(hnh^{-1})n^{-1}\in N$. Similarly, since $h\in H$, then $h^{-1}\in H$. Additionally, since $H$ is normal in $G$, then $nh^{-1}n^{-1}\in H$. By closure, it follows that $h(nh^{-1}n^{-1})\in H$. Thus, $hnh^{-1}n^{-1}\in N\cap H$. However, since $N\cap H=\{e\}$, then it follows that $hnh^{-1}n^{-1}=e$. Thus, $hn=nh$.
                \item Let $f\colon N\times H\rightarrow NH$ be defined by $f\big((n,h)\big)=nh$. We want to show that $f$ is an isomorphism. First we will show that $f$ is well defined. Let $(a,b)=(c,d)$. Then $f\big((a,b)\big)=ab$ and $f\big((c,d)\big)=cd$. Since $a=c$ and $b=d$, by assumption, then $ab=cd$. Thus, $f\big((a,b)\big)=f\big((c,d)\big)$. Thus, $f$ is a function.\par Now we will show that $f$ is a homomorphism. Let $(a,b),(c,d)\in N\times H$. Then $f\big((a,b)\odot(c,d)\big)=f\big((ac,bd)\big)=(ac)(bd)$. However, since $cb=bc$ by (a), then $(ac)(bd)=(ab)(cd)=f\big((a,b)\big)f\big((c,d)\big)$. Thus, $f$ is a homomorphism.\par\hspace{4mm}Now we will show that $f$ is onto. Let $nh\in NH$, then $f\big((n,h)\big)=nh$. Thus, $f$ is onto.\par\hspace{4mm} Assume $f\big((a,b)\big)=f\big((c,d)\big)$. Then $ab=cd$. Thus, $c^{-1}a=db^{-1}$. Since $a,c\in N$, then $c^{-1}a\in N$ and since $b,d\in H$, then $db^{-1}\in H$. Thus, $c^{-1}a\in N\cap H$ and $db^{-1}\in N\cap H$. Thus, $c^{-1}a= e$ and so $a=c$. Similarly, $db^{-1}=e$ and so $d=b$. Thus, $(a,b)=(c,d)$. Thus, $f$ is 1-1. Therefore, $f$ is an isomorphism and so $N\times H\cong NH$.
            \end{enumerate}
        \end{proof}
        
    \begin{prop}\label{prop:1.13}
        Let $G$ be a group, $M\subseteq G$, $N\triangleleft G$, and $N\subseteq M$. Then $M/N\cong G/N$ iff $M\triangleleft G$.
    \end{prop}
        \begin{proof}
        
        \end{proof}
        
    \begin{prop}\label{prop:1.14}
        Assume that $G$ is a group, $N\triangleleft G$, and that $M\triangleleft G$. Then $NM/M\cong N/N\cap M$.
    \end{prop}
        \begin{proof}
        
        \end{proof}
        
    \begin{prop}\label{prop:1.15}
        Let $G$ be a group, $S$ be a set, and let $\varphi\colon G\times S\rightarrow S$ be an action of the group $G$ on $S$. Then the set of orbits of $\varphi$ partition $S$.
    \end{prop}
        \begin{proof}
            Let $O=\{G(s)\mid s\in S\}$. Since $s=\varphi(e,s)$, then $s\in G(s)$ and $G(s)\subseteq \bigcup O$. Thus, $s\in \bigcup O$ and $\bigcup O\neq\varnothing$. Let $s\in S$. Then it follows that $s=\varphi(e,s)\in G(s)$. Since $G(s)\subseteq \bigcup O$, then $s\in\bigcup O$. Let $\varphi(g,s)\in\bigcup O$. Then by Definition \ref{df:1.1}, $\varphi\colon G\times S\rightarrow S$, then $\varphi(g,s)\in S$. Thus, $\bigcup O\subseteq S$. Hence, $\bigcup O=S$.\par\hspace{4mm} Assume $G(s)\cap G(t)\neq\varnothing$. Then there exists $g,h\in G$ such that $\varphi(g,s)=\varphi(h,t)$. By (ii) of Definition \ref{df:1.1}, $\varphi(e,s)=s$. Thus, $\varphi(g^{-1},\varphi(g,s))=\varphi(g^{-1},\varphi(h,t))$. Thus, $\varphi(g^{-1}h,t)=s$. Now let $x\in G(s)$, then there exists $f\in G$, such that $x=\varphi(f,s)$. Thus, $x=\varphi(f,s)=\varphi(f,\varphi(g^{-1}h,t))=\varphi(fg^{-1}h,t)\in G(t)$. Thus, $G(s)\subseteq G(t)$. Similarly, $G(t)\subseteq G(s)$. Thus, $G(s)=G(t)$. Therefore, $O$ is a partition on $S$.
        \end{proof}
        
    \begin{theorem}[Orbit-Stabilizer Theorem]\label{thm:1.1}
        Let $\varphi\colon G\times S\rightarrow S$ be an action of the group $G$ on the set $S$. Then for all $s\in S$, 
        
        \begin{equation*}
            \abs{G(s)}=\frac{\abs{G}}{\abs{G_s}}.
        \end{equation*}
    \end{theorem}
        \begin{proof}
            Define $\theta\colon G/G_s\rightarrow G(s)$ by $\theta(G_sa)=\varphi(a^{-1},s)$. Let $G_sa,G_sb\in G/G_s$ and assume that $G_sa=G_sb$. Then $\theta(G_sa)=\varphi(a^{-1},s)$ and $\theta(G_sb)=\varphi(b^{-1},s)$. Recall that, by (??), $G_sa=G_sb$ iff $ab^{-1}\in G_s$. Thus, $\varphi(ab^{-1},s)=s$. Substituting in for $s$, we get that $\varphi(a^{-1},s)=\varphi(a^{-1},\varphi(ab^{-1},s))$. Since $\varphi$ is an action, then $\varphi(a^{-1},s)=\varphi(a^{-1}ab^{-1},s)=\varphi(b^{-1},s)$. Thus, $\varphi(a^{-1}, S)=\varphi(b^{-1},s)$. Thus, $\theta(G_sa)=\theta(G_sb)$. Hence, $\theta$ is well-defined.\par\hspace{4mm} Now assume that $\theta(G_sa)=\theta(G_sb)$. Then $\varphi(a^{-1},s)=\varphi(b^{-1},s)$. Since $\varphi$ is an action, then $\varphi(e,s)=s$. Thus, $\varphi(aa^{-1},s)=s=\varphi(bb^{-1},s)$. Thus, we have that $\varphi(a,\varphi(a^{-1},s))=\varphi(b,\varphi((b^{-1},s))$. Thus, $\varphi(a,\varphi(b^{-1},s))=\varphi(ab^{-1},s)=s$. Thus, $ab^{-1}\in G_s$.  Thus, $G_sa=G_sb$. Hence, $\theta$ is 1-1.\par\hspace{4mm} Let $\varphi(g,s)\in G(s)$. Then $\theta(G_sg^{-1})=\varphi((g^{-1})^{-1},s)=\varphi(g,s)$. Thus, $\theta$ is onto. Thus, $\theta$ is a well-defined bijection from $G/G_s$ to $G(s)$. Therefore, 
            
            \begin{equation*}
                \abs{G(s)}=\frac{\abs{G}}{\abs{G_s}}.
            \end{equation*}
        \end{proof}
        
    \newpage
        
    \begin{prop}\label{prop:1.16}
        If $\varphi\colon G\times S\rightarrow S$ is an action of a group $G$ on a set $S$. Then $G_s\subseteq_g G$.
    \end{prop}
        \begin{proof}
            It follows from the (ii) of Definition \ref{df:1.1} that $\varphi(e,s)=s$ and so $e\in G_s$. Thus, $G_s\neq\varnothing$. Now let $a,b\in G_s$. Then we have that $\varphi(a,s)=s=\varphi(b,s)$. Thus, $\varphi(e,s)=s$ and so $\varphi(b^{-1}b,s)=\varphi(b^{-1},\varphi(b,s))=\varphi(b^{-1},\varphi(a,s))=\varphi(b^{-1}a,s)=s$. Thus, $\varphi(a,s)=\varphi(a,\varphi(b^{-1}a,s))=\varphi(ab^{-1},\varphi(a,s))=\varphi(ab^{-1},s)=s$. Thus, $ab^{-1}\in G_s$. Therefore, $G_s\subseteq_g G$.
        \end{proof}
            
    \begin{prop}\label{prop:1.17}
        Let $G$ be a group of order $p^n$ and let $\varphi\colon G\times S\rightarrow S$ be an action of the group $G$ on the set $S$. Then given the set
            
        \begin{equation*}
            S_0=\{s\in S\mid\forall g\in G\colon\varphi(g,s)=s\},
        \end{equation*}
            
        the following holds
            
        \begin{equation*}
            \abs{S}\equiv\abs{S_0}(\text{mod }p).
        \end{equation*}
    \end{prop}
        \begin{proof}
            Consider the orbit of some $s\in S$. We have that $G(s)=\{\varphi(g,s)\mid g\in G\}$. Suppose that this set has one element. Then since $\varphi$ is an action, we know that $\varphi(e,s)=s\in G(s)$. Thus, it follows that for all $g\in G$, $\varphi(g,s)=s$. Thus, $s\in S_0$. Similarly, if $s\in S_0$, then $\abs{G(s)}=1$. Thus, $\abs{G(s)}=1$ iff $s\in S_0$.\par\hspace{4mm} Now by Theorem \ref{thm:1.1} we can write $S$ as the disjoint union of all of the orbits of $\varphi$. Thus, $S=S_0\cup G(s_1)\cup \cdots \cup G(s_n)$. With $\abs{G(s_i)}>1$ for all $i$. Hence $\abs{S}=\abs{S_0}+\abs{G(s_1)}+\cdots+\abs{G(s_n)}$. Note that by Theorem \ref{thm:1.1}, $\abs{G(s_i)}$ divides $\abs{G}$. Consequently, $p\mid\abs{G(s_i)}$ for each $i$. Therefore, $\abs{S}\equiv\abs{S_0}(\text{mod }p)$.
        \end{proof}
            
    \begin{theorem}[Cauchy's Theorem]\label{thm:1.2}
        If $G$ is a finite group whose order is divisible by a prime $p$, then $G$ contains an element of order $p$.
    \end{theorem}
        \begin{proof}
            Let $X$ be the set of $p$-tuples of groups elements
            
            \begin{equation*}
                X=\{(a_1,\dots,a_p)\in G^p\mid a_1\cdots a_p=e\}.
            \end{equation*}
            
            It follows that $a_p$ is uniquely determined as $(a_1\cdots a_{p-1})^{-1}$. Thus, $\abs{X}=n^{p-1}$, where $\abs{G}=n$. Since $p\mid n$, then $\abs{X}\equiv 0(\text{mod }p)$.\par\hspace{4mm} Let $\sigma$ be the cycle $(12\cdots p)$ in $S_p$. We can let $\sigma$ act on $X$ by 
            
            \begin{equation*}
                \varphi(\sigma,(a_1,\dots,a_p))=(a_{\sigma(1)},\dots,a_{\sigma(p)})=(a_2,a_3,\dots,a_p,a_1).
            \end{equation*}
            
            Note that $(a_2,\dots,a_p,a_1)\in X$, for $a_1(a_2\cdots a_p)=e$ implies that $a_1=(a_2\cdots a_p)^{-1}$, so $(a_2\cdots a_p)a_1=e$ also. Thus, $\sigma$ acts on $X$, and we consider the subgroup $\langle\sigma\rangle$ of $S_p$ to act on $X$ by iteration in the natural way.\par\hspace{4mm} Now $(a_1,a_2,\dots,a_p)\in S_0$ iff $a_1=\cdots=a_p$; clearly, $(e,e,\dots,e)\in S_0$. Thus, $\abs{S_0}\neq 0$. By Proposition \ref{prop:1.17}, $\abs{X}\equiv\abs{S_0}(\text{mod }p)$. However, $\abs{X}\equiv 0(\text{mod }p)$. Thus, $\abs{S_0}\equiv 0(\text{mod }p)$. Thus, $p\mid\abs{S_0}$. Thus, there exists $a\neq e$ such that $(a,a\dots,a)\in S_0$. Hence, $a^p=e$. Thus, $o(a)=p$.
        \end{proof}
        
    \newpage
    
    \begin{definition}\label{df:1.6}
        A group in which every element has order a power ($\geq 0$) of some fixed prime $p$ is called a \textbf{\textit{p-group}}. If $H$ is a subgroup of a group $G$ and $H$ is a $p$-group, $H$ is said to be a \textbf{\textit{p-subgroup}} of $G$.
    \end{definition}
        
    \begin{corollary}\label{cor:1.2.1}
        A finite group $G$ is a $p$-group if and only if $\abs{G}$ is a power of $p$.
    \end{corollary}
        \begin{proof}
            If $G$ is a $p$-group and $q$ is a prime which divides $\abs{G}$, then $G$ contains an element of order $q$ by Theorem \ref{thm:1.2}. Since every element of $G$ has order a power of $p$, then for the element of order $q$, it must be the case that $q=p^{\alpha}$. However, for any $1\leq\alpha$, it would follow that $q$ is not prime. Thus, $q=p$. Hence, $\abs{G}$ is a power of $p$.\par\hspace{4mm} Assume $\abs{G}$ is a power of $p$. Then take any $a\in G$ and consider the subgroup generated by $a$, $\langle a\rangle$. By Langrange's Theorem, it follows that $\abs{\langle a\rangle}\mid\abs{G}$. Thus, if we denote $\abs{G}=p^n$, then $\abs{\langle a\rangle}\mid p^n$. Thus, $\abs{\langle a\rangle}=p^m$, for some $0\leq m\leq n$. Therefore, every element of $G$ has order ($\geq 0$) of $p$ and $G$ is thereby a $p$-group.
        \end{proof}
        
    \begin{lemma}\label{lem:1.2.1}
        If $H$ is a $p$-subgroup of a finite group $G$, then
        
        \begin{equation*}
            [N_G(H)\colon H]=[G\colon H](\text{mod }p).
        \end{equation*}
    \end{lemma}
        \begin{proof}
            Let $S$ be the set of left cosets of $H$ in $G$ and define $\varphi\colon H\times S\rightarrow S$ by $\varphi(h,aH)=haH$. It is easily shown that $\varphi$ is an action. Then $\abs{S}=[G\colon H]$. Also, note that $xH\in S_0$ iff $hxH=xH$ for all $h\in H$. This is equivalent to $x^{-1}hxH=H$ for all $h\in H$. Moreover, this is equivalent to $x^{-1}hx\in H$ for all $h\in H$. Thus, $x^{-1}Hx=H$. Thus, $xHx^{-1}=H$. Hence, $x\in N_G(H)$. Therefore, $\abs{S_0}$ is the number of cosets $xH$ with $x\in N_G(H)$. That is, $\abs{S_0}=[N_G(H)\colon H]$. By Proposition \ref{prop:1.17}, $[N_G(H)\colon H]=\abs{S_0}\equiv\abs{S}=[G\colon H](\text{mod }p)$.
        \end{proof}
        
    \begin{corollary}\label{cor:1.2.2}
        If $H$ is a $p$-subgroup of a finite group $G$ such that $p$ divides $[G\colon H]$, then $N_G(H)\neq H$.
    \end{corollary}
        \begin{proof}
            Since $p\mid[G\colon H]$ then $[G\colon H]\equiv 0(\text{mod }p)$. Additionally, by Lemma \ref{lem:1.2.1}, we have that $[N_G(H)\colon H]\equiv[G\colon H](\text{mod }p)$. Thus, $[N_G(H)\colon H]\equiv 0(\text{mod }p)$. Hence, $p\mid [N_G(H)\colon H]$ and it is always the case that $[N_G(H)\colon H]\geq 1$ since $H\subseteq_g N_G(H)$. Thus, there are at least $p$ many cosets of $H$ in $N_G(H)$. Therefore, $N_G(H)\neq H$.
        \end{proof}
        
    \begin{corollary}\label{cor:1.2.3}
        If $H$ is a $p$-subgroup of a finite group $G$ such that $p\mid[G\colon H]$, then $N_G(H)\neq H$.
    \end{corollary}
        \begin{proof}
            By Lemma \ref{lem:1.2.1}, $[N_G(H)\colon H]\equiv[G\colon H](\text{mod p})$. However, $p\mid[G\colon H]$ and thus $[G\colon H]\equiv 0(\text{mod }p)$. Hence, $[N_G(H)\colon H]\equiv 0(\text{mod }p)$. Thus, $p\mid[N_G(H)\colon H]$. Since $H\subseteq_gN_G(H)$, then $H\in N_G(H)/H$ and $[N_G(H)\colon H]\geq 1$. Since $p\mid[N_G(H)\colon H]$, then $[N_G(H)\colon H]>1$ and thus, $N_G(H)\neq H$, otherwise $[N_G(H)\colon H]=1$.
        \end{proof}
        
    \begin{prop}\label{prop:1.18}
        If $N$ is a normal subgroup of a group $G$, then every subgroup of $G/N$ is of the form $K/N$, where $K$ is a subgroup of $G$ that contains $N$. Furthermore, $K/N$ is normal in $G/N$ if and only if $K$ is normal in $G$.
    \end{prop}
        \begin{proof}
        
        \end{proof}
        
    \begin{prop}\label{prop:1.19}
        Let $G$ be a group and $H$ be a subgroup of $G$. Then $N_G(H)\subseteq_g G$.
    \end{prop}
        \begin{proof}
            Let $\varphi\colon G\times S\rightarrow S$, where $S$ is the set of all subgroups of $G$, be the group action defined by conjugation, then for any $H\subseteq_g G$, $G_H=\{g\in G\colon gH=Hg\}=N_G(H)$. Then by Proposition \ref{prop:1.6}, $G_H\subseteq_g G$. Therefore, $N_G(H)\subseteq_g G$. 
        \end{proof}
        
    \begin{prop}\label{prop:1.20}
        If $G$ is a group and $H$ is a subgroup of $G$. Then $H$ is normal in $N_G(H)$.
    \end{prop}
        \begin{proof}
            Let $h\in H$. Then since $hH=H=Hh$, it follows that $H\subseteq N_G(H)$. Thus, since $H$ is a subgroup of $G$, and by Proposition \ref{prop:1.19} $N_G(H)$ is a subgroup of $G$, then $H\subseteq N_G(H)$ implies that $H\subseteq_g N_G(H)$. Now let $g\in N_G(H)$. Then $gH=Hg$, Thus, $H$ is normal in $N_G(H)$.
        \end{proof}
        
    \begin{theorem}[First Sylow Theorem]\label{thm:1.3}
        Let $G$ be a group of order $p^nm$, with $n\geq 1$, $p$ prime, and $(p,m)=1$. Then $G$ contains a subgroup of order $p^i$ for each $1\leq i\leq n$ and every subgroup of $G$ of order $p^i$ $(i<n)$ is normal in some subgroup of order $p^{i+1}$.
    \end{theorem}
        \begin{proof}
            Since $p\mid\abs{G}$, then by Theorem \ref{thm:1.2}, there exists $a\in G$ such that $o(a)=p$. Thus, we have that $\langle a\rangle\subseteq_g G$ and $o(\langle a\rangle)=p$. Proceeding by induction assume $H$ is a subgroup of $G$ of order $p^i$ $(1\leq i<n)$. Then since $G$ is finite, by Lagrange's Theorem , $[G\colon H]=\abs{G}/\abs{H}=p^{n-i}$, where $0<p-i$ and so $p\mid[G\colon H]$. Thus, by Lemma \ref{lem:1.2.1}, $[N_G(H)\colon H]\equiv[G\colon H]\equiv 0(\text{mod }p)$. Moreover, by Corollary \ref{cor:1.2.2}, $N_G(H)\neq H$. Thus, $[N_G(H)\colon H]>1$. Hence, $[N_G(H)\colon H]\equiv 0(\text{mod }p)$ implies that $p\mid[N_G(H)\colon H]$. Also note that by Proposition \ref{prop:1.20}, $H\triangleleft N_G(H)$. Thus, $N_G(H)/H$ is a group by Proposition ??. Since $p$ divides the order of this group, then by Theorem \ref{thm:1.2}, there exists an element and hence a subgroup (generated by that element) of order $p$. By Proposition \ref{prop:1.18}, this subgroup is of the form $H_1/H$ where $H_1$ is a subgroup of $N_G(H)$ containing $H$. Since $H$ is normal in $N_G(H)$, then it follows that $H$ is normal in $H_1$. Finally, $\abs{H_1}=\abs{H}\abs{H_1/H}=p^ip=p^{i+1}$.
        \end{proof}
        
    \begin{definition}\label{def:1.7}
        A subgroup $P$ of a group $G$ is said to be a \textbf{\textit{Sylow p-subgroup}} ($p$ prime) if $P$ is a maximal $p$ subgroup of $G$. That is, if $P\subseteq_g H\subseteq_g G$, then if $H$ is a $p$-subgroup, then $P=H$.
    \end{definition}
    
    \newpage
        
    \begin{corollary}\label{cor:1.3.1}
        Let $G$ be a group of order $p^nm$ with $p$ prime, $n\geq 1$ and $(m,p)=1$. Let $H$ be a $p$-subgroup of $G$.
        \begin{enumerate}[label=(\roman*)]
            \item $H$ is a Sylow $p$-subgroup of $G$ if and only if $\abs{H}=p^n$.
            \item Every conjugate of a Sylow $p$-subgroup is a Sylow $p$.
            \item If there is only one Sylow $p$-subgroup $P$, then $P$ is normal in $G$.
        \end{enumerate}
    \end{corollary}
        \begin{proof}\hfill\par
            \begin{enumerate}[label=(\roman*)]
                \item Assume that $H$ is a Sylow $p$-subgroup of $G$. Then by Definition \ref{def:1.7}, $H$ is a $p$-subgroup of $G$. Thus, by Definition \ref{df:1.6}, $\abs{H}=p^i$ for some $i\geq 0$. Now assume that $P$ is a subgroup of $G$ of order $p^n$ such that $H\subseteq_g P$. The subgroup $P$ exists by Theorem \ref{thm:1.3}. However, since $H$ is a Sylow $p$-subgorup, then $H=P$. Thus, $\abs{H}=p^n$. Now assume that $\abs{H}=p^n$. Then by Langrange's Theorem, $\abs{H}\mid\abs{G}$ and $\abs{G}/\abs{H}=m$. Since $(m,p)=1$, then $H$ is a maximal $p$-subgroup of $G$. Thus, $H$ is a Sylow $p$-subgroup of $G$.
                \item Let $P$ be a Sylow $p$-subgroup of $G$ and let $g\in G$. Then the set $gPg^{-1}\neq\varnothing$ since $e\in P$ and thus $geg^{-1}=e\in gPg^{-1}$. Now let $a,b\in G$ and assume that $a,b\in gPg^{-1}$. Then $a=gh_1g^{-1}$ and $b=gh_2g^{-1}$ for some $h_1,h_2\in P$. Thus, $b^{-1}=gh_2^{-1}g^{-1}$. Thus, $ab^{-1}=(gh_1g^{-1})(gh_2^{-1}g^{-1})=g(h_1h_2^{-1})g^{-1}$. Since $h_1h_2^{-1}\in P$, then $ab^{-1}\in gPg^{-1}$. Thus, $gPg^{-1}\subseteq_g G$.\par\hspace{4mm} Now let $f\colon P\rightarrow gPg^{-1}$ be defined by $f(h)=ghg^{-1}$. Let $h,t\in P$ and assume $h=t$. Then $f(h)=ghg^{-1}=gtg^{-1}=f(t)$. Thus, $f$ is well defined. Now assume that $f(h)=f(t)$. Then $ghg^{-1}=gtg^{-1}$. Thus, $gh=gt$ and so $h=t$. Hence, $f$ is 1-1. Now let $ghg^{-1}\in gPg^{-1}$. Then $f(g)=ghg^{-1}$ and thus $f$ is onto. Hence, $f$ is a bijection. Therefore, $\abs{P}=\abs{gPg^{-1}}$. Thus, given any Sylow $p$-subgroup, then for any $g\in G$, the conjugate $gPg^{-1}$ is also a Sylow $p$-subgroup.
                \item Assume that there is only one Sylow $p$-subgroup $P$. Then by (ii), for all $g\in G$, $gPg^{-1}$ is another Sylow $p$-subgroup. However, since there is only one, then it follows that $gPg^{-1}=P$. Thus, $P$ is normal in $G$.
            \end{enumerate}
        \end{proof}
        
    \newpage
        
    \begin{theorem}[Sylow 1, Version 2]\label{thm:1.4}
        Let $G$ be a group of order $n=p^{\alpha}m$, where $1\leq\alpha$, and $(p,m)=1$. Then there exists $H\subseteq_g G$ such that $o(H)=p^{\alpha}$. 
    \end{theorem}
        \begin{proof}
            Let $\mathscr{C}=\{C\subseteq G\mid \abs{C}=p^{\alpha}\}$. It follows from this definition that $\abs{\mathscr{C}}={n\choose p^{\alpha}}$. Expanding this out, we see that 
                
            \begin{equation*}
                \begin{split}
                    \abs{\mathscr{C}} &= \frac{n!}{(n-p^{\alpha})!(p^{\alpha})!}.
                \end{split}
            \end{equation*}
                
            After cancelling out the appropriate terms, we obtain
                
            \begin{equation*}
                \abs{\mathscr{C}}&=\frac{n(n-1)\cdots(n-p^{\alpha}+1)}{p^{\alpha}(p^{\alpha}-1)\cdots(p^{\alpha}-p^{\alpha}+1)}=\prod\limits_{k=0}^{p^{\alpha}-1}\frac{(n-k)}{(p^{\alpha}-k)}.
            \end{equation*}
                
            We will replace $n$ with $p^{\alpha}m$ and we note that for each $0\leq k\leq p^{\alpha}-1$, we can write $k=p^i L$, where $0\leq i<\alpha$ is the highest power of $p$ that occurs in the prime factorization of $k$. It also follows that $p\nmid L$. Using these substitutions, the above equality becomes
                
            \begin{equation*}
                \abs{\mathscr{C}} =\prod\limits_{k=0}^{p^{\alpha}-1}\frac{(p^{\alpha}m-p^iL)}{(p^{\alpha}-p^iL)} 
                = \prod\limits_{k=0}^{p^{\alpha}-1}\frac{(p^{\alpha-i}m-L)}{(p^{\alpha-i}-L)} 
                =\frac{\prod\limits_{k=0}^{p^{\alpha}-1}(p^{\alpha-i}m-L)}{\prod\limits_{k=0}^{p^{\alpha}-1}(p^{\alpha-i}-L)}.
            \end{equation*}
                
            Now let $p^{\alpha-i}m-L$ be any single term in the product $\prod\limits_{k=0}^{p^{\alpha}-1}(p^{\alpha-i}m-L)$. Since $i<\alpha$ and $1<\alpha$, then $1\leq \alpha-i$ and thus, $p\mid(p^{\alpha-i}m)$. However, as stated earlier, $p\nmid L$. Thus, $p\nmid (p^{\alpha-i}m-L)$. Since $p^{\alpha-i}m-L)$ was arbitrary, then it follows that
                
            \begin{equation*}
                p\nmid\bigg(\prod\limits_{k=0}^{p^{\alpha}-1}(p^{\alpha-i}m-L\bigg).
            \end{equation*}
                
            Thus, 
                
            \begin{equation*}
                p\nmid\bigg(\frac{\prod\limits_{k=0}^{p^{\alpha}-1}(p^{\alpha-i}m-L)}{\prod\limits_{k=0}^{p^{\alpha}-1}(p^{\alpha-i}-L)}\bigg).
            \end{equation*}
                
            Therefore, $p\nmid\abs{\mathscr{C}}$.\par\hspace{4mm} Now define the group action (prove) $\varphi\colon G\times\mathscr{C}\rightarrow\mathscr{C}$ by $\varphi(g,C)=gC$. By Proposition \ref{prop:1.15}, we can partition $\mathscr{C}$ by the collection of orbits of $\varphi$. Thus, it follows that
                
            \begin{equation*}
                \abs{\mathscr{C}}=\sum_{C\in\mathscr{C}}\abs{G(\hat{C})}.
            \end{equation*}
                
            Since $p\nmid\abs{\mathscr{C}}$, then there exists $\hat{C}\in\mathscr{C}$ such that $p\nmid\abs{G(\hat{C})}$. By Theorem \ref{thm:1.1}, 
            
            \begin{equation*}
                \abs{G(\hat{C})}=\frac{\abs{G}}{\abs{G_{\hat{C}}}}=\frac{p^{\alpha}m}{\abs{G_{\hat{C}}}}.
            \end{equation*}
            
            Thus, $\abs{G(\hat{C})}\abs{G_{\hat{C}}}=p^{\alpha}m$. However, since $p\nmid\abs{G(\hat{C})}$, then it follows that $p^{\alpha}\nmid\abs{G(\hat{C})}$. Thus, $p^{\alpha}\mid\abs{G_{\hat{C}}}$. Now let $x\in\hat{C}$, then it follows that $G_{\hat{C}}x\subseteq C$. Recall that $\abs{\hat{C}}=p^{\alpha}$. Thus, $\abs{G_{\hat{C}}x}=\abs{G_{\hat{C}}}\leq\abs{\hat{C}}=p^{\alpha}$. Thus, if $p^{\alpha}\mid\abs{G_{\hat{C}}}$ and $\abs{G_{\hat{C}}}\leq p^{\alpha}$, then it follows that $\abs{G_{\hat{C}}}=p^{\alpha}$. By Proposition \ref{prop:1.16}, $G_{\hat{C}}\subseteq_g G$. Therefore, there exists a subgroup, namely $G_{\hat{C}}$ of $G$ with order $p^{\alpha}$.    
        \end{proof}
        
    \begin{theorem}[Second Sylow Theorem]\label{thm:1.5}
        If $H$ is a $p$-subgroup of a finite group $G$, and $P$ is any Sylow $p$-subgroup of $G$, then there exists $x\in G$ such that $H\subseteq xPx^{-1}$. In particular, any two Sylow $p$-subgroups are conjugate.
    \end{theorem}
        \begin{proof}
            Let $S$ be the set of left cosets of $P$ in $G$ and let $\varphi\colon H\times S\rightarrow S$ be an action defined by $\varphi(h,aP)=haP$. Then by Proposition \ref{prop:1.17}, $\abs{S}\equiv\abs{S_0}(\text{mod }p)$. Since $\abs{S}=[G\colon P]$, then $\abs{S_0}\equiv[G\colon P](\text{mod }p)$. But since $P$ is a Sylow $p$-subgroup of $G$, then $\abs{P}=p^n$, where $p^n$ is the highest power of $p$ that occurs in the prime factorization of $\abs{G}$. Thus, $p\nmid[G\colon P]$. Hence, $p\nmid\abs{S_0}$. Thus, $\abs{S_0}\neq\varnothing$ and there exists $xP\in S_0$. Thus, by definition of $S_0$, $xP\in S$ and for all $h\in H$, $hxP=xP$. Thus, $x^{-1}hxP=P$ for all $h\in H$. Thus, since $x^{-1}hx\in x^{-1}Hx$, then $x^{-1}Hx\subseteq P$. Moreover, since $x^{-1}Hx\subseteq_g G$, then $x^{-1}Hx\subseteq_g P$. Thus, $H\subseteq_g xPx^{-1}$. Hence, if $H$ is a Sylow $p$-subgroup, $\abs{H}=\abs{P}=\abs{xPx^{-1}}$. Thus, $H=xPx^{-1}$.
        \end{proof}
        
    \begin{theorem}[Third Sylow Theorem]\label{thm:1.6}
        If $G$ is a finite group and $p$ is a prime, then the number of Sylow $p$-subgroups of $G$ divides $\abs{G}$ and is of the form $kp+1$ for some $k\geq 0$. 
    \end{theorem}
        \begin{proof}
            By Theorem \ref{thm:1.5}, the number of Sylow $p$-subgroups is the number of conjugates of any one of them, say $P$. But this number is $[G\colon N_G(P)]$. Since, $\abs{G}=(\abs{G}/\abs{N_G(P)})\abs{N_G(P)}$. Hence, $[G\colon N_G(P)]$ is a divisor of $\abs{G}$. Thus, the number of Sylow $p$-subgroups is a divisor of $G$. Now let $\gamma\colon P\times\text{Syl}_P(G)\rightarrow\text{Syl}_P(G)$ be the action of conjugation. Then assume $Q\in S_0$. Then $Q\in\text{Syl}_P(G)$ and for all $g\in P$, $gQg^{-1}=Q$. Now consider $N_G(Q)=\{g\in G\mid gQg^{-1}=Q\}$. It follows that $P\subseteq_g N_G(Q)$. Since both $P$ and $Q$ are Sylow $p$-subgroups of $G$ and hence of $N_G(Q)$ and are therefore conjugate in $N_G(Q)$. But since by Proposition \ref{prop:1.20}, $Q$ is normal in $N_G(Q)$. Thus, for all $g\in N_G(Q)$, $gQg^{-1}=Q$. Thus, $P=Q$. Hence, $S_0=\{P\}$. By Proposition \ref{prop:1.17}, $\abs{\text{Syl}_P(G)}\equiv\abs{S_0}=1(\text{mod }p)$. Thus, $\abs{\text{Syl}_P(G)}=kp+1$.
        \end{proof}
        
    \newpage
    
    \section{Results To Be Proven and Applications}
    
    \begin{prop}\label{prop:2.1}
        If $G$ is a finite group, $H\subseteq_g G$, $H\neq G$, and $o(G)\nmid[G\colon H]!$, then $H$ contains a nontrivial normal subgroup of $G$.
    \end{prop}
        \begin{proof}
        
        \end{proof}
        
    \begin{prop}\label{prop:2.2}
        If $G$ is a finite group, $H\subseteq_g G$, $[G\colon G]=p$, $p$ is prime, and $p$ is the smallest prime factor of $o(G)$, then $H\triangleleft G$. 
    \end{prop}
        \begin{proof}
        
        \end{proof}
        
    \begin{prop}\label{prop:2.3}
        If $H\subseteq_g G$ and $N\triangleleft G$, then $HN=NH$ is a subgroup of $G$, and $o(HN)=o(H)o(N)/o(H\cap N)$. Therefore, if $H\cap N=\{e\}$, then $o(HN)=o(H)o(N)$.
    \end{prop}
        \begin{proof}
        
        \end{proof}
        
    \begin{prop}\label{prop:2.4}
        If $N\triangleleft G$ and $M\triangleleft G$ and $N\cap M=\{e\}$, then $NM\cong N\times M$.
    \end{prop}
        \begin{proof}
        
        \end{proof}
        
    \begin{prop}\label{prop:2.5}
        Assume that $G=\langle a\rangle$, and $o(G)=n$. Prove that $\text{Aut}(G)\cong (\mathbb{Z}_{(n)},\odot)$.
    \end{prop}
        \begin{proof}
            Let $f\colon\text{Aut}(G)\rightarrow\mathbb{Z}_{(n)}$ be defined as $f(\theta)=[k]$ where $\theta(a)=a^k$, where $k\in\mathbb{Z}$ such that $(n,k)=1$. By definition, $f$ is defined over Aut$(G)$. Next, we will check if $f$ is well defined. Let $\theta,\gamma\in\text{Aut}(G)$ and assume $\theta=\gamma$. Then we want to show that $f(\theta)=f(\gamma)$. We have that $\theta(a)=a^i$ and $\gamma(a)=a^j$. Thus, $f(\theta)=[i]$ and $f(\gamma)=[j]$. Since $\theta=\gamma$, then $a^i=a^j$. Thus, $\theta(a)=a^j$. Hence, $f(\theta)=[j]=f(\gamma)$. So $f$ is well defined.\par\hspace{4mm} Now assume that $f(\theta)=f(\gamma)$. Then $[i]=[j]$. Thus, $i\in[j]$ and so there exists some $m\in\mathbb{Z}$ such that $i=mn+j$. Thus, $a^{i}=a^{mn+j}=a^{mn}a^j=a^j$. It follows that $\theta(a)=\gamma(a)$, and since $a$ generates $G$, then $\theta=\gamma$. Hence, $f$ is 1-1.\par\hspace{4mm} Now let $[k]\in\mathbb{Z}_{(n)}$. Now consider some $\varphi$ whose domain is $G$, where $\varphi(a)=a^k$. Then we want to show that $\varphi\in\text{Aut}(G)$. By definition, $\text{ran}(\varphi)\subseteq G$ and so $\varphi\colon G\rightarrow G$. Now assume that $x,y\in G$ and that $x=y$. Then since $G=\langle a\rangle$, then $x=a^s$ and $y=a^t$, for some $s,t\in\mathbb{Z}$. Thus, $\varphi(x)=\varphi(a^s)=(a^s)^k$, and $\varphi(y)=\varphi(a^t)=(a^t)^k$. But since $a^s=a^t$, then $(a^s)^k=(a^t)^k$. Thus, $\varphi(x)=\varphi(y)$. Thus, $\varphi$ is well defined. Now let $a^i,a^j\in G$. Then $\varphi(a^ia^j)=\varphi(a^{i+j})=(a^{i+j})^k=a^{ik}a^{jk}=\varphi(a^i)\varphi(a^j)$. Thus, $\varphi$ is a homomorphism. Now assume $\varphi(x)=\varphi(y)$. Thus, $(a^s)^k=(a^t)^k$, for some $s,t\in\mathbb{Z}$. Thus, $a^{k(s-t)}=e$. Then $n\mid k(s-t)$, Since $(n,k)=1$, then $n\mid(s-t)$. Thus, $s=qn+t$, for some $q\in\mathbb{Z}$. Thus, $a^s=a^{qn+t}=a^{qn}a^t=a^t$. Hence, $x=y$ and $\varphi$ is 1-1. Since $(n,k)=1$, then $\langle a\rangle=\langle a^k\rangle$. Thus, $\varphi$ is onto. Therefore, $\varphi\in\text{Aut}(G)$ and $f(\varphi)=[k]$. Thus, $f$ is onto.\par\hspace{4mm} Now we want to show that $f$ is a homomorphism. Let $\theta,\gamma\in\text{Aut}(G)$ and suppose $\theta(a)=a^i$ and $\gamma(a)=a^j$. Then we have that $(\theta\circ\gamma)(a)=\theta(\gamma(a))=\theta(a^j)=(a^j)^i=a^{ji}$. Then $f(\theta\circ\gamma)=[ji]=[j]\odot[i]=f(\gamma)\odot f(\theta)=f(\theta)\odot f(\gamma)$. Thus, $f$ is a homomorphism. Therefore, $f$ is an isomorphism. Hence, $\text{Aut}(G)\cong(\mathbb{Z}_{(n)},\odot)$. 
        \end{proof}
        
    \begin{prop}\label{prop:2.6}
        If $H\subseteq_g G$, $N\triangleleft G$, $H\cap N=\{e\}$, and $G=HN$, then for every $x\in G$, there exist unique elements $h\in H$ and $n\in N$ such that $x=hn$.
    \end{prop}
        \begin{proof}
        
        \end{proof}
        
    \begin{remark}
        By Proposition \ref{prop:2.6}, if there exist $h'\in H$, $n'\in N$ such that $x=h'n'$, then $hn=h'n'$. Thus, $h'^{-1}h=n'n^{-1}$. Since $h'^{-1}h\in H$ and $n'n^{-1}\in N$, then $h'^{-1}h\in H\cap N$ and $n'n^{-1}\in H\cap N$. But $H\cap N=\{e\}$. Thus, $h'^{-1}h=e$ and $n'n^{-1}=e$. Thus, $h=h'$ and $n=n'$.\par Thus, if $x,y\in G$, then there exist unique elements $g,h\in H$ and $m,n\in N$ such that $x=gm$ and $y=hn$, and $xy=gmhn=gh(h^{-1}mh)n$. Now define $\sigma_h\colon N\rightarrow N$ by $\sigma(h)=hnh^{-1}$, then by Proposition \ref{prop:1.11}, $\sigma_h\in\text{Aut}(N)$. Note that $xy$ can be written as $xy=gh(\sigma_{h^{-1}}(m))n$. Now define $\alpha\colon H\rightarrow\text{Aut}(N)$ by $\alpha(h)=\sigma_h$, then by Proposition \ref{prop:1.11}, $\alpha$ is a homomorphsim. Conversely, given any homormorphism $\theta\colon H\rightarrow\text{Aut}(N)$, we can determine the structure of $HN$ by determining the possible values of $\theta(h)$.\par\hspace{4mm} Assume that $o(a)=n$. Recall that by Proposition \ref{prop:2.5}, $\text{Aut}(\langle a\rangle)\cong\mathbb{Z}_{(n)}$. Thus, $\text{Aut}(\langle a\rangle)=\{\varphi_k\colon (n,k)=1\}$, where $\varphi_k\colon \langle a\rangle\rightarrow\langle a\rangle$ is defined by $\varphi_k(x)=x^k$.
    \end{remark}
    
    \begin{example}
        Let $N=\langle d\rangle$, $o(d)=m$, and let $H=\langle a\rangle$, $o(a)=2$. Then $\text{Aut}(N)=\{\varphi_k\colon (k,m)=1\}$, where $\varphi_k(x)=x^k$. Assume that $\theta\colon H\rightarrow\text{Aut}(N)$ is a homomorphism, and that $\theta(h)=\varphi_k$.\par If $h=e$, then since $\theta$ is a homomorphism, it must map identities to identities. Thus, $\theta(h)=i_N$. If $h=a$, then since $o(a)=2$, then $o(\theta(a))\mid 2$. Thus, $o(\theta(a))=1$ or $o(\theta(a))=2$. Thus, if $\theta(a)=\varphi_k$, then it follows that $\varphi_k^2=i_N$. But also $(\varphi_k)^2(x)=\varphi_k(\varphi_k(x))=\varphi_k(x^k)=(x^k)^k=x^{k^2}=\varphi_{k^2}(x)$. Thus, $\varphi_{k^{2}}=i_N$.\par If $m=3$, then $\text{Aut}(N)=\{\varphi_k\colon (3,k)=1\}=\{\varphi_1,\varphi_2\}$. Moreover, $\varphi_{1^2}=\varphi_1=i_N$ and $\varphi_{2^2}=\varphi_4=\varphi_1=i_N$. Thus, both $\varphi_1$ and $\varphi_2$ work.\par If $m=5$, then $\text{Aut}(N)=\{\varphi_1,\varphi_2,\varphi_3,\varphi_4\}$. However, $\varphi_{2^2}=\varphi_4\neq i_N$, and $\varphi_{3^2}=\varphi_9=\varphi=\varphi_4\neq i_N$. Thus, only $\varphi_1$ and $\varphi_4$ work.
    \end{example}
    
    \begin{example}
        Let $G$ be a group of order 6. Then since $6=2\times 3$, by Theorem \ref{thm:1.6} the number of Sylow $2$-subgroups, $n_2$ is congruent to 1 modulo $2$ and the number of Sylow 3-subgroups is congruent to 1 modulo 3. Additionally, the number of Sylow 2-subgroups divides $\abs{G}=6$ and the number of Sylow 3-subgroups divides $\abs{G}=6$. Thus, $n_2\equiv 1(\text{mod }2)$, $n_2\mid 6$, $n_3\equiv 1(\text{mod }3)$, and $n_3\mid 6$. Thus, Since 6 has factors 1, 2, 3, and 6. Then amongst these, 1 and 3 are congruent to 1 modulo 2. Thus, $n_2=1$ or $n_2=3$. Similarly, amongst the factors 1, 2, 3, and 6, those of which are congruent to 1 modulo 3 is just 1. Thus, $n_3=1$. Hence, by (iii) of Corollary \ref{cor:1.3.1}, $P_3\triangleleft G$, where $P_3$ is the Sylow 3-subgroup.\par Assume that $n_2=1$, then $P_2\triangleleft G$, then by Proposition \ref{prop:2.4}, $G\cong\mathbb{Z}_2\times\mathbb{Z}_3\cong\mathbb{Z}_6$ since any groups whose order is less than or equal to 5 is abelian, and any finite abelian group of order $n$ is isomorphic to $\mathbb{Z}_n$. Thus, $G$ is cyclic.\par Now assume that $n_2=3$. Then since $P_2\cap P_3=\{e\}$ (since $P_2$ contains the identity and an element of order 2, and $P_3$ contains the identity and 2 elements of order 3) and $G=P_2P_3$ by Proposition ??. Additionally, $\text{Aut}(P_3)=\{\varphi_1,\varphi_2\}$. Let $P_2=\langle a\rangle$ and $P_3=\langle b\rangle$. Then since $P_3\triangleleft G$, $aba^{-1}\in\langle b\rangle$. Thus, $\varpi_1(aba^{-1})=\varphi_1(a)b\varphi_1(a^{-1})=\varphi_1(a)b(\varphi_1(a))^{-1}$. We have three choices for $\varphi_1(a)$. It can map to $e, b$, or $b^2$. If it maps to $e$, then that would imply $a=e$ which is not the case since it generates $P_2$. It can map to $b$, but that $\varphi_1(a)=b=\varphi_1(b)$ and since $\varphi_1$ is 1-1, then $a=b$ but this too is a contradiction. Thus, $\varphi_1(a)=b^2$. Hence, $\varphi_1(aba^{-1})=b^2bb^{-2}=b$. Thus, $aba^{-1}=b$ which implies that $ab=ba$. Therefore, $G$ is abelian. Thus, $P_2\triangleleft G$ and $G\cong\mathbb{Z}_6$. $\varphi_2$ gives us $aba^{-1}=b^2$ and thus, $ab=b^{-1}a$. Thus, $G\cong D_6$.
    \end{example}
    
    \begin{theorem}\label{thm:2.1}
        For each $n\geq 3$, the dihedral group $D_n$ is a group of order $2n$ whose generators $a$ and $b$ satisfy:
        \begin{enumerate}[label=(\roman*)]
            \item $a^n=(1)$; $a^k\neq(1)$ if $0<k<n$;
            \item $ba=a^{-1}b$.
        \end{enumerate}
        
        \noindent Any group which is generated by element s$a,b\in G$ satisfying (i) and (ii) for some $n\geq 3$ (which $e\in G$ in place of $(1)$) is isomorphic to $D_n$.
    \end{theorem}
        \begin{proof}
    
        \end{proof}
        
    \begin{prop}\label{prop:2.7}
        Let $p$ and $q$ be primes such that $p>q$. If $q\nmid p-1$, then every group of order $pq$ is isomoarphic to the cyclic group $\mathbb{Z}_{pq}$. If $q\nmid p-1$, then there are (up to isomorphism) exactly two distinct groups of order $pq$: the cyclic group $\mathbb{Z}_{pq}$ and a non-abelian group $K$ generated by elements $c$ and $d$ such that 
        
        \begin{equation*}
            \abs{c}=p;\quad\abs{d}=q;\quad dc=c^sd,
        \end{equation*}
        
        \noindent where $s\not\equiv 1(\text{mod p})$ and $s^q\equiv 1(\text{mod }p)$.
    \end{prop}
        \begin{proof}
        
        \end{proof}
        
    \newpage
    
    
    \begin{example}
        Let $G$ be a group of order 6. Then by Cauchy's Theorem, $G$ contains elements $a$ and $b$ of order $2$ and $3$ respectively. Thus, $G$ has subgroups $\langle a\rangle$ and $\langle b\rangle$ of orders 2 and 3. It is clear that $\langle a\rangle$ is a 2-Sylow subgroup of $G$ and $\langle b\rangle$ is a 3-Sylow subgroup of $G$. By Theorem \ref{thm:1.6}, $n_2\equiv 1(\text{mod }2)$, $n_2\mid 3$, $n_3\equiv 1(\text{mod }3)$, and $n_3\mid 2$. From these relations it follows that $n_2=1$ or $n_2=3$, and $n_3=1$. Thus, letting $\langle a\rangle $ denote a 2-Sylow subgroup and $\langle b\rangle$ denote the 3-Sylow subgroup, from $n_3=1$ it follows that $\langle b\rangle\triangleleft G$. Note that for any $x\in\langle a\rangle\cap\langle b\rangle$, the order of $x$ must divide both 2 and 3. Thus, $o(x)=1$ and $\langle a\rangle\cap\langle b\rangle=\{e\}$. Hence, by Proposition \ref{prop:2.3}, $o(\langle a\rangle\langle b\rangle)=6$, and since $\langle a\rangle\langle b\rangle\subseteq_g G$, then $G=\langle a\rangle\langle b\rangle$.\par\hspace{4mm} Now assume that $\theta\colon \langle a\rangle\rightarrow\text{Aut}(\langle b\rangle)$ is a homomorphsism, where $\theta(a^k)=\varphi_k$ and $\varphi_k(b)=a^kba^{-k}$. TBC ...
    \end{example}
    
    \newpage
              
\end{document}