\documentclass{article}
\usepackage{graphicx}
\usepackage{tikz}
\usepackage{amsmath}
\usepackage{authblk}
\usepackage{titlesec}
\usepackage{amsthm}
\usepackage{amsfonts}
\usepackage{amssymb}
\usepackage{array}
\usepackage{booktabs}
\usepackage{ragged2e}
\usepackage{enumerate}
\usepackage{enumitem}
\usepackage{cleveref}
\usepackage{slashed}
\usepackage{commath}
\usepackage{lipsum}
\usepackage{colonequals}
\usepackage{addfont}
\usepackage{enumitem}
\usepackage{sectsty}
\usepackage{mathtools}
\usepackage{mathrsfs}

\usepackage{hyperref}
\hypersetup{
    colorlinks=true,
    linkcolor=blue,
    filecolor=magenta,      
    urlcolor=cyan,
}

\usetikzlibrary{decorations.pathreplacing}
\usetikzlibrary{arrows.meta}


%\subsectionfont{\itshape}

\newtheorem{theorem}{Theorem}[section]
\newtheorem{corollary}{Corollary}[theorem]
\newtheorem{lemma}{Lemma}[theorem]
\theoremstyle{definition}
\newtheorem{prop}{Proposition}[section]
\newtheorem{definition}{Definition}[section]
\theoremstyle{remark}
\newtheorem*{remark}{Remark}

\let\oldproofname=\proofname
\renewcommand{\proofname}{\bf{\textit{\oldproofname}}}

\newcommand{\closure}[2][3]{%
  {}\mkern#1mu\overline{\mkern-#1mu#2}}

\theoremstyle{definition}
\newtheorem{example}{Example}[section]

\newtheorem*{discussion}{Discussion}

\makeatletter
\renewenvironment{proof}[1][\proofname]{\par
  \pushQED{\qed}%
  \normalfont \topsep6\p@\@plus6\p@\relax
  \list{}{\leftmargin=0mm
          \rightmargin=0mm
          \settowidth{\itemindent}{\itshape#1}%
          \labelwidth=4mm
          \parsep=0pt \listparindent=0mm%\parindent 
  }
  \item[\hskip\labelsep
        \itshape
    #1\@addpunct{.}]\ignorespaces
}{%
  \popQED\endlist\@endpefalse
}

\makeatletter
\renewenvironment{proof*}[1][\proofname]{\par
  \pushQED{\qed}%
  \normalfont \topsep6\p@\@plus6\p@\relax
  \list{}{\leftmargin=0mm
          \rightmargin=0mm
          \settowidth{\itemindent}{\itshape#1}%
          \labelwidth=\itemindent
          \parsep=0pt \listparindent=0mm%\parindent 
  }
  \item[\hskip\labelsep
        \itshape
    #1\@addpunct{.}]\ignorespaces
}{%
  \popQED\endlist\@endpefalse
}

\newenvironment{solution}[1][\bf{\textit{Solution}}]{\par
  
  \normalfont \topsep6\p@\@plus6\p@\relax
  \list{}{\leftmargin=0mm
          \rightmargin=0mm
          \settowidth{\itemindent}{\itshape#1}%
          \labelwidth=\itemindent
          \parsep=0pt \listparindent=\parindent 
  }
  \item[\hskip\labelsep
        \itshape
    #1\@addpunct{.}]\ignorespaces
}{%
  \popQED\endlist\@endpefalse
}


\begin{document}
    \section{Exam 2 Review Questions}
    
    \begin{enumerate}[leftmargin=*]
        \item Recall from HW4, that if $G$ is a finite group, $H$ is a subgroup of $G$, then $\varphi\colon G\times G/H\rightarrow G/H$ by $\varphi(a,Hg)=Hga^{-1}$ is an action. Also recall that for each $g\in G$, $\sigma_g\colon S\rightarrow S$ by $\sigma_g(t)=\varphi(g,t)$ is a permutation of $S$, and $\theta\colon G\rightarrow A(S)$ by $\theta(g)=\sigma_g$ is a homomorphism. On HW4 we proved that $\ker\theta=\cap\{xHx^{-1}\colon x\in G\}$, and $\ker\theta\subseteq_g H$.
            \begin{enumerate}[label=(\alph*)]
                \item Assume $o(G)=12$, and that $P$ is a 3-Sylow subgroup of $G$. Using the above, prove that either $G\cong A_4$ or $\ker\theta=P$.
                    \begin{proof*}
                        Let $\varphi\colon G\times G/P\rightarrow G/P$ be the action described above, and let $\theta\colon G\rightarrow A(G/P)$ be the permutation representation associated with the action. Then we have that $\ker\theta\subseteq_g P$. Thus, by Lagrange's theorem, $o(\ker\theta)\mid 3$ which implies that $o(\ker\theta)=1$ or $o(\ker\theta)=3$.\par\hspace{4mm} If $o(\ker\theta)=1$, then $\ker\theta=\{e\}$. Note that in this case $G/\ker\theta\cong G$. Thus, by the FHT
                            \begin{equation*}
                                G\cong G/\ker\theta\cong\theta(G)\subseteq_g A(G/P)\cong S_4.
                            \end{equation*}
                        Thus, $G$ is isomorphic to a subgroup of $S_4$. Since $o(G)=12$ and the only subgroup of $S_4$ is $A_4$, then $G\cong A_4$.\par\hspace{4mm} If $o(\ker\theta)=3$, then since $\ker\theta\subseteq P$ and $\abs{\ker\theta}=\abs{P}$, then it follows that $\ker\theta=P$.
                    \end{proof*}
                \item Assume that $G\not\cong A_4$. Let $P=\langle t\rangle$, and prove that $[G\colon N(t)]=1$ or 2.
                    \begin{proof*}
                        Let $\psi\colon G\times G\rightarrow G$ be the action given by $\psi(g,h)=ghg^{-1}$. Then the orbit of $t$, $G(t)=\{gtg^{-1}\colon g\in G\}$, is equal to the conjugacy class of $t$, $c(t)$. Also the stabilizer of $t$, $G_t=\{g\in G\colon gtg^{-1}=t\}$, is equal to the normalizer of $t$, $N(t)$. Since $o(t)=3$, then any conjugate of $t$ must also have order 3. In part (a) we showed that $P=\ker\theta$ and so $P\triangleleft G$. Thus, there is only one 3-Sylow subgroup of $G$. This implies that there are only two elements of $G$ with order 3. Hence, $\abs{c(t)}=1$ or $\abs{c(t)}=2$. By the orbit-stabilizer theorem, we have that 
                            \begin{equation*}
                                \abs{c(t)}=\frac{\abs{G}}{\abs{N(t)}}=[G\colon N(t)].
                            \end{equation*}
                        Therefore, $[G\colon N(t)]=1$ or 2.
                    \end{proof*}
                \item Explain why it follows from (b) that there exists $g\in N(t)$ such that $o(g)=2$.
                    \begin{proof*}
                        If $[G\colon N(t)]=1$, then $\abs{N(t)}=12$ and since $2\mid 12$, then by Cauchy's Theorem, there exists $g\in N(t)$ such that $o(g)=2$. If $[G\colon N(t)]=2$, then $\abs{N(t)}=6$. Since $2\mid 6$, then by Cauchy's Theorem, there exists $g\in N(t)$ such that $o(g)=2$.
                    \end{proof*}
    \newpage
                \item Explain why $o(tg)=6$.
                    \begin{proof*}
                        Since for all $x\in N(t)$ we have that $xtx^{-1}=t$ which gives $xt=tx$, then for the $g\in N(t)$ from part (c), we have that $gt=tg$. Thus, from Exam 1, $o(tg)=o(t)o(g)=3\cdot 2=6$.
                    \end{proof*}
            \end{enumerate}
        \item Assume that $o(G)=p^2q$, where $p$ and $q$ are distinct odd primes. Prove that $G$ contains a Sylow subgroup that is normal in $G$, and prove that $G$ is solvable.
            \begin{proof*}
                By Sylow I, $G$ contains a Sylow subgroup, $P$, of order $p^2$ and a Sylow subgroup, $Q$, of order $q$. Assume that $p>q$. By Sylow III, $n_p\equiv 1(\text{mod }p)$ and $n_p\mid q$. The first relation implies that $n_p=1$ or $n_p>p$. The second relation implies that $n_p\leq q$. If $n_p\neq 1$, then $n_p>p$ and $n_p\leq q$, but since $p>q$, then this cannot occur. Thus, if $p>q$, then $n_p=1$ and hence $P\triangleleft G$.\par\hspace{4mm} Assume that $p<q$. By Sylow III, $n_q\equiv 1(\text{mod }q)$ and $n_q\mid p^2$. The first relation implies that $n_q=1$ or $n_q>q$. The second relation implies that $n_q=1$, $n_q=p$, or $n_q=p^2$. If $n_q=p$, then $n_q\neq 1$ and thus $n_q>q$ which implies $p>q$ and this contradicts our assumption. Thus, $n_q=1$ or $n_q=p^2$. If $n_q=p^2$, then $p^2\equiv 1(\text{mod }q)$. Thus, $q\mid (p^2-1)$. Thus, $q\mid (p-1)(p+1)$. Since $q$ is prime, then $q\mid(p-1)$ or $q\mid (p+1)$. Since $p<q$, then $q\nmid p$ and so $q\nmid (p-1)$. Thus, $q\mid (p+1)$. Moreover, since $q\nmid p$ and $q\mid (p+1)$, then $q=p+1$. However, since $p$ is an odd prime, then $p+1$ is even and thus $q=p+1$ is even. This contradicts our assumption that $q$ is an odd prime. Thus, $n_q\neq p^2$. Therefore, $n_q=1$ which implies $Q\triangleleft G$.\par\hspace{4mm} If $P\triangleleft G$, then $G/P$ is a group in which $o(G/P)=q$. Thus, $G/P$ is abelian. Hence, $\{e\}\triangleleft P\triangleleft G$ is a normal series of $G$ in which each factor is abelian. Thus, $G$ is solvable. If $Q\triangleleft G$, then $G/Q$ is a group in which $o(G/Q)=p^2$ and by (4) of HW6, $G/Q$ is abelian. Thus, $\{e\}\triangleleft Q\triangleleft G$ is a normal series of $G$ in which each factor is abelian. Thus, $G$ is solvable.
            \end{proof*}
        \item Assume that $P$ is $p$-Sylow subgroup of $G$. Prove that if $N\triangleleft G$, $N\neq G$, and $NP\neq N$, then $NP/N$ is a $p$-Sylow subgroup of $G/N$.
            \begin{proof*}
                Let $o(P)=p^k$. Since $N\cap P\subseteq_g P$, then $o(N\cap P)\mid p^k$. Let $o(N\cap P)=p^i$. Since $P\subseteq_g G$ and $N\triangleleft G$, then $NP\subseteq_g G$. Thus, $N\triangleleft NP$ and $NP/N$ is a group. It follows that 
                    \begin{equation*}
                        o(NP/N)=\frac{o(NP)}{o(N)}=\frac{\frac{o(N)o(P)}{o(N\cap P)}}{o(N)}=\frac{o(P)}{o(N\cap P)}=\frac{p^k}{p^i}=p^{k-i}.
                    \end{equation*}
                Note that if $i=k$, then $o(N\cap P)=p^k$ which would imply that $N=P$ and thus $NP=N$ which contradicts our assumption. Thus, $i<k$ and $k-i>0$. If $i=0$, then $o(NP/N)=p^k$ and thus $NP/N$ is a $p$-Sylow subgroup of $G/N$. Assume that $i>0$ and that $Q/N\subseteq_g G/N$ such that $o(Q/N)=p^j$ where $j>k-i$. Then $o(Q)=p^jo(N)$. However, since $N\cap P\subseteq_g N$, then $p^i\mid o(N)$. Thus, $p^{i+j}\mid o(Q)$. However, $i+j>k$ and so no such $Q$ can exist. Thus, $NP/P$ is a $p$-Sylow subgroup of $G/N$.
            \end{proof*}
    \newpage
        \item Assume that $o(G)=108$. Prove that $G$ has a normal subgroup of order 9, or of order 27.
            \begin{proof*}
                To begin, first note that $108=2^2\cdot3^3$. By Sylow I, $G$ contains a 3-Sylow subgroup of order 27. By Sylow III, $n_3\equiv 1(\text{mod }3)$ and $n_3\mid 4$. Thus, $n_3=1$ or $n_3=4$. Assume that $n_3=1$. Then the 3-Sylow subgroup is normal in $G$ which implies that $G$ contains a normal subgroup of order 27.\par\hspace{4mm} Assume that $n_3=4$. Then let $P$ and $Q$ be two distinct 3-Sylow subgroups of $G$. Since $PQ\subseteq G$, then it follows that 
                    \begin{equation*}
                        \abs{PQ}=\frac{\abs{P}\abs{Q}}{\abs{P\cap Q}}\leq \abs{G}\Leftrightarrow \frac{729}{108}\leq \abs{P\cap Q}.
                    \end{equation*}
                With this relation and the fact that $\abs{P\cap Q}\mid P$, then $\abs{P\cap Q}=9$ or $\abs{P\cap Q}=27$. If $\abs{P\cap Q}=27$, then $P=Q$ and this would contradict our assumption that $P$ and $Q$ distinct 3-Sylow subgroups of $G$. Thus, $\abs{P\cap Q}=9$. Since $o(P/(P\cap Q))=3$ and 3 is the smallest prime factor of $o(P)$, then $P\cap Q\triangleleft P$. Similarly, $P\cap Q\triangleleft Q$. It follows then that $P\subseteq N(P\cap Q)$ and $Q\subseteq N(P\cap Q)$. Thus, $PQ\subseteq N(P\cap Q)$. However, since $\abs{PQ}=81$ and $o(N(P\cap Q))\mid o(G)$, then $o(N(P\cap Q))=o(G)$, which implies that $N(P\cap Q)=G$. Finally, since $P\cap Q\triangleleft N(P\cap Q)$, then $P\cap Q\triangleleft G$. Therefore, $G$ contains a normal subgroup of order 9.
            \end{proof*}
        \item Determine the structure of all groups of order 57.
            \begin{proof*}
                We begin by first noting that $57=3\cdot 19$. Thus, Sylow I guarantees us that $o(G)=57$, then $G$ contains a 3-Sylow subgroup of order 3 and a 19-Sylow subgroup of order 19. From Sylow III it follows that $n_3\equiv 1(\text{mod }3)$ and $n_3\mid 19$. Thus, $n_3=1$ or $n_3=19$. Similarly, $n_{19}\equiv 1\text{mod }19)$ and $n_{19}\mid 3$. Thus, $n_{19}=1$. Hence, the 19-Sylow subgroup is normal in $G$. Let $\langle a\rangle$ denote a 3-Sylow subgroup of $G$ and let $\langle b\rangle$ denote the 19-Sylow subgroup of $G$. Assume that $\theta\colon \langle a\rangle\rightarrow\text{Aut}(\langle b\rangle)$ be a homomorphism defined by $\theta(g)=\sigma_k$ and $\sigma_k(x)=x^k$. Each map $\sigma_k$ corresponds to a relation $aba^{-1}=b^k$. So we must determine for which values of $k$ this relation holds. Note that by the properties of homomorphisms, $o(\theta(a))\mid o(a)$. Hence, $o(\varphi_k)\mid 3$. Thus, $o(\varphi_k)=1$ or $o(\varphi_k)=3$. If $o(\varphi_k)=1$, then $\varphi_k$ is the identity map and $k=1$. This then corresponds to $aba^{-1}=b$. If $o(\varphi_k)=3$, then $(\varphi_k)^3=\varphi_{k^3}=\varphi_1$. This implies that for all $x\in\langle b\rangle$, $\varphi_{k^3}(x)=x^{k^3}=x$. Thus, $x^{k^3-1}=e$ for all $x\in\langle b\rangle$. Hence, $o(x)\mid (k^3-1)$. And so $19\mid(k^3-1)$. Then we are looking for solutions to $k^3\equiv 1(\text{mod }19)$, of which there are 3. Checking all $1\leq k\leq 19$, we find that $1, 7,$ and 11 are solutions to this. However, taking $k=7$, we find that $7^2\equiv 11(\text{mod }19)$ and $7^3\equiv 1(\text{mod }19)$. This implies that $\varphi_7$ and $\varphi_{11}$ both yield the same structure.\par\hspace{4mm} Finally, if $k=1$, then $aba^{-1}=b$, thus $ab=ba$, thus $G$ is Abelian. Hence, $\langle a\rangle\triangleleft G$, $\langle a\rangle\cap \langle b\rangle=\{e\}$, and $G=\langle a\rangle$. Thus, $G\cong \mathbb{Z}_3\times\mathbb{Z}_{19}\cong\mathbb{Z}_{57}$.\par\hspace{4mm} If $k=7$, then $G$ is a non-abelian group of order 57 defined by $o(a)=3$; $o(b)=19$; $aba^{-1}=b^7$.
            \end{proof*}
    \newpage
        \item Assume that $G$ is a group, $p$ is a prime, $N\triangleleft G$, $o(N)=m$, $p\nmid m$, and $o(Ng)=p$. Prove that $o(g^m)=p$. 
            \begin{proof*}
                Since $o(Ng)=p$, then $(Ng)^p=Ng^p=N$. Thus, $g^p\in N$. Since $o(N)=m$, then $(g^p)^m=(g^m)^p=e$. Now assume that for some $k\in\mathbb{Z}^{+}$, $(g^m)^k=e$. Then $g^{mk}=e$. It then follows that $(Ng)^{mk}=Ng^{mk}=N$. Since $o(N)=p$, then $p\mid mk$. However, since $p$ is prime and $p\nmid m$, then $(p,m)=1$ and thus $p\mid k$. Thus, $p\leq k$. Therefore, $(g^m)^p=e$ and for all $k\in\mathbb{Z}^{+}$, if $(g^m)^k=e$, then $p\leq k$. Hence, $o(g^m)=p$.
            \end{proof*}
        \item Assume that $G$ is a finite Abelian group, and that $G$ is not cyclic. Prove that there exists a prime $p$ such that $G$ has a subgroup isomorphic to $\mathbb{Z}_p\times\mathbb{Z}_p$. 
            \begin{proof*}
                By the fundamental theorem of finite abelian groups, $G$ is isomorphic to a product of cyclic groups. In particular, if $o(G)=p_1^{r_1}\cdots p_n^{r_n}$ is the prime decomposition of the order of $G$, then $G\cong\mathbb{Z}_{p_1^{r_1}}\times\cdots\times\mathbb{Z}_{p_n^{r_n}}$. Now consider $\mathbb{Z}_{p_i^{r_i}}\times\mathbb{Z}_{p_j^{r_j}}$, this is clearly in the above product and $\mathbb{Z}_p\times\mathbb{Z}_p$ is a subgroup of this product. Thus, $G$ has a subgroup isomorphic to $\mathbb{Z}_p\times\mathbb{Z}_p$.
            \end{proof*}
        \item Assume that $G\cong A\times B$ and $\theta$ is an isomorphism from $G$ onto $A\times B$. Let $N=\{g\in G\colon \theta(g)\in A\times\{e\}\}$. Prove that $N$ is a normal subgroup of $G$ and that $N\cong A$. Let $M=\{g\in G\colon \theta(g)\in \{e\}\times B\}$. Explain how to express $G$ as an internal direct product.
            \begin{proof*}
                It is easily shown that both $N$ and $M$ are subgroups of $G$. Let $n\in N$ and $g\in G$. We want to show that $gng^{-1}\in N$. Thus, we need to show that $\theta(gng^{-1})\in A\times\{e\}$. Let $\theta(n)=(a,e)$ and let $\theta(g)=(x,y)$.
                    \begin{equation*}
                        \theta(gng^{-1})=\theta(g)\theta(n)\theta(g^{-1})=(x,y)(a,e)(x^{-1},y^{-1})=(xax^{-1},e).
                    \end{equation*}
                Thus, $\theta(gng^{-1})\in A\times\{e\}$ and so $gng^{-1}\in N$. Thus, $N\triangleleft G$. By a similar argument, $M\triangleleft G$.\par\hspace{4mm} We have $N\cap M=\{e\}$, since if $x\in N\cap M$, then $\theta(x)=(a,e)$ and $\theta(x)=(e,b)$ and so $\theta(x)=(e,e)$. Furthermore, since $\theta$ is an isomorphism, then $\ker\theta=\{e\}$ and so $x=e$. Thus, $G=NM\cong N\times M$.
            \end{proof*}
        \item Determine, with explanation, if the following are always true. 
            \begin{enumerate}[label=(\alph*)]
                \item If $N\triangleleft G$ and $M\triangleleft G$, then $NM\triangleleft G$.
                    \begin{proof*}
                        We have that $NM\subseteq_g G$. Let $x\in NM$ and let $g\in G$. Then we want to show that $gxg^{-1}\in NM$. Since $x\in NM$, then for some $n\in N$ and $m\in M$, we can write $x=nm$. Thus,
                            \begin{equation*}
                                gxg^{-1}=gnmg^{-1}=(gng^{-1})(gmg^{-1}).
                            \end{equation*}
                        Since both $N$ and $M$ are normal in $G$, then the left term is an element of $N$ and the right term is an element of $M$. Thus, the product is an element of $NM$. Therefore, $NM\triangleleft G$.
                    \end{proof*}
    \newpage
                \item If $M$ and $N$ are Abelian subgroups of $G$, $N\cap M=\{e\}$, and $G=NM$, then $G$ is Abelian. 
                    \begin{proof*}
                        This is not true. Not enough time to think of a counter example!
                    \end{proof*}
                \item If $I$ and $J$ are ideals of $R$, then $I+J$ is an ideal of $R$.
                    \begin{proof*}
                        In order for $I+J$ to be an ideal of $R$, we need $(I+J,+)$ to be a subgroup of $R$ and if $a\in I+J$ and $r\in R$, then $a\cdot r\in I+J$ and $r\cdot a\in I+J$. Clearly, $I+J\subseteq R$. So then let $a,b\in I+J$. Then $a=g_1+h_1$ and $b=g_2+h_2$. Thus, $a-b=(g_1-g_2)+(h_1-h_2)\in I+J$. Thus, $I+J$ is a subgroup of $R$. Now let $a\in I+J$ and $r\in R$. Then call $a=x+y$. Then $r\cdot(x+y)=r\cdot x+r\cdot y\in I+J$ and $(x+y)\cdot r=x\cdot r+y\cdot r\in I+J$. Thus, $I+J$ is an ideal of $R$.
                    \end{proof*}
            \end{enumerate}
        \item Assume that $R$ is a commutative ring with identity, and $J=\{a\in R\colon a^{-1}\in R\}$. Prove that $(J,\cdot)$ is a group. Determine $J$ for $R=\mathbb{Z}_n$ and for $R=$ a field.
            \begin{proof*}
                First we must show that $J$ is non-empty. Since $1\in R$ and $1\cdot 1=1$, then $1\in J$. Now let $a,b\in J$. Then we want to show that $a\cdot b\in J$. Since $R$ is a ring, then $a\cdot b\in R$ and $b^{-1}\cdot a^{-1}\in R$. Next, since $(a\cdot b)\cdot(b^{-1}\cdot a^{-1})=1$, then $(a\cdot b)^{-1}=b^{-1}\cdot a^{-1}$. Therefore, $a\cdot b\in J$. Now let $a,b,c\in J$. Then since $a\cdot b\in R$ and $b^{-1}\cdot a^{-1}\in R$, and $c\in R$ and $c^{-1}\in R$, then $(a\cdot b)\cdot c\in J$. Similarly, $a\cdot(b\cdot c)\in J$. Thus, $((a\cdot b)\cdot c)\cdot(a\cdot(b\cdot c))^{-1}=((a\cdot b)\cdot c)\cdot c^{-1}\cdot b^{-1}\cdot a^{-1}=1$. Thus, $(a\cdot b)\cdot c=a\cdot(b\cdot c)$. Finally, since $a\in J$, then $a^{-1}\in J$ since $(a^{-1})^{-1}=a$.\par\hspace{4mm} Now let $R=\mathbb{Z}_n$. Then by HW 1, $J=\mathbb{Z}_{(n)}$. Lastly, if $R$ is a field, then $J=R-\{0\}$.
            \end{proof*}   
    \end{enumerate}    
    \section{Ring Theory Results}
        \begin{enumerate}[leftmargin=*]
            \item Assume that $(R,+,\cdot)$ is a commutative ring with identity, and that $n1_R=0$. Then $\text{char}(R)\mid n$.
                \begin{proof*}
                    Let $\text{char(R)}=m$. By the Quotient Remainder Theorem, there exists $q,r\in\mathbb{Z}$ such that $n=qm+r$ and $0\leq r <m$. Then 
                        \begin{equation*}
                            n1_r=(qm+r)1_R=qm1_r+r1_r=q(m1_R)_r1_R=0+r1_R=0.
                        \end{equation*}
                    Since $r<m$, then $r=0$. Thus, $n=qm$ and, therefore, $m\mid n$.
                \end{proof*}
                
            \item Assume that $(R,+,\cdot)$ is a commutative ring with identity and $a\in R$. Let $(a)_i=\{r\cdot a\colon r\in R\}$. Then $(a)_i$ is an ideal of $R$, and $(a)_i$ is the smallest ideal of $R$ that contains $a$.
                \begin{proof*}
                    First we must prove that $((a)_i, +)$ is a subgroup of $(R,+)$. Note that $1\in R$ and so $1\cdot a\in (a)_i$. Thus $(a)_i\neq \varnothing$. It is also clear that $(a)_i\subseteq R$. Now let $x,y\in (a)_i$. Then for some $r_1,r_2\in R$, $x=r_1\cdot a$ and $y=r_2\cdot a$. Additionally, since $r_2\in R$, then $-r_2\in R$. Moreover, since $-r_2\cdot y=-(r_2\cdot y)=-y$, then $-y\in R$. Thus, $x-y=r_1\cdot a+-(r_2\cdot a)=(r_1-r_2)\cdot a\in (a)_i$. Thus, $((a)_i,+)$ is a subgroup of $(R,+)$. Let $r\in R$ and let $x\in(a)_i$, then for some $r'\in R$, $x=r'\cdot a$ and $r\cdot x=r\cdot r'\cdot a\in(a)_i$. Similarly, $x\cdot r=r'\cdot a\cdot r$ 
                \end{proof*}
        \end{enumerate}
        
\end{document}