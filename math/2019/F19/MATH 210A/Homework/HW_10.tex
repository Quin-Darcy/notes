\documentclass[12pt]{article}
\usepackage[margin=1in]{geometry} 
\usepackage{graphicx}
\usepackage{amsmath}
\usepackage{authblk}
\usepackage{titlesec}
\usepackage{amsthm}
\usepackage{amsfonts}
\usepackage{amssymb}
\usepackage{array}
\usepackage{booktabs}
\usepackage{ragged2e}
\usepackage{enumerate}
\usepackage{enumitem}
\usepackage{cleveref}
\usepackage{slashed}
\usepackage{commath}
\usepackage{lipsum}
\usepackage{colonequals}
\usepackage{addfont}
\usepackage{enumitem}
\usepackage{sectsty}
\usepackage{lastpage}
\usepackage{fancyhdr}
\usepackage{accents}
\usepackage{xcolor}
\usepackage[inline]{enumitem}
\pagestyle{fancy}
\setlength{\headheight}{10pt}

\subsectionfont{\itshape}

\newtheorem{theorem}{Theorem}[section]
\newtheorem{corollary}{Corollary}[theorem]
\newtheorem{prop}{Proposition}[section]
\newtheorem{lemma}[theorem]{Lemma}
\theoremstyle{definition}
\newtheorem{definition}{Definition}[section]
\theoremstyle{remark}
\newtheorem*{remark}{Remark}
 
\makeatletter
\renewenvironment{proof}[1][\proofname]{\par
  \pushQED{\qed}%
  \normalfont \topsep6\p@\@plus6\p@\relax
  \list{}{\leftmargin=0mm
          \rightmargin=4mm
          \settowidth{\itemindent}{\itshape#1}%
          \labelwidth=\itemindent
          \parsep=0pt \listparindent=\parindent 
  }
  \item[\hskip\labelsep
        \itshape
    #1\@addpunct{.}]\ignorespaces
}{%
  \popQED\endlist\@endpefalse
}

\newenvironment{solution}[1][\bf{\textit{Solution}}]{\par
  
  \normalfont \topsep6\p@\@plus6\p@\relax
  \list{}{\leftmargin=0mm
          \rightmargin=0mm
          \settowidth{\itemindent}{\itshape#1}%
          \labelwidth=\itemindent
          \parsep=0pt \listparindent=\parindent 
  }
  \item[\hskip\labelsep
        \itshape
    #1\@addpunct{.}]\ignorespaces
}{%
  \popQED\endlist\@endpefalse
}

\let\oldproofname=\proofname
\renewcommand{\proofname}{\bf{\textit{\oldproofname}}}


\newlist{mylist}{enumerate*}{1}
\setlist[mylist]{label=(\alph*)}

\begin{document}

\begin{center}
	\vspace{.4cm} {\textbf { \large MATH 210A}}
\end{center}
{\textbf{Name:}\ Quin Darcy \hspace{\fill} \textbf{Due Date:} 11/27/19   \\
{ \textbf{Instructor:}}\ Dr. Shannon \hspace{\fill} \textbf{Assignment:} Homework 10 \\ \hrule}

\justifying

    \begin{enumerate}[leftmargin=*]
        \item[5.] Assume that $D$ is an integral domain, and $\text{char}(D)$ is finite. Prove that $\text{char(D)}$ is prime.
            \begin{proof}
                Let $\text{char}(D)=n$. Assume $n=kr$ for $1<k<n$ and $1<r<n$. In class we proved that $\text{char}(D)$ is the least positive integer for which $n1=0$. Additionally, by part 4.(b), we have that $(kr)1=(k1)\cdot(r1)=0$. Since there are no zero divisors, then either $k1=0$ or $r1=0$. However, since $k,r<n$, then this contradicts $\text{char}(D)=n$. Therefore, $n$ is not composite and must be prime. 
            \end{proof}
            
        \item[6.] Prove that if $F$ is a field, then the only ideals of $F$ are $\{0\}$ and $F$.
            \begin{proof}
                Suppose $a\neq 0$ and let $(a)\subseteq_i F$. Let $x\in F$. Then since $a^{-1}\in F$, then $a^{-1}\cdot x\in F$. Thus, $a\cdot(a^{-1}\cdot x)=x\in (a)$. Thus, $F\subseteq (a)$ and, by definition $(a)\subseteq F$. Hence, $F=(a)$. Now suppose $a=0$. Then $(a)=\{0\}$. Therefore, the only ideals of $F$ are $\{0\}$ and $F$. 
            \end{proof}
            
        \item[7.] Assume that $(R,+,\cdot)$ is a field, $(S,\star,\#)$ is a ring, and that $\alpha\colon R\rightarrow S$ is a ring homomorphism of $R$ onto $S$. Prove that either $\alpha$ is an isomorphism or $S=\{0\}$. 
            \begin{proof}
                Let $x,y\in R$ such that $x\neq y$. Assume that $f(x)=f(y)$. Then $f(x)-f(y)=f(x-y)=0$. Since $x\neq y$, then $x-y\neq 0$ and thus there exists $(x-y)^{-1}\in R$. Hence, $f(x-y)\#f((x-y)^{-1})=f((x-y)\cdot(x-y)^{-1})=f(1_R)=0$. Thus, $1_R\in\ker\alpha$. Since $\ker\alpha$ is an ideal of $R$, and $1_R\in R$, then $\ker\alpha=R$. Moreover, since $\alpha$ is onto, then $\ker\alpha=R$ implies $S=\{0\}$. Now suppose that $\ker\alpha\neq R$. Then since $R$ is a field and $\ker\alpha$ is an ideal of $R$, then $\ker\alpha=\{0\}$. Thus, $\alpha$ is an isomorphism.
            \end{proof}
            
        \item [8.] Assume that $\theta\colon R\rightarrow S$ is an isomorphism onto $S$, and $1_R$ is an identity for $R$. 
            \begin{enumerate}[label=(\alph*)]
                \item Prove that $\theta(1_R)$ is an identity for $S$.
                    \begin{proof}
                        Consider $\theta(1_R)\in S$. Let $s\in S$. Since $\theta$ is an isomorphism, then there exists $r\in R$ such that $\theta(r)=s$. Then
                            \begin{equation*}
                                s\cdot\theta(1_R)=\theta(r)\cdot\theta(1_R)=\theta(r\cdot 1_R)=\theta(r)=s.
                            \end{equation*}
                        Similarly, we get that $\theta(1_R)\cdot s=s$. Thus, for all $s\in S$, $\theta(1_R)\cdot s=s\cdot\theta(1_R)=s$. Finally, assume there exists $x\in S$ such that for all $s\in S$, $x\cdot s=s\cdot s=s$. Then $x=x\cdot\theta(1_R)=\theta(1_R)$. Hence, $\theta(1_R)$ is unique. Therefore, $\theta(1_R)$ is an identity for $S$.
                    \end{proof}
            \newpage
                \item Prove that if $x^2=1_R+1_R$ has a solution in $R$, then $x^2=1_S+1_S$ has a solution in $S$. 
                    \begin{proof}
                        Let $a\in R$ denote a solution to $x^2=1_R+1_R$. Then $a^2=1_R+1_R$. Thus, $\theta(a^2)=\big(\theta(a)\big)^2=\theta(1_R)+\theta(1_R)=1_S+1_S$. Therefore, $\theta(a)$ is a solution in $S$ to $x^2=1_S+1_S$.
                    \end{proof}
                \item Prove that $\mathbb{Q}[\sqrt{2}]$ and $\mathbb{Q}[\sqrt{3}]$ are not isomorphic.
                    \begin{proof}
                        Assume that $\theta\colon \mathbb{Q}[\sqrt{2}]\rightarrow\mathbb{Q}[\sqrt{3}]$ is an isomorphism. In $\mathbb{Q}[\sqrt{2}]$ there is an element which satisfies $x^2-1=0$. Namely, $x=\sqrt{2}$. Then since $\theta$ is an isomorphism, then 
                            \begin{equation*}
                                \begin{split}
                                    \theta(x^2-2)&=\theta(x^2)-\theta(2) \\
                                    &=\big(\theta(x)\big)^2-\big\theta(1+1) \\
                                    &=\big(\theta(x)\big)^2-\big(\theta(1)+\theta(1)\big) \\
                                    &=\big(\theta(x)\big)^2-2.
                                \end{split}
                            \end{equation*}
                        Letting $x=\sqrt{2}$, we get that $\big(\theta(\sqrt{2})\big)^2-2=0$. This implies that there exists an element, namely, $\theta(\sqrt{2})\in\mathbb{Q}[\sqrt{3}]$ whose square is 2. However, since no such element exists in $\mathbb{Q}[\sqrt{3}]$. Therefore, $\mathbb{Q}[\sqrt{2}]\not\cong\mathbb{Q}[\sqrt{3}]$. 
                    \end{proof}
            \end{enumerate}
        
        \item[9.] Assume that $R$ is an integral domain. Prove that is $a$ is prime, then $a$ is irreducible. Prove that if $R$ is a PID, then the converse hold.
            \begin{proof}
                Let $a\neq 0$ be prime. Assume that $a=bc$ for some $b,c\in R$. Then $a\mid bc$. Thus, $a\mid b$ or $a\mid c$. Without loss of generality, suppose $a\mid b$. Then there exists $k\in\mathbb{Z}$ such that $b=ak$. Thus, $a=bc=akc$. Thus, $a(1-kc)=0$. Since $R$ is an integral domain, then either $a=0$ or $1-kc=0$. Since we assumed $a\neq 0$, then it follows that $1-kc=0$ and so $1=kc$. Therefore, $c\mid 1$ and $a$ is irreducible.\par\hspace{4mm} Assume $R$ is a PID and that $a$ is irreducible. Then for a contradiction, assume there exists an ideal, $I$, of $R$ such that $(a)\varsubsetneq I\varsubsetneq R$. Since $R$ is a PID, then for some $u\in R$, $I=(u)$. Since $a\in I$, then $a=ur$ for some $r\in R$. Since $a$ is irreducible, then either $u\mid 1$ or $r\mid 1$. If $u\mid 1$, then $u\cdot u^{-1}=1\in (u)$ which would imply $(u)=R$ and so this cannot occur. Hence, $r\mid 1$. Thus, $a\cdot r^{-1}=u$. Thus, $(u)=(a)=I$. Thus, $(a)$ is maximal and by pg. 6, $a$ is therefore prime.  
            \end{proof}
        
        \item[10.] Assume that $R$ is a Euclidean domain, and that $J$ is an ideal of $R$. Prove that there exists $b\in R$ such that $J=(b)$.
            \begin{proof}
                Let $J\subseteq R$ be a nonzero ideal of $R$. Let $a\in J$ such that $a\neq 0$ and $v(a)$ is a minimum of the set $\{v(x)\colon x\in J\}$. Since $a\in J$, then $aR\subseteq J$. Now suppose $b\in J$ such that $b\neq 0$. Then since $R$ is a Euclidean domain, then there exists $q,r\in R$ such that $b=aq+r$ and $r=0$ or $v(r)<v(a)$. Thus, $r=b-aq$. Since $b\in J$ and $-aq\in J$, then $r\in J$. Since $v(a)$ was assumed to be a minimum in $J$, then $v(a)\leq v(r)$. Thus, $v(r)\not< v(a)$. Hence, $r=0$ and $b=aq$. Thus, $b\in aR$ which implies $J\subseteq aR$. Thus, $J=aR=(a)$. Therefore, $J$ is a prime ideal. 
            \end{proof}
    \end{enumerate}
    
\end{document}