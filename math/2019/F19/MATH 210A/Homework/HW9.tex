\documentclass[12pt]{article}
\usepackage[margin=1in]{geometry} 
\usepackage{graphicx}
\usepackage{amsmath}
\usepackage{authblk}
\usepackage{titlesec}
\usepackage{amsthm}
\usepackage{amsfonts}
\usepackage{amssymb}
\usepackage{array}
\usepackage{booktabs}
\usepackage{ragged2e}
\usepackage{enumerate}
\usepackage{enumitem}
\usepackage{cleveref}
\usepackage{slashed}
\usepackage{commath}
\usepackage{lipsum}
\usepackage{colonequals}
\usepackage{addfont}
\usepackage{enumitem}
\usepackage{sectsty}
\usepackage{lastpage}
\usepackage{fancyhdr}
\usepackage{accents}
\usepackage{xcolor}
\usepackage[inline]{enumitem}
\pagestyle{fancy}
\setlength{\headheight}{10pt}

\subsectionfont{\itshape}

\newtheorem{theorem}{Theorem}[section]
\newtheorem{corollary}{Corollary}[theorem]
\newtheorem{prop}{Proposition}[section]
\newtheorem{lemma}[theorem]{Lemma}
\theoremstyle{definition}
\newtheorem{definition}{Definition}[section]
\theoremstyle{remark}
\newtheorem*{remark}{Remark}
 
\makeatletter
\renewenvironment{proof}[1][\proofname]{\par
  \pushQED{\qed}%
  \normalfont \topsep6\p@\@plus6\p@\relax
  \list{}{\leftmargin=0mm
          \rightmargin=4mm
          \settowidth{\itemindent}{\itshape#1}%
          \labelwidth=\itemindent
          \parsep=0pt \listparindent=\parindent 
  }
  \item[\hskip\labelsep
        \itshape
    #1\@addpunct{.}]\ignorespaces
}{%
  \popQED\endlist\@endpefalse
}

\newenvironment{solution}[1][\bf{\textit{Solution}}]{\par
  
  \normalfont \topsep6\p@\@plus6\p@\relax
  \list{}{\leftmargin=0mm
          \rightmargin=0mm
          \settowidth{\itemindent}{\itshape#1}%
          \labelwidth=\itemindent
          \parsep=0pt \listparindent=\parindent 
  }
  \item[\hskip\labelsep
        \itshape
    #1\@addpunct{.}]\ignorespaces
}{%
  \popQED\endlist\@endpefalse
}

\let\oldproofname=\proofname
\renewcommand{\proofname}{\bf{\textit{\oldproofname}}}


\newlist{mylist}{enumerate*}{1}
\setlist[mylist]{label=(\alph*)}

\begin{document}

\begin{center}
	\vspace{.4cm} {\textbf { \large MATH 210A}}
\end{center}
{\textbf{Name:}\ Quin Darcy \hspace{\fill} \textbf{Due Date:} 11/06/19   \\
{ \textbf{Instructor:}}\ Dr. Shannon \hspace{\fill} \textbf{Assignment:} Homework 9 \\ \hrule}

\justifying

    \begin{enumerate}[leftmargin=*]
        \item[2.] Assume that $G$ is a finite group, and $b\in G-Z(G)$, $o(b)=p$, where $p$ is prime. Prove that $\langle b\rangle\cap Z(G)=\{e\}$.
            \begin{proof}
                Let $x\in\langle b\rangle\cap Z(G)$. Then $x\in \langle b\rangle$ and $x\in Z(G)$. It follows from $x\in\langle b\rangle$ that $o(x)\mid o(\langle b\rangle)$. Thus, $o(x)\mid p$. Since $p$ is prime then either $o(x)=1$ or $o(x)=p$.\par\hspace{4mm} If $o(x)=1$, then $x=e$ and $\langle b\rangle\cap Z(G)=\{e\}$. If $o(x)=p$, then $o(\langle x\rangle)=p$ and since $\langle x\rangle\subseteq_g\langle b\rangle$, then $\langle x\rangle=\langle b\rangle$. Additionally, since $x\in\langle b\rangle\cap Z(G)$, then $x\in Z(G)$. Thus, by closure, $\langle x\rangle\subseteq_g Z(G)$ and since $\langle x\rangle =\langle b\rangle$, then $\langle b\rangle\subseteq_g Z(G)$. Thus, $b\in Z(G)$. Hence, $b\notin G-Z(G)$ and this is a contradiction. Therefore, for all $x\in\langle b\rangle\cap Z(G)$, it follows that $o(x)=1$ and $x=e$. Thus, $\langle b\rangle\cap Z(G)=\{e\}$.
            \end{proof}
            
        \item[3.] Without simply citing the results that we proved for groups of order $pq$, determine the structure of all groups of order 55.
            \begin{proof}
                Assume that $o(G)=55$. We have $n_5\equiv 1(\text{mod }5)$ and $n_5\mid 11$. Thus, $n_5=1$ or $n_5=11$. Similarly, $n_{11}\equiv 1(\text{mod }11)$ and $n_{11}\mid 5$. Thus, $n_{11}=1$. Hence, $P_{11}\triangleleft G$. Let $\langle b\rangle$ denote the 11-Sylow subgroup and let $\langle a\rangle$ denote the 5-Sylow subgroup. We have that $\langle a\rangle\cap \langle b\rangle=\{e\}$ and $o(G)=o(\langle a\rangle)o(\langle b\rangle)$. Thus, $G=\langle a\rangle\langle b\rangle$.\par\hspace{4mm} Assume that $\theta\colon\langle a\rangle\rightarrow\text{Aut}(\langle b\rangle)$ is a homomorphism where $\theta(h)=\varphi_k$ and that $\varphi_k(x)=x^k$. Because each $\varphi_k$ corresponds to $aba^{-1}=b^k$, then we must determine which values of $k$ work. If $h\in\langle a\rangle$, then $o(\theta(h))\mid 5$, thus $o(\varphi_k)\mid 5$. Then $o(\varphi_k)=1$ or $o(\varphi_k)=5$. Hence, either $\varphi_k=\varphi_1$ or $(\varphi_k)^5=\varphi_{k^5}=\varphi_1$. The latter case implies that $x^{k^5}=x$ for all $x\in\langle b\rangle$ and so $x^{k^5-1}=e$ for all $x\in\langle b\rangle$. It follows from this that we need $11\mid k^5-1$. Hence, we are looking for solutions to $k^5\equiv 1(\text{mod }11)$. There are 5 solutions to this. Namely, $k=1,3,4,5,9$. However, if we take $k=3$ we have that $\varphi_3$ corresponds to $aba^{-1}=b^3$ and from this we get the following relations
                
                    \begin{equation*}
                        \begin{split}
                            ab^3a^{-1}&=(aba^{-1})^4=(b^3)^3=b^9 \\
                            ab^9a^{-1}&=(aba^{-1})^9=(b^3)^9=b^{27}=b^5 \\
                            ab^5a^{-1}&=(aba^{-1})^5=(b^3)^5=b^{15}=b^4.
                        \end{split}
                    \end{equation*}
                
                Thus, $\varphi_3,\varphi_4,\varphi_5$, and $\varphi_9$ all correspond to the same structure. Therefore, there are 2 groups of order 55. We have that $G=\langle a\rangle\langle b\rangle\cong\langle a\rangle\times\langle b\rangle\cong\mathbb{Z}_5\times\mathbb{Z}_{11}\cong\mathbb{Z}_{55}$. This is the case when $n_5=1$. Then we have the nonabelian group, $G=\langle a\rangle\langle b\rangle$, of order 55 whose structure is defined by the following relations
                
                    \begin{equation*}
                        o(a)=5;\quad o(b)=11;\quad aba^{-1}=b^3.
                    \end{equation*}        
            \end{proof}
                
    \newpage
    
        \item[5.] Assume that $Q$ is a $p$-Sylow subgroup of $G$, $M\triangleleft G$, and that $M\cap Q\neq\{e\}$. Prove that $M\cap Q$ is a $p$-Sylow subgroup of $M$.
            \begin{proof}
                We know that $M\cap Q\subseteq_g M$ and $M\cap Q\subseteq_g Q$. Thus, by Lagrange's Theorem, $o(M\cap Q)\mid o(Q)$ and $o(M\cap Q)\mid o(M)$. Since $Q$ is a $p$-Sylow subgroup, then $M\cap Q$ must have order of $p$ to some power and thus $M\cap Q$ is a $p$-subgroup of $M$. By Sylow II, there exists a $p$-Sylow subgroup, $P$, of $M$ such that $M\cap Q\subseteq_g P$. Additionally, by Sylow II, there is some $p$-Sylow subgroup of $G$ for which $P$ is a subgroup of and since any two $p$-Sylow subgroups are conjugtes, then there exists some $g\in G$ such that $P\subseteq_g gQg^{-1}$. Since $M$ is normal in $G$, then $gMg^{-1}=M$ and thus $P\subseteq_g gMg^{-1}$. Note that for any $x\in P$, there exists $a\in M$ and $b\in Q$ such that $x=gag^{-1}$ and $x=gbg^{-1}$. Thus, $g^{-1}xg=a$ and $g^{-1}xg=b$. Thus, $g^{-1}Pg\subseteq_g M$ and $g^{-1}Pg\subseteq_g Q$. Hence, $g^{-1}Pg\subseteq_g M\cap Q$. Finally, since $\abs{g^{-1}Pg}=\abs{P}$ and both $P$ and $g^{-1}Pg$ are subgroups of $M$, then we have that $M\cap Q$ is a subgroup of the $p$-Sylow subgroup $P$ of $M$ and we have that $g^{-1}Pg$ is a subgroup of $M$ which is the same size as $P$. Thus, $\abs{M\cap Q}=\abs{P}$. Therefore, $M\cap Q$ is a $p$-Sylow subgroup of $M$.
            \end{proof}
            
        \item[6.] Determine with explanation, if the following are always true.
            \begin{enumerate}[label=(\alph*)]
                \item If $P$ and $Q$ are each $p$-Sylow subgroups of a group, $G$, then either $P=Q$ or $P\cap Q=\{e\}$.
                    \begin{proof}
                        This is not true. Let $G=S_5$. Here the order of $G$ is 5!. Now consider the two following subgroups
                        
                        \begin{equation*}
                            \begin{split}
                                 &\{(1),(13),(24),(13)(24),(12)(34),(14)(23),(1234),(1432)\} \\
                                 &\{(2),(24),(35),(24)(35),(23)(45),(25)(34),(2345),(2543)\}.
                            \end{split}
                        \end{equation*}
                        
                        Both these subgroups have the same structure as $D_8$ and are 2-Sylow subgroups of $S_5$. The identity and $(24)$ would be present in their intersection. 
                    \end{proof}
                    
                \item If $o(G)=2n$, $o(b)=n$, $a\in G-\langle b\rangle$, $G=\langle a\rangle\langle b\rangle$, and $aba^{-1}=b^{-1}$, then $G\cong D_{2n}$.
                    \begin{proof}
                        This description fully defines $D_{2n}$ and so any group $G$ with these properties is isomorphic to $D_{2n}$.
                    \end{proof}
            \end{enumerate}
            
        \item[7.] Assume that $R$ is a ring, and that $Z=\{a\in R\colon ax=xa$ for all $x\in R\}$. Prove that $Z$ is a subring of $R$.
            \begin{proof}
                We want to show that $Z\neq\varnothing$, for all $a,b\in Z$, $a+b\in Z$, $-a\in Z$, and $ab\in Z$. Since $1$ commutes with itself, then $1\in Z$ and thus $Z\neq\varnothing$. Now let $a,b\in R$ then $ax=xa$ and $by=yb$ for all $x,y\in R$. Let $x\in R$, then $ax+bx=xa+xb$. Thus, $(a+b)x=x(a+b)$ for all $x\in R$. Thus, $a+b\in Z$. Since $a\in Z$, then for all $x\in R$, $ax=xa$ and since $(-1)(ax)=(-1)(xa)$, then $(-a)x=x(-a)$. Thus, $-a\in Z$. Now consider $abx=axb=xab$. Thus, $ab\in Z$ and $Z$ is therefore a subring of $R$.
            \end{proof}
            
    \newpage
        
        \item[8.] Find, with explanation, the smallest subring, $S$, of $\mathbb{R}$ such that $1/2\in S$. 
            \begin{proof}
                Let $S=\{\frac{a}{2^k}\mid a\in\mathbb{Z}\wedge k\in\mathbb{N}\wedge (2,a)=1\}$. To begin we must first show that $S$ is a subring of $\mathbb{R}$ and that $\frac{1}{2}\in S$. Since $1\in\mathbb{Z}$ and $1\in\mathbb{N}$, then $\frac{1}{2^1}\in S$ and thus $S$ is not empty and it contains $\frac{1}{2}$. Now let $x,y\in S$. Then for some $a,b\in\mathbb{Z}$ and $k,m\in\mathbb{N}$ we have that $x=\frac{a}{2^k}$ and $y=\frac{b}{2^m}$. Without loss of generality, assume $k\leq m$. We then check closure under + by 
                    \begin{equation*}
                        \begin{split}
                            x+y &= \frac{a}{2^k}+\frac{b}{2^m} \\
                            &= \frac{2^ma+2^kb}{2^{k+m}} \\
                            &= \frac{2^k(2^{m-k}a+b)}{2^{k+m}} \\ 
                            &= \frac{2^{m-k}a+b}{2^m}.
                        \end{split}
                    \end{equation*}
                    
                Note that the numerator is of the form of $2t+r$, where $r$ is an odd number and so $2^{m-k}a+b$ is itself an odd number. Hence, $(2,2^{m-k}a+b)=1$. Thus, $x+y\in S$. Now consider
                
                    \begin{equation*}
                        x\cdot y = \bigg(\frac{a}{2^k}\bigg)\bigg(\frac{b}{2^m}\bigg) 
                            = \frac{ab}{2^{k+m}}.
                    \end{equation*}
                    
                Since $2\nmid a$ and $2\nmid b$, then $2\nmid ab$ (also $ab\nmid 2$) and thus $(2,ab)=1$. Hence, $x\cdot y\in S$. The last thing we must show is the existence of additive inverses. Consider the same $x$ as before. Since $-a\in\mathbb{Z}$ and $-x=\frac{-a}{2^k}$, then $-x\in S$. Therefore, $S$ is a subring of $\mathbb{R}$.\par\hspace{4mm} Now assume that $T\subseteq_r\mathbb{R}$ and $\frac{1}{2}\in T$. Consider the same $x\in S$ as before. Since $T$ is closed under + and $\frac{1}{2}\in T$, then we can take $\frac{1}{2}$ and operate on it with itself, under +, $a$ many times to obtain $\frac{a}{2}\in T$. Since $T$ is closed under $\cdot$, then we can operate on $\frac{1}{2}$ with itself, under $\cdot$, $k-1$ many times to obtain $\frac{1}{2^{k-1}}\in T$. Finally, since $T$ is closed under $\cdot$, then 
                
                    \begin{equation*}
                        \bigg(\frac{a}{2}\bigg)\cdot\bigg(\frac{1}{2^{k-1}}\bigg)=\frac{a}{2^k}=x.
                    \end{equation*}
                    
                Thus, $x\in T$. Hence, $S\subseteq T$. Therefore, $S$ is the smallest subring of $\mathbb{R}$ that contains 1/2. 
            \end{proof}
            
        \item[9.] Let $m\in\mathbb{Z}_n$. Prove that $[m]\neq[0]$ is a zero-divisor iff $(m,n)\neq1$.
            \begin{proof}
                We will argue the first direction by proving the contrapositive. Assume $[m]\neq 0$, $(m,n)=1$, and that for some $[s]\in\mathbb{Z}_n$. $[m]\cdot[s]=[0]$. Then $[ms]=0$. Thus, $n\mid ms$. However, since $(m,n)=1$, then $n\mid s$ and $[s]=0$. Thus, if $(m,n)=1$, then $[m]$ is not a zero-divisor.\par\hspace{4mm} Now assume that $(m,n)=d>1$. Then $d\mid m$ and $d\mid n$. Thus, $[m]\cdot[\frac{n}{d}]=[n]\cdot[\frac{m}{d}]=[0]\cdot[\frac{m}{d}]=[0]$. Thus, $[m]\cdot[\frac{n}{d}]=[0]$ and since $[m]\neq[0]$ and $[\frac{n}{d}]\neq [0]$, then this implies that $[m]$ is a zero-divisor.
            \end{proof}
    \end{enumerate}
        
        
    
\end{document}