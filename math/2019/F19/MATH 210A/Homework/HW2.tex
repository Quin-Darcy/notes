\documentclass[12pt]{article}
\usepackage[margin=1in]{geometry} 
\usepackage{graphicx}
\usepackage{amsmath}
\usepackage{authblk}
\usepackage{titlesec}
\usepackage{amsthm}
\usepackage{amsfonts}
\usepackage{amssymb}
\usepackage{array}
\usepackage{booktabs}
\usepackage{ragged2e}
\usepackage{enumerate}
\usepackage{enumitem}
\usepackage{cleveref}
\usepackage{slashed}
\usepackage{commath}
\usepackage{lipsum}
\usepackage{colonequals}
\usepackage{addfont}
\usepackage{enumitem}
\usepackage{sectsty}
\usepackage{lastpage}
\usepackage{fancyhdr}
\usepackage{accents}
\usepackage{xcolor}
\usepackage[inline]{enumitem}
\pagestyle{fancy}
\setlength{\headheight}{10pt}

\subsectionfont{\itshape}

\newtheorem{theorem}{Theorem}[section]
\newtheorem{corollary}{Corollary}[theorem]
\newtheorem{prop}{Proposition}[section]
\newtheorem{lemma}[theorem]{Lemma}
\theoremstyle{definition}
\newtheorem{definition}{Definition}[section]
\theoremstyle{remark}
\newtheorem*{remark}{Remark}
 
\makeatletter
\renewenvironment{proof}[1][\proofname]{\par
  \pushQED{\qed}%
  \normalfont \topsep6\p@\@plus6\p@\relax
  \list{}{\leftmargin=0mm
          \rightmargin=4mm
          \settowidth{\itemindent}{\itshape#1}%
          \labelwidth=\itemindent
          \parsep=0pt \listparindent=\parindent 
  }
  \item[\hskip\labelsep
        \itshape
    #1\@addpunct{.}]\ignorespaces
}{%
  \popQED\endlist\@endpefalse
}

\newenvironment{solution}[1][\bf{\textit{Solution}}]{\par
  
  \normalfont \topsep6\p@\@plus6\p@\relax
  \list{}{\leftmargin=0mm
          \rightmargin=0mm
          \settowidth{\itemindent}{\itshape#1}%
          \labelwidth=\itemindent
          \parsep=0pt \listparindent=\parindent 
  }
  \item[\hskip\labelsep
        \itshape
    #1\@addpunct{.}]\ignorespaces
}{%
  \popQED\endlist\@endpefalse
}

\let\oldproofname=\proofname
\renewcommand{\proofname}{\bf{\textit{\oldproofname}}}


\newlist{mylist}{enumerate*}{1}
\setlist[mylist]{label=(\alph*)}

\begin{document}

\begin{center}
	\vspace{.4cm} {\textbf { \large MATH 210A}}
\end{center}
{\textbf{Name:}\ Quin Darcy \hspace{\fill} \textbf{Due Date:} 9/11/19   \\
{ \textbf{Instructor:}} \ Dr. Shannon \hspace{\fill} \textbf{Assignment:} Homework 2 \\ \hrule}

\justifying

\begin{enumerate}[leftmargin=*]

    \item Assume $(G,\star)$ is a group, $a\in G$, $o(a)=n$, $m\in\mathbb{Z}^{+}$, and $d=(m.n)$.
    
        \begin{enumerate}[label=(\alph*)]
            \item Prove that $o(a^m)=o(a^d)$.
                \begin{proof}
                    Let $y=o(a^m)$ and $t=o(a^d)$. Then in order to show that $y=t$, we must show that $t\mid y$ and $y\mid t$. Since $d=(m,n)$, then there exists $\alpha,\beta\in\mathbb{Z}$ such that $d=\alpha m+\beta n$. Thus,
                    
                    \begin{equation*}
                            t = o(a^d) 
                            = o(a^{\alpha m+\beta n}) 
                            = o(a^{\alpha m}\star a^{\beta n}) 
                            = o(a^{\alpha m}\star(a^n)^{\beta}) 
                            = o(a^{\alpha m}\star e^{\beta}) 
                            = o(a^{\alpha m}).
                    \end{equation*}
                    
                    Hence, $(a^{\alpha m})^t=e$. This can be rearranged to say $(a^m)^{\alpha t}=e$. Additionally, we also have that $(a^m)^y=e$. Thus, $(a^m)^y=(a^m)^{\alpha t}$. Thus, $(a^m)^{y-\alpha t}=e$. Since $y$ is the smallest number such that $(a^m)^y=e$ and $y-\alpha t<y$, then $y-\alpha t=0$. Thus, $y=\alpha t$. This implies that $t\mid y$. In the case that $\alpha<0$, then since $(a^m)^{y+\alpha t}=e$, then $y+\alpha t<y$ would still imply that $y+\alpha t=0$ and thus $y=-\alpha t$. So we still have that $t\mid y$.\par\hspace{4mm} Since $d=(m,n)$, then $d\mid m$ and thus there exists some $k\in\mathbb{Z}$ such that $m=kd$. Thus, we have that $(a^d)^t=e$ and $(a^m)^y=(a^{kd})^y=(a^d)^{ky}=e$. Hence, $(a^d)^t=(a^d)^{ky}$. Multiplying both sides by $(a^d)^{-ky}$ yields $(a^d)^{t-ky}=e$. Since $t-ky<t$, then $t-ky=0$. Thus, $t=ky$ which implies $y\mid t$. Thus, we have that $t\mid y$ and $y\mid t$. Thus, $t=y$. Therefore, $o(a^m)=o(a^d)$.
                \end{proof}
                
            \item Prove that $\langle a^m\rangle=\langle a^d\rangle$.
                \begin{proof}
                    Let $p\in\langle a^d\rangle$. Then for some $k\in\mathbb{Z}$, $p=(a^d)^k$. Note that in part (a) we showed that $a^d=a^{\alpha m}$ for some $\alpha\in\mathbb{Z}$. So then $(a^d)^k=(a^{\alpha m})^k=(a^m)^{\alpha k}$. Since $\alpha k\in\mathbb{Z}$, then $(a^m)^{\alpha k}\in\langle a^m\rangle$. Thus, $p\in\langle a^m\rangle$. Thus, $\langle a^d\rangle\subseteq\langle a^m\rangle$.\par\hspace{4mm} Now let $p\in\langle a^m\rangle$. Then for some $k\in\mathbb{Z}$, $p=(a^m)^k$. However, since $d\mid m$, then for some $q\in\mathbb{Z}$, we have that $m=qd$. Thus, $(a^m)^k=(a^{qd})^k=(a^d)^{kq}$. Since $kq\in\mathbb{Z}$, then $(a^d)^{kq}\in\langle a^d\rangle$. Thus, $p\in\langle a^d\rangle$. Thus,  $\langle a^m\rangle\subseteq\langle a^d\rangle$. Therefore, $\langle a^m\rangle =\langle a^d\rangle$.
                \end{proof}
            
        \end{enumerate}
        
    \item Assume that $(G,\star)$ is a cyclic group, $G=\langle a\rangle$, $(H,\star)$ is a subgroup of $(G,\star)$, and $H\neq\{e\}$.
    
        \begin{enumerate}[label=(\alph*)]
            \item Prove that $H=\langle a^k\rangle$, where $k$ is the smallest positive integer such that $a^k\in H$.
            
                \begin{proof}
                    We are given that $(H,\star)$ is a subgroup of $(G,\star)$ and that $a^k\in H$. It follows from the definition of subgroup that $H$ is closed under $\star$. So then let $g\in\langle a^k\rangle$. Then for some $m\in\mathbb{Z}$, $g=a^{mk}$. Since $a^{mk}$ is the result of applying $\star$ on $a^k$ $m$-many times with itself, then $a^{mk}\in H$ by the closure of $H$ under $\star$. Thus, $g\in H$. Thus, $\langle a^k\rangle\subseteq H$.\par\hspace{4mm} Let $h\in H$. Since $H\subseteq G=\langle a\rangle$, then $h\in \langle a\rangle$ and so for some $m\in\mathbb{Z}$, $h=a^m$. Since $k$ is the smallest positive integer such that $a^k\in H$, then $k\leq m$. Moreover, by the division algorithm, there exists $q,r\in\mathbb{Z}$, with $0\leq r<k$ such that $m=qk+r$. Thus, $a^m=a^{qk+r}=a^{qk}a^r$. Assume that $r\neq 0$, then since $r<k$, it follows that $a^r\notin H$. Now consider
                    
                    \begin{equation*}
                        a^m=a^{qk}a^r=(a^k)^qa^r\Leftrightarrow (a^k)^{-q}a^m=a^r.
                    \end{equation*}
                    
                    Since $(a^k)^q\in H$ and $H$ is a subgroup, then $(a^k)^{-q}\in H$. Thus, $a^r$ is the `product' of two elements of $H$, namely $(a^k)^{-q}$ and $a^m$. Thus, $a^r\in H$. This is a contradiction. Thus, $r=0$ and $a^m=a^{kq}=(a^k)^q$, which implies $a^m\in\langle a^k\rangle$. Thus, $h\in\langle a^k\rangle$. Thus, $H\subseteq\langle a^k\rangle$. Therefore, $H=\langle a^k\rangle$.
                \end{proof}
                
            \item $G/H$ is cyclic (since $G$ is cyclic). Assume $[G\colon H]=t$. Prove that $t=k$.
                \begin{proof}
                
                    If $[G\colon H]=t$, then there are $t$ many distinct right (or left) cosets of $H$ in $G$. Now consider the following cosets of $H$ in $G$
                    
                    \begin{equation*}
                        \begin{split}
                            H        &= \{a^k, a^{2k}, a^{3k},\dots, e\} \\
                            Ha       &= \{a^{k+1}, a^{2k+1}, a^{3k+1}, \dots, a\} \\
                            Ha^2     &= \{a^{k+2}, a^{2k+2}, a^{3k+2}, \dots, a^2\} \\
                                     &\;\;\vdots \\
                            Ha^{k-1} &= \{a^{2k-1}, a^{3k-1}, a^{4k-1},\dots, a^{k-1}\}
                        \end{split}
                    \end{equation*}
                    
                    Clearly, there are $k$ many of these cosets. Now we want to show that this set
                    of cosets forms a partition on $G$. If we can show this, then it would imply that the list above is equal to $G/H$ and that $t=k$.\par\hspace{4mm} First, let $A=\{H,Ha,Ha^2,\dots,Ha^{k-1}\}$. To show that $A$ forms a partition on $G$, we will take two arbitrary elements of $A$, call them $Ha^i$ and $Ha^j$, and show that if $i=j$, then $Ha^i=Ha^j$ and if $i\neq j$, then $Ha^i\cap Ha^j=\varnothing$. The first case follows immediately, for if $i=j$, then $Ha^i=Ha^i=Ha^j$.\par\hspace{4mm} Now assume that $i\neq j$. Let $a^s\in Ha^i\cap Ha^j$. Then we have that $a^s\in Ha^i$. Thus, $a^s=a^{lk+i}$, for some $0\leq l<k$. Similarly, $a^s\in Ha^j$. Thus, $a^s=a^{mk+j}$, for some $0\leq m<k$. Thus, $a^{lk+i}=a^{mk+j}$, which can be written as $a^{lk}\star a^i=a^{mk}\star a^j$. Since $0\leq i,j<k$ and $i\neq j$, then $a^i\neq a^j$. Moreover, because $0\leq l,m<k$, then if $l=m$ this would imply that $a^{mk}\star a^i=a^{mk}\star a^j$ and thus $a^i=a^j$ which is a contradiction. Thus, if $l\neq m$, then $a^{lk}\neq a^{mk}$. Thus, $a^{lk+i}\neq a^{mk+j}$, which is a contradiction. Thus, $a^s\notin Ha^i\cap Ha^j$. Therefore, $Ha^i\cap Ha^j=\varnothing$ and the elements of $A$ are mutually disjoint.\par\hspace{4mm} Now we must show that $G=\cup A$. Clearly, $\cup A\subseteq G$ since $A$ contains, as its elements, subsets of $G$. Let $a^y\in G$. Then either $0\leq y<k$, or $k\leq y<o(a)$. If $0\leq y<k$, then $a^y\in Ha^y\in A$ and thus $a^y\in\cup A$. If $k\leq y<o(a)$, then $y=nk+m$ for some $m,n\in\mathbb{Z}^+$ such that $m<k$ and $nk+m<o(a)$. Thus, $a^y=a^{nk+m}=(a^k)^n\star a^m\in Ha^m$. Thus, $a^y\in \cup A$ and hence $G\subseteq\cup A$. Therefore, $G=\cup A$ and $A$ forms a partition on $G$. Thus, $A=G/H$ and $[G\colon H]=k$.
                \end{proof}
        \end{enumerate}
        
    \item Assume that $(G,\star)$ is a group. $Z(G)=\{a\in G\mid\forall g\in G\colon a\star g=g\star a\}$, and if $x\in G$, then $N(x)=\{y\in G\mid y\star x=x\star y\}$. Prove that $Z(G)\subseteq_g G$, and that $a\in Z(G)$ iff $N(a)=G$.
    
        \begin{proof}
        
            To prove that $Z(G)\subseteq_g G$, we must show, by the corollary on pg.4, that $Z(G)\neq\varnothing$ and $a\star b^{-1}\in Z(G)$ whenever $a,b\in Z(G)$. We can see that $e\in Z(G)$ since for all $g\in G$, $e\star g=g=g\star e$. Thus, $Z(G)\neq\varnothing$. Now let $a,b\in Z(G)$. Then since $a\in Z(G)$, it follows that $\forall g\in G$, $a\star g=g\star a$ and since $b^{-1}\in G$, then $a\star b^{-1}=b^{-1}\star a$. Similarly, since $b\in Z(G)$, then $b\star g=g\star b$. Multiplying both sides on the left by $b^{-1}$ we get that $g=b^{-1}\star g\star b$. Then multiplying both sides on the right by $b^{-1}$, we get $g\star b^{-1}=b^{-1}\star g$. Thus, $b^{-1}\in Z(G)$. Finally, letting $g\in G$ and using all the previous equalities we obtain
            
            \begin{equation*}
                \begin{split}
                    (a\star b^{-1})\star g&=a\star(b^{-1}\star g) \\
                    &= a\star(g\star b^{-1}) \\
                    &= (a\star g)\star b^{-1} \\
                    &= (g\star a)\star b^{-1} \\
                    &= g\star (a\star b^{-1}).
                \end{split}
            \end{equation*}
        
            Thus, $a\star b^{-1}\in Z(G)$. Thus, $Z(G)\subseteq_g G$.\par\hspace{4mm} Assume $a\in Z(G)$. Then $a\star g=g\star a$ for all $g\in G$. Thus, the set of all elements which commute with $a$ is every element of $G$. Hence, $N(a)=G$. Now assume $N(a)=G$. Then $a\star g=g\star a$ for all $g\in G$. Thus, $a\in Z(G)$. Therefore, $a\in Z(G)$ iff $N(a)=G$.
        \end{proof}
        
    \item Assume that $(G,\star)$ is a group, $H\subseteq_g G$, $a\in G$, and $aHa^{-1}=\{a\star h\star a^{-1}\colon h\in H\}$.
        \begin{enumerate}[label=(\alph*)]
            \item Prove that $H\triangleleft G$ iff for all $g\in G$, $gHg^{-1}=H$, iff for all $g\in G$, $Hg=gH$.
                \begin{proof}
                    $(\Rightarrow)$ Assume $H\triangleleft G$. Then by definition, for all $g\in G$ and for all $h\in H$, $g\star h\star g^{-1}\in H$. We want to show that this implies that for all $g\in G$, $gHg^{-1}=H$. First, we will show that for any $g\in G$, $gHg^{-1}\subseteq H$. Let $g\in G$ and let $a\in gHg^{-1}$. Then for some $h'\in H$, $a=g\star h'\star g^{-1}$. However, by assumption we have that for all $g\in G$ and all $h\in H$, $g\star h\star g^{-1}\in H$. Thus, $a=g\star h'\star g^{-1}\in H$. Hence $gHg^{-1}\subseteq H$. Now we want to show that for any $g\in G$, $H\subseteq gHg^{-1}$. Let $g\in G$ and let $h'\in H$. Then by assumption, $g^{-1}\star h'\star g\in H$, since $g^{-1}\star h'\star g=(g^{-1})\star h'\star (g^{-1})^{-1}$. Thus, if we let $h\in H$ such that $h=g^{-1}\star h'\star g$, then $h'=g\star h\star g^{-1}$. Thus, $h'\in gHg^{-1}$. Hence, $H\subseteq gHg^{-1}$. Therefore, for all $g\in G$, $gHg^{-1}=H$.\par\hspace{4mm}
                    $(\Rightarrow)$ Assume that for all $g\in G$, $gHg^{-1}=H$. We want to show that this implies that for all $g\in G$, $Hg=gH$. First, we will show that for all $g\in G$, $Hg\subseteq gH$. Let $g\in G$ and let $a\in Hg$. Then $a=h'\star g$, for some $h'\in H$. Since $h'\in H$, then our assumption implies that for some $h\in H$, $h'=g\star h\star g^{-1}$. Thus, multiplying both sides on the right by $g$, we obtain $h'\star g=g\star h$ and since $a=h'\star g$, then $a=g\star h$. Thus, $a\in gH$. Hence, $Hg\subseteq gH$. Now let $a\in gH$. Then for some $h'\in H$, $a=g\star h'$. Thus, $a\star g^{-1}=h'\in H$. Thus, $a=(a\star g^{-1})\star g\in Hg$. Hence, $gH\subseteq Hg$. Therefore, for all $g\in G$, $Hg=gH$.\par\hspace{4mm} $(\Rightarrow)$ Assume that for all $g\in G$, $Hg=gH$. We want to show that this implies that $H\triangleleft G$. Thus, we need to show that for all $g\in G$ and for all $h\in H$, $g\star h\star g^{-1}\in H$. Let $g\in G$ and $h\in H$. Then for some $h'\in H$, we have that $g\star h=h'\star g$. Multiplying both sides by $g^{-1}$, we obtain that $g\star h\star g^{-1}=h'\in H$. Thus, for all $g\in G$ and for all $h\in H$, $g\star h\star g^{-1}\in H$. Therefore, $H\triangleleft G$.
                \end{proof}
            \item Prove that $aHa^{-1}$ is a subgroup of $G$, and prove that $o(aHa^{-1})=o(H)$.
                \begin{proof}
                    To prove $aHa^{-1}\subseteq_g G$, we must show that $aHa^{-1}\neq\varnothing$ and we must show that $x\star y^{-1}\in aHa^{-1}$ whenever $x,y\in aHa^{-1}$. We know that $e\in H$, and since $a\star e\star a^{-1}=e$, then $e\in aHa^{-1}$. Thus, $aHa^{-1}\neq\varnothing$. Now let $x,y\in aHa^{-1}$. Then for some $h_1,h_2\in H$, we have that $x=a\star h_1\star a^{-1}$ and $y=a\star h_2\star a^{-1}$. Note that $y^{-1}=a\star h_2^{-1}\star a^{-1}$. Thus, 
                    
                    \begin{equation*}
                        \begin{split}
                            x\star y^{-1} &= (a\star h_1\star a^{-1})\star(a\star h_2^{-1}\star a^{-1}) \\
                            &= a\star h_1\star(a^{-1}\star a)\star h_2^{-1}\star a^{-1} \\
                            &= a\star(h_1\star h_2^{-1})\star a^{-1}.
                        \end{split}
                    \end{equation*}
                    
                    Since $h_1\star h_2^{-1}\in H$, then $a\star(h_1\star h_2^{-1})\star a^{-1}\in aHa^{-1}$. Thus, $x\star y^{-1}\in aHa^{-1}$. Therefore, $aHa^{-1}\subseteq_g G$.\par\hspace{4mm} To prove that $o(aHa^{-1})=o(H)$, we will show that there exists a bijective map from $aHa^{-1}$ to $H$. Let $f\colon aHa^{-1}\rightarrow H$ be a map defined as 
                    
                    \begin{equation*}
                        f(x)=a^{-1}\star x\star a
                    \end{equation*}
                    
                    for all $x\in aHa^{-1}$. To show that this map is 1-1, let $x_1,x_2\in aHa^{-1}$ such that $f(x_1)=f(x_2)$. Then we need to show that this implies $x_1=x_2$. Since $x_1,x_2\in aHa^{-1}$, then for some $h_1,h_2\in H$, we have that $x_1=a\star h_1\star a^{-1}$ and that $x_2=a\star h_2\star a^{-1}$. It follows from the definition of $f$ that
                    
                    \begin{equation*}
                        \begin{split}
                            f(x_1)&=f(a\star h_1\star a^{-1}) \\
                            &=a^{-1}\star(a\star h_1\star a^{-1})\star a \\
                            &= (a^{-1}\star a)\star h_1\star (a^{-1}\star a) \\
                            &= e\star h_1\star e \\
                            &= h_1.
                        \end{split}
                    \end{equation*}
                    
                    Similarly, 
                    
                    \begin{equation*}
                        \begin{split}
                            f(x_2) &=f(a\star h_2\star a^{-1}) \\
                            &=a^{-1}\star(a\star h_2\star a^{-1})\star a \\
                            &= (a^{-1}\star a)\star h_2\star (a^{-1}\star a) \\
                            &= e\star h_2\star e \\
                            &= h_2.
                        \end{split}
                    \end{equation*}
                    
                    Thus, $f(x_1)=f(x_2)$ implies that $h_1=h_2$. Multiplying both sides on the right by $a$ and both sides on the left by $a^{-1}$, We get that $a\star h_1\star a^{-1}=a\star h_2\star a^{-1}$. Hence, $x_1=x_2$ and $f$ is thereby injective.\par\hspace{4mm} To prove that $f$ is surjective, we must show that for all $h\in H$, there exists some $x\in aHa^{-1}$, such that $f(x)=h$. Let $h\in H$. Then let $x=a\star h\star a^{-1}$. Clearly, $x\in aHa^{-1}$, and we also find that $f(x)=f(a\star h\star a^{-1})=h$. Hence, $f$ is surjective. Thus, $f$ is a bijective map from $aHa^{-1}$ to $H$. Therefore, $o(aHa^{-1})=o(H)$.
                \end{proof}
            \item Assume that $H$ is the only subgroup of $G$ whose order is $o(H)$. Prove that $H\triangleleft G$.
                \begin{proof}
                    From the previous result, we have that for any $a\in G$, $aHa^{-1}\subseteq_g G$ and that $o(aHa^{-1})=o(H)$. Thus, it follows that $H=aHa^{-1}$. Moreover, from the result obtained in 4.(a), this implies that $H\triangleleft G$.
                \end{proof}
        \end{enumerate}
        
    \item Assume that $(G,\star)$ is a group, $H\subseteq_g G$, and that $[G\colon H]=2$. Prove that $H\triangleleft G$.
        \begin{proof}
           Let $a\in G$. Then $a\in H$ or $a\notin H$. If $a\in H$, then $aH\subseteq H$ since for any $x\in aH$, we have that $x=a\star h$, where both $a,h\in H$ and so $a\in H$. Similarly, for any $h\in H$, we have that $h=a\star(a^{-1}\star h)$ and thus, $x\in aH$. Thus, $H\subseteq aH$. Hence, $aH=H=Ha$. If $a\notin H$, then $a\in G-H$. Thus, $aH=G-H=Ha$. Thus, for any $a\in G$, $Ha=aH$. Therefore, $H\triangleleft G$.      
        \end{proof}
        
    \item Assume that $(G,\star)$ is a group, $(H,\star)$ is a subgroup of $G$, and $R$ is defined on $G$ by $aRb$ iff $a\star b^{-1}\in H$. Recall that $R$ is an equivalence relation on $G$.
        \begin{enumerate}[label=(\alph*)]
            \item If $G=S_4$, and $H=\{(1),(12),(34),(12)(34)\}$, prove that $R$ is not a congruence relation.
                \begin{proof}
                    Consider the elements $(13)$ and $(132)$. Note that $(132)^{-1}=(123)$ and so $(13)\star(123)=(12)$. Thus, $(13)\star(132)^{-1}\in H$ and so $(13)R(132)$. Next, consider the elements $(14)$ and $(142)$. Note that $(142)^{-1}=(124)$. Thus, $(14)\star(124)=(12)$. Thus, $(14)\star(142)^{-1}\in H$. Thus, $(14)R(142)$. Lastly, consider the products $(13)\star(14)=(143)$ and $(132)\star(142)=(14)\star(23)$. Also note that $((14)\star(23))^{-1}=(14)\star(23)$. Thus, $(143)\star(14)\star(23)=(132)$. We see that $(132)\notin H$. Thus, it is not true that for all $a,b,c,d\in G$, $aRb$ and $cRd$ implies $(a\star c)R(b\star d)$. Therefore, $R$ is not a congruence relation.
                \end{proof}
                
            \item Prove that if $H\triangleleft G$, then $R$ is a congruence relation.
                \begin{proof}
                    Assume that $H\triangleleft G$. Then for $a,b,c,d\in G$, assume $aRb$ and $cRd$. Then we have that $a\star b^{-1}\in H$ and $c\star d^{-1}\in H$. We want to show that $(a\star c)\star(b\star d)^{-1}\in H$. We see that $(a\star c)\star(b\star d)^{-1}=a\star(c\star d^{-1})\star b^{-1}$. Additionally, since $c\star d^{-1}\in H$, let $h=c\star d^{-1}$. Then we have 
                    
                    \begin{equation*}
                        \begin{split}
                            a\star h\star b^{-1} &= (a\star h\star b^{-1})\star (b\star a^{-1})\star (a\star b^{-1})^{-1} \\
                            &= (a\star h\star a^{-1})\star (b\star a^{-1}).
                        \end{split}
                    \end{equation*}
                    
                    Since $H$ is normal, then $aHa^{-1}=H$ and so $a\star h\star a^{-1}\in H$. Moreover, since $a\star b^{-1}\in H$, then $(a\star b^{-1})^{-1}=(b\star a^{-1})\in H$. Thus, $(a\star h\star a^{-1})\star(b\star a^{-1})\in H$. Hence, 
                    
                    \begin{equation*}
                        a\star h\star b^{-1}=a\star(c\star d^{-1})\star b^{-1}=(a\star c)\star(b\star d)^{-1}\in H.
                    \end{equation*}
                    
                    Thus, $aRb$ and $cRd$ implies $(a\star c)R(b\star d)$, for all $a,b,c,d\in G$. Therefore, $R$ is a congruence relation.
                \end{proof}
        \end{enumerate}
        
\end{enumerate}



\end{document}