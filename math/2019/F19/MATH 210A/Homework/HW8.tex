\documentclass[12pt]{article}
\usepackage[margin=1in]{geometry} 
\usepackage{graphicx}
\usepackage{amsmath}
\usepackage{authblk}
\usepackage{titlesec}
\usepackage{amsthm}
\usepackage{amsfonts}
\usepackage{amssymb}
\usepackage{array}
\usepackage{booktabs}
\usepackage{ragged2e}
\usepackage{enumerate}
\usepackage{enumitem}
\usepackage{cleveref}
\usepackage{slashed}
\usepackage{commath}
\usepackage{lipsum}
\usepackage{colonequals}
\usepackage{addfont}
\usepackage{enumitem}
\usepackage{sectsty}
\usepackage{lastpage}
\usepackage{fancyhdr}
\usepackage{accents}
\usepackage{xcolor}
\usepackage[inline]{enumitem}
\pagestyle{fancy}
\setlength{\headheight}{10pt}

\subsectionfont{\itshape}

\newtheorem{theorem}{Theorem}[section]
\newtheorem{corollary}{Corollary}[theorem]
\newtheorem{prop}{Proposition}[section]
\newtheorem{lemma}[theorem]{Lemma}
\theoremstyle{definition}
\newtheorem{definition}{Definition}[section]
\theoremstyle{remark}
\newtheorem*{remark}{Remark}
 
\makeatletter
\renewenvironment{proof}[1][\proofname]{\par
  \pushQED{\qed}%
  \normalfont \topsep6\p@\@plus6\p@\relax
  \list{}{\leftmargin=0mm
          \rightmargin=4mm
          \settowidth{\itemindent}{\itshape#1}%
          \labelwidth=\itemindent
          \parsep=0pt \listparindent=\parindent 
  }
  \item[\hskip\labelsep
        \itshape
    #1\@addpunct{.}]\ignorespaces
}{%
  \popQED\endlist\@endpefalse
}

\newenvironment{solution}[1][\bf{\textit{Solution}}]{\par
  
  \normalfont \topsep6\p@\@plus6\p@\relax
  \list{}{\leftmargin=0mm
          \rightmargin=0mm
          \settowidth{\itemindent}{\itshape#1}%
          \labelwidth=\itemindent
          \parsep=0pt \listparindent=\parindent 
  }
  \item[\hskip\labelsep
        \itshape
    #1\@addpunct{.}]\ignorespaces
}{%
  \popQED\endlist\@endpefalse
}

\let\oldproofname=\proofname
\renewcommand{\proofname}{\bf{\textit{\oldproofname}}}


\newlist{mylist}{enumerate*}{1}
\setlist[mylist]{label=(\alph*)}

\begin{document}

\begin{center}
	\vspace{.4cm} {\textbf { \large MATH 210A}}
\end{center}
{\textbf{Name:}\ Quin Darcy \hspace{\fill} \textbf{Due Date:} 10/30/19   \\
{ \textbf{Instructor:}}\ Dr. Shannon \hspace{\fill} \textbf{Assignment:} Homework 8 \\ \hrule}

\justifying

    \begin{enumerate}[leftmargin=*]
        \item Find the conjugacy classes of the elements of $A_4$, and verify that the class equation holds for $A_4$. (recall that $c(a)=o(G)/o(N(a))$, and as part of this process, find $N((12)(34))$)
            \begin{solution}
                In $S_4$ we know that given any permutation $\sigma\in S_4$, the conjugacy class of $\sigma$ will contain all other permutations with the same cycle structure as $\sigma$. Hence, in $S_4$, we have that 
                
                \begin{equation*}
                    \begin{split}
                        c((1))&=\{(1)\} \\
                        c((12))&=\{(12),(13),(14),(23),(24),(34)\} \\
                        c(((123))&=\{(123),(132),(124),(142),(134),(143),(234),(243)\} \\
                        c((1234))&=\{(1234),(1243),(1324),(1342),(1423),(1432)\} \\
                        c((12)(34))&=\{(12)(34),(13)(24),(14)(23)\}.
                    \end{split}
                \end{equation*}
                
                Since $A_4$ contains only even permutations, then it is only $c((1)),c((123)),$ and $c((12)(34))$ that we are interested in. By simple calculations, we find that $c((123))$ in $S_4$ splits in $A_4$ whereas $c((1))$ and $c((12)(34))$ remain the same. Thus, the conjugacy classes in $A_4$ are 
                
                \begin{equation*}
                    \begin{split}
                        c((1))&=\{(1)\} \\
                        c((123))&=\{(123),(134),(142),(243)\} \\
                        c((132))&=\{(132),(124),(143),(234)\} \\
                        c((12)(34))&=\{(12)(34),(13)(24),(14)(23)\}. 
                    \end{split}
                \end{equation*}
                
                We can now use the fact that $c(a)=o(G)/o(N(a))$. Note that $o(A_4)=12$. It follows that
                
                \begin{align*}
                    o(N((1)))&=\frac{12}{12}=1 & o(N((123)))=\frac{12}{4}=3 \\
                    o(N((132)))&=\frac{12}{4}=3 & o(N((12)(34)))=\frac{12}{3}=4.
                \end{align*}
                
                Thus, 
                
                \begin{equation*}
                    \begin{split}
                        \abs{A_4}&=\abs{Z(A_4)}+\sum_{N(\sigma)\neq A_4}\frac{\abs{A_4}}{\abs{N(\sigma)}} \\
                        &= \abs{\{(1)\}}+\frac{12}{3}+\frac{12}{3}+\frac{12}{4} \\
                        &= 1+4+4+3 \\
                        &=12.
                    \end{split}
                \end{equation*}
                
                Lastly, we have that $N((12)(34))=\{(1),(12)(34),(13)(24),(14)(23)\}$.
            \end{solution}
            
        \newpage
        
        \item[3.] Let $o(G)=3\times 5\times 7$.\hfill\par 
            \begin{enumerate}[label=(\alph*)]
                \item Prove that $G$ is not simple. 
                    \begin{proof}
                        We want to show that $G$ contains a proper normal subgroup. By Sylow I, $G$ contains subgroups of orders $3,5,$ and $7$, respectively. Since $3,5,$ and $7$ are the highest powers of these primes that occur in the order of $G$, then it follows that those subgroups are Sylow subgroups. Thus, by Sylow III, we have that 
                        
                        \begin{equation*}
                            n_3\equiv 1(\text{mod }3)&\wedge n_3\mid 35,\quad 
                            n_5\equiv 1(\text{mod }5)&\wedge n_5\mid 21,\quad
                            n_7\equiv 1(\text{mod }7)&\wedge n_7\mid 15.
                        \end{equation*}
                        
                        From these relations we conclude that $n_3=1$ or $n_3=7$; $n_5=1$; $n_7=1$. Thus, $P_5\triangleleft G$ and $P_7\triangleleft G$. Finally, since $\abs{P_5}=5<105=o(G)$. Thus, $P_5\subset G$. Thus, $G$ contains a proper normal subgroup. Therefore, $G$ is not simple.
                    \end{proof}
                    
                \item Prove that $G$ has a normal subgroup of order 35.
                    \begin{proof}
                        From (a) we have that $P_5\triangleleft G$ and $P_7\triangleleft G$. Additionally, we have that $P_5\cap P_7=\{e\}$. Thus, $P_5P_7\cong P_5\times P_7\cong \mathbb{Z}_5\times\mathbb{Z}_7\cong\mathbb{Z}_{35}$. Since $P_5P_7\subseteq_g G$, then it follows that $G$ contains a subgroup of order 35. We now only need to show that $P_5P_7\triangleleft G$. So let $g\in G$, $x\in P_5$, and $y\in P_7$. Then $g(xy)g^{-1}=(gxg^{-1})(gyg^{-1})$ and since $P_5\triangleleft G$ and $P_7\triangleleft G$, then $gxg^{-1}\in P_5$ and $gyg^{-1}\in P_7$. Thus, $(gxg^{-1})(gyg^{-1})\in P_5P_7$. Thus, $g(xy)g^{-1}\in P_5P_7$. Thus, $P_5P_7\triangleleft G$. Therefore, $G$ contains a normal subgroup of order 35.
                    \end{proof}
                    
                \item Prove that both $P_5\triangleleft G$ and $P_7\triangleleft G$.
                    \begin{proof}
                        This follows from part (a).
                    \end{proof}
                    
                \item Prove that if $P_3\triangleleft G$, then $G$ is Abelian.
                    \begin{proof}
                        Assume that $P_3\triangleleft G$. Then since $P_3\cap P_5=\{e\}$, $P_3\cap P_7=\{e\}$, and $P_5\cap P_7=\{e\}$, then $G=P_3P_5P_7\cong P_3\times P_5\times P_7\cong \mathbb{Z}_3\times\mathbb{Z}_5\times \mathbb{Z}_7\cong\mathbb{Z}_{105}$. Therefore, $G$ is Abelian.
                    \end{proof}
                    
                \item Prove that there are only two groups of order 105. 
                    \begin{proof}
                        Let $P_3=\langle a\rangle$, $P_5=\langle b\rangle$, and $P_7=\langle c\rangle$. Then since $P_7\triangleleft G$ and $P_3\subseteq_g G$, then $\langle a\rangle\langle c\rangle\subseteq_g G$. Let $H=\langle a\rangle\langle c\rangle$. Now assume that $\theta\colon H\rightarrow\text{Aut}(\langle b\rangle)$ is a homomorphism, where $\theta(h)=\varphi_k$ and $\varphi_k(x)=x^k$. From class we know that each $\varphi_k$ corresponds to $hbh^{-1}=b^k$, for $h\in H$. Additionally, we must also satisfy $o(\theta(h))\mid o(h)$. Thus, since $\langle a\rangle\cap\langle c\rangle=\{e\}$, then $o(\langle a\rangle\langle c\rangle)=21$ and so $o(\theta(h))\mid 21$. Note that it must also hold that $o(\theta(h))\mid o(\text{Aut}(\langle b\rangle))=4$. Thus, $o(\theta(h))=1$ and $hbh^{-1}=b$. Finally, since $G=H\langle b\rangle$ (since $H\cap\langle b\rangle=\{e\})$ and from hw 7, there are two possibilities for $H$ ($\mathbb{Z}_{21}$ and the nonabelian group), then there are only two possible groups of order 105. 
                    \end{proof}
            \end{enumerate}
            
        \newpage
            
            \item[4.] Determine if the following are always true. If the statement if always true, then prove it, and if it is not always, then give a counter example.
                \begin{enumerate}[label=(\alph*)]
                    \item If $G$ is a group, $N\triangleleft G$, $G/N$ is Abelian, and $N$ is Abelian, then $G$ is Abelian.
                        \begin{proof}
                            This is not always true. Suppose $o(G)=6$, then $G$ has a 3-Sylow subgroup which is normal in $G$. Additionally, we have that $\abs{G/P_3}=2$ and it is therefore Abelian. However, by the results on pg. 22, $G$ is either isomorphic to $\mathbb{Z}_6$ or isomorphic to $D_6$. The latter case is an instance of a nonabelian group. Therefore, we have that $P_3\triangleleft G$, $G/P_3$ is Abelian, $P_3$ is Abelian, but $G$ is not Abelian.
                        \end{proof}
                    
                    \item If $G$ is a group, $N\triangleleft G$, $G/N\cong M$, then $G\cong N\times M$.
                        \begin{proof}
                            This is not always true. Let $G=S_3$ and let $N=A_3$. Since $\abs{A_3}=3$, then $[S_3\colon A_3]=2$ and thus by hw 2, $A_3\triangleleft G$. Now define $\theta\colon S_3/A_3\rightarrow\mathbb{Z}_2$, by 
                            
                           \begin{equation*}
                                \theta(\sigma A_3)=
                                \begin{cases} 
                                    [0] & \quad\text{if}\quad \sigma\in A_3 \\
                                    [1] & \quad\text{if}\quad \sigma\notin A_3, \\
                                \end{cases}
                            \end{equation*}
                            
                            By definition, $\theta(S_3/A_3)\subseteq\mathbb{Z}_2$. Now let $\sigma A_3,\gamma A_3\in S_3/A_3$ and to show $\theta$ is well defined assume that $\sigma A_3=\gamma A_3$. Then if $\sigma\in A_3$, then $\sigma A_3=\gamma A_3$ implies that for $\delta,\lambda\in A_3$, $\sigma\delta=\gamma\lambda$. Thus, $\gamma=\sigma\delta\lambda^{-1}\in A_3$. Thus, if $\sigma\in A_3$ and $\sigma A_3=\gamma A_3$, then $\gamma\in A_3$. Thus, $\theta(\sigma A_3)=[0]=\theta(\gamma A_3)$. If $\sigma\notin A_3$ and $\sigma A_3=\gamma A_3$, then by the same reasoning, $\gamma=\sigma\delta\lambda^{-1}\notin A_3$. Thus, $\theta(\sigma A_3)=[1]=\theta(\gamma A_3)$. Hence, $\theta$ is well defined.\par\hspace{4mm} Now assume $\theta(\sigma A_3)=(\gamma A_3)$. Then either $\sigma,\gamma\in A_3$ or $\sigma,\gamma\notin A_3$. In the former case, we have that $\sigma,\gamma\in A_3$ and thus $\sigma A_3=A_3=\gamma A_3$. In the latter case where $\sigma,\gamma\notin A_3$, then since $[S_3\colon A_3]=2$ it follows that $\sigma A_3=\gamma A_3\neq A_3$. Thus, $\theta$ is 1-1.\par\hspace{4mm} We can see that $\theta(A_3)=[0]$ and $\theta((12)A_3)=[1]$ and thus $\theta$ is onto.\par Finally, we see that for $\sigma A_3,\gamma A_3\in S_3/A_3$ we have three cases:
                            
                            \begin{enumerate}[label=(\roman*)]
                                \item $\sigma,\gamma\in A_3$. In this case we have that 
                                    \begin{equation*}
                                        \theta(\sigma A_3\gamma A_3)=\theta((\sigma\gamma)A_3)=[0]=[0]+[0]=\theta(\sigma A_3)+\theta(\gamma A_3).
                                    \end{equation*}
                                \item $\sigma\in A_3$, $\gamma\notin A_3$. In this case we have that
                                    \begin{equation*}
                                        \theta(\sigma A_3\gamma A_3)=\theta((\sigma\gamma)A_3)=[1]=[0]+[1]=\theta(\sigma A_3)+\theta(\gamma A_3).
                                    \end{equation*}
                                \item $\sigma,\gamma\notin A_3$. In this case we have that
                                    \begin{equation*}
                                        \theta(\sigma A_3\gamma A_3)=\theta((\sigma\gamma)A_3)=[0]=[1]+[1]=\theta(\sigma A_3)+\theta(\gamma A_3).
                                    \end{equation*}
                            \end{enumerate}
                            
                            Therefore, $\theta$ is an isomorphism and $S_3/A_3\cong\mathbb{Z}_2$. Now note that since $o(A_3)=3$, then by hw 1, $A_3$ is Abelian. Also, since $\mathbb{Z}_2$ is Abelien, then $A_3\times\mathbb{Z}_2$ is Abelian. However, $S_3$ is not Abelian. Thus, $S_3\not\cong A_3\times\mathbb{Z}_2$.
                        \end{proof}
                    \newpage
                    
                    \item If $G$ is cyclic then $\text{Aut}(G)$ is cyclic.
                        \begin{proof}
                            This is not always the case. Let $G$ be a cyclic group of order 21. Then from hw 5, $\text{Aut}(G)\cong\mathbb{Z}_{(21)}$. We have that $o(\mathbb{Z}_{(21)})=12$. In order for $\text{Aut}(G)$ to be cyclic, it must be the case that $\mathbb{Z}_{(21)}$ is cyclic. However, we have that 
                            
                            \begin{align*}
                                \langle[1]\rangle &= \{[1]\} &\langle[11]\rangle&=\{[1],[2],[4],[8],[11],[16]\} \\
                                \langle[2]\rangle &= \{[1],[2],[4],[8],[11],[16]\} &\langle[13]\rangle&=\{[1],[13]\} \\
                                \langle[4]\rangle &= \{[1],[4],[16]\} &\langle[16]\rangle&=\{[1],[4],[16]\} \\
                                \langle[5]\rangle &= \{[1],[4],[5],[16],[17],[20]\} &\langle[17]\rangle&=\{[1],[4],[5],[16],[17],[20]\} \\
                                \langle[8]\rangle &= \{[1],[8]\} &\langle[19]\rangle&=\{[1],[4],[10],[13],[16],[19]\} \\
                                \langle[10]\rangle &= \{[1],[4],[10],[13],[16],[19]\} &\langle[20]\rangle&=\{[1],[20]\}. \\
                            \end{align*}
                            
                            From this we can see that $\mathbb{Z}_{(21)}$ has no generator and is therefore not cyclic. Thus, $\text{Aut}(G)$ is not cyclic.
                        \end{proof}
                        
                    \item If $G/Z(G)$ is cyclic, then $G$ is Abelian.
                        \begin{proof}
                            Since $G/Z(G)$ is cyclic, then for some $g\in G$, $\langle gZ(G)\rangle=G/Z(G)$. Now let $x,y\in G$, then for some $k,m\in\mathbb{Z}^{+}$, and $h,h'\in Z(G)$, we have $x=g^kh$ and $y=g^mh'$. Thus, 
                            
                            \begin{equation*}
                                xy=g^khg^mh'=g^kg^mhh'=g^{k+m}h'h=g^{m+k}h'h=g^mg^kh'h=g^mh'g^kh=yx.
                            \end{equation*}
                            
                            Therefore, $G$ is Abelian.
                        \end{proof}
                        
                    \item If $o(G)=n$, $m>0$ and $m\mid n$, then $G$ has a subgroup $H$, with $o(H)=m$.
                        \begin{proof}
                            This is not always the case. As we saw in part (c), $o(\mathbb{Z}_{(21)})=12$ and we have that $4\mid 12$. However, $\mathbb{Z}_{(21)}$ does not have a subgroup of order 4.
                        \end{proof}
                        
                    \item If $k>0$ and $k\mid n$, then $\mathbb{Z}_n$ has a unique subgroup of order $k$.
                        \begin{proof}
                            This is not always the case. Let $G=\mathbb{Z}_{12}$. Then we have that $3\mid o(\mathbb{Z}_{12})$, however, $o(\langle 4\rangle)=3=o(\langle 8\rangle)$.
                        \end{proof}
                \end{enumerate}
                
            \item[5.] Assume that $G$ is a finite group, $p$ is prime, $p\mid o(G)$, and for all $a,b\in G$, $(ab)^p=a^pb^p$. 
                \begin{enumerate}[label=(\alph*)]
                    \item Prove, by induction, that for all $a,b\in G$, and for all $k\in\mathbb{Z}^{+}$, $(ab)^{p^k}=a^{p^k}b^{p^k}$.
                        \begin{proof}
                            Let $P(n):=(ab)^{p^k}=a^{p^k}b^{p^k}$. The case where $k=1$ is given by our assumption.\par\vspace{2mm}\underline{\textsc{Base Case:}} Let $k=2$. Then $(ab)^{p^2}=\big((ab)^p\big)^p=\big(a^pb^p)^p$. Since $a^p,b^p\in G$, then by our assumption we have that $\big(a^pb^p\big)^p=(a^p)^p(b^p)^p=a^{p^2}b^{p^2}$. Therefore, $P(2)$ holds.
                            
                            \par\vspace{2mm}\underline{\textsc{Inductive Step:}} Let $k>2$ and assume that $P(k)$ holds. Then we have that $(ab)^{p^k}=a^{p^k}b^{p^k}$. We can then consider when we have $k+1$ and see that 
                            
                            \begin{equation*}
                                (ab)^{p^{k+1}}=(ab)^{p^kp}=\big((ab)^{p^k}\big)^p=(a^{p^k}b^{p^k})^p=a^{p^kp}b^{p^kp}=a^{p^{k+1}}b^{p^{k+1}}.
                            \end{equation*}
                            
                            Thus, if $P(k)$ holds then $P(k+1)$ holds. Therefore, $P(n)$ holds for all $a,b\in G$ and for all $n\in\mathbb{Z}^{+}$, $P(n)$ holds.
                        \end{proof}
                        
                    \item Let $o(G)=p^mt$, where $(p,t)=1$. Define $\theta\colon G\rightarrow G$ by $\theta(g)=g^{p^m}$. Prove that $\theta$ is a homomorphism.
                        \begin{proof}
                            Since $G$ is closed, then for each $g\in G$, we have that $g^{p^m}\in G$. Thus, $\theta(G)\subseteq G$. Now let $h,g\in G$, then $\theta(gh)=(gh)^{p^m}$. Since $p\mid o(G)$ and $g,h\in G$, then by (a) it follows that $(gh)^{p^m}=g^{p^m}h^{p^m}=\theta(g)\theta(h)$. Therefore, $\theta$ is a homomorphsim.
                        \end{proof}
                        
                    \item Prove that $o(\ker\theta)=p^m$.
                        \begin{proof}
                            Let $g\in\ker\theta$. Then $\theta(g)=g^{p^m}=e$. Thus, $o(g)\mid p^m$. Now let $g\in G$ such that $o(g)\mid p^m$. Then $g^{p^m}=e$ and thus $\theta(g)=e$, and so $g\in\ker\theta$. Therefore, $g\in\ker\theta$ iff $o(g)\mid p^m$.\par\hspace{4mm} Let $P$ be a $p$-Sylow subgroup of $G$. Then $o(P)=p^m$ and for every $x\in P$, we have that $o(x)\mid p^m$. Thus, for any $x\in P$, it follows that $\theta(x)=x^{p^m}=e$. Hence, $x\in\ker\theta$. Therefore, $P\subseteq\ker\theta$.\par\hspace{4mm} If we let $a,b\in\ker\theta$ such that $a,b\in P$, then since $P$ is a group, we have that $ab^{-1}\in P$. Thus, $P\subseteq_g\ker\theta$. By Lagrange's Theorem, $o(P)\mid o(\ker\theta)$. Thus, $o(\ker\theta)=kp^m$. Since $o(G)=p^mt$ and $(p,t)=1$, then it cannot be the case that $p\mid k$. Otherwise, $o(\ker\theta)$ would contain a power of $p$ greater than that which occurs in the  order of $G$ and this is not possible since $o(\ker\theta)\mid p^mt$.\par\hspace{4mm} If $q$ is a prime distinct from $p$ and $q\mid k$, then by Cauchy's Theorem, there exists some $y\in\ker\theta$ such that $o(y)=q$. However, since $(q,p)=1$, then $q\nmid p^m$. Thus, $o(y)\nmid p^m$ and so by the above equivalence, $y\notin\ker\theta$ and $k$ contains no powers of $p$, nor does it contain any prime factors distinct from $p$. Hence, $k=1$. Therefore, $\ker\theta=p^m$.
                        \end{proof}
                    
                    \item Prove that if $P$ is a $p$-Sylow subgroup of $G$, then $P\triangleleft G$.
                        \begin{proof}
                            In (c) we showed that if $P$ is any $p$-Sylow subgroup of $G$, then $P\subseteq_g\ker\theta$ and $o(P)=o(\ker\theta)$. Thus, $P=\ker\theta$. Additionally, we know that if $f\colon G\rightarrow H$ is a group homomorphism, then $\ker f\triangleleft G$. Thus, since $\theta$ is a group homomorphism, then $\ker\theta\triangleleft G$ and so $P\triangleleft G$.
                        \end{proof}
                        
                    \item Let $N=\theta(G)$. Prove that if $N\triangleleft G$ and that $G$ is the direct product of $P$ and $N$.
                        \begin{proof}
                            By the FHT, $G/\ker\theta\cong\theta(G)$ and thus $o(G/\ker\theta)=o(\theta(G))=t$. Now let $x\in G$ and take $g^{p^m}\in\theta(G)$. Then $(xgx^{-1})^{p^m}=(xgx^{-1})(xgx^{-1})\cdots(xgx^{-1})$. Cancelling out the $x$'s, we find that $(xgx^{-1})^{p^m}=xg^{p^m}x^{-1}$. Thus, since $(xgx^{-1})^{p^m}\in N$, then $xg^{p^m}x^{-1}\in N$. Hence, $N\triangleleft G$. Now note that for any $g\in P$ such that $g\neq e$, we have that $o(g)\mid p^m$ and for any $h\in N$ such that $h\neq e$, we have that $o(h)\mid t$. Now suppose there exists $x\in N\cap P$ such that $x\neq e$. Then $o(x)\mid p^m$. Thus, there exists $k\in\mathbb{Z}$ such that $0<k\leq m$ and $o(x)=p^k$. We also have that $o(x)\mid t$. Thus, $p^k\mid t$. Hence, for some $r\in\mathbb{Z}$, $t=p^kr=p(p^{k-1}r)$. Thus, $p\mid t$. This is not possible since $(p,t)=1$. Thus, $N\cap P=\{e\}$. Therefore, $G=NP$.
                        \end{proof}
                \end{enumerate}
                
              \item Assume that $G$ is an Abelian group, $o(G)=3^8$, and that if $g\in G$, then $o(g)\leq 3^4$.
                        \begin{enumerate}[label=(\alph*)]
                            \item Find all the possibilities for $G$. 
                                \begin{proof}
                                    By the restriction that for all $g\in G$, $o(g)\leq 3^4$, we have precluded the possibility of $G\cong \mathbb{Z}_{3^8}$. We now appeal to the Fundamental Theorem of Finite Abelian Groups we proved on pages 26-28. From this theorem the possibilities of $G$ can be first written as 
                                    
                                    \begin{align*}
                                        &(3^4,3^4)& &(3^4,3^3,3)&  &(3^4,3^2,3^2)& &(3^4,3^2,3,3)& \\ &(3^4,3,3,3, 3)& 
                                        &(3^3,3^3,3^2)& &(3^3,3^3,3,3)&  &(3^3,3^2,3^2,3)& \\ &(3^3,3^2,3,3,3)& 
                                        &(3^3,3,3,3,3,3)& &(3^2,3^2,3^2,3^2)& &(3^2,3^2,3^2,3,3)& \\ &(3^2,3^2,3,3,3,3)& 
                                        &(3^2,3,3,3,3,3,3)& &(3,3,3,3,3,3,3,3)&.
                                    \end{align*}
                                    
                                    Thus, the possibilities for $G$ are 
                                    
                                    \begin{align*}
                                        &\mathbb{Z}_{81}\times\mathbb{Z}_{81}& \\ &\mathbb{Z}_{81}\times\mathbb{Z}_{27}\times\mathbb{Z}_3& \\ &\mathbb{Z}_{81}\times\mathbb{Z}_9\times\mathbb{Z}_9& \\
                                        &\mathbb{Z}_{81}\times\mathbb{Z}_9\times\mathbb{Z}_3\times\mathbb{Z}_3& \\ &\mathbb{Z}_{81}\times\mathbb{Z}_3\times\mathbb{Z}_3\times\mathbb{Z}_3\times\mathbb{Z}_3& \\ &\mathbb{Z}_{27}\times\mathbb{Z}_{27}\times\mathbb{Z}_9& \\
                                        &\mathbb{Z}_{27}\times\mathbb{Z}_{27}\times\mathbb{Z}_3\times\mathbb{Z}_3&\\ &\mathbb{Z}_{27}\times\mathbb{Z}_9\times\mathbb{Z}_9\times\mathbb{Z}_3&\\ &\mathbb{Z}_{27}\times\mathbb{Z}_9\times\mathbb{Z}_3\times\mathbb{Z}_3\times\mathbb{Z}_3& \\ &\mathbb{Z}_{27}\times\mathbb{Z}_3\times\mathbb{Z}_3\times\mathbb{Z}_3\times\mathbb{Z}_3\times\mathbb{Z}_3&\\ &\mathbb{Z}_9\times\mathbb{Z}_9\times\mathbb{Z}_9\times\mathbb{Z}_9&\\ &\mathbb{Z}_9\times\mathbb{Z}_9\times\mathbb{Z}_9\times\mathbb{Z}_3\times\mathbb{Z}_3& \\ &\mathbb{Z}_9\times\mathbb{Z}_9\times\mathbb{Z}_3\times\mathbb{Z}_3\times\mathbb{Z}_3\times\mathbb{Z}_3& \\ &\mathbb{Z}_9\times\mathbb{Z}_3\times\mathbb{Z}_3\times\mathbb{Z}_3\times\mathbb{Z}_3\times\mathbb{Z}_3\times\mathbb{Z}_3&\\
                                        &\mathbb{Z}_3\times\mathbb{Z}_3\times\mathbb{Z}_3\times\mathbb{Z}_3\times\mathbb{Z}_3\times\mathbb{Z}_3\times\mathbb{Z}_3\times\mathbb{Z}_3&.
                                    \end{align*}
                                \end{proof}
                                
                            \item Determine the number and the types of all Abelian groups of order $2^65^27^3$.
                                \begin{proof}
                                    To determine this number, we count the number of ways we can partition 6, 2, and 3 and the answer will be the product of these 3 numbers. Thus, for 6 we have $6:(6),(5,1),(4,2),(4,1,1),(3,3),(3,2,1),(3,1,1,1),(2,2,2)$,\par $(2,2,1,1),(2,1,1,1,1,1,1),(1,1,1,1,1,1)$. So there are 11 partitions for 6. For 2, we have $(2), (1,1)$ so there are 2 partitions of 2. Lastly, for 3, we have $(3), (2,1),(1,1,1)$ and so there are 3. Hence, the number of Abelian groups of order $2^65^27^3$ is $11\times2\times3=66$.
                                \end{proof}
                        \end{enumerate}
    \end{enumerate}
    
\end{document}