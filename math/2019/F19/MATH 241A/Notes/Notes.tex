\documentclass{article}
\usepackage{graphicx}
\usepackage{tikz}
\usepackage{amsmath}
\usepackage{authblk}
\usepackage{titlesec}
\usepackage{amsthm}
\usepackage{amsfonts}
\usepackage{amssymb}
\usepackage{array}
\usepackage{booktabs}
\usepackage{ragged2e}
\usepackage{enumerate}
\usepackage{enumitem}
\usepackage{cleveref}
\usepackage{slashed}
\usepackage{commath}
\usepackage{lipsum}
\usepackage{colonequals}
\usepackage{addfont}
\usepackage{enumitem}
\usepackage{sectsty}
\usepackage{mathtools}

\usepackage{hyperref}
\hypersetup{
    colorlinks=true,
    linkcolor=blue,
    filecolor=magenta,      
    urlcolor=cyan,
}

\usetikzlibrary{decorations.pathreplacing}
\usetikzlibrary{arrows.meta}


%\subsectionfont{\itshape}

\newtheorem{theorem}{Theorem}[section]
\newtheorem{corollary}{Corollary}[theorem]
\newtheorem{lemma}[theorem]{Lemma}
\theoremstyle{definition}
\newtheorem{prop}{Proposition}[section]
\newtheorem{definition}{Definition}[section]
\theoremstyle{remark}
\newtheorem*{remark}{Remark}

\let\oldproofname=\proofname
\renewcommand{\proofname}{\bf{\textit{\oldproofname}}}

\newcommand{\closure}[2][3]{%
  {}\mkern#1mu\overline{\mkern-#1mu#2}}

\theoremstyle{definition}
\newtheorem{example}{Example}[section]

\newtheorem*{discussion}{Discussion}

\makeatletter
\renewenvironment{proof}[1][\proofname]{\par
  \pushQED{\qed}%
  \normalfont \topsep6\p@\@plus6\p@\relax
  \list{}{\leftmargin=0mm
          \rightmargin=0mm
          \settowidth{\itemindent}{\itshape#1}%
          \labelwidth=\itemindent
          \parsep=0pt \listparindent=0mm%\parindent 
  }
  \item[\hskip\labelsep
        \itshape
    #1\@addpunct{.}]\ignorespaces
}{%
  \popQED\endlist\@endpefalse
}

\newenvironment{solution}[1][\bf{\textit{Solution}}]{\par
  
  \normalfont \topsep6\p@\@plus6\p@\relax
  \list{}{\leftmargin=0mm
          \rightmargin=0mm
          \settowidth{\itemindent}{\itshape#1}%
          \labelwidth=\itemindent
          \parsep=0pt \listparindent=\parindent 
  }
  \item[\hskip\labelsep
        \itshape
    #1\@addpunct{.}]\ignorespaces
}{%
  \popQED\endlist\@endpefalse
}


\begin{document}

\title{Notes for Applied Mathematics}
\author{Quin Darcy}
\date{May 18, 2019}
\affil{\small{California State University, Sacramento}}
\maketitle

\section{The Pi Theorem}

In general, the Pi Theorem states that a physical law 

\begin{equation}
    f(q_1, q_2, \dots, q_m)=0
\end{equation}

\noindent relating $m$ dimensional quantities $q_1, q_2, \dots, q_m$, is equivalent to a physical law 

\begin{equation}
    F(\pi_1, \pi_2, \dots, \pi_k)=0
\end{equation}

\noindent relating $k$ dimensionless quantities $\pi_1, \dots, \pi_k$ that can be formed from $q_1, \dots, q_m$. By dimensioned quantities $q_1, \dots, q_m$, we mean these can be expressed in a natural way in terms of a minimal set of fundamental dimensions $L_1, l_2, \dots, L_n$ $(n<m)$, appropriate to the problem being studied. Examples of these fundamental dimensions would be be time $T$, mass $M$, length  $L$, etc.\par In general, the dimensions $q_i$, denoted by the square bracket notation $[q_i]$, can be written in terms of the fundamental dimensions as 

\begin{equation}
    [q_i]=L_1^{a_1}l_2^{a_2}\cdots L_n^{a_n}
\end{equation}

\noindent for some choice of exponents $a_1, \dots, a_n$. If $[q_i]=1$, then the quantity $q_i$ is said to be \textbf{dimensionless}.\par We proceed as follows, if $\pi$ is a quantity of the form 

\begin{equation}
    \pi=q_1^{p_1}q_2^{p_2}\cdots q_m^{p_m},
\end{equation}

\noindent a monomial in the dimensioned quantities, we want to find all exponents $p_1, \dots, p_m$ for which $[\pi]=1$. Then 

\begin{equation}
    \begin{split}
        [\pi] &= [q_1]^{p_1}[q_2]^{p_2}\cdots[q_m]^{p_m} \\
        &= \big(L_1^{a_{11}}L_2^{a_{21}}\cdots L_n^{a_{n1}}\big)^{p_1}\cdots\big(L_1^{a_{1m}}L_2^{a_{2m}}\cdots L_n^{a_{nm}}\big)^{p_m} \\
        &= 1.
    \end{split}
\end{equation}

\noindent This means that the exponents on each $L_i$ must sum to 0. Thus, simplifying (24), we obtain 

\begin{equation}
    \begin{split}
        1 &= L_1^{a_{11}p_1+a_{12}p_2+\cdots+a_{1m}p_m}\cdots L_n^{a_{n1}p_1+a_{n2}p_2+\cdots+a_{nm}p_m} \\
        &= \prod_{i=1}^{n}L_i^{\sum_{j=1}^{m}a_{ij}p_j}
    \end{split}
\end{equation}


\noindent Since the exponents on each $L_i$ must sum to zero, then we get the
homogeneous system of $n$ linear equations with $m$ unknowns

\begin{equation*}
    \begin{bmatrix} 0 \\ \vdots \\ 0\end{bmatrix} = \begin{bmatrix} a_{11}p_1+a_{12}p_2+\cdots+a_{1m}p_m \\ \vdots \\ a_{n1}p_1+a_{n2}p_2+\cdots+a_{nm}p_m\end{bmatrix}=\begin{bmatrix} a_{11} & a_{12} &\cdots&a_{1m} \\ a_{21} & a_{22} & \cdots & a_{2m} \\ \vdots & \vdots & \ddots & \vdots \\ a_{n1} & a_{n2} & \cdots & a_{nm} \end{bmatrix}  \begin{bmatrix} p_1 \\ p_2 \\ \vdots \\ p_m \end{bmatrix} = A\mathbf{p}. 
\end{equation*}

\noindent The matrix seen here is referred to as the dimension matrix and the elements of the $i$th column give the exponents for $q_i$ in terms of the powers of $L_1,\dots,L_n$. It follows from a key result in linear algebra that the number of independent solutions is $m-r$, where $r$ is the rank of the matrix. Recall that the rank of a matrix is the number of linearly independent rows, which, when the matrix is reduced to row echelon form, is the number of nonzero rows. So, the number of independent dimensionless variables that can be formed from $q_1,\dots,q_m$ is $m-r$.\par We assume that (1) is \textit{unit free} in the sense that it is independent of the particular set of units chosen to express the quantities $q_1,\dots,q_m$. Any fundamental dimension $L_i$ has the property that its units can be changed upon multiplication by the appropriate conversion factor $\lambda_i>0$ to obtain $\overline{L}_i$ in a new system of units. We write

\begin{equation*}
    \overline{L}_i=\labmda_i L_i,\quad i=1,\dots,n.
\end{equation*}

The units of derived quantities $q$ can be changed in a similar fashion. If

\begin{equation*}
    [q]=L_1^{b_1}L_2^{b_2}\cdots L_n^{b_n},
\end{equation*}

\noindent then

\begin{equation*}
    \overline{q}=\lambda_^{b_1}\lambda_2^{b_2}\cdots\lambda_n^{b_n}q
\end{equation*}


\begin{definition}
    The physical law (1) is \textbf{unit-free} if for all choices of real positive numbers $\lamda_1,\dots,\lambda_n$ we have $f(\overline{q}_1,\dots,\overline{q}_m)=0$, if, and only if, $f(q_1,\dots,q_m)=0$.
\end{definition}

\newpage

\begin{example}
    The physical law 
    
    \begin{equation}
        f(x,t,g)\equiv x-\frac{1}{2}gt^2=0
    \end{equation}
    
    \noindent relates the distance $x$ a body falls in a constant gravitational field $g$ to the time $t$. In the cgs system of units, $x$ is given in centimeters (cm), $t$ in seconds, and $g$ in cm/sec$^2$. If we change units for the fundamental quantities $x$ and $t$ to inches and minutes, then in the new system of units
    
    \begin{equation*}
        \overline{x}=\lambda_1x,\quad \overline{t}=\lambda_2t,
    \end{equation*}
    
    where $\lambda_1=\frac{1}{2.54}$ (in/cm) and $\lambda_2=\frac{1}{60}$ (min/sec). Because $[g]=LT^{-2}$, we have 
    
    \begin{equation*}
        \overline{g}=\lambda_1\lambda_2^{-2}g.
    \end{equation*}
    
    \noindent Then
    
    \begin{equation*}
        f(\overline{x},\overline{t},\overline{g})=\overline{x}-\frac{1}{2}\overline{g}\overline{t}^2=\lambda_1x-\frac{1}{2}(\lambda_1\lambda_2^{-2}g)(\lambda_2t)^2=\lambda_1(x-\frac{1}{2}gt^2).
    \end{equation*}
    
    \noindent Therefore, (7) is unit-free. 
\end{example}

\newpage

\section{Perturbation Theory}

What does it mean to solve a polynomial equation by a perturbation method? Suppose the problem is
    \begin{equation}
        x^2-3.99x+3.02=0.
    \end{equation}
\noindent Of course, this can be solved by the quadratic formula. But yo approach it by perturbation theory, there are four steps:
    \begin{enumerate}
        \item Notice that since $-3.99=-4+0.01$ and $3.02=3+0.02$, then equation (8) is almost the same as $x^2-4x+3=0$ which can be factored as $(x-1)(x-3)=0$, giving two roots $x^{(1)}=1$, $x^{(2)}=3$.
        \item The next step is to create a family of problems that lay between the factorable problem and the original problem. If we let $\varepsilon=0.01$, then $-3.99=-4+\varepsilon$ and $3.02=3+2\varepsilon$. Then equation (8) can be written as 
            \begin{equation}
                x^2+(\varepsilon-4)x+(3+2\varepsilon)=0.
            \end{equation}
        By doing this we no longer obtain a single equation, but rather a family of equations, one for each $\varepsilon$. When $\varepsilon=0$, then we get the factorable problem, and when $\varepsilon=0.01$ then we obtain the ``target problem'' equation (8). For $0<\varepsilon<0.01$, it is midway between the two. Equation (9) is called a \textit{perturbation family}, a family of problems depending on a small parameter $\varepsilon$ which is easily solvable when $\varepsilon=0$.
        \item Find approximate solutions of equation (9), in the form of polynomials in the small parameter $\varepsilon$. Suitable solutions turn out to be 
            \begin{equation*}
                \begin{split}
                    x^{(1)}&\approx 1+\frac{3}{2}\varepsilon+\frac{15}{8}\varepsilon^2, \\
                    x^{(2)}&\approx 3-\frac{5}{2}\varepsilon-\frac{15}{8}\varepsilon^2.
                \end{split}
            \end{equation*}
        Evaluating this solutions when $\varepsilon=0.01$ gives an approximate solution to (8), namely $x^{(1)}\approx 1.0151875$, $x^{(2)}\approx 2.9748125$.
        \item The last step is to say something about the amount of error in these approximations.
    \end{enumerate}
    
    It is useful to think of a perturbation family as a \textit{path} leading from the easy \textit{reduced problem} to the given \textit{target problem}. To clarify this idea of a path, think of (8) as a special case of a general quadratic equation $ax^2+bx+c=0$. If a set of three perpendicular coordinate axes, labeled $a,b,$ and $c$, is constructed, then to each point $(a,b,c)$ there corresponds a specific quadratic equation. Hence, $(1,-3.99,3.02)$ corresponds to (8), and $(1,-4,3)$ corresponds to the factorable quadratic equation in step 1. Thus, the three equations $a=1, b=\varepsilon-4$, and $c=3+2\varepsilon$ are the parametric equations of a straight line in this coordinate system, joining the reduced and target problem. 
    \newpage
    It is important to realize that perturbation families must be created and do not arise automatically. As an example of the choices involved in formulating a perturbation problem, suppose the target problem is 
        \begin{equation}
            
        \end{equation}
    


\end{document}