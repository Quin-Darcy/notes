\documentclass{article}
\usepackage{graphicx}
\usepackage{tikz}
\usepackage{amsmath}
\usepackage{authblk}
\usepackage{titlesec}
\usepackage{amsthm}
\usepackage{amsfonts}
\usepackage{amssymb}
\usepackage{array}
\usepackage{booktabs}
\usepackage{ragged2e}
\usepackage{enumerate}
\usepackage{enumitem}
\usepackage{cleveref}
\usepackage{slashed}
\usepackage{commath}
\usepackage{lipsum}
\usepackage{colonequals}
\usepackage{addfont}
\usepackage{enumitem}
\usepackage{sectsty}
\usepackage{mathtools}
\usepackage{mathrsfs}

\usepackage{hyperref}
\hypersetup{
    colorlinks=true,
    linkcolor=blue,
    filecolor=magenta,      
    urlcolor=cyan,
}

\usetikzlibrary{decorations.pathreplacing}
\usetikzlibrary{arrows.meta}

\newtheorem{theorem}{\hspace{6.5mm}Theorem}[section]
\newtheorem{corollary}{\hspace{6.5mm}Corollary}[theorem]
\newtheorem{lemma}{Lemma}[theorem]
\theoremstyle{definition}
\newtheorem{prop}{Proposition}[section]
\newtheorem{definition}{Definition}[section]
\theoremstyle{remark}
\newtheorem*{remark}{Remark}

\let\oldproofname=\proofname
\renewcommand{\proofname}{\bf{\textit{\oldproofname}}}

\newcommand{\closure}[2][3]{%
  {}\mkern#1mu\overline{\mkern-#1mu#2}}

\theoremstyle{definition}
\newtheorem{example}{Example}[section]

\newtheorem*{discussion}{Discussion}

\makeatletter
\renewenvironment{proof}[1][\proofname]{\par
  \pushQED{\qed}%
  \normalfont \topsep6\p@\@plus6\p@\relax
  \list{}{\leftmargin=0mm
          \rightmargin=0mm
          \settowidth{\itemindent}{\itshape#1}%
          \labelwidth=4mm
          \parsep=0pt \listparindent=0mm%\parindent 
  }
  \item[\hskip\labelsep
        \itshape
    #1\@addpunct{.}]\ignorespaces
}{%
  \popQED\endlist\@endpefalse
}

\makeatletter
\renewenvironment{proof*}[1][\proofname]{\par
  \pushQED{\qed}%
  \normalfont \topsep6\p@\@plus6\p@\relax
  \list{}{\leftmargin=0mm
          \rightmargin=0mm
          \settowidth{\itemindent}{\itshape#1}%
          \labelwidth=\itemindent
          \parsep=0pt \listparindent=0mm%\parindent 
  }
  \item[\hskip\labelsep
        \itshape
    #1\@addpunct{.}]\ignorespaces
}{%
  \popQED\endlist\@endpefalse
}

\newenvironment{solution}[1][\bf{\textit{Solution}}]{\par
  
  \normalfont \topsep6\p@\@plus6\p@\relax
  \list{}{\leftmargin=0mm
          \rightmargin=0mm
          \settowidth{\itemindent}{\itshape#1}%
          \labelwidth=\itemindent
          \parsep=0pt \listparindent=\parindent 
  }
  \item[\hskip\labelsep
        \itshape
    #1\@addpunct{.}]\ignorespaces
}{%
  \popQED\endlist\@endpefalse
}



\begin{document}

\title{Notes for Number Theory}
\author{Quin Darcy}
\date{Dec 17, 2019}
\affil{\small{California State University, Sacramento}}
\maketitle

\section{The Basis Representation Theorem}

\begin{theorem}\label{thm:1.1}
    The sum of the first $n$ positive integers is $\frac{n(n+1)}{2}$.
\end{theorem}

\begin{proof}
    By induction on $n$. Let $P(n)\coloneqq\sum\limits_{i=1}^n i=\frac{n(n+1)}{2}$. For the base case we let $n=1$ and it follows that $\sum\limits_{i=1}^1 i=1$ and $1(1+1)/2=1$. Thus, $P(1)$ holds. Now assume $P(k)$ holds for some positive integer $k>1$. Then $\sum\limits_{i=1}^k i=\frac{k(k+1)}{2}$. Adding $k+1$ to both sides we obtain
        \begin{equation*}
            \sum\limits_{i=1}^k i+(k+1)=\frac{k(k+1)}{2}+(k+1),
        \end{equation*}
    \noindent which simplifies to
        \begin{equation*}
            \sum\limits_{i=1}^{k+1}i=\frac{(k+1)\big((k+1)+1\big)}{2}.
        \end{equation*}
    \noindent Thus, $P(k+1)$ holds. Therefore, the sum of the first $n$ positive integers is as claimed.
\end{proof}

\begin{theorem}\label{thm:1.2}
    If $x$ is any real number other than 1, then $\sum\limits_{j=0}^{n-1} x^j=\frac{x^n-1}{x-1}$. 
\end{theorem}
    \begin{proof}
        By induction on $n$. Define $P(n)\coloneqq\sum\limits_{j=0}^{n-1}x^j=\frac{x^n-1}{x-1}$. Then if $n=1$, we have that $\sum\limits_{j=0}^{1-1}x^j=1$ and we also have that $\frac{x^1-1}{x-1}=1$. Thus, the two expressions are equal and $P(1)$ holds. Now assume that $P(k)$ holds for some $k>1$. Then we have that $\sum\limits_{j=0}^{k-1}x^j=\frac{x^k-1}{x-1}$. Adding $x^k$ to both sides we obtain
            \begin{equation*}
                \sum\limits_{j=0}^{k-1}x^j+x^k=\frac{x^k-1}{x-1}+x^k,
            \end{equation*}
        \noindent which simplifies to 
            \begin{equation*}
                \sum\limits_{j=0}^k x^j=\frac{x^k-1+x^{k+1}-x^k}{x-1}=\frac{x^{k+1}-1}{x-1}.
            \end{equation*}
        \noindent Thus, $P(k+1)$ holds. Therefore, we have established the theorem.
    \end{proof}
    
\begin{corollary}\label{cor:1.2.1}
    If $m$ and $n$ are positive integers and if $m>1$, then $n<m^n$.
\end{corollary}

\begin{theorem}\label{thm:1.3}
    Let $k$ be any integer larger than 1. Then, for each positive integer $n$, there exists a representation 
        \begin{equation*}
            n=a_0k^s+a_1k^{s-1}+\cdots+a_s,
        \end{equation*}
    where $a_0\neq 0$, and where each $a_i$ is nonnegative and less than $k$. Furthermore, this representation of $n$ is unique; it is called the representation of $n$ to the base $k$.
\end{theorem}

\begin{proof}
    Let $b_k(n)$ denote the number of representations of $n$ to the base $k$. Then we want to show that $b_k(n)=1$ for all $k$. Suppose that 
        \begin{equation*}
            n=a_0k^s+a_1k^{s-1}+\cdots+ a_{s-t}k^t
        \end{equation*}
    \noindent where neither $a_0$ nor $a_{s-t}$ equals zero. Then 
        \begin{equation*}
            \begin{split}
                n-1&=a_0k^s+\cdots+ a_{s-t}k^t-1 \\
                &= a_0k^s+\cdots+ a_{s-t}k^t+(k^t-k^t)-1 \\
                &= a_0k^s+\cdots+ (a_{s-t}-1)k^t+k^t-1 \\
                &= a_0k^s+\cdots+(a_{s-t}-1)k^t+\sum\limits_{j=0}^{t-1}(k-1)k^j.
            \end{split}
        \end{equation*}
\end{proof}



\end{document}