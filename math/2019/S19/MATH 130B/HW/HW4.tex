\documentclass[12pt, a4paper]{article}
\usepackage[margin=1in]{geometry}
\usepackage[latin1]{inputenc}
\usepackage{titlesec}
\usepackage{amsmath}
\usepackage{amsthm}
\usepackage{amsfonts}
\usepackage{amssymb}
\usepackage{array}
\usepackage{booktabs}
\usepackage{ragged2e}
\usepackage{enumerate}
\usepackage{enumitem}
\usepackage{cleveref}
\usepackage{slashed}
\usepackage{commath}
\usepackage{lipsum}
\usepackage{colonequals}
\usepackage{addfont}
\addfont{OT1}{rsfs10}{\rsfs}
\renewcommand{\baselinestretch}{1.1}
\usepackage[mathscr]{euscript}
\let\euscr\mathscr \let\mathscr\relax
\usepackage[scr]{rsfso}
\newcommand{\powerset}{\raisebox{.15\baselineskip}{\Large\ensuremath{\wp}}}
\usepackage{longtable}
\usepackage{multirow}
\usepackage{multicol}
\usepackage{calligra}
\usepackage[T1]{fontenc}
\newcounter{proofc}
\renewcommand\theproofc{(\arabic{proofc})}
\DeclareRobustCommand\stepproofc{\refstepcounter{proofc}\theproofc}
\usepackage{fancyhdr}
\pagestyle{fancy}

\renewcommand{\headrulewidth}{0pt}
\fancyhead[R]{}
\usepackage{enumitem}
\usepackage{tikz}
\usepackage{commath}
\usepackage{colonequals}
\usepackage{bm}
\usepackage{tikz-cd}
\renewcommand{\baselinestretch}{1.1}
\usepackage[mathscr]{euscript}
\let\euscr\mathscr \let\mathscr\relax
\usepackage[scr]{rsfso}
\usepackage{titlesec}
\usepackage{scrextend}
\usepackage{lscape}
\usepackage{relsize}

\usepackage[english]{babel}
\usepackage{blindtext}
\usepackage{polynom}



\newcommand*{\logeq}{\ratio\Leftrightarrow}

\titleformat{\section}
  {\normalfont\Large\bfseries}{\thesection}{1em}{}[{\titlerule[0.8pt]}]
  
\setlist[description]{leftmargin=6mm,labelindent=8mm, rightmargin=6mm}
  
\begin{document}
  
\begin{flushleft}
  
    Quin Darcy\par
    Dr. Cummings\par
    MATH 130B\par
    2/19/19
  
\end{flushleft}
  
\centerline{\boxed{\text{Homework 4}}}
 
\vspace{4mm}
 
\noindent\textsc{Section: }\par
 
\justifying
 
\vspace{1mm}
 
\hline
 
\vspace{6mm}

\noindent\textbf{1.} Prove that if $f\colon A\rightarrow\mathbb{R}$ is uniformly continuous and bounded, then $f(A)$ is bounded. 

\begin{description}
    \item\textit{\textbf{Proof.}} Assume $A\subseteq\mathbb{R}$ is bounded, and that $f\colon A\rightarrow\mathbb{R}$ is uniformly continuous. Let us begin by quickly reviewing the meaning behind each of these assumptions. 
    
    \begin{enumerate}[label=(\roman*)]
        \item To assume $A\subseteq\mathbb{R}$ is bounded is to say that there exists some positive non-zero real number $M$ such that for all elements $a$, if $a\in A$, then $\abs{a}\leq M$.
        
        \item To assume $f\colon A\rightarrow\mathbb{R}$ is uniformly continuous is to say that given any $\varepsilon>0$, there exists $\delta>0$ such that for all $x,y\in A$, if\hspace{1mm} $\abs{x-y}<\delta$, then $\abs{f(x)-f(y)}<\varepsilon$.     
    \end{enumerate}
    
    \item Let $\varepsilon>0$. Then (ii) assures us that there is some $\delta>0$ such that for all $x,y\in A$, if $\abs{x-y}<\delta$, then $\abs{f(x)-f(y)}<\varepsilon$. Because $A$ is bounded, then we can cover $A$ with finitely many $\delta$-neighborhoods $V_{\delta}(x_i)$, where $0\leq i\leq N$ for some $N\in\mathbb{N}$ and $x_i\in A$. To the set of these $x_i$'s about which each $\delta$-neighborhood is formed, let $D=\min\{f(x_i)\}$ and $E=\max\{f(x_i)\}$, for $0\leq i\leq N$. Now it is clear that for each $x\in A$, there is some natural number $m\leq N$ which corresponds to a particular neighborhood in which $x\in V_{\delta}(x_m)$. Which means $\abs{x-x_m}<\delta$ and this implies $\abs{f(x)-f(x_m)}<\varepsilon$. Rewriting this inequality we obtain $f(x_m)-\varepsilon<f(x)<f(x_m)+\varepsilon$. Now note that since $D=\min\{f(x_i)\}$, then $f(x_m)\geq D$. Similarly, since $E=\max\{f(x_i)\}$, then $f(x_m)\leq E$. Thus, it follows that
    
    \begin{equation*}
        D-\varepsilon<f(x)<E+\varepsilon.
    \end{equation*}
    
    \vspace{4mm}
    
    Since $x$ was arbitrary, then the previous applies to all $x\in A$. Thus, for all $x\in A$, $f(x)\in(D-\varepsilon,E+\varepsilon)$. Therefore, $f(A)$ is bounded.\hspace{51mm}\square
    
\end{description}

\noindent\textbf{2.} For each of the below, you do \textit{not} need to prove that your answer works.

\begin{enumerate}[label=(\alph*)]
    \item Give an example of an unbounded continuous function on $(0,1)$ that is not uniformly continuous.
    
    \begin{description}
        \item\textbf{Example.} 
                \begin{equation*}
                    f(x)=\frac{1}{x}.
                \end{equation*}
    \end{description}
    
    \item Give an example of a bounded continuous function on $(0,1)$ that is not uniformly continuous.
    
    \begin{description}
        \item\textbf{Example.}
        
        \begin{equation*}
            f(x)=\sin(\frac{1}{x}).
        \end{equation*}
    \end{description}
\end{enumerate}

\noindent\textbf{3.} Define the \textit{derivative} of a function $f$ at a point $c$.

\begin{description}
    \item\textbf{Definition:} Let $I\subseteq\mathbb{R}$. Then given a function $f\colon I\rightarrow\mathbb{R}$ and a point $c\in I$. If 
    
    \begin{equation*}
        \lim\limits_{x\rightarrow c}\frac{f(x)-f(c)}{x-c}\in\mathbb{R},
    \end{equation*}
    
    \vspace{2mm}
    
    \item then $f$ is \textit{differentiable at c}.
\end{description}

\end{document}