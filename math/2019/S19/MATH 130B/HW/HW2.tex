\documentclass[12pt, a4paper]{article}
\usepackage[margin=1in]{geometry}
\usepackage[latin1]{inputenc}
\usepackage{titlesec}
\usepackage{amsmath}
\usepackage{amsthm}
\usepackage{amsfonts}
\usepackage{amssymb}
\usepackage{array}
\usepackage{booktabs}
\usepackage{ragged2e}
\usepackage{enumerate}
\usepackage{enumitem}
\usepackage{cleveref}
\usepackage{slashed}
\usepackage{commath}
\usepackage{lipsum}
\usepackage{colonequals}
\usepackage{addfont}
\addfont{OT1}{rsfs10}{\rsfs}
\renewcommand{\baselinestretch}{1.1}
\usepackage[mathscr]{euscript}
\let\euscr\mathscr \let\mathscr\relax
\usepackage[scr]{rsfso}
\newcommand{\powerset}{\raisebox{.15\baselineskip}{\Large\ensuremath{\wp}}}
\usepackage{longtable}
\usepackage{multirow}
\usepackage{multicol}
\usepackage{calligra}
\usepackage[T1]{fontenc}
\newcounter{proofc}
\renewcommand\theproofc{(\arabic{proofc})}
\DeclareRobustCommand\stepproofc{\refstepcounter{proofc}\theproofc}
\usepackage{fancyhdr}
\pagestyle{fancy}

\renewcommand{\headrulewidth}{0pt}
\fancyhead[R]{}
\usepackage{enumitem}
\usepackage{tikz}
\usepackage{commath}
\usepackage{colonequals}
\usepackage{bm}
\usepackage{tikz-cd}
\renewcommand{\baselinestretch}{1.1}
\usepackage[mathscr]{euscript}
\let\euscr\mathscr \let\mathscr\relax
\usepackage[scr]{rsfso}
\usepackage{titlesec}
\usepackage{scrextend}
\usepackage{lscape}
\usepackage{relsize}

\usepackage[english]{babel}
\usepackage{blindtext}
\usepackage{polynom}



\newcommand*{\logeq}{\ratio\Leftrightarrow}

\titleformat{\section}
  {\normalfont\Large\bfseries}{\thesection}{1em}{}[{\titlerule[0.8pt]}]
  
\setlist[description]{leftmargin=6mm,labelindent=4mm}
  
\begin{document}
  
\begin{flushleft}
  
    Quin Darcy\par
    Dr. Cummings\par
    MATH 130B\par
    1/25/19
  
\end{flushleft}
  
\centerline{\boxed{\text{Homework 1}}}
 
\vspace{4mm}
 
\noindent\textsc{Section: }\par
 
\justifying
 
\vspace{1mm}
 
\hline
 
\vspace{6mm}

\noindent\textbf{1. } State the func-y squeeze theorem.

\begin{description}
    \item(\textit{Func-y Limit Laws}) Let $f$ and $g$ be functions from $A\subseteq\mathbb{R}$ to $\mathbb{R}$ and let $c$ be a limit point of $A$. Lastly, assume that 
    
    \vspace{3mm}
    
    \centerline{$\lim\limits_{x\rightarrow c}f(x)=L$\hspace{4mm}and\hspace{4mm}$\lim\limits_{x\rightarrow c}g(x)=M$.}
    
    \vspace{3mm}
    
    \item Then
    
    \begin{description}
        \item(i) $\lim\limits_{x\rightarrow c}[k\cdot f(x)]=k\cdot L$ for any $k\in\mathbb{R}$,
        \item(ii) $\lim\limits_{x\rightarrow c}[f(x)+g(x)]=L+M$,
        \item(iii) $\lim\limits_{x\rightarrow c}[f(x)\cdot g(x)]=L\cdot M$, and
        \item(iv) $\lim\limits_{x\rightarrow c}\mathlarger{\frac{f(x)}{g(x)}}=\mathlarger{\frac{L}{M}}$, provided $M\neq 0$ and $g(x)\neq 0$ for any $x\in A$.
    \end{description}
\end{description}

\vspace{2mm}

\noindent\textbf{2. } Assume that $f\colon\mathbb{R}\rightarrow\mathbb{R}$ is continuous at $c$, and $f(c)>0$. Prove that there exists some $\delta>0$ such that $f(x)>0$ for all $x$ in the $\delta$-neighborhood of $c$.

\begin{description}
    \item\textit{\textbf{Proof.}} For a contradiction, assume the following (I apologize in advance for how tedious this proof is)
    
    \begin{description}
        \item(A1) The function $f\colon\mathbb{R}\rightarrow\mathbb{R}$ is continuous at $c$. This means that for all $\varepsilon>0$, there exists $\delta>0$ such that for all $x$, if $\abs{x-c}<\delta$, then $\abs{f(x)-f(c)}<\varepsilon$. 
        \item(A2) $f(c)>0$.
        \vspace{3mm}
        \item(A3) For all $\delta>0$, there exists some $x$ for which $\abs{x-c}<\delta$ and $f(x)\leq 0$.
    \end{description}
    
    \item Since (A2) assures us that $f(c)>0$, then let $\varepsilon=f(c)$ [1]. It follows from (A1) that there exists some particular $\hat{\delta}>0$  [2]  such that if $\abs{x-c}<\hat{\delta}$, then $\abs{f(x)-f(c)}<f(c)$. From (A3) we may select any $\delta>0$, and so let $\delta=\hat{\delta}$, the same $\hat{\delta}$ from [2]. It continues to follow from (A3) that there exists some $\hat{x}$ [3] for which $\abs{\hat{x}-c}<\hat{\delta}$ and $f(\hat{x})\leq 0$.\par
    \hspace{4.5mm}Thus, from the line previous to the last, we have that $\abs{\hat{x}-c}<\hat{\delta}$ and $f(\hat{x})\leq 0$ [4]. Additionally, because we have instantiated a particular $\varepsilon$ in [1], and we have $\delta=\hat{\delta}$ from [2], then from (A1) we may choose any $x$ which satisfies $\abs{x-c}<\hat{\delta}$. Thus, let $x=\hat{x}$, the same $\hat{x}$ from [3]. Therefore, from [4] we have $\abs{\hat{x}-c}<\hat{\delta}$ and by \textit{modus ponens} with the conditional in (A1), we have $\abs{f(\hat{x})-f(c)}<f(c)$ [5]. Moreover, also from [4], we have that $f(\hat{x})\leq 0$, which is equivalent to $f(\hat{x})<0$ or $f(\hat{x})=0$.
    
    \newpage
    
    \hspace{4.5mm} Assume $f(\hat{x})<0$. Substituting in for [5], we have $-f(c)<f(\hat{x})-f(c)<f(c)$, which implies $0<f(\hat{x})<2f(\hat{x})$. Thus, $0<f(\hat{x})$. This contradicts the assumption, therefore $f(\hat{x})\not<0$.\par
    \hspace{4.5mm} Assume $f(\hat{x})=0$. Then [5] simplifies to $\abs{-f(c)}<f(c)$. Since $f(c)>0$ by (A2), then we can write $f(c)<f(c)$. However, because $f(c)\in\mathbb{R}$, then $f(c)<f(c)$ is false and thus $f(\hat{x})\neq 0$.\par
    \hspace{4.5mm} Since the assumption of (A3) led to a contradiction in either case, then (A3) must be false and the negation of (A3) is true. Thus, there exists some $\delta>0$ such that $f(x)>0$ for all $x$ in the $\delta$-neighborhood of $c$.\hspace{68mm}\square
     
\end{description}

\noindent\textbf{3.} Given an example of a Cauchy sequence $(a_n)$ and a continuous function $f\colon(0,1)\rightarrow\mathbb{R}$ for which the sequence $(f(a_n))$ is \textit{not} Cauchy. You do \textit{not} need to prove your answer.

\begin{description}
    \item\textbf{Solution: } Let $\{a_n\}_{n=2}^{\infty}$ be a sequence where $a_i=1/i$, for all $i\geq 2$. We know from 130A that this is a sequence is Cauchy since the same sequence starting from $n=1$ is Cauchy. Now let $f\colon(0,1)\rightarrow\mathbb{R}$ be a function defined as $f(x)=1/x$, for all $x\in(0,1)$. Then since for any $i\geq 2$, we have that $0<1/i<1$, then $a_i\in(0,1)$ for all $i$. Thus, for every $a_i\in(0,1)$ we have
    
    \begin{equation*}
        f(a_i)=f(\frac{1}{i})=(\frac{1}{i})^{-1}=i.
    \end{equation*}
    
    \item Therefore, $\{f(a_n)\}:=n$ for $n\geq 2$, and is thus not convergent and thereby not Cauchy.
\end{description}

\noindent\textbf{4.} Define \textit{Thomae's function} to be

\vspace{2mm}

\[ h(x)=  \left\{
\begin{array}{ll}
      1 & \text{if } x=0 \\
      1/n & \text{if } x\in\mathbb{Q} \text{ and $x=m/n$ is in lowest terms with $n>0$} \\
      0 & \text{if } x\notin\mathbb{Q}. \\
\end{array} 
\right. \]

\vspace{4mm}

Prove that $h$ is continuous at every irrational number and discontinuous at every rational number.

\begin{description}
    \item\textit{\textbf{Proof.} } Let $\varepsilon>0$. Then by the Archimedean Principle, there exists some $n_0\in\mathbb{N}$, such that $1/n_0<\varepsilon$. Let $\delta=1/n_0$ and suppose $c\notin\mathbb{Q}$. Now let $x\in(c-\delta,c+\delta)$. It follows that either $x\in\mathbb{Q}$ or $x\notin\mathbb{Q}$. If $x\notin\mathbb{Q}$, then $h(x)=0$ and $\abs{h(x)-h(c)}=0<\varepsilon$. So suppose $x\in\mathbb{Q}$. Then there exists $m,n\in\mathbb{Z}$ such that $x=m/n$ where $m/n$ is in lowest terms and $n\neq0$. Thus, we may write
    
    \begin{equation}
        \abs{x-c}<\delta\Rightarrow\abs{\frac{m}{n}-c}<\frac{1}{n_0}\Rightarrow c-\frac{1}{n_0}<\frac{m}{n}<c+\frac{1}{n_0}.
    \end{equation}
\end{description}

\end{document}