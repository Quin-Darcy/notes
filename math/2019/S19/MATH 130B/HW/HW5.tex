\documentclass[12pt]{article}
\usepackage[margin=0.97in]{geometry} 
\usepackage{graphicx}
\usepackage{amsmath}
\usepackage{authblk}
\usepackage{titlesec}
\usepackage{amsthm}
\usepackage{amsfonts}
\usepackage{amssymb}
\usepackage{array}
\usepackage{booktabs}
\usepackage{ragged2e}
\usepackage{enumerate}
\usepackage{enumitem}
\usepackage{cleveref}
\usepackage{slashed}
\usepackage{commath}
\usepackage{lipsum}
\usepackage{colonequals}
\usepackage{addfont}
\usepackage{enumitem}
\usepackage{sectsty}
\usepackage{lastpage}
\usepackage{fancyhdr}
\usepackage{accents}
\usepackage[inline]{enumitem}
\pagestyle{fancy}
\setlength{\headheight}{10pt}

\subsectionfont{\itshape}

\newtheorem{theorem}{Theorem}[section]
\newtheorem{corollary}{Corollary}[theorem]
\newtheorem{lemma}[theorem]{Lemma}
\theoremstyle{definition}
\newtheorem{definition}{Definition}[section]
\theoremstyle{remark}
\newtheorem*{remark}{Remark}
 
\makeatletter
\renewenvironment{proof}[1][\proofname]{\par
  \pushQED{\qed}%
  \normalfont \topsep6\p@\@plus6\p@\relax
  \list{}{\leftmargin=0mm
          \rightmargin=0mm
          \settowidth{\itemindent}{\itshape#1}%
          \labelwidth=\itemindent
          \parsep=0pt \listparindent=\parindent 
  }
  \item[\hskip\labelsep
        \itshape
    #1\@addpunct{.}]\ignorespaces
}{%
  \popQED\endlist\@endpefalse
}

\newenvironment{solution}[1][\bf{\textit{Solution}}]{\par
  
  \normalfont \topsep6\p@\@plus6\p@\relax
  \list{}{\leftmargin=0mm
          \rightmargin=0mm
          \settowidth{\itemindent}{\itshape#1}%
          \labelwidth=\itemindent
          \parsep=0pt \listparindent=\parindent 
  }
  \item[\hskip\labelsep
        \itshape
    #1\@addpunct{.}]\ignorespaces
}{%
  \popQED\endlist\@endpefalse
}

\let\oldproofname=\proofname
\renewcommand{\proofname}{\bf{\textit{\oldproofname}}}


\newlist{mylist}{enumerate*}{1}
\setlist[mylist]{label=(\alph*)}

\begin{document}

\begin{center}
	\vspace{.4cm} {\textbf { \large MATH 130B}}
\end{center}
{\textbf{Name:}\ Quin Darcy \hspace{\fill} \textbf{Due Date:} 03/14/19   \\
{ \textbf{Instructor:}} \ Dr. Cummings \hspace{\fill} \textbf{Assignment:} Homework 5 \\ \hrule}

\justifying

% Document begins here

\begin{enumerate}[leftmargin=*]

    \item State the \textit{product}, \textit{quotient}, and \textit{chain rules}. 
        \begin{theorem}
            (The product rule). Let $I$ be an interval and let $f,g\colon I\rightarrow\mathbb{R}$ be differentiable at $c$. Then
            \begin{equation*}
                (fg)'(c)=f'(c)g(c)+f(c)g'(c).
            \end{equation*}
        \end{theorem}
        \begin{theorem}
            (The quotient rule). Let $I$ be an interval and let $f,g\colon I\rightarrow\mathbb{R}$ be differentiable at $c$. Then 
            \begin{equation*}
                \bigg(\frac{f}{g}\bigg)'(c)=\frac{f'(c)g(c)-f(c)g'(c)}{[g(c)]^2},
            \end{equation*}
            provided that $g(c)\neq 0$.
        \end{theorem}
        \begin{theorem}
            (The chain rule). Let $I_1$ and $I_2$ be intervals and $g\colon I_1\rightarrow I_2$ and $f\colon I_2\rightarrow\mathbb{R}$. If $g$ is differentiable at $c\in I_1$ and $f$ is differentiable at $g(c)\in I_2$, then
            \begin{equation*}
                (f\circ g)'(c)=f'(g(c))\dot g'(c).
            \end{equation*}
            
        \end{theorem}
    
    \item Prove the \tetit{quotient rule}.
    
    \begin{proof}
        Let $f$ and $g$ be functions defined on an interval $I$, and assume both are differentiable at some point $c\in I$. Additionally, assume $g(c)\neq 0$. Then we have that
        
        \begin{equation*}
            \begin{split}
                \frac{d}{dx}\bigg(\frac{f(x)}{g(x)}\bigg)\Bigr|_c &= \lim_{x\rightarrow c} \frac{\frac{f(x)}{g(x)}-\frac{f(c)}{g(c)}}{x-c} \\
                &= \lim_{x\rightarrow c}\frac{\frac{f(x)g(c)-f(c)g(x)}{g(x)g(c)}}{x-c} \\
                &= \lim_{x\rightarrow c}\frac{f(x)g(c)-f(c)g(x)}{(g(x)g(c))(x-c)} \\
                &= \lim_{x\rightarrow c}\frac{f(x)g(c)-f(c)g(c)+f(c)g(c)-f(c)g(x)}{(g(x)g(c))(x-c)} \\
                &= \lim_{x\rightarrow c}\bigg(\frac{f(x)g(c)-f(c)g(c)}{g(x)g(c)(x-c)}-\frac{f(c)g(x)-f(c)g(c)}{g(x)g(c)(x-c)}\bigg) \\
                &= \lim_{x\rightarrow c}\bigg[\bigg(\frac{f(x)-f(c)}{x-c}\bigg)\bigg(\frac{1}{g(x)}\bigg)-\bigg(\frac{f(c)}{g(x)g(c)}\bigg)\bigg(\frac{g(x)-g(c)}{x-c}\bigg)\bigg] \\
                &= \bigg(\lim_{x\rightarrow c}\frac{f(x)-f(c)}{x-c}\bigg)\bigg(\lim_{x\rightarrow c}\frac{1}{g(x)}\bigg)-\bigg(\lim_{x\rightarrow c}\frac{f(c)}{g(x)g(c)}\bigg)\bigg(\lim_{x\rightarrow c}\frac{g(x)-g(c)}{x-c}\bigg) \\
                &= f'(c)\bigg(\frac{1}{g(c)}\bigg)-\bigg(\frac{f(c)}{[g(c)]^2}\bigg)g'(c) \\
                &= \frac{f'(c)g(c)-f(c)g'(c)}{[g(c)]^2}.
            \end{split}
        \end{equation*}
    \end{proof}
    
    \item Is
        \begin{equation*}
            f(x)=
            \begin{cases} 
                \frac{1}{2}x & \quad\text{if}\quad x\in\mathbb{Q} \\
                x & \quad\text{if}\quad x\notin\mathbb{Q} \\
            \end{cases}
        \end{equation*}
        differentiable at 0?
        
        \begin{proof}
            Since both $\mathbb{Q}$ and $\mathbb{Q}^c$ are dense in $\mathbb{R}$, then we can find sequences $(x_n)\subseteq\mathbb{Q}$ and $(y_n)\subseteq\mathbb{Q}^c$ such that $\lim x_n=0$ and $\lim y_n=0$. Moreover, both $\frac{1}{2}x$ and $x$ are continuous functions over all of $\mathbb{R}$. So then 
            
            \begin{equation*}
                \lim_{n\rightarrow\infty}\frac{f(x_n)-f(0)}{x_n-0}=\lim_{n\rightarrow\infty}\frac{\frac{1}{2}x_n}{x_n}=\frac{1}{2},
            \end{equation*}
            
            \noindent and
            
            \begin{equation*}
                \lim_{n\rightarrow\infty}\frac{f(y_n)-f(0)}{y_n-0}=\lim_{n\rightarrow\infty}\frac{y_n}{y_n}=1.
            \end{equation*}
            
            \noindent Thus, the limit does not exist and $f(x)$ is not differentiable at 0.
        \end{proof}
        
    \item Use induction to prove $\frac{d}{dx}x^n=nx^{n-1}$ for all $n\in\mathbb{N}$.
    
    \begin{proof}
        Let $n\in\mathbb{N}$, and define $P(n):=\frac{d}{dx}x^n=nx^{n-1}$.
        
        \begin{enumerate}[label=(\roman*)]
            \item\textsc{Base Case:} Let $n=1$. Let $c$ be any arbitrary point in $\mathbb{R}$. Then 
            
            \begin{equation*}
                \begin{split}
                    P(1)=\frac{d}{dx}x\mid_c &= \lim_{x\rightarrow c}\frac{x-c}{x-c} \\
                    &= \lim_{x\rightarrow c} 1 \\
                    &= 1 \\
                    &= 1\cdot x^0.
                \end{split}
            \end{equation*}
            
            \noindent Thus, $P(1)$ holds.
            
            \item\textsc{inductive step:} Assume $P(k)$ holds for some $k\in\mathbb{N}$, where $k>1$. Let $c$ be any arbitrary point in $\mathbb{R}$. Then assuming we have the product rule, 
            
            \begin{equation*}
                \begin{split}
                    P(k+1)=\frac{d}{dx}x^{k+1}\mid_c &= \frac{d}{dx}(x^k\cdot x) \\
                    &= \bigg(\frac{d}{dx}x^k\mid_c\bigg)\cdot c+c^k\cdot\bigg(\frac{d}{dx}x\mid_c\bigg) \\
                    &= (kc^{k-1})\cdot c+c^k\cdot(1) \\
                    &= kc^k+c^k \\
                    &= (k+1)c^k.
                \end{split}
            \end{equation*}
            
            \noindent Thus, $P(k+1)$ holds. Therefore, for all $n\in\mathbb{N}$, $P(n)$ holds.
        \end{enumerate}
    \end{proof}
    
    \newpage
    
    \item Assume that $I$ is an interval and $f\colon I\rightarrow\mathbb{R}$ is differentiable. Prove that if $f'$ is bounded, then $f$ is uniformly continuous.
        
    \begin{proof}
        Denote $I=[a,b]$ as the interval over which $f$ is differentiable. Let $\varepsilon>0$, and let $\delta>0$ such that $M\delta<\varepsilon$, where $M\in\mathbb{R}$ is the real number that bounds $f'$. Now select $x,y\in (a,b)$ such that $\abs{y-x}<\delta_1$. Then because $f$ is differentiable over $I$, then it is continuous over $I$. Moreover, it is the case that $f$ is differentiable over $(x,y)$ since $(x,y)\subset I$. Thus, by the Mean Value Theorem, there exists $c\in(x,y)$ such that
        
        \begin{equation*}
            f'(c)=\frac{f(y)-f(x)}{y-x}.
        \end{equation*}
        
        \noindent However, since $f'$ is bounded, then it follows that
        
        \begin{equation*}
            f'(c)=\frac{f(y)-f(x)}{y-x}\leq M,
        \end{equation*}
        
        \noindent which implies that $f(y)-f(x)\leq(y-x)M$, which further implies $f(y)-f(x)<\delta_1 M<\varepsilon$. Thus, for all $\varepsilon>0$, there exists $\delta_1>0$ such that for all $x,y\in(a,b)$, if $\abs{y-x}<\delta_1$, then $\abs{f(y)-f(x)}<\varepsilon$. This means that $f$ is uniformly continuous over $(a,b)$.\par 
        \hspace{8mm}To show that $f$ is uniformly continuous on $I$ we must address the endpoints of $I$, $a$ and $b$. Taking the same $\varepsilon>0$, we know there exists $\delta_2>0$ such that for all $x\in I$, if $\abs{x-a}<\delta_2$, then $\abs{f(x)-f(a)}<\varepsilon$. We know this because $f$ is continuous at $a$. So then assume $\delta_2<\delta_1$. Thus, if $\varepsilon>0$ is any value, then let $\delta_2<\delta_1$. It follows then that if $x,y\in[a,b)$ such that $\abs{y-x}<\delta_2$, then $\abs{f(y)-f(x)}<\varepsilon$. \par 
        \hspace{8mm}By the same argument, we can find $\delta_3$ such that for any $x,y\in(a,b]$ such that $\abs{y-x}<\delta_3$, then $\abs{f(y)-f(x)}<\varepsilon$, for all $\varepsilon>0$. Finally, let $\varepsilon>0$, then let $\delta=\min\{\delta_2,\delta_3\}$, then for all $x,y\in I$, if $\abs{y-x}<\delta$, then $\abs{f(y)-f(x)}<\varepsilon$. Therefore, $f$ is uniformly continuous over $I$. 
    \end{proof}
              
\end{enumerate}

\end{document}