\documentclass[12pt, a4paper]{article}
\usepackage[margin=1in]{geometry}
\usepackage[latin1]{inputenc}
\usepackage{titlesec}
\usepackage{amsmath}
\usepackage{amsthm}
\usepackage{amsfonts}
\usepackage{amssymb}
\usepackage{array}
\usepackage{booktabs}
\usepackage{ragged2e}
\usepackage{enumerate}
\usepackage{enumitem}
\usepackage{cleveref}
\usepackage{slashed}
\usepackage{commath}
\usepackage{lipsum}
\usepackage{colonequals}
\usepackage{addfont}
\addfont{OT1}{rsfs10}{\rsfs}
\renewcommand{\baselinestretch}{1.1}
\usepackage[mathscr]{euscript}
\let\euscr\mathscr \let\mathscr\relax
\usepackage[scr]{rsfso}
\newcommand{\powerset}{\raisebox{.15\baselineskip}{\Large\ensuremath{\wp}}}
\usepackage{longtable}
\usepackage{multirow}
\usepackage{multicol}
\usepackage{calligra}
\usepackage[T1]{fontenc}
\newcounter{proofc}
\renewcommand\theproofc{(\arabic{proofc})}
\DeclareRobustCommand\stepproofc{\refstepcounter{proofc}\theproofc}
\usepackage{fancyhdr}
\pagestyle{fancy}

\renewcommand{\headrulewidth}{0pt}
\fancyhead[R]{}
\usepackage{enumitem}
\usepackage{tikz}
\usepackage{commath}
\usepackage{colonequals}
\usepackage{bm}
\usepackage{tikz-cd}
\renewcommand{\baselinestretch}{1.1}
\usepackage[mathscr]{euscript}
\let\euscr\mathscr \let\mathscr\relax
\usepackage[scr]{rsfso}
\usepackage{titlesec}
\usepackage{scrextend}
\usepackage{lscape}
\usepackage{relsize}
\usepackage{enumitem,kantlipsum}

\usepackage[english]{babel}
\usepackage{blindtext}
\usepackage{polynom}



\newcommand*{\logeq}{\ratio\Leftrightarrow}

\titleformat{\section}
  {\normalfont\Large\bfseries}{\thesection}{1em}{}[{\titlerule[0.8pt]}]
  
\setlist[description]{leftmargin=6mm,labelindent=8mm, rightmargin=6mm}
  
\begin{document}
  
\begin{flushleft}
  
    Quin Darcy\par
    Misc. Problems\par
    Real Analysis\par
    03/08/19
  
\end{flushleft}
  
\centerline{\boxed{\text{Fun Problems}}}
 
\vspace{4mm}
 
\noindent\textsc{Section: Topology}\par
 
\justifying
 
\vspace{1mm}
 
\hline

\noindent

\begin{enumerate}[leftmargin=*]
    \item Let $\{r_1, r_2, r_3, \dots\}$ be an enumeration of the rational numbers and for each $n\in\mathbb{N}$ set $\varepsilon_n=1/2^n$. Define $O=\bigcup_{n=1}^{\infty}V_{\varepsilon_n}(r_n)$, and $F=O^c$.
    
    \begin{enumerate}[label=(\alph*), rightmargin=10mm]
        \item Argue that $F$ is a closed, nonempty set consisting of only irrational numbers.
        \item Does $F$ contain any nonempty open intervals? If $F$ totally disconnected?
        \item Is it possible to know whether $F$ is perfect? If not, can we modify the construction to produce a nonempty perfect set of irrational numbers?
    \end{enumerate}
    
    \vspace{4mm}
    
    \underline{\textsc{Discussion}}
    
    \begin{enumerate}[label=(\alph*), listparindent=1.5em, labelsep=0.5em, itemindent=1.5em, rightmargin=10mm]
        \item \noindent Since both $O$ and $F$ are subsets of $\mathbb{R}$, then we will try to tackle this problem by showing that $O\neq\mathbb{R}$. What this will do is prove that $F\neq\varnothing$. To do this, let us assume for a contradiction, that $\mathbb{R}\subseteq O$. Then it follows that for all $x$, $x\in\mathbb{R}$ implies $x\in O$. Take $\sqrt{2}\in\mathbb{R}$. We have that $\sqrt{2}\in O$. Thus, there exists $n\in\mathbb{N}$ such that $\sqrt{2}\in V_{\varepsilon_n}(r_n)$. Or more precisely,
        
        \begin{equation*}
            \sqrt{2}\in\{a\in\mathbb{R}\colon r_n-\frac{1}{2^n}<a<r_n+\frac{1}{2^n}\}.
        \end{equation*}
        
        \noindent Thus, 
        
        \begin{equation*}
            
        \end{equation*}
    \end{enumerate}
    
\end{enumerate}

\end{document}