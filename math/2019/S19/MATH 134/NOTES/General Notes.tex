=]\documentclass{article}
\usepackage{graphicx}
\usepackage{amsmath}
\usepackage{authblk}
\usepackage{titlesec}
\usepackage{amsthm}
\usepackage{amsfonts}
\usepackage{amssymb}
\usepackage{array}
\usepackage{booktabs}
\usepackage{ragged2e}
\usepackage{enumerate}
\usepackage{enumitem}
\usepackage{cleveref}
\usepackage{slashed}
\usepackage{commath}
\usepackage{lipsum}
\usepackage{colonequals}
\usepackage{addfont}
\usepackage{enumitem}
\usepackage{sectsty}

\subsectionfont{\itshape}

\newtheorem{theorem}{Theorem}[section]
\newtheorem{corollary}{Corollary}[theorem]
\newtheorem{lemma}[theorem]{Lemma}
\theoremstyle{definition}
\newtheorem{definition}{Definition}[section]
\theoremstyle{remark}
\newtheorem*{remark}{Remark}

\let\oldproofname=\proofname
\renewcommand{\proofname}{\bf{\textit{\oldproofname}}}

\theoremstyle{definition}
\newtheorem{example}{Example}[section]

\newtheorem*{discussion}{Discussion}



\begin{document}

\title{Quick Notes for 134}
\author{Quin Darcy}
\date{09 March 2019\\\small{(Last Edited : 09 March 2019)}}
\affil{\small{California State University Sacramento}}
\maketitle

\section{Complex Inegration}
 The following is an example of the simplest type of complex integration. Namely, a complex values function $f$ being integrated over a finite interval $[a,b]$. Splitting $f$ into its real and imaginary parts $u$ and $v$, then this case can be reduced to integration of real valued functions. 
 
\begin{definition}
    Let $f=u+iv\colon[a,b]\rightarrow\mathbb{C}$ be a continuous function. Then the integral of $f$ over $[a,b]$ is defined by
 
    \begin{equation*}
        \int_{a}^{b}f(t)\;dt:=\int_{a}^{b}u(t)\;dt+i\int_{a}^{b}v(t)\;dt.
    \end{equation*}
 
    \noindent The continuity assumption was only made for simplicity. 
\end{definition}
 
\begin{example}
    Let $k\in\mathbb{Z}$. Then 
    
    \begin{equation*}
        \int_{0}^{2\pi} e^{ikt}\;dt=\int_{0}^{2\pi}\cos{kt}\;dt+i\int_{0}^{2\pi}\sin{kt}\;dt=\begin{cases} 2\pi &\quad\text{if}\quad k=0 \\ 0 &\quad\text{if}\quad k\neq 0. \end{cases}
    \end{equation*}
\end{example}

\begin{definition}
    A \textbf{path} in the complex plane is a continuous function 
    
    \begin{equation*}
        \gamma\colon[a,b]\subseteq\mathbb{R}\rightarrow\mathbb{C}.
    \end{equation*}         
\end{definition}

\begin{definition}
    A \textbf{curve} $C$ from $z_1,z_2\in\mathbb{C}$ is a subset of $\mathbb{C}$ that is equal to the image of a path $\gamma$, im$_{\gamma}([a,b])$, where $\gamma(a)=z_1$ and $\gamma(b)=z_2$.
\end{definition}

\begin{definition}
    Given a curve $C$, a \textbf{parameterization} of $C$ is a path $\sigma\colon[c,d]\rightarrow \mathbb{C}$, such that $C$ is the image of $\sigma$, where $\sigma(c)=z_1$ and $\sigma(d)=z_2$
\end{definition}

\begin{definition}
    A path $\gamma\colon[a,b]\rightarrow\mathbb{C}$ is \textbf{smooth} if $\gamma'$ exists and is continuous throughout all of $[a,b]$.
\end{definition}

\newpage

\subsection{Exercises}





\end{document}