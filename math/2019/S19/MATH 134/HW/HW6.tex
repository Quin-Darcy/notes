\documentclass[12pt, a4paper]{article}
\usepackage[margin=1in]{geometry}
\usepackage[latin1]{inputenc}
\usepackage{titlesec}
\usepackage{amsmath}
\usepackage{amsthm}
\usepackage{amsfonts}
\usepackage{amssymb}
\usepackage{array}
\usepackage{booktabs}
\usepackage{ragged2e}
\usepackage{enumerate}
\usepackage{enumitem}
\usepackage{cleveref}
\usepackage{slashed}
\usepackage{commath}
\usepackage{lipsum}
\usepackage{colonequals}
\usepackage{addfont}
\addfont{OT1}{rsfs10}{\rsfs}
\renewcommand{\baselinestretch}{1.1}
\usepackage[mathscr]{euscript}
\let\euscr\mathscr \let\mathscr\relax
\usepackage[scr]{rsfso}
\newcommand{\powerset}{\raisebox{.15\baselineskip}{\Large\ensuremath{\wp}}}
\usepackage{longtable}
\usepackage{multirow}
\usepackage{multicol}
\usepackage{calligra}
\usepackage[T1]{fontenc}
\newcounter{proofc}
\renewcommand\theproofc{(\arabic{proofc})}
\DeclareRobustCommand\stepproofc{\refstepcounter{proofc}\theproofc}
\usepackage{fancyhdr}
\pagestyle{fancy}

\renewcommand{\headrulewidth}{0pt}
\fancyhead[R]{}
\usepackage{enumitem}
\usepackage{tikz}
\usepackage{commath}
\usepackage{colonequals}
\usepackage{bm}
\usepackage{tikz-cd}
\renewcommand{\baselinestretch}{1.1}
\usepackage[mathscr]{euscript}
\let\euscr\mathscr \let\mathscr\relax
\usepackage[scr]{rsfso}
\usepackage{titlesec}
\usepackage{scrextend}
\usepackage{lscape}
\usepackage{relsize}

\usepackage[english]{babel}
\usepackage{blindtext}
\usepackage{polynom}



\newcommand*{\logeq}{\ratio\Leftrightarrow}

\titleformat{\section}
  {\normalfont\Large\bfseries}{\thesection}{1em}{}[{\titlerule[0.8pt]}]
  
\setlist[description]{leftmargin=6mm,labelindent=8mm, rightmargin=6mm}
  
\begin{document}
  
\begin{flushleft}
  
    Quin Darcy\par
    Dr. Krauel\par
    MATH 134\par
    03/05/19
  
\end{flushleft}
  
\centerline{\boxed{\text{Homework 6}}}
 
\vspace{4mm}
 
\noindent\textsc{Section: }\par
 
\justifying
 
\vspace{1mm}
 
\hline
 
\vspace{6mm}

\noindent\textbf{2.21} Show that

\begin{enumerate}[label=(\alph*)]
    \item Log$(-ei)=1-\frac{\pi}{2}i$;
    \item Log$(1-i)=\frac{1}{2}\ln2-\frac{\pi}{4}i.$
\end{enumerate}

\textbf{Solutions:}\par
\vspace{2mm}
\begin{enumerate}[label=(\alph*)]
    \item Log$(-ei)=\ln\abs{-ei}+i$Arg$(-ei)=\lne+i$Arg$(-ei)=1-\frac{\pi}{2}i$.
    \item Log$(1-i)=\ln\abs{1-i}+i$Arg$(1-i)=\ln(\sqrt{1^2+(-i)^2})+itan^{-1}(-1)=\frac{1}{2}\ln2-\frac{\pi}{4}i$.
\end{enumerate}

\vspace{2mm}

\noindent\textbf{2.22} Suppose that the point $z=x+iy$ lies in the horizontal strip $\alpha<y<\alph+2\pi$. Show that when the branch $\log z=\ln r+i\theta$, $(r>0,\alpha<\theta<\alpha+2\pi)$ of the logarithmic function, then $\log(e^z)=z$.

\par\vspace{4mm}
\textbf{Solution:} Given a region $R$, we know that any continuous function Log$\colon R\rightarrow\mathbb{C}$ that satisfies $\exp(\text{Log}(z))=z$ is a branch of the logorithm on $R$. So then if we take $\alpha\in\mathbb{R}$ and let $z\in\{y\in\mathbb{R}\colon \alpha<y<\alpha+2\pi\}$, and let $\log z=\ln r+i\theta$, where $r>0$ and $\alpha<\theta<\alpha+2\pi$, be a branch of the logarithm on the region $










\end{document}