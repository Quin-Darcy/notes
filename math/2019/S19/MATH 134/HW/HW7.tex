\documentclass[12pt]{article}
\usepackage[margin=0.95in]{geometry} 
\usepackage{graphicx}
\usepackage{amsmath}
\usepackage{authblk}
\usepackage{titlesec}
\usepackage{amsthm}
\usepackage{amsfonts}
\usepackage{amssymb}
\usepackage{array}
\usepackage{booktabs}
\usepackage{ragged2e}
\usepackage{enumerate}
\usepackage{enumitem}
\usepackage{cleveref}
\usepackage{slashed}
\usepackage{commath}
\usepackage{lipsum}
\usepackage{colonequals}
\usepackage{addfont}
\usepackage{enumitem}
\usepackage{sectsty}
\usepackage{lastpage}
\usepackage{fancyhdr}
\usepackage{accents}
\usepackage[inline]{enumitem}
\pagestyle{fancy}
\setlength{\headheight}{10pt}

\subsectionfont{\itshape}

\newtheorem{theorem}{Theorem}[section]
\newtheorem{corollary}{Corollary}[theorem]
\newtheorem{lemma}[theorem]{Lemma}
\theoremstyle{definition}
\newtheorem{definition}{Definition}[section]
\theoremstyle{remark}
\newtheorem*{remark}{Remark}

%\newenvironment{solution}
 % {\begin{proof}[Solution]}
 % {\end{proof}}
 
 \makeatletter
\renewenvironment{proof}[1][\proofname]{\par
  \pushQED{\qed}%
  \normalfont \topsep6\p@\@plus6\p@\relax
  \list{}{\leftmargin=0mm
          \rightmargin=6mm
          \settowidth{\itemindent}{\itshape#1}%
          \labelwidth=\itemindent
          % the following line is not needed with amsart, but might be with other classes
          \parsep=0pt \listparindent=\parindent 
  }
  \item[\hskip\labelsep
        \itshape
    #1\@addpunct{.}]\ignorespaces
}{%
  \popQED\endlist\@endpefalse
}

\newenvironment{solution}[1][\bf{\textit{Solution}}]{\par
  
  \normalfont \topsep6\p@\@plus6\p@\relax
  \list{}{\leftmargin=0mm
          \rightmargin=6mm
          \settowidth{\itemindent}{\itshape#1}%
          \labelwidth=\itemindent
          \parsep=0pt \listparindent=\parindent 
  }
  \item[\hskip\labelsep
        \itshape
    #1\@addpunct{.}]\ignorespaces
}{%
  \popQED\endlist\@endpefalse
}

\let\oldproofname=\proofname
\renewcommand{\proofname}{\bf{\textit{\oldproofname}}}

\newlist{mylist}{enumerate*}{1}
\setlist[mylist]{label=(\alph*)}

\begin{document}

\begin{center}
	\vspace{.4cm} {\textbf { \large MATH 134}}
\end{center}
{\textbf{Name:}\ Quin Darcy \hspace{\fill} \textbf{Due Date:} 03/14/19   \\
{ \textbf{Instructor:}} \ Dr. Krauel \hspace{\fill} \textbf{Assignment:} Homework 7 \\ \hrule}

\justifying

% This is where document starts 

\begin{enumerate}[leftmargin=*]
\setcounter{enumi}{23}

    \item Find all the values of $z$ such that
        \begin{center}
            \begin{mylist}
                \item $e^z=-2$; \item $e^z=1+\sqrt{3}i$.
            \end{mylist}
        \end{center}
        
        \begin{solution}\hfill
            \begin{enumerate}[label=(\alph*), rightmargin=10mm]
                \item By expressing $-2$ in it's polar form we get that $-2=e^xe^{i(\pi+2n\pi)}$. This means that $e^x=2$ and $y=\pi+2n\pi$. Thus, $x=\ln(2)$ and all the values of $z$ for which $e^z=-2$ are
                
                \begin{equation*}
                    z=\ln(2)+i(\pi+2n\pi)
                \end{equation*}
                
                \noindent for $n\in\mathbb{Z}$.
                
                \item Doing the same thing as before, let us write $1+\sqrt{3}i$ in polar form. We get that $1+\sqrt{3}i=2e^{i(\pi/3+2n\pi)}$. Thus, $e^x=2$ and $y=\pi/3+2n\pi$. Since $x=\ln(2)$ then we can express all the values $z$ for which $e^z=1+\sqrt{3}i$ in the following form
                
                \begin{equation*}
                    z=\ln(2)+i(\frac{\pi}{3}+2n\pi),
                \end{equation*}
                
                \noindent for $n\in\mathbb{Z}$.
            \end{enumerate}
        \end{solution}
        
    \item Show that if Re$(z_1)>0$ and Re$(z_2)>0$, then 
        \begin{equation*}
            \text{Log}(z_1z_2)=\text{Log}(z_1)+\text{Log}(z_2).
        \end{equation*}
        
        \begin{proof}
            Let $D=\{z\in\mathbb{C}\colon\text{Re}(z)>0\}$ be the set of all complex numbers whose real part is strictly greater than 0. Then for any $z\in D$, it follows that $\abs{\text{Arg}(z)}<\pi/2$. Thus, if $z_1,z_2\in D$, then because the multiplication of complex numbers results in the addition of their arguments, then we have that $\abs{\text{Arg}(z_1z_2)}=\abs{\text{Arg}(z_1)+\text{Arg}(z_2)}<\pi$. These means that there is only one instance of each argument, and Log$(z_1z_2)$ is single valued for  $z_1,z_2\in D$. Completing the computation, we get 
            
            \begin{equation*}
                \begin{split}
                    z_1z_2 &= \abs{z_1}e^{i\text{Arg}(z_1)}\abs{z_2}e^{i\text{Arg}(z_2)} \\
                    &= \abs{z_1z_2}e^{i(\text{Arg}(z_1)+\text{Arg}(z_2))}.
                \end{split}
            \end{equation*}
            
            \noindent Taking the Log of both sides of this equality we obtain
            
            \begin{equation*}
                \begin{split}
                    \text{Log}(z_1z_2) &= \text{Log}(\abs{z_1z_2}e^{i(\text{Arg}(z_1)+\text{Arg}(z_2))}) \\
                    &= \ln\abs{z_1z_2}+i(\text{Arg}(z_1)+\text{Arg}(z_2)) \\
                    &= \ln\abs{z_1}+\ln\abs{z_2}+i\text{Arg}(z_1)+i\text{Arg}(z_2) \\
                    &= (\ln\abs{z_1}+i\text{Arg}(z_1))+(\ln\abs{z_2}+\text{Arg}(z_2)) \\
                    &= \text{Log}(z_1)+\text{Log}(z_2).
                \end{split}
            \end{equation*}
        \end{proof}
        
    \item Show that, for any two nonzero complex numbers $z_1$ and $z_2$,
        \begin{equation*}
            \text{Log}(z_1z_2)=\text{Log}(z_1)+\text{Log}(z_2)+2N\pi i,
        \end{equation*}
        where $N$ has one of the values $0,\pm1$. (Compare with previous exercise.)
        
        \begin{proof}
            Unlike in the previous problem, there is no restriction on the complex numbers we are evaluating, besides that they are nonzero. So similar to before, we find that for any $z\in\mathbb{C}$ such that $\abs{z}\neq 0$, it follows that $\pi<\text{Arg}(z)\leq\pi$, by definition. Thus, given $z_1,z_2\in\mathbb{C}$ where $\abs{z_1}\neq 0$ and $\abs{z_2}\neq 0$, then $\abs{\text{Arg}(z_1z_2)}\leq 2\pi$ and $\abs{\text{Arg}(z_1)+\text{Arg}(z_2)}\leq 2\pi$. Thus, for all $z_1,z_2\in\mathbb{C}$
            
            \begin{equation*}
                \text{Log}(z_1z_2)=\text{Log}(z_1)+\text{Log}(z_2)+2N\pi i,
            \end{equation*}
            
            \noindent for $N=0,\pm 1$.
        \end{proof}
        
    \item Show that
        \begin{enumerate}[label=(\alph*)]
        
            \item the set of values of $\log(i^{1/2})$ is
                \begin{equation*}
                    \big(n+\frac{1}{4}\big)\pi i\qquad (n=0,\pm1,\pm2,\dots)
                \end{equation*}
                and that the same is true for $1/2\log(i)$;
                
                \begin{solution}
                    By rewriting $z=i^{1/2}$ as $z^2=i$, then $\abs{z}^2(\cos(2\theta)+i\sin(2\theta))=i$. Thus, $\abs{z}=1$ and $\theta=\pi/4+n\pi$ for $n\in\mathbb{Z}$. This gives us the two following complex numbers
                    
                    \begin{equation*}
                        z_1=e^{\frac{i\pi}{4}}\quad\text{and}\quad z_2=e^{i(\frac{\pi}{4}+\pi)}.
                    \end{equation*}
                    
                    \noindent Thus, 
                    
                    \begin{equation*}
                        \begin{split}
                            \log(z_1) &= \ln\abs{z_1}+i\arg(z_1) \\
                            &= \ln(1)+i(\pi/4+n\pi) \\
                            &= (n+\frac{1}{4})\pi i,
                        \end{split}
                    \end{equation*}
                    
                    \noindent for $n\in\mathbb{Z}$. Similarly, for $z_2$. Now since $\log(i)=(\pi/2+2n\pi)i$, for $n\in\mathbb{Z}$, then $1/2\log(i)=(\pi/4+n\pi)i=(n+1/4)\pi i$, for $n\in\mathbb{Z}$. Thus, as sets, $\log(i^{1/2})=1/2\log(i)$, as desired.
                    
                    
                \end{solution}
                
                \newpage
                
            \item the set of values of $\log(i^2)$ is not the same as $2\log(i)$.
            
            \begin{solution}
                Since $i^2=-1$ by definition, then it should be the case that $\log(i^2)=\log(-1)$. Thus, the latter set is 
                
                \begin{equation*}
                    \begin{split}
                        \log(-1) &= \ln(1)+i\arg(-1) \\
                        &= i(\pi+2n\pi)  \\
                        &= (1+2n)\pi i,
                    \end{split}
                \end{equation*}
                
                \noindent for $n\in\mathbb{Z}$. However, $2\log(i)=2(\pi/2+2n\pi)=(1+4n)\pi i$, for $n\in\mathbb{Z}$. Therefore, $\log(i^2)$ and $2\log(i)$ do not yield the same set.
            \end{solution}
        \end{enumerate}
        
\end{enumerate}






\end{document}