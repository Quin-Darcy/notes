\documentclass[12pt]{article}
\usepackage[margin=1.2in]{geometry} 
\usepackage{graphicx}
\usepackage{amsmath}
\usepackage{authblk}
\usepackage{titlesec}
\usepackage{amsthm}
\usepackage{amsfonts}
\usepackage{amssymb}
\usepackage{array}
\usepackage{booktabs}
\usepackage{ragged2e}
\usepackage{enumerate}
\usepackage{enumitem}
\usepackage{cleveref}
\usepackage{slashed}
\usepackage{commath}
\usepackage{lipsum}
\usepackage{colonequals}
\usepackage{addfont}
\usepackage{enumitem}
\usepackage{sectsty}
\usepackage{lastpage}
\usepackage{fancyhdr}
\usepackage{accents}
\usepackage[inline]{enumitem}
\pagestyle{fancy}
\setlength{\headheight}{10pt}

\subsectionfont{\itshape}

\newtheorem{theorem}{Theorem}[section]
\newtheorem{corollary}{Corollary}[theorem]
\newtheorem{lemma}[theorem]{Lemma}
\theoremstyle{definition}
\newtheorem{definition}{Definition}[section]
\theoremstyle{remark}
\newtheorem*{remark}{Remark}
 
\makeatletter
\renewenvironment{proof}[1][\proofname]{\par
  \pushQED{\qed}%
  \normalfont \topsep6\p@\@plus6\p@\relax
  \list{}{\leftmargin=0mm
          \rightmargin=0mm
          \settowidth{\itemindent}{\itshape#1}%
          \labelwidth=\itemindent
          \parsep=0pt \listparindent=\parindent 
  }
  \item[\hskip\labelsep
        \itshape
    #1\@addpunct{.}]\ignorespaces
}{%
  \popQED\endlist\@endpefalse
}

\newenvironment{solution}[1][\bf{\textit{Solution}}]{\par
  
  \normalfont \topsep6\p@\@plus6\p@\relax
  \list{}{\leftmargin=0mm
          \rightmargin=0mm
          \settowidth{\itemindent}{\itshape#1}%
          \labelwidth=\itemindent
          \parsep=0pt \listparindent=\parindent 
  }
  \item[\hskip\labelsep
        \itshape
    #1\@addpunct{.}]\ignorespaces
}{%
  \popQED\endlist\@endpefalse
}

\let\oldproofname=\proofname
\renewcommand{\proofname}{\bf{\textit{\oldproofname}}}


\newlist{mylist}{enumerate*}{1}
\setlist[mylist]{label=(\alph*)}

\begin{document}

\begin{center}
	\vspace{.4cm} {\textbf { \large MATH 134}}
\end{center}
{\textbf{Name:}\ Quin Darcy \hspace{\fill} \textbf{Due Date:} 4/25/19   \\
{ \textbf{Instructor:}} \ Dr. Krauel \hspace{\fill} \textbf{Assignment:} Homework 10 \\ \hrule}

\justifying

% Document begins here

    \begin{enumerate}[leftmargin=*]
        \item Consider the function $f(z)=z^3+z^2+z+1$.
            \begin{enumerate}[label=(\alph*)]
                \item Find the Taylor series for $f(z)$ about $z_0=2$.
                \item Expand the Taylor series found in part (a) and write it in the form               \begin{equation*}
                        c_3z^3+c_2z^2+c_1z+c_0,
                    \end{equation*} 
                \noindent for constants $c_0,c_1,c_2,c_3\in\mathbb{C}$.
            \end{enumerate}
            
            \begin{solution}\hfill\par
                \begin{enumerate}[label=(\alph*), leftmargin=9mm]
                    \item Since $f$ is a polynomial, we know that it is entire and can be sure that $f^{(k)}(z)$ exists for all $k\in\mathbb{Z}$. To begin, we will calculate the first few derivatives of $f$, then evaluate them at $z_0$.
                    
                    \begin{align*}
                        f^{(0)}(z) &= z^3+z^2 + z + 1  &f^{(0)}(2) &= 15 &\frac{f^{(0)}(2)}{0!} &= 15\\
                        f^{(1)}(z) &= 3z^2 + 2z + 1 &f^{(1)}(2) &= 17 &\frac{f^{(1)}(2)}{1!} &= 17\\
                        f^{(2)}(z) &= 6z + 2 &f^{(2)}(2) &= 14 &\frac{f^{(2)}(2)}{2!} &= 7\\
                        f^{(3)}(z) &= 6 &f^{(3)}(2) &= 6 &\frac{f^{(3)}(2)}{3!} &= 1.
                    \end{align*}
                    
                    \noindent Thus, the Taylor series for $f(z)$ about $z_0=2$ is
                    
                    \begin{equation*}
                        15+17(z-2)+7(z-2)^2+(z-2)^3.
                    \end{equation*}
                    
                    \item Expanding out the polynomial we obtained in (a) we get
                    
                    \begin{equation*}
                        \begin{split}
                            &15+17(z-2)+7(z^2-4z+4)+(z^3-6z^2+12z-8) \\
                            &= 15 + 17z-34+7z^2-28z+28+z^3-6z^2+12z-8 \\
                            &= z^3+z^2+z+1.
                        \end{split}
                    \end{equation*}
                    
                    \noindent Thus, our constants $c_0,c_1,c_2,c_3\in\mathbb{C}$ are $c_0=c_1=c_2=c_3=1$. So we have obtained our original function $f$.
                \end{enumerate}
            \end{solution}
        
        \item Obtain the Taylor series 
        
            \begin{equation*}
                e^z=\sum_{n=0}^{\infty}\frac{(z-1)^n}{n!},\quad\quad(\;\abs{z-1}<\infty\;)
            \end{equation*}
        
            \noindent for the function $f(z)=e^z$ by
            
            \begin{enumerate}[label=(\alph*)]
                \item using $f^{(n)}(1)$ $(n=0,1,2,\dots)$; and
                \item writing $e^z=e^{z-1}e$ and then using the identity $e^{z-1}=\sum_{n=0}^{\infty}\frac{1}{n!}(z-1)^n$.
            \end{enumerate}
            
            \begin{solution}\hfill\par
                \begin{enumerate}[label=(\alph*),leftmargin=9mm]
                    \item Since $f^{(n)}=e^z$ for all $n\in\mathbb{N}$, then for all $n\in\mathbb{N}$ we have $f^{(n)}(1)=e$. Thus, the Taylor series for $f(z)=e^z$ about $z_0=1$ is
                    
                    \begin{equation*}
                        \sum_{n=0}^{\infty}\frac{f^{(n)}(1)}{n!}(z-1)^n = e\sum_{n=0}^{\infty}\frac{(z-1)^n}{n!}.
                    \end{equation*}
                    \item Using the given identity
                    
                    \begin{equation}
                        e^{z-1}=\sum_{n=0}^{\infty}\frac{(z-1)^n}{n!}
                    \end{equation}
                    
                    \noindent and the result from part (a), we find that by multiplying both sides of (1) by $e$, we obtain
                    
                    \begin{equation*}
                        e\cdot e^{z-1} = e^z=e\sum_{n=0}^{\infty}\frac{(z-1)^n}{n!}, 
                    \end{equation*}
                    
                    \noindent which is the, by part (a), the Taylor series of $f(z)=e^z$ about the point $z_0=1$.
                \end{enumerate}            
            \end{solution}
            
        \item Without using derivatives, find the Maclaurin series expansion of the function 
        
            \begin{equation*}
                f(z)=\frac{z}{z^4+9}=\frac{1}{z}\cdot\frac{1}{1+(z^4/9)}.
            \end{equation*}
            
            \begin{solution} We begin by using the identity 
            
            \begin{equation*}
                \sum_{n=0}^{\infty} z^n=\frac{1}{1-z}.
            \end{equation*}
            
            \noindent This identity holds whenever $\abs{z}<1$. So we will first re-write the right side of our given product as
            
            \begin{equation*}
                \frac{1}{z}\cdot\frac{1}{1+(z^4/9)}=\frac{1}{z}\cdot\frac{1}{1-(-z^4/9)}.
            \end{equation*}
            
            \noindent Thus, we have that
            
            \begin{equation*}
                \begin{split}
                    \frac{1}{z}\cdot\frac{1}{1+(z^4/9)} &=\frac{1}{z}\sum_{n=0}^{\infty}(-\frac{z^4}{9})^n \\
                    &= \frac{1}{z}\sum_{n=0}^{\infty}(-1)^n(\frac{z^4}{9})^n\\
                    &= \sum_{n=0}^{\infty}(-\frac{1}{9})^nz^{4n-1}.
                \end{split}
            \end{equation*}
                
            \end{solution}
            
            
        
    \end{enumerate}
    
    
    
\end{document}