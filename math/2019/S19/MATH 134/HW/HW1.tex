\documentclass[12pt, a4paper]{article}
\usepackage[margin=1in]{geometry}
\usepackage[latin1]{inputenc}
\usepackage{titlesec}
\usepackage{amsmath}
\usepackage{amsthm}
\usepackage{amsfonts}
\usepackage{amssymb}
\usepackage{array}
\usepackage{booktabs}
\usepackage{ragged2e}
\usepackage{enumerate}
\usepackage{enumitem}
\usepackage{cleveref}
\usepackage{slashed}
\usepackage{commath}
\usepackage{lipsum}
\usepackage{colonequals}
\usepackage{addfont}
\addfont{OT1}{rsfs10}{\rsfs}
\renewcommand{\baselinestretch}{1.1}
\usepackage[mathscr]{euscript}
\let\euscr\mathscr \let\mathscr\relax
\usepackage[scr]{rsfso}
\newcommand{\powerset}{\raisebox{.15\baselineskip}{\Large\ensuremath{\wp}}}
\usepackage{longtable}
\usepackage{multirow}
\usepackage{multicol}
\usepackage{calligra}
\usepackage[T1]{fontenc}
\newcounter{proofc}
\renewcommand\theproofc{(\arabic{proofc})}
\DeclareRobustCommand\stepproofc{\refstepcounter{proofc}\theproofc}
\usepackage{fancyhdr}
\pagestyle{fancy}

\renewcommand{\headrulewidth}{0pt}
\fancyhead[R]{}
\usepackage{enumitem}
\usepackage{tikz}
\usepackage{commath}
\usepackage{colonequals}
\usepackage{bm}
\usepackage{tikz-cd}
\renewcommand{\baselinestretch}{1.1}
\usepackage[mathscr]{euscript}
\let\euscr\mathscr \let\mathscr\relax
\usepackage[scr]{rsfso}
\usepackage{titlesec}
\usepackage{scrextend}
\usepackage{lscape}
\usepackage{relsize}

\usepackage[english]{babel}
\usepackage{blindtext}
\usepackage{polynom}



\newcommand*{\logeq}{\ratio\Leftrightarrow}

\titleformat{\section}
  {\normalfont\Large\bfseries}{\thesection}{1em}{}[{\titlerule[0.8pt]}]
  
\setlist[description]{leftmargin=6mm,labelindent=4mm}
  
\begin{document}
  
\begin{flushleft}
  
    Quin Darcy\par
    Dr. Krauel\par
    MATH 134\par
    2/8/19
  
\end{flushleft}
  
\centerline{\boxed{\text{Homework 1}}}
 
\vspace{4mm}
 
\noindent\textsc{Section: }\par
 
\justifying
 
\vspace{1mm}
 
\hline
 
\vspace{6mm}

\noindent\textbf{1.11.} Prove parts (a)-(c) of the following proposition. 

\begin{description}
    \item\textsc{Proposition 1.18} \textit{Let $z,w\in\mathbb{C}$. We have}
    
    \begin{description}
        \item(a) $\overline{\overline{z}}=z$,
        \item(b) $\overline{z+w}=\overline{z}+\overline{w}$,
        \item(c) $\overline{wz}=\overline{w}\overline{z}$.
    \end{description}
    
    \vspace{4mm}
    
    \item\textit{\textbf{Proof.}} (a) Let $z=a+bi$. Then 
    
    \begin{equation*}
        \overline{\overline{z}} = \overline{a-bi} = a-(-b)i = a+bi = z.
    \end{equation*}
    
    \item\textit{\textbf{Proof.}} (b) Let $z=a+bi$ and $w=c+di$. Then
    
    \begin{equation*}
        \begin{split}
            \overline{z+w} &=\overline{(a+bi)+(c+di)} \\
            &=\overline{(a+c)+(b+d)i} \\
            &= (a+c)-(b+d)i \\
            &= (a-bi)+(c-di) \\
            &= \overline{z}+\overline{w}.
        \end{split}
    \end{equation*}
    
    \item\textit{\textbf{Proof.}} (c) Let $z=a+bi$ and $w=c+di$. Then
    
    \begin{equation*}
        \begin{split}
            \overline{zw} &= \overline{(a+bi)(c+di)} \\
            &= \overline{ac+adi+bci-bd} \\
            &= \overline{(ac-bd)+(ad+bc)i} \\
            &= (ac-bd)-(ad+bc)i \\
            &= ac-bd-adi-bci \\
            &= (a-bi)(c-di) \\
            &= \overline{z}\overline{w}.
        \end{split}
    \end{equation*}
    
\end{description}

\newpage

\noindent\textbf{1.13} Let $n\in\mathbb{Z}$ and $\theta\in\mathbb{R}$. Then 

\begin{equation*}
    (\cos(\theta)+i\sin(\theta))^n = \cos(n\theta)+i\sin(n\theta).
\end{equation*}

\begin{description}
    \item\textit{\textbf{Proof.}} 
    
    \begin{description}
        \item\textsc{Base case:} Let $n=1$. Then
        
        \begin{equation*}
            (\cos(\theta)+i\sin(\theta))^1=\cos(1\theta)+i\sin(1\theta).
        \end{equation*}
        
        \vspace{2mm}
        
        \item\textsc{Inductive step:} Assume for some $k\in\mathbb{Z}$, where $k>1$,
        
        \begin{equation*}
            (\cos(\theta)+i\sin(\theta))^k = \cos(k\theta)+i\sin(k\theta).
        \end{equation*}
        
        \vspace{2mm}
        
        \item Then rewriting the above in exponential form we get
        
        \begin{equation*}
            (e^{i\theta})^k=e^{ik\theta}.
        \end{equation*}
        
        \vspace{2mm}
        
        \item Multiplying both sides by $e^{i\theta}$, we get
        
        \begin{equation*}
            \begin{split}
                (e^{i\theta})^{k+1} &= e^{ik\theta}\cdot e^{i\theta} \\
                &= e^{ik\theta+i\theta} \\
                &= e^{i\theta(k+1)}. \\
            \end{split}
        \end{equation*}
        
        \item Thus, rewriting the first and last equality in normal polar form, we obtain
        
        \begin{equation*}
            (e^{i\theta})^{k+1}=(\cos(\theta)+i\sin(\theta))^{k+1}=\cos((k+1)\theta)+i\sin((k+1)\theta)=e^{i\theta(k+1)}.
        \end{equation*}
        
        \vspace{2mm}
        
        \item Therefore, for all $n\in\mathbb{Z}$ and $\theta\in\mathbb{R}$
        
        \begin{equation*}
            (\cos(\theta)+i\sin(\theta))^n = \cos(n\theta)+i\sin(n\theta),
        \end{equation*}
        
        \item as desired.
        
    \end{description}    
\end{description}

\newpage

\noindent\textbf{1.14} Find the $7th$ roots of unity. Express them in exponential form and plot them.

\vspace{4mm}

\begin{description}
    \textbf{Solution: } Let $z\in\mathbb{C}$ such that $z^7=1$. Then let us denote $z=\cos(\theta)+i\sin(\theta)$, then by problem 1.13
    
    \begin{equation*}
        z^7 = (\cos(\theta)+i\sin(\theta))^7 = \cos(7\theta)+i\sin(7\theta) = 1.
    \end{equation*}
    
    \item Since there is $\im(z)=0$, then it follows that $\cos(7\theta)=1$. Moreover, we have that the $\arccos{(1)}=\{\varphi\in\mathbb{R}|\exists k(k\in\mathbb{Z}\wedge \varphi=2k\pi$)\}. Thus, for any $\varphi$ in this set we can take it and set it equal to $7\theta$. This yields that each $\theta$ in $\arg(z)$ is one such that $\theta=\varphi/7=2k\pi/7$. Thus, consider
    
    \begin{align*}
        \theta_1 &= 0 & \theta_2 &= \frac{2\pi}{7} & \theta_3 &= \frac{4\pi}{7} \\
        \theta_4 &= \frac{6\pi}{7} & \theta_5 &= \frac{8\pi}{7} & \theta_6 &= \frac{10\pi}{7} \\
        \theta_7 &= \frac{12\pi}{7}.
    \end{align*}
    
    \item Thus, the 7 roots in exponential form are
    
    \begin{align*}
        z_1 &= 1 & z_2 &= \exp{(i\frac{2\pi}{7})} & z_3 &= \exp{(i\frac{4\pi}{7})} \\
        z_4 &= \exp{(i\frac{6\pi}{7})} & z_5 &= \exp{(i\frac{8\pi}{7})} & z_6 &= \exp{(i\frac{10\pi}{7})} \\ z_7 &= \exp{(i\frac{12\pi}{7})}.
    \end{align*}

\end{description}

\noindent\textbf{1.15} Find the cubed roots of $1+\sqrt{3}i$. Express them in normal form.

\begin{description}
    \item\textbf{Solution: } Let $z=1+\sqrt{3}i$. Then $\abs{z}=2$ and arg$(z)=\{\pi/3+2\pi k|k\in\mathbb{Z}\}$. Thus, the 3 cube roots will be of the form 
    
    \begin{equation*}
        (2)^{\frac{1}{3}}e^{i(\frac{\pi}{3}+2\pi k)}
    \end{equation*}
    
    \item for $k\in\mathbb{Z}$. We find three distinct elements are
    
    \begin{align*}
        w_1 &= (2)^{\frac{1}{3}}e^{\frac{i\pi}{3}} & w_2 &= (2)^{\frac{1}{3}}e^{\frac{i7\pi}{3}} & w_3 &= (2)^{\frac{1}{3}}e^{\frac{i13\pi}{3}}.
    \end{align*}
    
    \item Written in normal form we have
    
    \begin{equation*}
        \begin{split}
            &w_1 = (2)^{\frac{1}{3}}e^{\frac{i\pi}{3}} = (2)^{\frac{1}{3}}\cos{(\frac{\pi}{3})}+i(2)^{\frac{1}{3}}\sin{(\frac{\pi}{3})} \\
            &w_2 = (2)^{\frac{1}{3}}e^{\frac{i7\pi}{3}} = (2)^{\frac{1}{3}}\cos{(\frac{7\pi}{3})}+i(2)^{\frac{1}{3}}\sin{(\frac{7\pi}{3})} \\
            &w_3 = (2)^{\frac{1}{3}}e^{\frac{i13\pi}{3}} = (2)^{\frac{1}{3}}\cos{(\frac{13\pi}{3})}+i(2)^{\frac{1}{3}}\sin{(\frac{13\pi}{3})}. 
        \end{split}
    \end{equation*}
\end{description}

\newpage

\noindent\textbf{1.16} Are the following statements true or false?

\begin{description}
    \item(1) A set is closed if it contains its boundary. \textbf{TRUE}
    \item(2) A set is open if it does not contain its boundary. \textbf{FALSE}
    \item(3) An open set may contain some boundary points, but not all. \textbf{FALSE}
\end{description}


\end{document}