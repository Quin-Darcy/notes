\documentclass[12pt]{article}
\usepackage[margin=1in]{geometry}
\usepackage{graphicx}
\usepackage{amsmath}
\usepackage{amsthm}
\usepackage{amsfonts}
\usepackage{amssymb}
\usepackage{array}
\usepackage{enumerate}
\usepackage{slashed}
\usepackage{colonequals}
\usepackage{fancyhdr}
\usepackage{enumitem}

\pagestyle{fancy}
\fancyhf{}
\rhead{Darcy}
\lhead{MATH \#}
\rfoot{\thepage}
\setlength{\headheight}{10pt}

\newtheorem{theorem}{Theorem}[section]
\newtheorem{corollary}{Corollary}[theorem]
\newtheorem{prop}{Proposition}[section]
\newtheorem{lemma}[theorem]{Lemma}
\theoremstyle{definition}
\newtheorem{definition}{Definition}[section]
%\theoremstyle{remark}
%\newtheorem*{remark}{Remark}

\newcommand{\abs}[1]{\lvert #1 \rvert}
\newcommand{\bigabs}[1]{\Bigl \lvert #1 \Bigr \rvert}
\newcommand{\bigbracket}[1]{\Bigl [ #1 \Bigr ]}
\newcommand{\bigparen}[1]{\Bigl ( #1 \Bigr )}
\newcommand{\ceil}[1]{\lceil #1 \rceil}
\newcommand{\bigceil}[1]{\Bigl \lceil #1 \Bigr \rceil}
\newcommand{\floor}[1]{\lfloor #1 \rfloor}
\newcommand{\bigfloor}[1]{\Bigl \lfloor #1 \Bigr \rfloor}
\newcommand{\norm}[1]{\| #1 \|}
\newcommand{\bignorm}[1]{\Bigl \| #1 \Bigr \| #1}
\newcommand{\inner}[1]{\langle #1 \rangle}
\newcommand{\set}[1]{{ #1 }}

\begin{document}
    \thispagestyle{empty}\hrule

    \begin{center}
        \vspace{.4cm} { \large MATH \#}
    \end{center}
    {Name:\ Quin Darcy \hspace{\fill} Due Date: \#/\#/\# \\
    { Instructor:}\ Dr.\# \hspace{\fill} Assignment:
    Homework \# \\ \hrule}

    \begin{enumerate}
        \item[16.1] Show that if $Y$ is a subspace of $X$, and $A$ is a subset
            of $Y$, then the topology $A$ inherits as a subspace of $Y$ is the
            same as the topology it inherits as a subspace of $X$.
            \begin{proof}
                Given a basis $\mathcal{B}$ of $X$, we will let
                $\mathcal{B}_1=\{B\cap Y\mid B\in\mathcal{B}\}$. Similarly, we
                will let $\mathcal{B}_2=\{B\cap A\mid B\in\mathcal{B}_1\}$.
                Lastly, we let $\mathcal{B}_3=\{B\cap A\mid B\in\mathcal{B}\}$.
                By Lemma 16.1, $\mathcal{B}_1, \mathcal{B}_2$, and
                $\mathcal{B}_3$ are bases for the topologies $\mathcal{T}_1$
                for $Y$, $\mathcal{T}_2$ for $A$, and $\mathcal{T}_3$ for $A$,
                respectively. We want to show that
                $\mathcal{T}_2=\mathcal{T}_3$. \par\hspace{4mm}To this end, we let
                $B\in\mathcal{B}_2$ and choose any $x\in B$. Then we need to
                show that there exists $B'\in\mathcal{B}_3$ such that $x\in
                B'\subset B$. Since $\mathcal{B}_3$ is a basis for $A$ and
                $x\in A$, then by Definition 13.1, there exits
                $V\in\mathcal{B}_3$ such that $x\in V$. Note that $B=B_1\cap A$
                for $B_1\in\mathcal{B}_2$ and $V=B_2\cap A$ for
                $B_2\in\mathcal{B}_3$. Moreover, $B_1=T\cap Y$ for
                $T\in\mathcal{B}$ and $B_2=U\cap A$ for $U\in\mathcal{B}$.
                Hence
                \begin{equation*}
                    B\cap B'=(B_1\cap A)\cap(B_2\cap A)=((T\cap Y)\cap
                    A)\cap((U\cap A)\cap A)=(T\cap U)\cap A.
                \end{equation*}
                Seeing as $x\in(T\cap U)$, then there exists
                $\hat{B}\in\mathcal{B}$ such that $x\in\hat{B}\subset T\cap U$.
                It follows that $x\in\hat{B}\cap A\subset B\cap B'\subset
                B$. Hence, $\mathcal{T}_2\subset\mathcal{T}_3$. The other
                inclusion is shown in a similar way, and so we can conclude
                that $\mathcal{T}_2=\mathcal{T}_3$.
            \end{proof}
        \item[16.4] A map $f:X\to Y$ is said to be an \textbf{\textit{open
            map}} if for every open set $U$ of $X$, the set $f(U)$ is open in
            $Y$. Show that $\pi_1:X\times Y\to X$ and $\pi_2:X\times Y\to Y$
            are open maps.
            \begin{proof}
                Let $\mathcal{B}_1$ be a basis for $X$ and $\mathcal{B}_2$ be
                a basis for $Y$. Then by Theorem 15.1,
                $\mathcal{B}_1\times\mathcal{B}_2$ is a basis for $X\times Y$.
                Note that if $U\subset X$ is open, then it is equal to the
                union of some collection of elements $B\in\mathcal{B}_1$.
                Seeing as for any such $(B,
                B')\in\mathcal{B}_1\times\mathcal{B}_2$, we have that
                $\pi_1((B, B'))=B$, which is open in $X$, and the fact that
                maps preserve unions, it then follows that if $U=\bigcup_{B\in
                C}B$, then $\pi_1((U, V))=U$. Hence, if $W$ is open in $X\times
                Y$, then $\pi_1(W)$ is open in $X$. The same argument is used
                to show that $\pi_2$ is an open map.
            \end{proof}
        \item[17.2] Show that if $A$ is closed in $Y$ and $Y$ is closed in $X$,
            then $A$ is closed in $X$.
            \begin{proof}
                By assumption, $Y-A$ is open in $Y$ and $X-Y$ is open in $X$.
                We want to show that $X-A$ is open in $X$.
            \end{proof}
    \end{enumerate}
\end{document}