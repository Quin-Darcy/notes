\documentclass[12pt]{article}
\usepackage[margin=1in]{geometry}
\usepackage{graphicx}
\usepackage{amsmath}
\usepackage{amsthm}
\usepackage{amsfonts}
\usepackage{amssymb}
\usepackage{array}
\usepackage{enumerate}
\usepackage{slashed}
\usepackage{colonequals}
\usepackage{fancyhdr}

\pagestyle{fancy}
\fancyhf{}
\rhead{Darcy}
\lhead{MATH 220A}
\rfoot{\thepage}
\setlength{\headheight}{10pt}

\newtheorem{theorem}{Theorem}[section]
\newtheorem{corollary}{Corollary}[theorem]
\newtheorem{prop}{Proposition}[section]
\newtheorem{lemma}[theorem]{Lemma}
\theoremstyle{definition}
\newtheorem{definition}{Definition}[section]
%\theoremstyle{remark}
%\newtheorem*{remark}{Remark}

\newcommand{\abs}[1]{\lvert #1 \rvert}
\newcommand{\bigabs}[1]{\Bigl \lvert #1 \Bigr \rvert}
\newcommand{\bigbracket}[1]{\Bigl [ #1 \Bigr ]}
\newcommand{\bigparen}[1]{\Bigl ( #1 \Bigr )}
\newcommand{\ceil}[1]{\lceil #1 \rceil}
\newcommand{\bigceil}[1]{\Bigl \lceil #1 \Bigr \rceil}
\newcommand{\floor}[1]{\lfloor #1 \rfloor}
\newcommand{\bigfloor}[1]{\Bigl \lfloor #1 \Bigr \rfloor}
\newcommand{\norm}[1]{\| #1 \|}
\newcommand{\bignorm}[1]{\Bigl \| #1 \Bigr \| #1}
\newcommand{\inner}[1]{\langle #1 \rangle}
\newcommand{\set}[1]{{ #1 }}

\begin{document}
    \thispagestyle{empty}\hrule

    \begin{center}
        \vspace{.4cm} { \large MATH 220A}
    \end{center}
    {Name:\ Quin Darcy \hspace{\fill} Due Date: 2/19/21  \\
    { Instructor:}\ Dr.Martins \hspace{\fill} Assignment:
    Homework 2 \\ \hrule}

    \begin{enumerate}
        \item Show that if $\mathcal{A}$ is a basis for a topology on $X$, then
            the topology generated by $\mathcal{A}$ equals the intersection of
            all topologies on $X$ that contain $\mathcal{A}$. Prove the same if
            $\mathcal{A}$ is a subbasis.
                \begin{proof}
                    Let $\mathfrak{J}$ denote the topology generated by $\mathcal{A}$, and let $\mathcal{T}$ be any topology on $X$
                    which contains $\mathcal{A}$. Suppose $S\in\mathfrak{J}$. Then by Lemma 13.1, $S=\bigcup B_{\alpha}$ for $B_{\alpha}\in\mathcal{A}$. Furthermore, since $\mathcal{A}\subseteq\mathcal{T}$ and $\mathcal{T}$ is closed under unions, then $S\in\mathcal{T}$. Hence, $S$ is in the intersection of all topologies containing $\mathcal{A}$. Thus, $\mathfrak{J}\subseteq\mathcal{T}$, which implies $\mathfrak{J}\subseteq\bigcap\mathcal{T}$. \par\hspace{4mm} To obtain the other inclusion, we note that $\mathfrak{J}$ is a topology which contains $\mathcal{A}$ and so $\bigcap\mathcal{T}\subseteq\mathfrak{J}$. Thus proving the equality.\par\hspace{4mm} If $\mathcal{A}$ is a subbasis and $\mathfrak{J}$ is the topology generated by $\mathcal{A}$. Then if $S\in\mathfrak{J}$, we can write $S$ as a union of finite intersections of elements of $\mathcal{A}$. Thus, as $\mathcal{A}\subseteq\mathcal{T}$, then $S\in\mathcal{T}$ since $\mathcal{T}$ is closed under finite intersections. Hence, $\mathfrak{J}\subseteq\bigcap\mathcal{T}$. \par\hspace{4mm} To prove the other inclusion, we can use the same argument as above. Namely, with $\mathfrak{J}$ being a topology which contains $\mathcal{A}$, it is then in the intersection of all topologies containing $\mathcal{A}$. Thus, $\bigcap{T}\subseteq\mathfrak{J}$.
                \end{proof}
        \item[7.] Consider the following topologies on $\mathbb{R}$.
            \begin{equation*}
                \begin{split}
                    \mathfrak{J}_1&=\text{the standard topology.} \\
                    \mathfrak{J}_3&=\text{the finite complement topology.} \\
                    \mathfrak{J}_4&=\text{the upper limit topology, having all sets $(a, b]$ as basis.} \\
                    \mathfrak{J}_5&=\text{the topology having all sets $(-\infty, a)=\{x\mid x<a\}$ as basis.} 
                \end{split}
            \end{equation*}
            Determine, for each of these topologies, which of the others it
            contains.
                \begin{proof}
                    
                \end{proof}
        \item[8.] 
            \begin{enumerate}
                \item Apply Lemma 13.2 to show that the countable collection
                    \begin{equation*}
                        \mathcal{B}=\{(a, b)\mid a<b, a\text{ and  }b \text{
                        rational }\} 
                    \end{equation*}
                is a basis that generates the standard topology on
                $\mathbb{R}$.
                    \begin{proof}
                        
                    \end{proof}
                \item Show that the collection
                    \begin{equation*}
                        \mathcal{C}=\{[a, b)\mid a< b, a\text{ and }b \text{
                        rational }\}. 
                    \end{equation*}
                    \begin{proof}
                        
                    \end{proof}
            \end{enumerate}
        \item[1.] Show that if $Y$ is a subspace of $X$, and $A$ is a subset of
            $Y$, then the topology $A$ inherits as a subspace of $Y$ is the
            same as the topology it inherits as a subspace of $X$.
                \begin{proof}
                    
                \end{proof}
    \end{enumerate}
\end{document}
