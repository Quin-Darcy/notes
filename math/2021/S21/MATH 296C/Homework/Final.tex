\documentclass[12pt]{article}
\usepackage[margin=1in]{geometry}
\usepackage{graphicx}
\usepackage{amsmath}
\usepackage{amsthm}
\usepackage{amsfonts}
\usepackage{amssymb}
\usepackage{array}
\usepackage{enumerate}
\usepackage{slashed}
\usepackage{colonequals}
\usepackage{fancyhdr}
\usepackage{enumitem}

\pagestyle{fancy}
\fancyhf{}
\rhead{Darcy}
\lhead{MATH 296C}
\rfoot{\thepage}
\setlength{\headheight}{10pt}

\newtheorem{theorem}{Theorem}[section]
\newtheorem{corollary}{Corollary}[theorem]
\newtheorem{prop}{Proposition}[section]
\newtheorem{lemma}[theorem]{Lemma}
\theoremstyle{definition}
\newtheorem{definition}{Definition}[section]
%\theoremstyle{remark}
%\newtheorem*{remark}{Remark}
\newenvironment{solution}
{\renewcommand\qedsymbol{$\blacksquare$}\begin{proof}[Solution]}
{\end{proof}}

\newcommand{\abs}[1]{\lvert #1 \rvert}
\newcommand{\bigabs}[1]{\Bigl \lvert #1 \Bigr \rvert}
\newcommand{\bigbracket}[1]{\Bigl [ #1 \Bigr ]}
\newcommand{\bigparen}[1]{\Bigl ( #1 \Bigr )}
\newcommand{\ceil}[1]{\lceil #1 \rceil}
\newcommand{\bigceil}[1]{\Bigl \lceil #1 \Bigr \rceil}
\newcommand{\floor}[1]{\lfloor #1 \rfloor}
\newcommand{\bigfloor}[1]{\Bigl \lfloor #1 \Bigr \rfloor}
\newcommand{\norm}[1]{\| #1 \|}
\newcommand{\bignorm}[1]{\Bigl \| #1 \Bigr \| #1}
\newcommand{\inner}[1]{\langle #1 \rangle}
\newcommand{\set}[1]{{ #1 }}

\begin{document}
    \thispagestyle{empty}\hrule

    \begin{center}
        \vspace{.4cm} { \large MATH 296C}
    \end{center}
    {Name:\ Quin Darcy \hspace{\fill} Due Date: 5/17/21   \\
    { Instructor:}\ Dr. Krauel \hspace{\fill} Assignment:
    Final Exam \\ \hrule}

    \begin{enumerate}
        \item Explain your work for everything done.
        \begin{enumerate}[label=\textbf{\alph*}.]
            \item Do one of the following:
            \begin{enumerate}[label=(\Roman*)]
                \item Construct a rank 2 root system with associated angle $\theta=3\pi/4$. Include a diagram of the roots, along with their labels.
                \begin{solution}
                    Taking $\alpha,\beta\in\Phi$ such that $3\pi/4$ is the angle between them, then since this is an obtuse angle, we have that $\alpha+\beta\in\Phi$. Additionally, we have that $-\alpha, -\beta, -\alpha-\beta\in\Phi$. The angle between $\alpha$ and $\alpha+\beta$ is $\pi/4$
                \end{solution}
            \end{enumerate}
        \end{enumerate}
    \end{enumerate}
\end{document}