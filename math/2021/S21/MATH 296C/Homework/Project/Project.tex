\documentclass[12pt]{article}
\usepackage[margin=1in]{geometry}
\usepackage{graphicx}
\usepackage{amsmath}
\usepackage{amsthm}
\usepackage{amsfonts}
\usepackage{amssymb}
\usepackage{array}
\usepackage{enumerate}
\usepackage{slashed}
\usepackage{colonequals}
\usepackage{fancyhdr}
\usepackage{import}
\usepackage{xifthen}
\usepackage{pdfpages}
\usepackage{transparent}

\newcommand{\incfig}[1]{%
    \def\svgwidth{\columnwidth}
    \import{}{#1.pdf_tex}
}

\pagestyle{fancy}
\fancyhf{}
\rhead{}
\lhead{}
\rfoot{\thepage}
\setlength{\headheight}{10pt}

\newtheorem{theorem}{Theorem}[section]
\newtheorem{corollary}{Corollary}[theorem]
\newtheorem{prop}{Proposition}[section]
\newtheorem{lemma}[theorem]{Lemma}
\theoremstyle{definition}
\newtheorem{definition}{Definition}[section]
\theoremstyle{definition}
\newtheorem{exmp}{Example}[section]

\newcommand{\abs}[1]{\lvert #1 \rvert}
\newcommand{\bigabs}[1]{\Bigl \lvert #1 \Bigr \rvert}
\newcommand{\bigbracket}[1]{\Bigl [ #1 \Bigr ]}
\newcommand{\bigparen}[1]{\Bigl ( #1 \Bigr )}
\newcommand{\ceil}[1]{\lceil #1 \rceil}
\newcommand{\bigceil}[1]{\Bigl \lceil #1 \Bigr \rceil}
\newcommand{\floor}[1]{\lfloor #1 \rfloor}
\newcommand{\bigfloor}[1]{\Bigl \lfloor #1 \Bigr \rfloor}
\newcommand{\norm}[1]{\| #1 \|}
\newcommand{\bignorm}[1]{\Bigl \| #1 \Bigr \| #1}
\newcommand{\inner}[1]{\langle #1 \rangle}
\newcommand{\set}[1]{{ #1 }}


\begin{document}
\title{Plotting eigenvalues of $\mathfrak{sl}_n(\mathbb{C})$ and its generating roots}
\author{Quin Darcy}
\date{April, 15 2021}
\maketitle
    \section{How the roots are calculated}
    \begin{enumerate}
        \item Generate basis for $\mathfrak{sl}_n(\mathbb{C})$, given some
            $n\in\mathbb{Z}_{\geq 1}$: This is done in 3 parts.
            \begin{enumerate}
                \item Calculate dimension of $\mathfrak{sl}_n(\mathbb{C})$. We
                    use the formula 
                    \begin{equation*}
                        \text{dim}(\mathfrak{sl}_n(\mathbb{C}))=n-1+2\sum_{i=1}^{n-1}i,
                    \end{equation*}
                    where the $n-1$ term is a count of how many basis elements have
                    entries only on the main diagonal, and the sum is a count of
                    all the basis elements with entries off the main diagonal.
                \item As the formula suggests, we simply fill an array with 1's at
                    all entries $a_{ij}$, where $i\neq j$, i.e., 
                    \begin{equation*}
                        \begin{split}
                            &\texttt{for i in range(n):} \\
                            &\quad\quad\texttt{for j in range(n):} \\
                            &\quad\quad\quad\quad\texttt{if (i != j)} \\
                            &\quad\quad\quad\quad\quad\quad\texttt{B[matnum][i][j] = 1} \\ 
                            &\quad\quad\quad\quad\quad\quad\texttt{matnum++}.
                        \end{split} 
                    \end{equation*}
                    Similarly, for the diagonal elements, we use 
                    \begin{equation*}
                        \begin{split}
                            &\texttt{for i in range(n-1):} \\
                            &\quad\quad\texttt{B[matnum][i][i] = 1} \\
                            &\quad\quad\texttt{B[matnum][i+1][i+1] = -1} \\
                            &\quad\quad\texttt{matnum++}.
                        \end{split} 
                    \end{equation*}
                    Here \texttt{n} is the number the user entered and
                    \texttt{matnum} is a count of the matrices since \texttt{B}
                    is an array containing as many matrices as there are
                    dimensions in $\mathfrak{sl}_n(\mathbb{C})$, and each
                    matrix is $n\times n$.
            \end{enumerate}
    \end{enumerate}
    \subsection{A Quick Visual Detour}
        We stop here after step 1 for a moment since there is something
        interesting that we can look at with what we have so far. With a basis
        in hand, we can, in theory, generate all of
        $\mathfrak{sl}_n(\mathbb{C})$. However, this is not possible on
        a physical computer, as it
        would require taking the span of the basis elements which is an
        uncountably infinite set.\par\hspace{4mm} We can instead consider a finite spanning
        set. Suppose for instance that we want to see how the eigenvalues of
        $\mathfrak{sl}_n(\mathbb{C})$ are distributed in $\mathbb{C}$ as we
        range over $n$. To do this, we can, as stated before, take a particular
        finite spanning set of the basis, and plot all of the complex
        eigenvalues of each element in this spanning set.\par\hspace{4mm} But how do we choose
        the elements which populate our finite spanning set? In other words,
        any finite selection would in a sense be arbitrary. I could not think
        of a way around this aside from making the choice as natural and
        uniform as possible. With this I ask you, dear reader, what is more
        uniform and natural that the complex unit circle? Each point is unit
        length, and it occupies all 4 quadrants. As you ponder the
        answer to that question, let us just roll with it (``roll" $\dots$
        ``circle"$\dots$, you get it) and see where it takes
        us.\par\hspace{4mm} Let $N\in\mathbb{N}_{\geq 0}$, then we can choose $N$ 
        many points from the complex unit circle by letting $\theta
        = 2\pi/N$ and taking $M=\{c\in\mathbb{C}\mid \cos(\theta
        k)+i\sin(\theta k)=c, \text{ for }0\leq k<N\}$. We can now thing of the set 
        $M$ as the set containing all the multiples which we shall use to form our 
        linear combinations. For example, suppose $M=\{\alpha_1, \alpha_2\}$ and 
        our basis $B=\{x, y, h\}$. Then we want to calculate the following linear 
        combinations (which we'll denote $A_i$):
        \begin{align*}
            &A_{1}=\alpha_1x+\alpha_1y+\alpha_1h& 
            &A_{2}=\alpha_1x+\alpha_1y+\alpha_2h& \\ 
            &A_{3}=\alpha_1x+\alpha_2y+\alpha_1h& 
            &A_{4}=\alpha_1x+\alpha_2y+\alpha_2h& \\ 
            &A_5=\alpha_2x+\alpha_1y+\alpha_1h&   
            &A_6=\alpha_2x+\alpha_1y+\alpha_2h& \\
            &A_7=\alpha_2x+\alpha_2y+\alpha_1h&   &A_8=\alpha_2x+\alpha_2y+\alpha_2h&
        \end{align*}
        If it is not already clear, given a basis with $b$ many elements, and a 
        set of $N$ many multiples, then we will have a spanning set containing 
        $N^b$ many elements. So in our particular case our spanning set will contain
        \begin{equation*}
            N^{\text{dim}(\mathfrak{sl}_n(\mathbb{C}))}
        \end{equation*}
        many elements.\par\hspace{4mm} Alright, now with our linear combinations, 
        we need only calculate the eigenvalues for each $A_i$. So finally, supposing 
        all such eigenvalues are collected neatly into a set, then the last step is 
        to plot each element of this set. Below are the results of this.\newpage
        \begin{figure}[htp!]
            \centering
            \incfig{sl_22}
            \caption{14,706,125 2x2 matrices; 29,411,760 distinct eigenvalues}
            \label{fig:14,706,125 2x2 matrices}
        \end{figure}\newpage
        \begin{figure}[htb!]
            \centering
            \incfig{sl_33}
            \caption{16,777,216 3x3 matrices; 50,322,992 distinct eigenvalues}
            \label{fig:2}
        \end{figure}\newpage
        \begin{figure}[htb!]
            \centering
            \incfig{sl_44}
            \caption{14,348,907 4x4 matrices; 57,138,453 distinct eigenvalues}
            \label{fig:}
        \end{figure}\newpage
        \begin{figure}[htb!]
            \centering
            \incfig{sl_55}
            \caption{16,777,216 5x5 matrices; 81,914,833 distinct eigenvalues}
            \label{fig:}
        \end{figure}\newpage
    \subsection{Continuing where we left off}
    \begin{enumerate}
        \item[3.] Obtain the Cartan subalgebra. This step is actually quite
            easy in the case of $\mathfrak{sl}_n(\mathbb{C})$ since a basis for this
            subalgebra can be found by simply selecting, from the elements of the
            entire basis, those elements with entries on the main diagonal.
        \item[4.] Calculate the adjoint representation of each basis element of
            the Cartan subalgebra. The reason we are doing this is because we
            can think of the roots of the Lie algebra as the nonzero weights of
            the adjoint representation. Calculating the adjoint representaion
            is relatively simple on paper, but quite dry in code. Essentially,
            we take each (Cartan) basis element, bracket it with every (entire)
            basis element, express the result as a linear combination of the
            (entire) basis elements, and use the coefficients in this linear
            combination as a row in the adjoint representaion matrix.
        \item[5.] Calculate the eigenvectors of each adjoint representation.
            The weight vectors are those who are eigenvectors to every adjoint
            representation matrix. Hence, the weights are the eigenvalues of
            such eigenvectors.
        \item[6.] Filter through these eigenvalues, removing those with 0 in
            ever component and voila you have your roots!
    \end{enumerate}
\end{document}