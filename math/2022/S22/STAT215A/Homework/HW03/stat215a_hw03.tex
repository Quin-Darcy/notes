\documentclass[12pt]{article}
\usepackage[margin=1in]{geometry}
\usepackage{graphicx}
\usepackage{amsmath}
\usepackage{amsthm}
\usepackage{amsfonts}
\usepackage{amssymb}
\usepackage{array}
\usepackage{enumerate}
\usepackage{fancyhdr}
\pagestyle{fancy}
\fancyhf{}
\rhead{Darcy}
\lhead{STAT 215A}
\rfoot{\thepage}
\setlength{\headheight}{10pt}

\newenvironment{solution}
{\renewcommand\qedsymbol{$\blacksquare$}\begin{proof}[Solution]}
{\end{proof}}
\newenvironment{psmall}{\left(\begin{smallmatrix}}{\end{smallmatrix}\right)}

\begin{document}
    \thispagestyle{empty}\hrule

    \begin{center}
        \vspace{.4cm} { \large STAT 215A}
    \end{center}
    {Name:\ Quin Darcy \hspace{\fill} Due Date:\ 03/08/2022   \\
    { Instructor:}\ Dr. Cetin \hspace{\fill} Assignment:\
    Homework 03 \\ \hrule}

    \begin{enumerate} 
        \item Consider the experiment of rolling a balanced six-sided die
            repeatedly until a six occurs for the first time. Let $Y$ represent
            the variable for the number of the rolls before the first six
            appears. So, the possible values of $Y$ are $0, 1, 2, \dots, $ ($Y$
            is unbounded).
            \begin{enumerate}[(a)]
                \item Determine the pmf of $Y$.
                    \begin{solution}
                        We begin by noting that in this experiment 
                        \begin{equation*}
                            \Omega=\{6, 16, 26, 36, 46, 56, 116, \dots\}.
                        \end{equation*}
                        And so $Y$ is a discrete random variable
                        $Y:\Omega\to\mathbb{R}$ with $\text{range}(Y)=\{1, 2,
                        \dots\}$. Then the pmf of $Y$ is defined as
                        \begin{equation*}
                            p_Y(k)=\mathbb{P}(\{\omega\in\Omega\mid
                            X(\omega)=k\}).
                        \end{equation*}
                        This means that for any $k\in\mathbb{N}$, $p_Y(k)$ is
                        the probability that we roll a die $k-1$ times in a row and
                        never get a 6, but on the $k$th time we do roll a 6.
                        Observe that to roll a die once and not obtain a 6 has
                        a probability $\frac{5}{6}$ and to roll a die and
                        obtain a 6 has probability $\frac{1}{6}$. Since each
                        dice roll is independent from the previous roll, then
                        it follows that the probability of rolling a non-six
                        value $k-1$ times in a row, followed by a 6 on the
                        $k$th roll has probability 
                        \begin{equation*}
                            p_Y(k)=\bigg(\frac{5}{6}\bigg)^{k-1}\bigg(\frac{1}{6}\bigg)
                            =\frac{5^{k-1}}{6^k}.
                        \end{equation*}
                        Note that for all $k\in\{1, 2, \dots, \}$, $0\leq
                        p_Y(k)\leq 1$, and 
                        \begin{equation*}
                            \sum_{k=1}^{\infty}p_Y(k)=\sum_{k=1}^{\infty}\frac{5^{k-1}}{6^k}=1.
                        \end{equation*}
                        This shows that $p_Y(k)$ satisfies both conditions of
                        a pmf. 
                    \end{solution}
                \item Compute $P(Y>1+k|Y>1)$, where $k\in\mathbb{N}$.
                    \begin{solution}
                        For any $k\in\mathbb{N}$, let $A$ denote the event that
                        $Y>1+k$, and let $B$ denote the event that $Y>1$.
                        Seeing as $k$ starts at 1, then clearly $A\subseteq B$.
                        Hence $A\cap B = A$, and so 
                        \begin{equation*}
                            \mathbb{P}(Y>1+k\mid Y>1)=\frac{\mathbb{A\cap
                            B}}{\mathbb{P}(B)}=\frac{\mathbb{P}(A)}{\mathbb{P}(B)}.
                        \end{equation*}
                        To determine $\mathbb{P}(A)$, we first want to compute the
                        probability of $A^c$ which is the event that we get
                        a 6 in the first $1+k$ rolls. 
                        \begin{equation*}
                            \mathbb{P}(Y\leq
                            1+k)=\sum_{i=1}^{1+k}p_Y(i)=
                            \sum_{i=1}^{1+k}\frac{5^{i-1}}{6^i}=1-\bigg(\frac{6}{5}\bigg)^{-k-1}
                            =1-\bigg(\frac{5}{6}\bigg)^{k+1}.
                        \end{equation*}
                        Hence
                        \begin{equation*}
                            \mathbb{P}(A)=1-\mathbb{P}(A^c)=\bigg(\frac{5}{6}\bigg)^{k+1}.
                        \end{equation*}
                        Similarly, $\mathbb{P}(B)=1-\mathbb{P}(B^c)$. And $B^c$
                        is the event that $Y\leq 1$. Seeing as $Y$ can be 1 at
                        the smallest, then this is equivalent to
                        $\mathbb{P}(Y=1)=\frac{1}{6}$. Thus
                        $\mathbb{P}(B)=1-\frac{1}{6}=\frac{5}{6}$. Therefore
                        \begin{equation*}
                            \mathbb{P}(Y>1+k\mid
                            Y>1)=\frac{\mathbb{P}(A)}{\mathbb{P}(B)}
                            =\frac{\big(\frac{5}{6}\big)^{k+1}}{\big(\frac{5}{6})}
                            =\bigg(\frac{5}{6}\bigg)^k.
                        \end{equation*}
                    \end{solution}
            \end{enumerate}
        \item Let $X$ be a r.v. in a probability space $(\Omega, \mathcal{F},
            \mathbb{P})$ such that $P(X\geq 0)=1$. Moreover, suppose that
            $Y=1-e^{-2X}$.
            \begin{enumerate}[(a)]
                \item Explain why $Y$ is also a r.v. on $(\Omega, \mathcal{F},
                    \mathbb{P})$, and then describe the cdf of $Y$ in terms of
                    the cdf of $X$. 
                    \begin{solution}
                        
                    \end{solution}
                \item If $X\sim Exp(2)$, then determine the pdf of $Y$. 
                    \begin{solution}
                        
                    \end{solution}
            \end{enumerate}
    \end{enumerate}
\end{document}
