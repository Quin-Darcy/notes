\documentclass[12pt]{article}
\usepackage[margin=1in]{geometry}
\usepackage{graphicx}
\usepackage{amsmath}
\usepackage{amsthm}
\usepackage{amsfonts}
\usepackage{amssymb}
\usepackage{array}
\usepackage{enumerate}
%\usepackage{slashed}
%\usepackage{colonequals}
\usepackage{fancyhdr}
%\usepackage{import}
%\usepackage{xifthen}
%\usepackage{pdfpages}
%\usepackage{transparent}

%\newcommand{\incfig}[1]{%
%    \def\svgwidth{\columnwidth}
%    \import{/home/arbegla/figures/}{#1.pdf_tex}
%}

\pagestyle{fancy}
\fancyhf{}
\rhead{}
\lhead{}
\rfoot{\thepage}
\setlength{\headheight}{10pt}

\newtheorem{theorem}{Theorem}[section]
\newtheorem{corollary}{Corollary}[theorem]
\newtheorem{prop}{Proposition}[section]
\newtheorem{lemma}[theorem]{Lemma}
\theoremstyle{definition}
\newtheorem{definition}{Definition}[section]
\theoremstyle{definition}
\newtheorem{exmp}{Example}[section]

\newcommand{\abs}[1]{\lvert #1 \rvert}
\newcommand{\bigabs}[1]{\Bigl \lvert #1 \Bigr \rvert}
\newcommand{\bigbracket}[1]{\Bigl [ #1 \Bigr ]}
\newcommand{\bigparen}[1]{\Bigl ( #1 \Bigr )}
\newcommand{\ceil}[1]{\lceil #1 \rceil}
\newcommand{\bigceil}[1]{\Bigl \lceil #1 \Bigr \rceil}
\newcommand{\floor}[1]{\lfloor #1 \rfloor}
\newcommand{\bigfloor}[1]{\Bigl \lfloor #1 \Bigr \rfloor}
\newcommand{\norm}[1]{\| #1 \|}
\newcommand{\bignorm}[1]{\Bigl \| #1 \Bigr \| #1}
\newcommand{\inner}[1]{\langle #1 \rangle}
\newcommand{\set}[1]{{ #1 }}


\begin{document}
\title{Measure Theory Notes}
\author{Quin Darcy}
\date{31 March 2022}
\maketitle
\section{General Measure Spaces}
    \begin{definition}
        Let $M$ be a nonempty set. Then a collection of subsets
        $\sigma\subseteq\mathcal{P}(M)$ is called a $\sigma$-algebra if 
        \begin{enumerate}
            \item $M\in\sigma$
            \item $A\in\sigma\to M\backslash A\in\sigma$
            \item $A_1, A_2, \dots\in\sigma\to\bigcup_{n=1}^{\infty}A_n\in\sigma$
        \end{enumerate}
        The pair $(M, \sigma)$ is called a measurable space. 
    \end{definition}
    \begin{definition}
        A measure $\mu:\sigma\to \overline{R}$ on a measure space $(M, \sigma)$ is
        a map satisfying
        \begin{enumerate}
            \item $\mu(\varnothing)=0$
            \item $A_1, A_2, \dots\in\sigma$, $A_i\cap A_j=\varnothing$ when $i\neq
            j$
                \begin{equation*}
                    \mu\bigg(\bigcup_{n\geq 1}A_n\bigg)=\sum_{n\geq 1}\mu(A_n).
                \end{equation*}
        \end{enumerate}
        The tuple $(M, \sigma, \mu)$ is called a measure space. 
    \end{definition}
\section{Properties of a Meausre}
    Given a measure space $(M, \sigma, \mu)$
    \begin{enumerate}
        \item $A_1, A_2\in\sigma$ and $A_1\subset A_2$, then $\mu(A_1)\leq\mu(A_2)$.
        \item $A_1, A_2, \dots\in\sigma$ then
            \begin{equation*}
                \mu\bigg(\bigcup_{n\geq 1}A_n\bigg)\leq\sum_{n\geq 1}\mu(A_n).
            \end{equation*}
        \item Contiuity from below. An increasing sequence of measurable sets
            $A_1\subseteq A_2\subseteq\cdots$ where $\bigcup_{n\geq 1}A_n=A$,
            then 
            \begin{equation*}
                \lim_{n\to\infty}\mu(A_n)=\mu(A).
            \end{equation*}
        \item A decreasing sequence of measurable sets $A_1\supseteq
            A_2\supseteq\cdots$ where $\bigcap_{n\geq 1}A_n=A$ and
            $\mu(A_1)<\infty$, then 
            \begin{equation*}
                \lim_{n\to\infty}\mu(A_n)=\mu(A). 
            \end{equation*}
    \end{enumerate}
\end{document}
