\documentclass[12pt]{article}
\usepackage[margin=1in]{geometry}
\usepackage{graphicx}
\usepackage{amsmath}
\usepackage{amsthm}
\usepackage{amsfonts}
\usepackage{amssymb}
\usepackage{array}
\usepackage{enumerate}
\usepackage{slashed}
\usepackage{colonequals}
\usepackage{fancyhdr}
\usepackage{import}
\usepackage{xifthen}
\usepackage{pdfpages}
\usepackage{transparent}

\newcommand{\incfig}[1]{%
    \def\svgwidth{\columnwidth}
    \import{/home/arbegla/figures/}{#1.pdf_tex}
}

\pagestyle{fancy}
\fancyhf{}
\rhead{}
\lhead{}
\rfoot{\thepage}
\setlength{\headheight}{10pt}

\newtheorem{theorem}{Theorem}
\newtheorem{corollary}{Corollary}[theorem]
\newtheorem{prop}{Proposition}[section]
\newtheorem{lemma}[theorem]{Lemma}
\theoremstyle{definition}
\newtheorem{definition}{Definition}[section]
\theoremstyle{definition}
\newtheorem{exmp}{Example}[section]


\begin{document}
\title{Taylor's Theorem Notes}
\author{Quin Darcy}
\date{Feb, 3 2022}
\maketitle
\section{Needed Results}
    \begin{theorem}[Mean value Theorem]
        If $f$ is a continuous function on $[a, b]$ which is differentiable in
        $(a, b)$, then there is a point $x\in(a, b)$ at which 
        \begin{equation*}
            f(b)-f(a)=(b-a)f'(x).
        \end{equation*}
    \end{theorem}
\section{Two Variations of Taylor's Theorem}
    \begin{theorem}[Rudin]
        Suppose $f$ is a real function of $[a, b]$, $n$ is a positive integer,
        $f^{(n-1)}$ is continuous on $[a, b]$, $f^{(n)}(t)$ exists for all
        $t\in(a, b)$. Let $\alpha, \beta$ be distinct points of $[a, b]$, and
        define
        \begin{equation}
            P(t)=\sum_{k=0}^{n-1}\frac{f^{(k)}(\alpha)}{k!}(t-\alpha)^k.
        \end{equation}
        Then there exists a point $x$ between $\alpha$ and $\beta$ such that 
        \begin{equation}
            f(\beta)=P(\beta)+\frac{f^{(n)}(x)}{n!}(\beta-\alpha)^n.
        \end{equation}
    \end{theorem}
    \begin{proof}
       Let $M$ be the number that satisfies
       \begin{equation}
           f(\beta)=P(\beta)+M(\beta-\alpha)^n.
       \end{equation}
       With this $M$ in hand, define 
       \begin{equation}
           g(t)=f(t)-P(t)-M(t-\alpha)^n\;\;\;\;(a\leq t\leq b).
       \end{equation}
       We have to show that $n!M=f^{(n)}(x)$ for some $x$ between $\alpha$ and
       $\beta$. Now consider
       \begin{align*}
           g(t)&=f(t)-\big(f(\alpha)+f'(\alpha)(t-\alpha)+\frac{f''(\alpha)}{2!}(t-\alpha)^2+\cdots
           +\frac{f^{(n-1)}(\alpha)}{(n-1)!}(t-\alpha)^{n-1}\big)-M(t-\alpha)^n
           \\
           &\Rightarrow
           g'(t)=f'(t)-\big(f'(\alpha)+\cdots
           +\frac{f^{(n-1)}(\alpha)}{(n-2)!}(t-\alpha)^{n-2}\big)-nM(t-\alpha)^{n-1}
           \\
           &\Rightarrow
           g''(t)=f''(t)-\big(f''(\alpha)+\cdots
           +\frac{f^{(n-1)}(\alpha)}{(n-3)!}(t-\alpha)^{n-3}\big)-n(n-1)M(t-\alpha)^{n-2}
           \\
           &\vdots \\
           &\Rightarrow
           g^{(n-1)}(t)=f^{(n-1)}(t)-\big(f^{(n-1)}(\alpha)\big)-n(n-1)\cdots(2)M(t-\alpha)
           \\
           &\Rightarrow g^{(n)}(t)=f^{(n)}(t)-n!M,
       \end{align*}
       For all $a<t<b$. Hence, the proof will be complete if we can show
       $g^{(n)}(x)=0$ for some $x$ between $\alpha$ and $\beta$.\par\hspace{4mm} Note that
       $P(\alpha)=f(\alpha)$ since all the other terms in the sum have
       a $(t-\alpha)^k$ attached, which goes to zero when evaluated at
       $\alpha$. Moreover, $P'(\alpha)=f'(\alpha)$ for the same reason, except
       the first $f(\alpha)$ is elimiated by the derivative. Generally, we have
       $P^{(k)}=f^{(k)}(\alpha)$, for each $k=0, 1, \dots,n-1$. Thus
       \begin{equation*}
           g(\alpha)=g'(\alpha)=\cdots=g^{(n-1)}(\alpha)=0
       \end{equation*}
       Also note that our choice of $M$ in (2) gives that 
       \begin{equation*}
           M =\frac{f(\beta)-P(\beta)}{(\beta-\alpha)^n}
       \end{equation*}
       which implies
       \begin{align*}
           g(\beta)&=f(\beta)-P(\beta)-\bigg(\frac{f(\beta)-P(\beta)}{(\beta-\alpha)^n}\bigg)
           (\beta-\alpha)^n \\
           &=f(\beta)-P(\beta)-f(\beta)+P(\beta) \\
           &= 0. 
       \end{align*}
       Thus we have that $g$ is a continuous function on $[\alpha, \beta]$
       which is differentiable in $(\alpha, \beta)$, and so by Theorem (MVT),
       there exists $x_1\in(\alpha, \beta)$ such that
       $g'(x_1)=(f(\beta)-f(\alpha))/(\beta-\alpha)=0$. Since $g'(x_1)=0$, then
       we can conclude that $g''(x_2)=0$ for some $x_2\in(\alpha, x_1)$.
       Continuing in this manner for $n$ steps, we arrive at some
       $x_n\in(\alpha, x_{n-1})$ such that $g^{(n)}(x_n)=0$. And since
       $x_n\in(\alpha, \beta)$, then we have shown what we needed. 
    \end{proof}
    \begin{theorem}[Bloch]
        Let $[a, b]\subseteq\mathbb{R}$ be a non-degenerate (i.e., $(a, b), [a,
        b], (a, b]$, etc.) closed bounded interval, let $c\in(a, b)$, let
        $f:[a, b]\to\mathbb{R}$ be a function and let
        $n\in\mathbb{N}\cup\{0\}$. Suppose that $f^{(k)}$ exists and is
        continuous on $[a, b]$ for each $k\in\{0, \dots, n\}$, and that
        $f^{(n+1)}$ exists on $(a, b)$. Let $x\in[a, b]$ then there is some
        $p$ strictly between $x$ and $c$ (except that $p=c$ when $x=c$) such
        that
        \begin{equation*}
            f(x)=\sum_{k=0}^{n}\frac{f^{(k)}(c)}{k!}(x-c)^k+\frac{f^{(n+1)}(p)}{(n+1)!}(x-c)^{n+1}.
        \end{equation*}
    \end{theorem}
    \begin{proof}
        For $x=c$, the theorem holds trivially since everything else zeros out
        leaving $f(c)=f(c)$.\par\hspace{4mm} Now suppose that $x\neq c$. Then
        there is a unique $B\in\mathbb{R}$ such that the following equation
        holds (simply solve for $B$):
        \begin{equation*}
            f(x)=\sum_{k=0}^{n}\frac{f^{(k)}(c)}{k!}(x-c)^k+B(x-c)^{n+1}.
        \end{equation*}
        To prove the theorem, we will show that there is some $p$ strictly
        between $x$ and $c$ such that 
        \begin{equation}
            B=\frac{f^{(n+1)}(p)}{(n+1)!}.
        \end{equation}
    \end{proof}
\end{document}
