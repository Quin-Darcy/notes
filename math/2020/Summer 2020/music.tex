\documentclass{article}
\usepackage{graphicx}
\usepackage{amsmath}
\usepackage{authblk}
\usepackage{titlesec}
\usepackage{amsthm}
\usepackage{amsfonts}
\usepackage{amssymb}
\usepackage{array}
\usepackage{booktabs}
\usepackage{ragged2e}
\usepackage{enumerate}
\usepackage{enumitem}
\usepackage{cleveref}
\usepackage{slashed}
\usepackage{commath}
\usepackage{lipsum}
\usepackage{colonequals}
\usepackage{addfont}
\usepackage{enumitem}
\usepackage{sectsty}

\subsectionfont{\itshape}

\newtheorem{theorem}{Theorem}[section]
\newtheorem{corollary}{Corollary}[theorem]
\newtheorem{lemma}[theorem]{Lemma}
\theoremstyle{definition}
\newtheorem{definition}{Definition}[section]
\theoremstyle{remark}
\newtheorem*{remark}{Remark}
\theoremstyle{example}
\newtheorem{example}{Example}[section]

\let\oldproofname=\proofname
\renewcommand{\proofname}{\bf{\textit{\oldproofname}}}





\begin{document}

\section{A quick review of modular arithmetic}
Below we will be referring to the \textit{integers modulo 4}, or the set $\mathbb{Z}_4$. To understand what this is just think of it containing the remainder of every integer after being divided by 4. For example, the number 5 gives remainder 1 after being divided by 4 since $5=1\cdot4+1$. Another example is 58. Think about how many times 4 goes into 58 (14 times since $4\cdot14=56$) and so the remainder is 2. This means that every integer is \textit{congruent} to either 0, 1, 2, 3, or 4. So we would write $5\equiv 1\pmod{4}$ and $58\equiv 2\pmod{4}$ to express ``5 is congruent to 1 mod 4'' and ``58 is congruent to 2 mod 4'', respectively. 

\section{Voicings}
Let $X$ be the root of a $X$ \textit{maj}7 chord. Then we can describe this chord as a sequence of four tones
    \begin{equation*}
        a_0=X,\;a_1=X+4,\;a_2=X+7,\;a_3=X+11,
    \end{equation*}
where what we are adding to $X$ is the number of semitones separating it from the next note in the chord. So if $X$ was the note $C$, then $x+7$ would refer to the note 7 semitones up from $C$ which is $G$. Keeping this in mind, then consider the chords produced by letting 
    \begin{align*}
        &LH:\;a_i,\; a_i+7,\; a_i+12 \\
        &RH:\;a_{i+1},\; a_{i+2},\;a_{i+3},\;a_{i+4}
    \end{align*}




\end{document}