\section{}
\documentclass[leqno]{article}
%------------------------------------------------------------
\usepackage{amsmath,amssymb,amsthm}
%------------------------------------------------------------
\usepackage[utf8]{inputenc}
\usepackage[T1]{fontenc}
\usepackage[table,xcdraw]{xcolor}
\usepackage[colorlinks=true,pagebackref=true]{hyperref}
\hypersetup{urlcolor=blue, citecolor=red, linkcolor=blue}
\usepackage[capitalise,noabbrev,nameinlink]{cleveref}

\usepackage{graphicx}
\usepackage{tikz}
\usepackage{authblk}
\usepackage{titlesec}
\usepackage{amsthm}
\usepackage{amsfonts}
\usepackage{amssymb}
\usepackage{array}
\usepackage{booktabs}
\usepackage{ragged2e}
\usepackage{enumerate}
\usepackage{enumitem}
\usepackage{cleveref}
\usepackage{slashed}
\usepackage{commath}
\usepackage{lipsum}
\usepackage{colonequals}
\usepackage{addfont}
\usepackage{enumitem}
\usepackage{sectsty}
\usepackage{mathtools}
\usepackage{mathrsfs}
\usepackage{biblatex}

\addbibresource{Analysis.bib}

\hypersetup{
    colorlinks=true,
    linkcolor=blue,
    filecolor=magenta,      
    urlcolor=cyan,
}

\newtheorem{theorem}{Theorem}[section]
\newtheorem{corollary}{Corollary}[theorem]
\newtheorem{lemma}{Lemma}[theorem]
\theoremstyle{definition}
\newtheorem{prop}{Proposition}[section]
\newtheorem{definition}{Definition}[section]
\theoremstyle{remark}
\newtheorem*{remark}{Remark}

\let\oldproofname=\proofname
\renewcommand{\proofname}{\bf{\textit{\oldproofname}}}

\newcommand{\closure}[2][3]{%
  {}\mkern#1mu\overline{\mkern-#1mu#2}}


\newtheorem{example}{Example}[section]

\newtheorem*{discussion}{Discussion}

\makeatletter
\renewenvironment{proof}[1][\proofname]{\par
  \pushQED{\qed}%
  \normalfont \topsep6\p@\@plus6\p@\relax
  \list{}{\leftmargin=0mm
          \rightmargin=0mm
          \settowidth{\itemindent}{\itshape#1}%
          \labelwidth=4mm
          \parsep=0pt \listparindent=0mm%\parindent 
  }
  \item[\hskip\labelsep
        \itshape
    #1\@addpunct{.}]\ignorespaces
}{%
  \popQED\endlist\@endpefalse
}

\makeatletter
\renewenvironment{proof*}[1][\proofname]{\par
  \pushQED{\qed}%
  \normalfont \topsep6\p@\@plus6\p@\relax
  \list{}{\leftmargin=0mm
          \rightmargin=0mm
          \settowidth{\itemindent}{\itshape#1}%
          \labelwidth=\itemindent
          \parsep=0pt \listparindent=0mm%\parindent 
  }
  \item[\hskip\labelsep
        \itshape
    #1\@addpunct{.}]\ignorespaces
}{%
  \popQED\endlist\@endpefalse
}

\newenvironment{solution}[1][\bf{\textit{Solution}}]{\par
  
  \normalfont \topsep6\p@\@plus6\p@\relax
  \list{}{\leftmargin=0mm
          \rightmargin=0mm
          \settowidth{\itemindent}{\itshape#1}%
          \labelwidth=\itemindent
          \parsep=0pt \listparindent=\parindent 
  }
  \item[\hskip\labelsep
        \itshape
    #1\@addpunct{.}]\ignorespaces
}{%
  \popQED\endlist\@endpefalse
}


\begin{document}
\title{Finance Notes}
\author{Quin Darcy}
\date{Aug, 8 2020}
\maketitle

\section{Interest}
    The initially deposited amount will be called the \textbf{principle amount} and will be denoted $P$. The annual interest rate will be denoted by the symbol $r$. The sum of the principle amount and any earned interest will be called the \textbf{capital} or \textbf{amount due}. The symbol $A$ will be used to denote the amount due. The relationship between $P$, $r$, and $A$ after a single year period is
        \begin{equation}
            A=P+Pr=P(1+r).
        \end{equation}
    In general, if the financial institution adds interest earned into the depositor's account, then after $t$ years, the capital $A$ will grow to
        \begin{equation}
            A=P(1+r)^t.
        \end{equation}
\section{Compound Interest}
    Whenever interest is allowed to earn interest itself, an investment is said to earn \textbf{compound interest}. We will let $n$ denote the number of compounding periods per year. Now it is necessary to the of the interest rate per compounding period, $r/n$. Additionally, the elapsed time should be thought of as some number of compounding periods rather than years. Thus, with $n$ compounding periods per year, the number of compounding periods in $t$ years is $nt$. Therefore, the formula for compound interest is
        \begin{equation}
            A=P\left(1+\frac{r}{n}\right)^{nt}
        \end{equation}
\section{Present Value}
    We wish to compare the value of a particular investment at the present time with the value of the investment at some point in the future. This is the comparison between the \textbf{present value} of an investment and the \textbf{future value} of the investment.\par The future value of  $t$ years from now of an invested amount $P$ subject to an annual interest rate $r$ compounded monthly is
        \begin{equation}
            A=P\left(1+\frac{r}{12}\right)^{12t}.
        \end{equation}
    By contrast the present value of $A$ in an environment of interest rate $r$ compounded monthly for $t$ many years is
        \begin{equation}
            P=A\left(1+\frac{r}{12}\right)^{-12t}.
        \end{equation}
    \begin{example}
        Suppose an investor will receive payments at the end of the next six years in the amounts shown in the table below.
            \begin{table}[htc]
                \centering
                \begin{tabular}{|c|c|c|c|c|c|c|}
                \hline
                    Year    & 1   & 2   & 3   & 4   & 5   & 6   \\ \hline
                    Payment & 465 & 233 & 632 & 365 & 334 & 248 \\ \hline
                \end{tabular}
            \end{table}\hfill\par
        If the interest rate is 3.99\% compounded monthly, what is the present value of the investments? Assuming the first payment will arrive one year from now, the present value is the sum
            \begin{equation*}
                \begin{split}
                    465&\left(1+\frac{0.0399}{12}\right)^{-12}+233\left(1+\frac{0.0399}{12}\right)^{-24}+632\left(1+\frac{0.0399}{12}\right)^{-36} \\ &+365\left(1+\frac{0.0399}{12}\right)^{-48}+334\left(1+\frac{0.0399}{12}\right)^{-60}+248\left(1+\frac{0.0399}{12}\right)^{-72} \\
                    &=2003.01.
                \end{split}
            \end{equation*}
    \end{example}
    Loans are always made under the assumptions of a prevailing interest rate, and amount to be borrowed, and the lifespan of the loan, \textit{i.e.} the time the borrower has to repay the loan. 
    \begin{example}
        Suppose a person is 27 years old and plans to retire at 57. For the next 30 years they plan to invest a percentage of their salary into their 401(k) which earns interest at the rate of 9.6\% with compound period $n$. After retirement, the person plans on receiving a monthly payment of \$1,516 for 33 years. 
    \end{example}
    Let $P$ be the principle investment, $r$ be the nominal annual interest rate, $n$ be the number of compounding periods in 1 year, $i$ be the annual rate of inflation, and $s$ be the number of times a deposit $x$ is made in 1 year. Then letting $A_t$ denote the capital after $t$ many years, we get that
        \begin{equation*}
            \begin{split}
                A_0&=P \\
                A_1&=\frac{P}{(1+i)}\left(1+\frac{r}{n}\right)^n+x
            \end{split}
        \end{equation*}
\end{document}