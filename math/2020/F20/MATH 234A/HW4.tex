\section{}
\documentclass[12pt]{article}
\usepackage[margin=1in]{geometry} 
\usepackage{graphicx}
\usepackage{amsmath}
\usepackage{authblk}
\usepackage{titlesec}
\usepackage{amsthm}
\usepackage{amsfonts}
\usepackage{amssymb}
\usepackage{array}
\usepackage{booktabs}
\usepackage{ragged2e}
\usepackage{enumerate}
\usepackage{enumitem}
\usepackage{cleveref}
\usepackage{slashed}
\usepackage{commath}
\usepackage{lipsum}
\usepackage{colonequals}
\usepackage{addfont}
\usepackage{enumitem}
\usepackage{sectsty}
\usepackage{lastpage}
\usepackage{fancyhdr}
\usepackage{accents}
\usepackage[table,xcdraw]{xcolor}
\usepackage[inline]{enumitem}
\usepackage{tikz-cd}
\pagestyle{fancy}

\fancyhf{}
\rhead{Darcy}
\lhead{MATH 234A}
\rfoot{\thepage}
\setlength{\headheight}{10pt}

\subsectionfont{\itshape}

\newtheorem{theorem}{Theorem}[section]
\newtheorem{corollary}{Corollary}[theorem]
\newtheorem{prop}{Proposition}[section]
\newtheorem{lemma}[theorem]{Lemma}
\theoremstyle{definition}
\newtheorem{definition}{Definition}[section]
\theoremstyle{remark}
\newtheorem*{remark}{Remark}
 
\makeatletter
\renewenvironment{proof}[1][\proofname]{\par
  \pushQED{\qed}%
  \normalfont \topsep6\p@\@plus6\p@\relax
  \list{}{\leftmargin=0mm
          \rightmargin=4mm
          \settowidth{\itemindent}{\itshape#1}%
          \labelwidth=\itemindent
          \parsep=0pt \listparindent=\parindent 
  }
  \item[\hskip\labelsep
        \itshape
    #1\@addpunct{.}]\ignorespaces
}{%
  \popQED\endlist\@endpefalse
}

\newenvironment{solution}[1][\bf{\textit{Solution}}]{\par
  
  \normalfont \topsep6\p@\@plus6\p@\relax
  \list{}{\leftmargin=0mm
          \rightmargin=4mm
          \settowidth{\itemindent}{\itshape#1}%
          \labelwidth=\itemindent
          \parsep=0pt \listparindent=\parindent 
  }
  \item[\hskip\labelsep
        \itshape
    #1\@addpunct{.}]\ignorespaces
}{%
  \popQED\endlist\@endpefalse
}

\let\oldproofname=\proofname
\renewcommand{\proofname}{\bf{\textit{\oldproofname}}}


\newlist{mylist}{enumerate*}{1}
\setlist[mylist]{label=(\alph*)}

\begin{document}\thispagestyle{empty}\hline

\begin{center}
	\vspace{.4cm} {\textbf { \large MATH 234A}}
\end{center}
{\textbf{Name:}\ Quin Darcy \hspace{\fill} \textbf{Due Date:} 11/17/20   \\
{ \textbf{Instructor:}}\ Dr. Bin Lu \hspace{\fill} \textbf{Assignment:} Homework 4 \\ \hrule}

\justifying
    \begin{enumerate}[leftmargin=*]
        \item Let $\gamma\colon[0,1]\rightarrow\mathbb{C}$ be any $C^1$ curve. Define
            \begin{equation*}
                f(z)=\oint_{\gamma}\frac{1}{\zeta-z}d\zeta.
            \end{equation*}
            Prove that $f$ is holomorphic on $\mathbb{C}\backslash\Tilde{\gamma}$, where $\Tilde{\gamma}=\{\gamma(t)\colon0\leq t\leq 1\}$. In case $\gamma(t)=t$, show that there is no way to extend $f$ to a continuous function on $\mathbb{C}$.
                \begin{proof}
                    
                \end{proof}
        \item Explain why the following string of inequalities is incorrect:
            \begin{equation*}
                \frac{d^2}{dx^2}\int_{-1}^{1}\log\abs{x-t}dt=\int_{-1}^{1}\frac{d^2}{dx^2}\log\abs{x-t}dt=\int_{-1}^{1}\frac{-1}{(x-t)^2}dt
            \end{equation*}
            \begin{solution}
                We begin by noting 
                    \begin{equation*}
                        \begin{split}
                            \int_{-1}^{1}\log\abs{x-t}dt &= -((x-t)\log\abs{x-t}-(x-t))\bigg|_{t=-1}^{1} \\
                            &= -2-(x-1)\log\abs{x-1}+(x+1)\log\abs{x+1}.
                        \end{split}
                    \end{equation*}
                And so 
                    \begin{equation*}
                        \begin{split}
                            \frac{d^2}{dx^2}\int_{-1}^{1}\log\abs{x-t}dt \\
                            &=\frac{d^2}{dx^2}(-2-(x-1)\log\abs{x-1}+(x+1)\log\abs{x+1}) \\
                            &= \frac{1}{x+1}-\frac{1}{x-1}.
                        \end{split}
                    \end{equation*}
                However, 
                    \begin{equation*}
                        \int_{-1}^{1}\frac{-1}{(x-t)^2}dt = \frac{1}{x-1}-\frac{1}{x+1}.
                    \end{equation*}
                Thus the equality does not hold.
            \end{solution}
        \item Use Morera's theorem to give another proof of Theorem 3.5.1: If $\{f_j\}$ is a sequence of holomorphic functions on a domain $U$ and if the sequence converges uniformly on compact subsets of $U$ to a limit function $f$, then $f$ is holomorphic on $U$.
            \begin{proof}
                Given that $\{f_j\}$ is a sequence of holomorphic functions, then for each $j$ and every closed curve $\gamma$ in $U$, we have that 
                    \begin{equation*}
                        \oint_{\gamma}f_jdz = 0.
                    \end{equation*}
                Thus if $\{f_j\}$ converges to $f$ uniformly, then 
                    \begin{equation*}
                        \oint_{\gamma}fdz=\oint_{\gamma}\lim_{j\rightarrow\infty}f_jdz=\lim_{j\rightarrow\infty}\oint_{\gamma}f_jdz=0.
                    \end{equation*}
                Therefore $f$ is holomorphic on $U$.
            \end{proof}
        \item[9.] Let $\sum_{k=0}^{\infty}a_kx^k$ and $\sum_{k=0}^{\infty}b_kx^k$ be real power series which converge for $\abs{x}<1$. Suppose that $\sum_{k=0}^{\infty}a_kx^k=\sum_{k=0}^{\infty}b_kx^k$ when $x=1/2, 1/3, 1/4, \dots$. Prove that $a_k=b_k$ for all $k$.
            \begin{proof}
                
            \end{proof}
    \end{enumerate}
\end{document}