\section{}
\documentclass[12pt]{article}
\usepackage[margin=1in]{geometry} 
\usepackage{graphicx}
\usepackage{amsmath}
\usepackage{authblk}
\usepackage{titlesec}
\usepackage{amsthm}
\usepackage{amsfonts}
\usepackage{amssymb}
\usepackage{array}
\usepackage{booktabs}
\usepackage{ragged2e}
\usepackage{enumerate}
\usepackage{enumitem}
\usepackage{cleveref}
\usepackage{slashed}
\usepackage{commath}
\usepackage{lipsum}
\usepackage{colonequals}
\usepackage{addfont}
\usepackage{enumitem}
\usepackage{sectsty}
\usepackage{lastpage}
\usepackage{fancyhdr}
\usepackage{accents}
\usepackage[table,xcdraw]{xcolor}
\usepackage[inline]{enumitem}
\usepackage{tikz-cd}
\pagestyle{fancy}

\fancyhf{}
\rhead{Darcy}
\lhead{MATH 234A}
\rfoot{\thepage}
\setlength{\headheight}{10pt}

\subsectionfont{\itshape}

\newtheorem{theorem}{Theorem}[section]
\newtheorem{corollary}{Corollary}[theorem]
\newtheorem{prop}{Proposition}[section]
\newtheorem{lemma}[theorem]{Lemma}
\theoremstyle{definition}
\newtheorem{definition}{Definition}[section]
\theoremstyle{remark}
\newtheorem*{remark}{Remark}

\newcommand{\abs}[1]{\lvert #1 \rvert}
\newcommand{\bigabs}[1]{\Bigl \lvert #1 \Bigr \rvert}
\newcommand{\bigbracket}[1]{\Bigl [ #1 \Bigr ]}
\newcommand{\bigparen}[1]{\Bigl ( #1 \Bigr )}
\newcommand{\ceil}[1]{\lceil #1 \rceil}
\newcommand{\bigceil}[1]{\Bigl \lceil #1 \Bigr \rceil}
\newcommand{\floor}[1]{\lfloor #1 \rfloor}
\newcommand{\bigfloor}[1]{\Bigl \lfloor #1 \Bigr \rfloor}
\newcommand{\norm}[1]{\| #1 \|}
\newcommand{\bignorm}[1]{\Bigl \| #1 \Bigr \| #1}
\newcommand{\inner}[1]{\langle #1 \rangle}
\newcommand{\set}[1]{\{ #1 \}}
 
\makeatletter
\renewenvironment{proof}[1][\proofname]{\par
  \pushQED{\qed}%
  \normalfont \topsep6\p@\@plus6\p@\relax
  \list{}{\leftmargin=0mm
          \rightmargin=4mm
          \settowidth{\itemindent}{\itshape#1}%
          \labelwidth=\itemindent
          \parsep=0pt \listparindent=\parindent 
  }
  \item[\hskip\labelsep
        \itshape
    #1\@addpunct{.}]\ignorespaces
}{%
  \popQED\endlist\@endpefalse
}

\newenvironment{solution}[1][\bf{\textit{Solution}}]{\par
  
  \normalfont \topsep6\p@\@plus6\p@\relax
  \list{}{\leftmargin=0mm
          \rightmargin=4mm
          \settowidth{\itemindent}{\itshape#1}%
          \labelwidth=\itemindent
          \parsep=0pt \listparindent=\parindent 
  }
  \item[\hskip\labelsep
        \itshape
    #1\@addpunct{.}]\ignorespaces
}{%
  \popQED\endlist\@endpefalse
}

\let\oldproofname=\proofname
\renewcommand{\proofname}{\bf{\textit{\oldproofname}}}


\newlist{mylist}{enumerate*}{1}
\setlist[mylist]{label=(\alph*)}

\begin{document}\thispagestyle{empty}\hline

\begin{center}
	\vspace{.4cm} {\textbf { \large MATH 234A}}
\end{center}
{\textbf{Name:}\ Quin Darcy \hspace{\fill} \textbf{Due Date:} 11/25/20   \\
{ \textbf{Instructor:}}\ Dr. Bin Lu \hspace{\fill} \textbf{Assignment:} Midterm 2 \\ \hrule}

\justifying
    \begin{enumerate}[leftmargin=*]
        \item
            \begin{enumerate}
                \item Find the radius of convergence $r$ of the series•
                    \begin{equation*}
                        \sum_{n=1}^{\infty}\frac{z^n}{n^2}.
                    \end{equation*}
                    \begin{proof}
                        By the ratio test we find that•
                            \begin{equation*}
                                \lim_{n\rightarrow\infty}\frac{\abs{a_n}}{\abs{a_{n+1}}}=\lim_{n\rightarrow\infty}\frac{\frac{1}{n^2}}{\frac{1}{(n+1)^2}}=\lim_{n\rightarrow\infty}\frac{n^2+2n+1}{n^2}=1.
                            \end{equation*}
                        Thus the radius of convergence $r=1$.
                    \end{proof}
                \item Show the series converges uniformly on $D(0,s)$ with $0<s<r$.
                    \begin{proof}
                        Since the radius of convergence $r=1$, then by
                        Proposition 3.2.9, if $0<s<r$, $\sum z^n/n^2$
                        converges uniformly on $\overline{D}(0,s)$ and since
                        $D(0,s)\subseteq\overline{D}(0,s)$, then $\sum
                        z^n/n^2$ converges uniformly on $D(0,s)$.
                    \end{proof}
                \item Does the series converge uniformly on $\overline{D}(0,r)$?
                    \begin{proof}
                        Yes. Since for any $\abs{z}=1$, we have that
                        $\abs{z^n/n^2}=1/n^2$ and since $\sum1/n^2$ converges,
                        so does $\sum\abs{z^n/n^2}$. Thus the series converges
                        absolutely on the boundary and thus the series
                        converges uniformly on $\overline{D}(r,0)$.
                    \end{proof}
                \item What is the radius of convergence $r_1$ of the derivative of the series? Does it converge uniformly on $\overline{D}(0, r_1)$?
                    \begin{proof}
                        We have that if $f(z)=\sum z^n/n^2$, then by Lemma
                        3.2.10,•
                            \begin{equation*}
                                f'(z)=\sum_{n=2}^{\infty}\frac{z^{n}}{n}.
                            \end{equation*}
                        By the ratio test, we get that $r_1=1$. Letting
                        $\{a_n\}=\{1/n\}$ denote a sequence of real numbers, and
                        $\{b_n\}=\{z^{n}\}$ denote a sequence of complex numbers,
                        then $a_n$ is monotone decreasing with limit 0. And•
                            \begin{equation*}
                                \bigabs{\sum_{n=1}^Nz^n}=\bigabs{\frac{z-z^{N+1}}{1-z}}\leq\frac{2}{\abs{1-z}}
                            \end{equation*}
                            for all $N\in\mathbb{N}$. Thus by Dirichlet's test, $\sum a_nb_n=\sum z^n/n$ converges for
                        all $\abs{z}=1$ except $z=1$. Therefore the derivative
                        does not converge uniformly on $\overline{D}(r_1,0)$.
                    \end{proof}
            \end{enumerate}
        \item Suppose that \( \displaystyle
            \sum_{n=-\infty}^{\infty}a_j(z-P)^j \) converges at two points.
            \begin{enumerate}
                \item Prove that there are unique $r_1,r_2\geq 0$ (possibly
                    $\infty$) such that the series converges absolutely for
                    all $z$ with $r_1<\abs{z-P}<r_2$.
                        \begin{proof}
                            Let $z_1, z_2$ denote the points that that the
                            series converges to. If $z_1\neq z_2$, then
                            either $\abs{z_1-P}<\abs{z_2-P}$, or vice
                            versa. WLOG, assume that
                            $\abs{z_1-P}<\abs{z_2-P}$. Then given that the
                            series converges at both these points, this
                            implies that
                            $\sum_{n=-\infty}^{\infty}a_k(z_1-P)^k$ and
                            $\sum_{n=-\infty}^{\infty}a_k(z_2-P)^k$ both
                            converge. Moreover, we have that the same two
                            sums converge while $n\rightarrow+\infty$. And
                            thus for every $z$ such that
                            $\abs{z-P}<\abs{z_1-P}$ and every $z$ such
                            that $\abs{z-P}<\abs{z_2-P}$ the series
                            B
                            converges. Thus for any $z$ such that
                            $\abs{z_1-P}<\abs{z-P}<\abs{z_2-P}$, the series
                            converges. Thus, letting $r_1=\abs{z_1-P}$ and
                            $r_2=\abs{z_2-P}$, then•
                        \end{proof}
                    \item Prove that if $r_1<r'_1\leq r'_2<r_2$, then the
                        series converges uniformly and absolutely for all $z$
                        with $r'_1\leq\abs{z-P}\leq r'_2$.
                        \begin{proof}

                        \end{proof}
            \end{enumerate}
        \item[3.] Prove a version of l'Hopital's Rule for holomorphic functions. If•
            \begin{equation*}
                \lim_{z\rightarrow P}\frac{f(z)}{f(z)}
            \end{equation*}
            is an indeterminate expression for $f$ and $g$ holomorphic near $P$, then the limit may be evaluated by•
            \begin{equation*}
                \lim_{z\rightarrow P}\frac{\partial f/\partial z}{\partial
                g/\partial z}.
            \end{equation*}
            Formulate a precise result and prove it.
            B
        \item Test
    \end{enumerate}
\end{document}

