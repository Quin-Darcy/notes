\section{}
\documentclass[12pt]{article}
\usepackage[margin=1in]{geometry} 
\usepackage{graphicx}
\usepackage{amsmath}
\usepackage{authblk}
\usepackage{titlesec}
\usepackage{amsthm}
\usepackage{amsfonts}
\usepackage{amssymb}
\usepackage{array}
\usepackage{booktabs}
\usepackage{ragged2e}
\usepackage{enumerate}
\usepackage{enumitem}
\usepackage{cleveref}
\usepackage{slashed}
\usepackage{commath}
\usepackage{lipsum}
\usepackage{colonequals}
\usepackage{addfont}
\usepackage{enumitem}
\usepackage{sectsty}
\usepackage{lastpage}
\usepackage{fancyhdr}
\usepackage{accents}
\usepackage[table,xcdraw]{xcolor}
\usepackage[inline]{enumitem}
\usepackage{tikz-cd}
\pagestyle{fancy}

\fancyhf{}
\rhead{Darcy}
\lhead{MATH 234A}
\rfoot{\thepage}
\setlength{\headheight}{10pt}

\subsectionfont{\itshape}

\newtheorem{theorem}{Theorem}[section]
\newtheorem{corollary}{Corollary}[theorem]
\newtheorem{prop}{Proposition}[section]
\newtheorem{lemma}[theorem]{Lemma}
\theoremstyle{definition}
\newtheorem{definition}{Definition}[section]
\theoremstyle{remark}
\newtheorem*{remark}{Remark}
 
\makeatletter
\renewenvironment{proof}[1][\proofname]{\par
  \pushQED{\qed}%
  \normalfont \topsep6\p@\@plus6\p@\relax
  \list{}{\leftmargin=0mm
          \rightmargin=4mm
          \settowidth{\itemindent}{\itshape#1}%
          \labelwidth=\itemindent
          \parsep=0pt \listparindent=\parindent 
  }
  \item[\hskip\labelsep
        \itshape
    #1\@addpunct{.}]\ignorespaces
}{%
  \popQED\endlist\@endpefalse
}

\newenvironment{solution}[1][\bf{\textit{Solution}}]{\par
  
  \normalfont \topsep6\p@\@plus6\p@\relax
  \list{}{\leftmargin=0mm
          \rightmargin=4mm
          \settowidth{\itemindent}{\itshape#1}%
          \labelwidth=\itemindent
          \parsep=0pt \listparindent=\parindent 
  }
  \item[\hskip\labelsep
        \itshape
    #1\@addpunct{.}]\ignorespaces
}{%
  \popQED\endlist\@endpefalse
}

\let\oldproofname=\proofname
\renewcommand{\proofname}{\bf{\textit{\oldproofname}}}


\newlist{mylist}{enumerate*}{1}
\setlist[mylist]{label=(\alph*)}

\begin{document}\thispagestyle{empty}\hline

\begin{center}
	\vspace{.4cm} {\textbf { \large MATH 234A}}
\end{center}
{\textbf{Name:}\ Quin Darcy \hspace{\fill} \textbf{Due Date:} 10/13/20   \\
{ \textbf{Instructor:}}\ Dr. Bin Lu \hspace{\fill} \textbf{Assignment:} Midterm 1 \\ \hrule}

\justifying
    \begin{enumerate}[leftmargin=*]
        \item Let $F,G\colon U\rightarrow\mathbb{C}$ be $C^1$ functions, where $U\subseteq\mathbb{C}$ is open. Prove the product rules:
            \begin{equation*}
                \frac{\partial}{\partial z}(F\cdot G)=\frac{\partial F}{\partial z}\cdot G+F\cdot\frac{\partial G}{\partial z}.
            \end{equation*}
            \begin{proof}
                Let $F=u+iv$ and $G=u'+iv'$. Then starting from the left side and expanding, we get
                    \begin{equation*}
                        \begin{split}
                            \frac{\partial}{\partial z}(F\cdot G) &=\frac{1}{2}\left(\frac{\partial}{\partial x}-i\frac{\partial}{\partial y}\right)[(u+iv)(u'+iv')] \\
                            &=\frac{1}{2}\left(\frac{\partial}{\partial x}-i\frac{\partial}{\partial y}\right)[(uu'-vv')+i(uv'+u'v)] \\
                            &=\frac{1}{2}\left[\frac{\partial}{\partial x}(uu'-vv')+i\frac{\partial}{\partial x}(uv'+u'v)-i\frac{\partial}{\partial y}(uu'-vv')+\frac{\partial}{\partial y}(uv'+u'v)\right] \\
                            &=\frac{1}{2}\bigg[\frac{\partial u}{\partial x}u'+u\frac{\partial u'}{\partial x}-\frac{\partial v}{\partial x}v'-v\frac{\partial v'}{\partial x}+i\frac{\partial u}{\partial x}v'+iu\frac{\partial v'}{\partial x}+i\frac{\partial u'}{\partial x}v+iu'\frac{\partial v}{\partial x}\\ &\quad\quad\quad\quad-i\frac{\partial u}{\partial y}u'-iu\frac{\partial u'}{\partial y}+i\frac{\partial v}{\partial y}v'+iv\frac{\partial v'}{\partial y}+\frac{\partial u}{\partial y}v'+u\frac{\partial v'}{\partial y}+\frac{\partial u'}{\partial y}v+u'\frac{\partial v}{\partial y} \bigg] \\
                            &=\frac{1}{2}\bigg[\frac{\partial u}{\partial x}u'+\frac{\partial v}{\partial y}u'+i\frac{\partial u}{\partial x}v'+i\frac{\partial v}{\partial y}v'+i\frac{\partial v}{\partial x}u'-i\frac{\partial u}{\partial y}u'-\frac{\partial v}{\partial x}v'+\frac{\partial u}{\partial y}v'\bigg] \\ &\quad\quad\quad+\frac{1}{2}\bigg[\frac{\partial u'}{\partial x}u+\frac{\partial v'}{\partial y}u+i\frac{\partial u'}{\partial x}v+i\frac{\partial v'}{\partial y}v+i\frac{\partial v'}{\partial x}u-i\frac{\partial u'}{\partial y}u-\frac{\partial v'}{\partial x}v+\frac{\partial u'}{\partial y}v\bigg] \\
                            &=\frac{1}{2}\bigg[\bigg(\frac{\partial u}{\partial x}+\frac{\partial v}{\partial y}\bigg)u'+\bigg(\frac{\partial u}{\partial x}+\frac{\partial v}{\partial y}\bigg)iv'+i\bigg(\frac{\partial v}{\partial x}-\frac{\partial u}{\partial y}\bigg)u'+i\bigg(\frac{\partial v}{\partial x}-\frac{\partial u}{\partial y}\bigg)iv'\bigg] \\
                            &\quad\quad\frac{1}{2}\bigg[\bigg(\frac{\partial u'}{\partial x}+\frac{\partial v'}{\partial y}\bigg)u+\bigg(\frac{\partial u'}{\partial x}+\frac{\partial v'}{\partial y}\bigg)iv+i\bigg(\frac{\partial v'}{\partial x}-\frac{\partial u'}{\partial y}\bigg)u+i\bigg(\frac{\partial v'}{\partial x}-\frac{\partial u'}{\partial y}\bigg)iv\bigg] \\
                            &= \frac{1}{2}\bigg[\bigg(\frac{\partial u}{\partial x}+\frac{\partial v}{\partial y}\bigg)+i\bigg(\frac{\partial v}{\partial x}-\frac{\partial u}{\partial y}\bigg)\bigg](u'+iv')\\&\quad\quad\quad+\frac{1}{2}\bigg[\bigg(\frac{\partial u'}{\partial x}+\frac{\partial v'}{\partial y}\bigg)+i\bigg(\frac{\partial v'}{\partial x}-\frac{\partial u'}{\partial y}\bigg)\bigg](u+iv) \\
                            &=\frac{\partial F}{\partial z}\cdot G+F\cdot\frac{\partial G}{\partial z}.
                        \end{split}
                    \end{equation*}
            \end{proof}\newpage
        \item If $f$ and $\overline{f}$ are both holomorphic on a connected open set $U\subseteq\mathbb{C}$, then prove that $f$ is identically constant.
            \begin{proof}
                Assume $f$ and $\overline{f}$ are holomorphic on $U\subseteq\mathbb{C}$. Then both of these functions satisfy the Cauchy-Riemann equations. Letting $f=u+iv$ and $\overline{f}=u-iv$, then it follows that 
                    \begin{equation*}
                        \frac{\partial u}{\partial x}=\frac{\partial v}{\partial y},\quad\frac{\partial u}{\partial y}=-\frac{\partial v}{\partial x},\quad \frac{\partial u}{\partial x}=-\frac{\partial v}{\partial y},\quad\text{and}\quad\frac{\partial u}{\partial y}=\frac{\partial v}{\partial x}.
                    \end{equation*}
                Hence
                    \begin{equation*}
                        \frac{\partial v}{\partial y}=-\frac{\partial v}{\partial y},\quad\text{and}\quad\frac{\partial v}{\partial x}=-\frac{\partial v}{\partial x}.
                    \end{equation*}
                After integrating, we get that $v(x,y)+C_1(y)=$ and $v(x,y)+C_2(x)=0$. Hence $C_1(y)=C_2(x)$ and since $C_1'(y)=0$, then $C_1(y)=c_1$ and similarly $C_2(x)=c_2$ for constants $c_1$ and $c_2$. Therefore $v(x,y)+c=0$ implies that $v(x,y)$ is constant. A similar argument shows that this is true for $u(x,y)$ and therefore $f(z)=u+iv\in\mathbb{C}$ and is thus identically constant.
            \end{proof}
        \item Prove that if $f$ is $C^2$, holomorphic, non-vanishing, then $\log\abs{f}$ is harmonic. 
            \begin{proof}
                Let $f=u+iv$. Then since $f$ is holomorphic, then 
                    \begin{equation*}
                        \frac{\partial f}{\partial\overline{z}}=0=\frac{\partial\overline{f}}{\partial z}.
                    \end{equation*}
                Additionally, we have that $\abs{f}^2=f\cdot\overline{f}$. Now note that 
                    \begin{equation*}
                        \frac{\partial}{\partial\overline{z}}\log(f\overline{f})=\frac{1}{f\overline{f}}\bigg(\frac{\partial f}{\partial\overline{z}}\overline{f}+f\frac{\partial\overline{f}}{\partial\overline{z}}\bigg)=\frac{1}{ff'}\bigg(0\cdot\overline{f}+f\frac{\partial\overline{f}}{\partial\overline{z}}\bigg)=\frac{1}{\overline{f}}\frac{\partial\overline{f}}{\partial\overline{z}}.
                    \end{equation*}
                Next we see that 
                    \begin{equation*}
                        \begin{split}
                            \frac{\partial}{\partial z}\bigg(\frac{1}{\overline{f}}\frac{\partial\overline{f}}{\partial\overline{z}}\bigg)&=-\frac{1}{\overline{f}^2}\frac{\partial^2\overline{f}}{\partial z\partial\overline{z}}+\frac{1}{\overline{f}}\frac{\partial^2\overline{f}}{\partial z\partial\overline{z}} \\
                            &=\bigg(\frac{1}{\overline{f}}-\frac{1}{\overline{f}^2}\bigg)\frac{\partial\overline{f}}{\partial\overline{z}}\frac{\partial\overline{f}}{\partial z} \\
                            &=\bigg(\frac{1}{\overline{f}}-\frac{1}{\overline{f}^2}\bigg)\frac{\partial\overline{f}}{\partial\overline{z}}\cdot 0\\
                            &=0.
                        \end{split}
                    \end{equation*}
                Thus $4\cdot\frac{\partial^2}{\partial\overline{z}\partial z}\log(f\overline{f})=4\cdot 0$. Hence $4\cdot\frac{\partial^2}{\partial\overline{z}\partial z}\log\abs{f}=4\cdot\frac{\partial^2}{\partial\overline{z}\partial z}\frac{1}{2}\log(f\overline{f})=0$. Thus $\log\abs{f}$ is harmonic.
            \end{proof}\newpage
        \item[5.] Prove that if $U\subseteq\mathbb{C}$ is open, and if $f\colon U\rightarrow\mathbb{C}$ has a complex derivative at each point of $U$, then $f$ is continuous at each point of $U$. 
            \begin{proof}
                We want to show that for all $z_0\in U$ and all $\varepsilon>0$, there exists $\deta>0$ such that if $\abs{z-z_0}<\delta$ then $\abs{f(z)-f(z_0)}<\varepsilon$. So let $\varepsilon>0$ and $z_0\in U$. Then since $f$ has a complex derivative at $z_0$, then 
                    \begin{equation*}
                        \lim_{z\rightarrow z_0}\frac{f(z)-f(z_0)}{z-z_0}=f'(z_0)
                    \end{equation*}
                exists. Thus 
                    \begin{equation*}
                        \begin{split}
                            \lim_{z\rightarrow z_0}f(z)-f(z_0)&=\lim_{z\rightarrow z_0}\bigg(\frac{f(z)-f(z_0)}{z-z_0}(z-z_0)\bigg)\\
                            &=\lim_{z\rightarrow z_0}\bigg(\frac{f(z)-f(z_0)}{z-z_0}\bigg)\lim_{z\rightarrow z_0}(z-z_0) \\
                            &=f'(z_0)\cdot 0 \\
                            &= 0.
                        \end{split}
                    \end{equation*}
                Hence $\lim_{z\rightarrow z_0}f(z)=f(z_0)$ and $f$ is continuous at $z_0\in U$, for all $z_0\in U$.
            \end{proof}
        \item[7.] Let $u(x,y)=y^3-3x^2y$ be defined on $\abs{z}<2$. Verify that $u$ is harmonic on this disc, and find a holomorphic function $f$ on the disc such that Re$f=u$.
            \begin{proof}
                To show that $u$ is harmonic on the disc, we must show that $\Delta u=0$. This calculation yields
                    \begin{equation*}
                        \begin{split}
                            \Delta u&=\bigg(\frac{\partial^2}{\partial x^2}+\frac{\partial^2}{\partial y^2}\bigg)(y^3-3x^2y) \\
                            &=\frac{\partial^2}{\partial x^2}(y^3-3x^2y)+\frac{\partial^2}{\partial y^2}(y^3-3x^2y) \\
                            &=\frac{\partial}{\partial x}(-6xy)+\frac{\partial}{\partial y}(3y^2-3x^2) \\
                            &= -6y+6y \\
                            &= 0.
                        \end{split}
                    \end{equation*}
                Thus $u$ is harmonic.\par\hspace{4mm} Letting 
                    \begin{equation*}
                        h=-\frac{\partial u}{\partial y}=3x^2-3y^2\quad\text{and}\quad g=\frac{\partial u}{\partial x}=-6xy
                    \end{equation*}
                then by the above calculation, it follows that 
                    \begin{equation*}
                        \frac{\partial h}{\partial y}=-6y=\frac{\partial g}{\partial x}.
                    \end{equation*}
                Since $f,g$ are both $C^1$ functions on the open disc, then by Theorem 1.5.1, there exists a real valued $C^2$ function $f$ such that 
                    \begin{equation*}
                        \frac{\partial f}{\partial x}\equiv h\quad\text{and}\quad\frac{\partial f}{\partial y}\equiv g.
                    \end{equation*}
                Then for $(x,y)\in\mathbb{R}$, set 
                    \begin{equation*}
                        \begin{split}
                            f(x,y&)=\int_{a}^{y}h(t,b)dt+\int_{b}^{x}g(y,s)ds \\
                            &= \int_{a}^{y}(3t^2-3b^2)dt+\int_{b}^{x}(-6ys)ds \\
                            &= t^3-3b^2\bigg|_{a}^{y}-3ys^2\bigg|_{b}^{x} \\
                            &= y^3-3b^2y-a^3+3b^2a-3x^2y+3yb^2.
                        \end{split}
                    \end{equation*}
                Since $a,b\in\mathbb{C}$, then the real part of $f(x,y)=y^3-3x^2y=u(x,y)$. Finally, by Corollary 1.5.2, $f$ is holomorphic.
            \end{proof}
    \end{enumerate}
\end{document}