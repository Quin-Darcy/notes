\section{}
\documentclass[12pt]{article}
\usepackage[margin=1in]{geometry} 
\usepackage{graphicx}
\usepackage{amsmath}
\usepackage{authblk}
\usepackage{titlesec}
\usepackage{amsthm}
\usepackage{amsfonts}
\usepackage{amssymb}
\usepackage{array}
\usepackage{booktabs}
\usepackage{ragged2e}
\usepackage{enumerate}
\usepackage{enumitem}
\usepackage{cleveref}
\usepackage{slashed}
\usepackage{commath}
\usepackage{lipsum}
\usepackage{colonequals}
\usepackage{addfont}
\usepackage{enumitem}
\usepackage{sectsty}
\usepackage{lastpage}
\usepackage{fancyhdr}
\usepackage{accents}
\usepackage[table,xcdraw]{xcolor}
\usepackage[inline]{enumitem}
\usepackage{tikz-cd}
\pagestyle{fancy}

\fancyhf{}
\rhead{Darcy}
\lhead{MATH 161}
\rfoot{\thepage}
\setlength{\headheight}{10pt}

\subsectionfont{\itshape}

\newtheorem{theorem}{Theorem}[section]
\newtheorem{corollary}{Corollary}[theorem]
\newtheorem{prop}{Proposition}[section]
\newtheorem{lemma}[theorem]{Lemma}
\theoremstyle{definition}
\newtheorem{definition}{Definition}[section]
\theoremstyle{remark}
\newtheorem*{remark}{Remark}
 
\makeatletter
\renewenvironment{proof}[1][\proofname]{\par
  \pushQED{\qed}%
  \normalfont \topsep6\p@\@plus6\p@\relax
  \list{}{\leftmargin=0mm
          \rightmargin=4mm
          \settowidth{\itemindent}{\itshape#1}%
          \labelwidth=\itemindent
          \parsep=0pt \listparindent=\parindent 
  }
  \item[\hskip\labelsep
        \itshape
    #1\@addpunct{.}]\ignorespaces
}{%
  \popQED\endlist\@endpefalse
}

\newenvironment{solution}[1][\bf{\textit{Solution}}]{\par
  
  \normalfont \topsep6\p@\@plus6\p@\relax
  \list{}{\leftmargin=0mm
          \rightmargin=4mm
          \settowidth{\itemindent}{\itshape#1}%
          \labelwidth=\itemindent
          \parsep=0pt \listparindent=\parindent 
  }
  \item[\hskip\labelsep
        \itshape
    #1\@addpunct{.}]\ignorespaces
}{%
  \popQED\endlist\@endpefalse
}

\let\oldproofname=\proofname
\renewcommand{\proofname}{\bf{\textit{\oldproofname}}}


\newlist{mylist}{enumerate*}{1}
\setlist[mylist]{label=(\alph*)}

\begin{document}\thispagestyle{empty}\hline

\begin{center}
	\vspace{.4cm} {\textbf { \large MATH 161}}
\end{center}
{\textbf{Name:}\ Quin Darcy \hspace{\fill} \textbf{Due Date:} 9/24/20   \\
{ \textbf{Instructor:}}\ Dr. Shannon \hspace{\fill} \textbf{Assignment:} Homework 3 \\ \hrule}

\justifying
    \begin{enumerate}[leftmargin=*]
        \item Without constructing truth tables, prove that $\varphi\land\gamma$ logically implies $\theta$ iff $\varphi$ logically implies $\gamma\rightarrow \theta$.
            \begin{proof}
                By definition, if $\varphi\land\gamma\Rightarrow\theta$, then $\varphi\land\gamma\rightarrow\theta$ is a tautology. Similarly, $\varphi\Rightarrow(\gamma\rightarrow\theta)$ iff $\varphi\rightarrow(\gamma\rightarrow\theta)$ is a tautology. Note that 
                    \begin{equation*}
                        \begin{split}
                            \varphi\land\gamma\rightarrow\theta &\equiv \sim\varphi\lor\sim\gamma\lor\theta
                        \end{split}
                    \end{equation*}
                and 
                    \begin{equation*}
                        \varphi\rightarrow(\gamma\rightarrow\theta)\equiv\sim\varphi\lor(\gamma\rightarrow\theta)\equiv\sim\varphi\lor\sim\gamma\lor\theta.
                    \end{equation*}
                Hence $\varphi\land\gamma\rightarrow\theta\equiv\varphi\rightarrow(\gamma\rightarrow\theta)$. Thus if the left hand side is a tautology, then the right hand side is a tautology and vice versa.
            \end{proof}
        \item Prove that $\vdash_L(\sim\varphi\rightarrow\sim\gamma)\rightarrow((\sim\varphi\rightarrow\gamma)\rightarrow\varphi)$.
            \begin{proof}
                Let $L=\Gamma$ and consider $\Gamma\cup\{\sim\varphi\rightarrow\sim\gamma\}$. Then 
                    \begin{align*}
                        &(1) \quad\sim\varphi\rightarrow\sim\gamma &&\text{axiom} \\
                        &(2) \quad(\sim\varphi\rightarrow\sim\gamma)\rightarrow\sim\varphi &&\text{A1} \\
                        &(3) \quad\sim\varphi && \text{MP on (1) and (2)} \\
                        &(4) \quad\sim\gamma && \text{MP on (1) and (3)} \\
                        &(5) \quad\sim\varphi\land\sim\gamma && \text{Conj. of (3) and (4)} \\
                        &(6) \quad(\sim\varphi\land\sim\gamma)\lor\varphi && \text{Logically equiv. to (5)} \\
                        &(7) \quad\sim(\varphi\lor\gamma)\lor\varphi && \text{Logically equiv. to (6)} \\
                        &(8) \quad\sim(\sim\varphi\rightarrow\gamma)\lor\varphi && \text{Logically equiv. to (7)} \\
                        &(9) \quad(\sim\varphi\rightarrow\gamma)\rightarrow\varphi && \text{Logically equiv. (8)}
                    \end{align*}
                Thus $\Gamma\cup\{\sim\varphi\rightarrow\sim\gamma\}\vdash(\sim\varphi\rightarrow\gamma)\rightarrow\varphi$. Hence, by the Deduction Theorem
                    \begin{equation*}
                        \vdash_L(\sim\varphi\rightarrow\sim\gamma)\rightarrow((\sim\varphi\rightarrow\gamma)\rightarrow\varphi).
                    \end{equation*}
            \end{proof}\newpage
        \item Explain why each line below is a line of a proof, and then complete the proof so that line (8) is $\varphi$. Explain why from this proof we can conclude $\sim\sim\varphi\rightarrow\varphi$.
            \begin{solution} 
                Consider $L\cup\{\sim\sim\varphi\}$.
                    \begin{align*}
                        &(1) \quad\sim\sim\varphi &&\text{axiom} \\
                        &(2) \quad\sim\sim\varphi\rightarrow(\sim\sim\sim\sim\varphi\rightarrow\sim\sim\varphi) &&\text{A1} \\
                        &(3) \quad\sim\sim\sim\sim\varphi\rightarrow\sim\sim\varphi && \text{MP on (1) and (2)} \\
                        &(4) \quad(\sim\sim\sim\sim\varphi\rightarrow\sim\sim\varphi)\rightarrow(\sim\varphi\rightarrow\sim\sim\sim\varphi) && \text{A3} \\
                        &(5) \quad\sim\varphi\rightarrow\sim\sim\sim\varphi && \text{MP on (3) and (4)} \\
                        &(6) \quad(\sim\varphi\rightarrow\sim\sim\sim\varphi)\rightarrow(\sim\sim\varphi\rightarrow\varphi) && \text{A3} \\
                        &(7) \quad\sim\sim\varphi\rightarrow\varphi && \text{MP on (5) and (6)} \\
                        &(8) \quad\varphi && \text{MP on (1) and (7)} \\
                    \end{align*}
                Thus, by the Deduction Theorem, $\vdash_L\sim\sim\varphi\rightarrow\varphi$.
            \end{solution}
        \item Explain why each line below is a line in a proof (and then $\vdash_L \varphi\rightarrow\sim\sim\varphi$).
            \begin{solution}
                    \begin{align*}
                        &(1) \quad(\sim\sim\sim\varphi\rightarrow\sim\varphi)\rightarrow((\sim\sim\sim\varphi\rightarrow\varphi)\rightarrow\sim\sim\varphi) &&\text{A2} \\
                        &(2) \quad\sim\sim\sim\varphi\rightarrow\sim\varphi&&\text{Theorem from question (3)} \\
                        &(3) \quad(\sim\sim\sim\varphi\rightarrow\varphi)\rightarrow\sim\sim\varphi&& \text{MP on (1) and (2)} \\
                        &(4) \quad\varphi\rightarrow(\sim\sim\sim\varphi\rightarrow\varphi) && \text{A1} \\
                        &(5) \quad\sim\varphi\rightarrow\sim\sim\sim\varphi && \text{?} \\
                    \end{align*}
                Since $\sim\varphi\rightarrow\sim\sim\sim\varphi$, then letting $\gamma:=\sim\varphi$, it follows that $\gamma\rightarrow\sim\sim\gamma$.
            \end{solution}
        \item Assume that $J$ is an extension of $L$, that $J\vdash\varphi$, and that $\varphi$ logically implies $\theta$. Explain why $J\vdash\theta$.
            \begin{solution}
                Since $\varphi\rightarrow\theta$ is a tautology in $L$, then by the Adequacy Theorem for Propositional Logic, $\vdash_L\varphi\rightarrow\theta$. Moreover, with $J$ an extension of $L$, it follows by definition that $J\vash\varphi\rightarrow\theta$. By the Deduction Theorem, $J\cup\{\varphi\}\vdash\theta$. But since $\varphi$ is a sentence letter of $L$ and $J$ is an extension of $L$, then $\varphi$ is a sentence letter of $J$. Thus $J\cup\{\varphi\}=J$. Hence, $J\vdash\theta$.
            \end{solution}\newpage
        \item Assume $N(x)$ denotes ``$x$ is a real number'', $R(x)$ denotes ``$x$ is rational'', and $S(x)$ denotes the square root of $x$ -- then express the following statements symbolically. Do each first without using $\exists$, and then again without using $\forall$. Try to simplify the symbolic expressions.
            \begin{enumerate}
                \item ``Every real number is either rational or irrational''.
                    \begin{solution}\hfill\par
                        \begin{enumerate}
                            \item $\forall x\colon N(x)\rightarrow(R(x)\lor\sim R(x))$
                            \item $\exists x\colon N(x)\land(\sim(R(x)\lor\sim R(x)))$
                        \end{enumerate}
                    \end{solution}
                \item ``There exists a real number whose square root is irrational''.
                    \begin{solution}\hfill\par 
                        \begin{enumerate}
                            \item $\forall x\colon \sim N(x)\lor\sim R(S(x))$
                            \item $\exists x\colon N(x)\land R(S(x))$
                        \end{enumerate}
                    \end{solution}
            \end{enumerate}
    \end{enumerate}
\end{document}