\section{}
\documentclass[12pt]{article}
\usepackage[margin=1in]{geometry} 
\usepackage{graphicx}
\usepackage{amsmath}
\usepackage{authblk}
\usepackage{titlesec}
\usepackage{amsthm}
\usepackage{amsfonts}
\usepackage{amssymb}
\usepackage{array}
\usepackage{booktabs}
\usepackage{ragged2e}
\usepackage{enumerate}
\usepackage{enumitem}
\usepackage{cleveref}
\usepackage{slashed}
\usepackage{commath}
\usepackage{lipsum}
\usepackage{colonequals}
\usepackage{addfont}
\usepackage{enumitem}
\usepackage{sectsty}
\usepackage{lastpage}
\usepackage{fancyhdr}
\usepackage{accents}
\usepackage[table,xcdraw]{xcolor}
\usepackage[inline]{enumitem}
\usepackage{tikz-cd}
\usepackage{calligra}
\usepackage[T1]{fontenc}
\pagestyle{fancy}

\fancyhf{}
\rhead{Darcy}
\lhead{MATH 161}
\rfoot{\thepage}
\setlength{\headheight}{10pt}

\subsectionfont{\itshape}

\newtheorem{theorem}{Theorem}[section]
\newtheorem{corollary}{Corollary}[theorem]
\newtheorem{prop}{Proposition}[section]
\newtheorem{lemma}[theorem]{Lemma}
\theoremstyle{definition}
\newtheorem{definition}{Definition}[section]
\theoremstyle{remark}
\newtheorem*{remark}{Remark}
 
\makeatletter
\renewenvironment{proof}[1][\proofname]{\par
  \pushQED{\qed}%
  \normalfont \topsep6\p@\@plus6\p@\relax
  \list{}{\leftmargin=0mm
          \rightmargin=4mm
          \settowidth{\itemindent}{\itshape#1}%
          \labelwidth=\itemindent
          \parsep=0pt \listparindent=\parindent 
  }
  \item[\hskip\labelsep
        \itshape
    #1\@addpunct{.}]\ignorespaces
}{%
  \popQED\endlist\@endpefalse
}

\newenvironment{solution}[1][\bf{\textit{Solution}}]{\par
  
  \normalfont \topsep6\p@\@plus6\p@\relax
  \list{}{\leftmargin=0mm
          \rightmargin=4mm
          \settowidth{\itemindent}{\itshape#1}%
          \labelwidth=\itemindent
          \parsep=0pt \listparindent=\parindent 
  }
  \item[\hskip\labelsep
        \itshape
    #1\@addpunct{.}]\ignorespaces
}{%
  \popQED\endlist\@endpefalse
}

\let\oldproofname=\proofname
\renewcommand{\proofname}{\bf{\textit{\oldproofname}}}


\newlist{mylist}{enumerate*}{1}
\setlist[mylist]{label=(\alph*)}

\begin{document}\thispagestyle{empty}\hline

\begin{center}
	\vspace{.4cm} {\textbf { \large MATH 161}}
\end{center}
{\textbf{Name:}\ Quin Darcy \hspace{\fill} \textbf{Due Date:} 9/31/20   \\
{ \textbf{Instructor:}}\ Dr. Shannon \hspace{\fill} \textbf{Assignment:} Homework 4 \\ \hrule}

\justifying
    \begin{enumerate}[leftmargin=*]
        \item Prove that $\vdash_L(\varphi\rightarrow\gamma)\rightarrow[(\sim(\gamma\rightarrow\theta)\rightarrow\sim\varphi)\rightarrow(\varphi\rightarrow\theta)]$.
            \begin{proof}
                With two applications of the Deduction Theorem, we will let\par $\Gamma=L\cup\{\varphi\rightarrow\gamma, \sim(\gamma\rightarrow\theta)\rightarrow\sim\varphi\}$. We want to show that
                    \begin{equation*}
                        \Gamma\vdash\varphi\rightarrow\theta
                    \end{equation*}
                    \begin{align*}
                        &(1) \quad\varphi\rightarrow\gamma &&\text{axiom} \\
                        &(2) \quad\sim(\gamma\rightarrow\theta)\rightarrow\sim\varphi &&\text{axiom} \\
                        &(3) \quad\varphi\rightarrow(\gamma\rightarrow\theta) &&\text{A3 on (2)} \\
                        &(4) \quad(\varphi\rightarrow(\gamma\rightarrow\theta))\rightarrow((\varphi\rightarrow\gamma)\rightarrow(\varphi\rightarrow\theta)) &&\text{A2} \\
                        &(5) \quad(\varphi\rightarrow\gamma)\rightarrow(\varphi\rightarrow\theta) &&\text{MP on (3) and (4)} \\
                        &(6) \quad\varphi\rightarrow\theta &&\text{MP on (1) and (5)}
                    \end{align*}
            \end{proof}
        \item[4.]\par\hfill
            \begin{enumerate}
                \item Let $\theta$ be the wff: $\exists y\forall x(R(y,x)\rightarrow\forall zR(h(y,z),z))$. Define $\mathcal{I}$ by $A=$ the set of whole numbers; $R^{\mathcal{I}}$ is defined by $x\leq y$, and $h^{\mathcal{I}}$ is defined by $x+y$. Explain why $\theta$ is true in $\mathcal{I}$.
                    \begin{solution}
                        We note that $\theta$ is satisfied in $\mathcal{I}$ under two conditions: Either for some $a,b\in A$ such that $v(x)=a$ and $v(y)=b$, we have that $\sim R(y,x)$; or for some $a,b\in A$ with $v(x)=a$ and $v(y)=b$, we have that $\forall zR(h(y,z),z)$. Specifically, if $y=0$, and $x,z\in A$, then $0\leq x$ and $0+z=z\leq z$. Additionally, if $y>0$, then for all $x\in A$, $y\leq x$ is false and thus $y\leq x\rightarrow\forall z(y+z\leq z)$ is true. 
                    \end{solution}
                \item In $\mathcal{I}$, $A=\mathbb{N}\cup\{0\}$, $c^{\mathcal{I}}=0$. Define $f^{\mathcal{I}}$, $g^{\mathcal{I}}$, and $R^{\mathcal{I}}$ so that\par $\mathcal{I}\models\exists y\forall x(R(f(x,y),g(x,y))\rightarrow g(c,y)=c)$.
                    \begin{solution}
                        Let $f^{\mathcal{I}}$ be defined by $xy$, $g^{\mathcal{I}}$ by $x+y$, and $R^{\mathcal{I}}$ as $x\leq y$. Then it must be the case that for some $b\in A$ and for all $a\in A$ that $v[x\mid a,y\mid b]$ yields $\sim R(f(x,y),g(x,y))$ or that $v[x\mid a,y\mid b]$ gives $g(c,y)=c$. For $b=0$, we see that $x\cdot0\leq x+0$ and that $0+0=0$ are both true. Hence, if $y=0$, then for all $x$, we have that $xy\leq x+y\rightarrow y+0=0$. Thus $\mathcal{I}\models\exists y\forall x(R(f(x,y),g(x,y))\rightarrow g(c,y)=c)$. 
                    \end{solution}
            \end{enumerate}\newpage
        \item[5.]
            \begin{enumerate}
                \item Prove that $v$ satisfies $\theta\land\varphi$ iff $v$ satisfies $\theta$ and $v$ satisfies $\varphi$.
                    \begin{proof}
                        First we note that $\theta\land\varphi\equiv\sim(\theta\rightarrow\sim\varphi)$. Now assume that $v$ satisfies $\theta\land\varphi$. Then $v$ satisfies $\sim(\theta\rightarrow\sim\varphi)$ iff $v$ does not satisfy $\theta\rightarrow\sim\varphi$ iff $v$ does not satisfy $\sim\theta$ and $v$ does not satisfy $\sim\varphi$ iff $v$ satisfies $\theta$ and $v$ satisfies $\varphi$.
                    \end{proof}
                \item Prove that $v$ satisfies $\varphi\rightarrow\gamma$ iff ``if $v$ satisfies $\varphi$, then $v$ satisfies $\gamma$''.
                    \begin{proof}
                        Assume that $v$ satisfies $\varphi\rightarrow\gamma$ and assume $v$ satisfies $\varphi$. Since $v$ satisfies $\varphi\rightarrow\gamma$, then by definition, $v$ satisfies $\sim\varphi$ or $v$ satisfies $\gamma$. We assumed that $v$ satisfies $\varphi$ and thus $v$ does not satisfy $\sim\varphi$ and thus $v$ satisfies $\gamma$.\par\hspace{4mm} Assume that ``if $v$ satisfies $\gamma$, then $v$ satisfies $\gamma$''. Either $v$ satisfies $\varphi$ or $v$ satisfies $\sim\varphi$. If $v$ satisfies $\varphi$, then, by assumption, $v$ satisfies $\gamma$. Thus $v$ satisfies $\varphi\rightarrow\gamma$. If $v$ does not satisfy $\varphi$, then $v$ satisfies $\sim\varphi$. Thus, by definition, $v$ satisfies $\varphi\rightarrow\gamma$.
                    \end{proof}
            \end{enumerate}
        \item[7.] Determine (with explanation) if the following holds (for all wffs, $\varphi,\theta$, and interpretations $\mathcal{I}$).
            \begin{equation*}
                \mathcal{I}\models\varphi\land\gamma\text{ iff }\mathcal{I}\models\varphi\text{ and }\mathcal{I}\models\gamma.
            \end{equation*}
            \begin{proof}
                Let $v$ be any valuation and assume $\mathcal{I}\models\varphi\land\gamma$ for some interpretation $\mathcal{I}$ where $v$ satisfies $\varphi\land\gamma$ in $\mathcal{I}$. Then by 5.(a), $v$ satisfies $\varphi$ and $v$ satisfies $\gamma$. Conversely, assume that $v$ satisfies $\varphi$ and $v$ satisfies $\gamma$. Then by 5.(a), $v$ satisfies $\varphi\land\gamma$ and thus $\mathcal{I}\models\varphi$ and $\mathcal{I}\models\gamma$. 
            \end{proof}
    \end{enumerate}
\end{document}