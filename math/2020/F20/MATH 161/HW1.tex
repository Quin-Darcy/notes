\section{}
\documentclass[12pt]{article}
\usepackage[margin=1in]{geometry} 
\usepackage{graphicx}
\usepackage{amsmath}
\usepackage{authblk}
\usepackage{titlesec}
\usepackage{amsthm}
\usepackage{amsfonts}
\usepackage{amssymb}
\usepackage{array}
\usepackage{booktabs}
\usepackage{ragged2e}
\usepackage{enumerate}
\usepackage{enumitem}
\usepackage{cleveref}
\usepackage{slashed}
\usepackage{commath}
\usepackage{lipsum}
\usepackage{colonequals}
\usepackage{addfont}
\usepackage{enumitem}
\usepackage{sectsty}
\usepackage{lastpage}
\usepackage{fancyhdr}
\usepackage{accents}
\usepackage[table,xcdraw]{xcolor}
\usepackage[inline]{enumitem}
\usepackage{tikz-cd}
\pagestyle{fancy}

\fancyhf{}
\rhead{Darcy}
\lhead{MATH 161}
\rfoot{\thepage}
\setlength{\headheight}{10pt}

\subsectionfont{\itshape}

\newtheorem{theorem}{Theorem}[section]
\newtheorem{corollary}{Corollary}[theorem]
\newtheorem{prop}{Proposition}[section]
\newtheorem{lemma}[theorem]{Lemma}
\theoremstyle{definition}
\newtheorem{definition}{Definition}[section]
\theoremstyle{remark}
\newtheorem*{remark}{Remark}
 
\makeatletter
\renewenvironment{proof}[1][\proofname]{\par
  \pushQED{\qed}%
  \normalfont \topsep6\p@\@plus6\p@\relax
  \list{}{\leftmargin=0mm
          \rightmargin=4mm
          \settowidth{\itemindent}{\itshape#1}%
          \labelwidth=\itemindent
          \parsep=0pt \listparindent=\parindent 
  }
  \item[\hskip\labelsep
        \itshape
    #1\@addpunct{.}]\ignorespaces
}{%
  \popQED\endlist\@endpefalse
}

\newenvironment{solution}[1][\bf{\textit{Solution}}]{\par
  
  \normalfont \topsep6\p@\@plus6\p@\relax
  \list{}{\leftmargin=0mm
          \rightmargin=4mm
          \settowidth{\itemindent}{\itshape#1}%
          \labelwidth=\itemindent
          \parsep=0pt \listparindent=\parindent 
  }
  \item[\hskip\labelsep
        \itshape
    #1\@addpunct{.}]\ignorespaces
}{%
  \popQED\endlist\@endpefalse
}

\let\oldproofname=\proofname
\renewcommand{\proofname}{\bf{\textit{\oldproofname}}}


\newlist{mylist}{enumerate*}{1}
\setlist[mylist]{label=(\alph*)}

\begin{document}\thispagestyle{empty}\hline

\begin{center}
	\vspace{.4cm} {\textbf { \large MATH 161}}
\end{center}
{\textbf{Name:}\ Quin Darcy \hspace{\fill} \textbf{Due Date:} 9/10/20   \\
{ \textbf{Instructor:}}\ Dr. Shannon \hspace{\fill} \textbf{Assignment:} Homework 1 \\ \hrule}

\justifying

    \begin{enumerate}[leftmargin=*]
        \item Determine, with explanation, if $\varphi\rightarrow(\gamma\rightarrow\varphi)$ is a tautology.
            \begin{solution}
                We will show that the given sentence is a tautology via a truth table.
                    \begin{landscape}
                        \begin{table}[htp]
                            \centering
                            \begin{tabular}{c|c|c|c|c}
                                \cline{1-4}
                                \multicolumn{1}{|l|}{$\varphi$} & $\gamma$ & $\gamma\rightarrow\varphi$ & $\varphi\rightarrow(\gamma\rightarrow\varphi)$ &  \\ \cline{1-4}
                                \multicolumn{1}{|l|}{$T$} & $T$ & $T$ & $T$ &  \\ \cline{1-4}
                                \multicolumn{1}{|l|}{$T$} & $F$ & $T$ & $T$ &  \\ \cline{1-4}
                                \multicolumn{1}{|l|}{$F$} & $T$ & $F$ & $T$ &  \\ \cline{1-4}
                                \multicolumn{1}{|l|}{$F$} & $F$ & $T$ & $T$ &  \\ \cline{1-4}
                            \end{tabular}
                        \end{table}
                    \end{landscape}\hfill\par
                Each cell in the right most column containing all true values indicates the given sentence is a tautology.
            \end{solution}
        \item Using induction, prove that for all positive integers, $n\geq 2$,
            \begin{equation}
                \sim(\varphi_1\land\varphi_2\land\cdots\land\varphi_n)=\sim\varphi_1\lor\sim\varphi_2\lor\cdots\lor\sim\varphi_n.
            \end{equation}
            \begin{proof}
                By PMI. Let $P(n):=\sim(\varphi_1\land\varphi_2\land\cdots\land\varphi_n)=\sim\varphi_1\lor\sim\varphi_2\lor\cdots\lor\sim\varphi_n$. Our base case is for $n=2$. We want to show the equality of (1) holds when $n=2$ and will do so via truth table.
                    \begin{landscape}
                        \begin{table}[htp]
                            \centering
                            \begin{tabular}{c|c|c|c|c|c|c|}
                                \cline{1-7}
                                \multicolumn{1}{|l|}{$\varphi_1$} & $\varphi_2$ & $\sim\varphi_1$ & $\sim\varphi_2$ & $\varphi_1\land\varphi_2$ & $\sim(\varphi_1\land\varphi_2)$ & $\sim\varphi_1\lor\sim\varphi_2$   \\ \cline{1-7}
                                \multicolumn{1}{|c|}{$T$} & $T$ & $F$ & $F$ & $T$ & $F$ & $F$  \\ \cline{1-7}
                                \multicolumn{1}{|c|}{$T$} & $F$ & $F$ & $T$ & $F$ & $T$ & $T$  \\ \cline{1-7}
                                \multicolumn{1}{|c|}{$F$} & $T$ & $T$ & $F$ & $F$ & $T$ & $T$  \\ \cline{1-7}
                                \multicolumn{1}{|c|}{$F$} & $F$ & $T$ & $T$ & $F$ & $T$ & $T$  \\ \cline{1-7}
                            \end{tabular}
                        \end{table}
                    \end{landscape}\hfill\par
                Seeing that columns 6 and 7 in the above table are equal, this proves the $P(2)$ holds.\par\hspace{4mm} Now assume $P(k)$ holds for some $k>2$. Consider
                    \begin{equation*}
                        \begin{split}
                            \sim(\varphi_1\land\cdots\land\varphi_k\land\varphi_{k+1}) &\equiv \sim\big((\varphi_1\land\cdots\land\varphi_k)\land\varphi_{k+1}\big) \\
                            &\equiv \sim(\alpha\land\varphi_{k+1}) \\
                            &\equiv (\sim\alpha)\lor(\sim\varphi_{k+1}) \\
                            &\equiv\sim(\varphi_1\land\cdots\land\varphi_k)\lor(\sim\varphi_{k+1}) \\
                            &\equiv (\sim\varphi_1)\lor\cdots\lor(\sim\varphi_k)\lor(\sim\varphi_{k+1}).
                        \end{split}
                    \end{equation*}
                Thus $P(k+1)$ is true. Therefore, $P(n)$ is true for all $n\geq 2$.
            \end{proof}\newpage
        \item[5.] For each of the following wffs, find (and show your process) logically equivalent wffs that contain only the logical symbols $\sim$, and $\lor$.
            \begin{enumerate}
                \item $p\rightarrow(q\rightarrow r)$.
                    \begin{solution}
                        We claim $p\rightarrow(q\rightarrow r)\equiv (\sim p)\lor\big((\sim q)\lor r\big)$. To show this, consider the following truth table:
                            \begin{landscape}
                                \begin{table}[htp]
                                    \centering
                                    \begin{tabular}{c|c|c|c|c|}
                                        \cline{1-5}
                                        \multicolumn{1}{|l|}{$p$} & $q$ & $\sim p$ & $\sim p\lor q$ & $p\rightarrow q$  \\ \cline{1-5}
                                        \multicolumn{1}{|c|}{$T$} & $T$ & $F$ & $T$ & $T$ \\ \cline{1-5}
                                        \multicolumn{1}{|c|}{$T$} & $F$ & $F$ & $F$ & $F$ \\ \cline{1-5}
                                        \multicolumn{1}{|c|}{$F$} & $T$ & $T$ & $T$ & $T$ \\ \cline{1-5}
                                        \multicolumn{1}{|c|}{$F$} & $F$ & $T$ & $T$ & $T$ \\ \cline{1-5}
                                    \end{tabular}
                                \end{table}
                            \end{landscape}\hfill\par
                        Hence, $p\rightarrow q\equiv \sim p\lor q$. Thus, 
                            \begin{equation*}
                                p\rightarrow(q\rightarrow r)\equiv (\sim p)\lor(q\rightarrow r)\equiv (\sim p)\lor\big((\sim q)\lor r\big),
                            \end{equation*}
                        as desired.
                    \end{solution}
                \item $p\leftrightarrow q$.
                    \begin{solution}
                        We begin by noting $p\leftrightarrow q\equiv (p\rightarrow q)\land(q\rightarrow p)$. Second, by part (a), it follows that 
                            \begin{equation*}
                                \begin{split}
                                    (p\rightarrow q)\land(q\rightarrow p)&\equiv\big((\sim p)\lor q\big)\land\big((\sim q)\lor p\big) \\
                                    &\equiv \big[\big((\sim p)\lor q\big)\land(\sim q)\big]\lor \big[\big((\sim p)\lor q\big)\land p\big] \\
                                    &\equiv\big[(\sim p\land \sim q)\lor (q\land \sim q)\big]\lor\big[(\sim p\land p)\lor(q\land p)\big] \\
                                    &\equiv (\sim p\land \sim q)\lor (q\land p) *\\
                                    &\equiv \sim(p\lor q)\lor(\sim q\lor\sim p).
                                \end{split}
                            \end{equation*}
                        Note that $*$ holds since the statement above it contained $(q\land \sim q)$ and $(p\land \sim p)$, which are both contradictions, and each were connected to their respective sentences by a disjunction; thus there removal results in an equivalent statement (since $T \equiv T\lor F$ and $F\equiv F\lor F$).
                    \end{solution}
            \end{enumerate}\newpage
        \item[6.] Write $(p\rightarrow(q\rightarrow r))\land((p\rightarrow q)\rightarrow r)$ in disjunctive normal form.
            \begin{solution}
                Following the suggested instructions in problem (6), the resulting truth table is
                    \begin{table}[htp]
                        \centering
                        \resizebox{\textwidth}{!}{%
                        \begin{tabular}{|c|c|c|c|c|c|c|c|c|}
                            \hline
                            $p$ & $q$ & $r$ & $q\rightarrow r$ & $p\rightarrow q$ & $p\rightarrow(q\rightarrow r)$ & $(p\rightarrow q)\rightarrow r$ & $(p\rightarrow(q\rightarrow r))\land((p\rightarrow q)\rightarrow r)$ &  \\ \hline
                            $T$ & $T$ & $T$ & $T$ & $T$ & $T$ & $T$ & $T$ & $p\land q\land r$ \\ \hline
                            $T$ & $T$ & $F$ & $F$ & $T$ & $F$ & $F$ & $F$ &  \\ \hline
                            $T$ & $F$ & $T$ & $T$ & $F$ & $T$ & $T$ & $T$ & $p\land\sim q\land r$ \\ \hline
                            $T$ & $F$ & $F$ & $T$ & $F$ & $F$ & $T$ & $F$ &  \\ \hline
                            $F$ & $T$ & $T$ & $T$ & $T$ & $T$ & $T$ & $T$ & $\sim p\land q\land r$ \\ \hline
                            $F$ & $T$ & $F$ & $F$ & $T$ & $T$ & $F$ & $F$ &  \\ \hline
                            $F$ & $F$ & $T$ & $T$ & $T$ & $T$ & $T$ & $T$ & $\sim p\land \sim q\land r$ \\ \hline
                            $F$ & $F$ & $F$ & $T$ & $T$ & $T$ & $F$ & $F$ &  \\ \hline
                        \end{tabular}%
                        }
                    \end{table}\hfill\par 
                From this table it follows that
                    \begin{equation*}
                        (p\rightarrow(q\rightarrow r))\land((p\rightarrow q)\rightarrow r)\equiv (p\land q\land r)\lor(p\land\sim q\land r)\lor(\sim p\land\sim q\land r).
                    \end{equation*}
            \end{solution}
    \end{enumerate}
\end{document}