% This is in AMSLaTeX.
\documentclass[10pt]{amsart}
\usepackage{amssymb}
% \renewcommand{\baselinestretch}{1.27}

\renewcommand{\S}{\subset}
\newcommand{\M}{\setminus}
\newcommand{\E}{\varnothing}
\renewcommand{\O}{\overline}
\newcommand{\I}{\infty}
\newcommand{\andeqn}{\qquad {\mbox{and}} \qquad}

\newenvironment{pff}{{\emph{Proof:}}}{\end{proof}}

\newcommand{\beq}{\begin{equation}}
\newcommand{\eeq}{\end{equation}}
\newcommand{\beqr}{\begin{eqnarray*}}
\newcommand{\eeqr}{\end{eqnarray*}}
\newcommand{\limi}[1]{\lim_{{#1} \to \infty}}

\newcommand{\af}{\alpha}
\newcommand{\bt}{\beta}
\newcommand{\gm}{\gamma}
\newcommand{\dt}{\delta}
\newcommand{\ep}{\varepsilon}
\newcommand{\zt}{\zeta}
\newcommand{\et}{\eta}
\newcommand{\ch}{\chi}
\newcommand{\io}{\iota}
\newcommand{\te}{\theta}
\newcommand{\ld}{\lambda}
\newcommand{\sm}{\sigma}
\newcommand{\kp}{\kappa}
\newcommand{\ph}{\varphi}
\newcommand{\ps}{\psi}
\newcommand{\rh}{\rho}
\newcommand{\om}{\omega}
\newcommand{\ta}{\tau}

\newcommand{\Gm}{\Gamma}
\newcommand{\Dt}{\Delta}
\newcommand{\Et}{\Eta}
\newcommand{\Th}{\Theta}
\newcommand{\Ld}{\Lambda}
\newcommand{\Sm}{\Sigma}
\newcommand{\Ph}{\Phi}
\newcommand{\Ps}{\Psi}
\newcommand{\Om}{\Omega}

\newcommand{\Q}{{\mathbb{Q}}}
\newcommand{\Z}{{\mathbb{Z}}}
\newcommand{\R}{{\mathbb{R}}}
\newcommand{\C}{{\mathbb{C}}}
% \newcommand{\N}{{\mathbb{N}}}
\newcommand{\N}{{\mathbb{Z}}_{> 0}}
\newcommand{\Nz}{{\mathbb{Z}}_{\geq 0}}

\pagenumbering{arabic}

\newcommand{\id}{{\mathrm{id}}}
\newcommand{\sint}{{\mathrm{int}}}
\newcommand{\diam}{{\mathrm{diam}}}
\newcommand{\dist}{{\mathrm{dist}}}

\newcommand{\A}{\qquad {\mbox{and}} \qquad}

\newcommand{\ct}{continuous}
\newcommand{\nbhd}{neighborhood}
\newcommand{\cpt}{compact}
\newcommand{\wolog}{without loss of generality}
\newcommand{\Wolog}{Without loss of generality}
\newcommand{\Tfae}{The following are equivalent}
\newcommand{\tfae}{the following are equivalent}
\newcommand{\ifo}{if and only if}
\newcommand{\ms}{metric space}
\newcommand{\cms}{compact metric space}
\newcommand{\cfn}{continuous function}

\newcommand{\wrt}{with respect to}

\title{Math 413 [513] (Phillips) Solutions to Homework 4}
\date{22 October 2001}

\begin{document}

% \setcounter{section}{-1}

\maketitle

Generally, a ``solution'' is something that would be acceptable
if turned in in the form presented here, although the solutions
given are often close to minimal in this respect.
A ``solution (sketch)'' is too sketchy to be considered a
complete solution if turned in;
varying amounts of detail would need to be filled in.

\vspace{3ex}

\noindent
{\textbf{Problem 3.1:}}
Prove that if $(s_n)_{n \in \N}$ converges,
then $( |s_n| )_{n \in \N}$ converges.
Is the converse true?

\vspace{1ex}

\begin{proof}[Solution (sketch)]
Use the inequality
$\left| \rule{0em}{2ex} |s_n| - |s| \right| \le |s_n - s|$ and
the definition of the limit.
The converse is false.
Take $s_n = (-1)^n$ for $n \in \N$.
(This requires proof, of course.)
\end{proof}

\vspace{2ex}

\noindent
{\textbf{Problem 3.2:}}
Calculate $\lim_{n \to \I} \left( \sqrt{n^2+1} - n \right)$.

\vspace{1ex}

\begin{proof}[Solution (sketch)]
\[
\sqrt{n^2 + 1} - n = \frac{n}{\sqrt{n^2 + 1} + n}
    = \frac{1}{\sqrt{1 + \frac{1}{n^2} } + 1} \to \frac{1}{2}.
\]
(Of course, the last step requires proof.
You will need to show that if $(y_n)_{n \in \N}$
is a sequence in $[0, \I)$
and $y_n \to 1$,
then $\sqrt{y_n} \to 1$.)
\end{proof}

\vspace{2ex}

\noindent
{\textbf{Problem 3.3:}}
Let $s_1 = \sqrt{2}$, and recursively define
\[
s_{n + 1} = \sqrt{2 + \sqrt{s_n}}
\]
for $n \in \N$.
Prove that $(s_n)_{n \in \N}$ converges,
and that $s_n < 2$ for all $n \in \N$.

\vspace{1ex}

\begin{proof}[Solution]
We know from earlier work that $x \mapsto \sqrt{x}$
is strictly increasing on $[0, \I)$,
and we use this fact without comment below.

By induction, it is immediate that $s_n > 0$ for all $n$, so that
$s_{n + 1}$ is always defined.

Next, we show by induction that $s_n < 2$ for all $n \in \N$.
This is clear for $n = 1$.
The computation for the induction step is
\[
s_{n + 1}
 = \sqrt{2 + \sqrt{s_n}}
 \leq \sqrt{2 + \sqrt{2}}
 < \sqrt{2 + 2}
 = 2.
\]
We have proved that $s_n < 2$ for all $n \in \N$.

We now claim that $(s_n)_{n \in \N}$ is strictly increasing.
We prove by induction on $n \in \N$ that
$s_{n+1} - s_n \geq 0$.
The base case is
\[
s_2
 = \sqrt{2 + \sqrt{s_1}}
 = \sqrt{2 + \sqrt{\sqrt{2}}}
 > \sqrt{2}
 = s_1.
\]
For the induction step,
let $n \in \N$ satisfy $n \geq 2$,
and assume $s_{n} - s_{n - 1} > 0$.

Then,
using the induction hypothesis at the last step
to see that $\sqrt{s_n} - \sqrt{s_{n - 1}} > 0$,
%
\begin{align*}
s_{n+1} - s_n
& = \sqrt{2 + \sqrt{s_n}} - \sqrt{2 + \sqrt{s_{n - 1}}}
\\
& = \left( \sqrt{2 + \sqrt{s_n}} - \sqrt{2 + \sqrt{s_{n - 1}}} \right)
    \left( \frac{\sqrt{2 + \sqrt{s_n}} + \sqrt{2 + \sqrt{s_{n - 1}}}}{
         \sqrt{2 + \sqrt{s_n}} + \sqrt{2 + \sqrt{s_{n - 1}}}} \right)
\\
& = \frac{\sqrt{s_n} - \sqrt{s_{n - 1}}}{\sqrt{2 + \sqrt{s_n}} +
                      \sqrt{2 + \sqrt{s_{n - 1}}}} > 0.
\end{align*}
%
The induction is complete,
and the claim is proved.

Convergence of $(s_n)_{n \in \N}$
now follows from Theorem~3.14
of Rudin's book
(bounded monotone sequences in~$\R$ converge).
\end{proof}

\vspace{2ex}

\noindent
{\textbf{Problem 3.4:}}
Let $s_1 = 0$, and recursively define
\[
s_{n + 1} = \left\{ \begin{array}{ll}
  {\textstyle{\frac{1}{2}}} + s_n & \hspace{3em} \mbox{$n$ is even} \\
  {\textstyle{\frac{1}{2}}} s_n & \hspace{3em} \mbox{$n$ is odd}
                                              \end{array} \right..   % }
\]
for $n \in \N$.
Find $\limsup_{n \to \I} s_n$ and $\liminf_{n \to \I} s_n$.

\vspace{1ex}

\begin{proof}[Solution (sketch)]
Use induction to show that
\[
s_{2 m} = \frac{2^{m - 1} - 1}{2^m} \andeqn
s_{2m + 1} = \frac{2^m - 1}{2^m}.
\]
It follows that
\[
\limsup_{n \to \I} s_n = 1 \andeqn
 \liminf_{n \to \I} s_n = {\textstyle{\frac{1}{2}}}.
\]
This completes the sketch.
\end{proof}

\vspace{2ex}

\noindent
{\textbf{Problem~3.5:}}
Let $(a_n)_{n \in \N}$ and $(b_n)_{n \in \N}$ be sequences in $\R$.
Prove that
\[
\limsup_{n \to \I} (a_n + b_n)
   \leq \limsup_{n \to \I} a_n + \limsup_{n \to \I} b_n,
\]
provided that the right hand side is defined, that is, not of the
form $\I - \I$ or $- \I + \I$.

\vspace{1ex}

Four solutions are presented or sketched.
The first is what I presume to be the solution Rudin intended.
The second is a variation of the first, which minimizes the
amount of work that must be done in different cases.
The third shows what must be done if one wants to work directly
from Rudin's definition.
The fourth is the ``traditional'' proof of the result, and proceeds
via the traditional definition.

\vspace{1ex}

\begin{proof}[Solution~1 (sketch)]
We give a complete solution for the case
\[
\limsup_{n \to \I} a_n \in (-\I, \I)
 \A \limsup_{n \to \I} b_n \in (-\I, \I).
\]
One needs to consider several other cases,
but the basic method is the same.

Define
\[
a = \limsup_{n \to \I} a_n \A b = \limsup_{n \to \I} b_n.
\]
Let $c > a + b$.
We show that $c$ is not a subsequential limit
of $(a_n + b_n)_{n \in \N}$.

Let $\ep = \frac{1}{3} (c - a - b) > 0$.
Use Theorem~3.17~(b) of Rudin to choose $N_1 \in \N$ such that
$n \geq N_1$ implies $a_n < a + \ep$, and also to
choose $N_2 \in \N$ such that $n \geq N_2$ implies $b_n < b + \ep$.
For $n \geq \max (N_1, N_2)$, we then have
$a_n + b_n < a + b + 2 \ep$.
It follows that every subsequential limit $l$ of
$(a_n + b_n)_{n \in \N}$ satisfies $l \leq a + b + 2 \ep$.
Since $c = a + b + 3 \ep > a + b + 2 \ep$, it follows that
$c$ is not a subsequential limit of $(a_n + b_n)_{n \in \N}$.

We conclude that $a + b$ is an upper bound for the set of
subsequential limits of $(a_n + b_n)_{n \in \N}$.
Therefore $\limsup_{n \to \I} (a_n + b_n) \leq a + b$.
\end{proof}

\vspace{1ex}

\begin{proof}[Solution~2]
This solution is a variation of Solution~1, designed to handle
all cases at once.
(You will see, though, that the case breakdown can't be avoided
entirely.)

As in  Solution~1, define
\[
a = \limsup_{n \to \I} a_n \A b = \limsup_{n \to \I} b_n,
\]
and let $c > a + b$.
We show that $c$ is not a subsequential limit
of $(a_n + b_n)_{n \in \N}$.

We first claim that there are $r, \, s, \, t \in \R$ such that
\[
a < r, \qquad b < s, \qquad t > 0, \A r + s + t \leq c.
\]

We prove the claim.
If $a = \I$ or $b = \I$, the claim is vacuous,
since no such $c$ can exist.
Next, suppose $a$ and $b$ are finite.
If $c = \I$, then
\[
r = a + 1, \qquad s = b + 1, \A t = 1
\]
will do.
Otherwise, let $\ep = \frac{1}{3} (c - a - b) > 0$, and take
\[
r = a + \ep, \qquad s = b + \ep, \A t = \ep.
\]
Finally, suppose at least one of $a$ and $b$ is $-\I$,
but neither is $\I$.
Exchanging the sequences if necessary, assume that $a = - \I$.
Choose any $s > b$, choose any $t > 0$, and set $r = c - s - t$,
which is certainly greater than $- \I$.
The claim is proved.

Having $r$, $s$, and $t$,
use Theorem~3.17~(b) of Rudin to choose $N_1 \in \N$ such that
$n \geq N_1$ implies $a_n < r$, and also to
choose $N_2 \in \N$ such that $n \geq N_2$ implies $b_n < s$.
For $n \geq \max (N_1, N_2)$, we then have
$a_n + b_n < r + s$.
It follows that every subsequential limit $l$ of
$(a_n + b_n)_{n \in \N}$ satisfies $l \leq r + s$.
Since $c = r + s + t > r + s$, it follows that
$c$ is not a subsequential limit of $(a_n + b_n)_{n \in \N}$.

We conclude that $a + b$ is an upper bound for the set of
subsequential limits of $(a_n + b_n)_{n \in \N}$.
Therefore $\limsup_{n \to \I} (a_n + b_n) \leq a + b$.
\end{proof}

\begin{proof}[Solution~3 (sketch)]
We only consider the case that both
$a = \limsup_{n \to \I} a_n$
and $b = \limsup_{n \to \I} b_n$ are finite.
Let $s = \limsup_{n \to \I} (a_n + b_n)_{n \in \N}$.
Then there is a subsequence $(a_{k (n)} + b_{k (n)})_{n \in \N}$
of $(a_{n} + b_{n})_{n \in \N}$ which converges to $s$.

We claim that $(a_{k (n)})_{n \in \N}$ is bounded.
To prove the claim,
observe that this sequence is bounded above
because $\limsup_{n \to \I} a_n$ is finite.
It is bounded below
because $(a_{k (n)} + b_{k (n)})_{n \in \N}$ is convergent,
hence bounded,
and $(b_{k (n)})_{n \in \N}$ is bounded above
(since $\limsup_{n \to \I} b_n$ is finite).
The claim is proved.

By the claim, there is a subsequence $(a_{l (n)})_{n \in \N}$
of $(a_{k (n)})_{n \in \N}$ which converges.
(That is, there is a strictly increasing function
$n \to r (n)$ such that
the sequence $(a_{k \circ r (n)})_{n \in \N}$ converges, and we let
$l = k \circ r  \colon \N \to \N$.
If we used traditional subsequence notation, we would have
the subsequence $j \mapsto a_{n_{k_{j}}}$ at this point.)
Let $c = \lim_{n \to \I} a_{l (n)}$.
By similar reasoning to that given above,
the sequence $(b_{l (n)})_{n \in \N}$
is bounded.
Therefore it has a convergent subsequence,
say $(b_{m (n)})_{n \in \N}$.
(With traditional subsequence notation, we would now have
the subsequence $i \mapsto a_{n_{k_{j_i}}}$.
You can see why I don't like traditional notation.)
Let $d = \lim_{n \to \I} b_{m (n)}$.
Since $(a_{m (n)})_{n \in \N}$
is a subsequence of $(a_{l (n)})_{n \in \N}$,
we still have $\lim_{n \to \I} a_{m (n)} = c$.
So $\lim_{n \to \I} ( a_{m (n)} + b_{m (n)} ) = c + d$.
But also $(a_{m (n)} + b_{m (n)})_{n \in \N}$ is a subsequence of
$(a_{k (n)} + b_{k (n)})_{n \in \N}$, and so converges to $s$.
Therefore $s = c + d$.
We have $c \leq a$ and $d \leq b$ by the definition of
$\limsup_{n \to \I} a_n$ and $\limsup_{n \to \I} b_n$,
giving the result.
\end{proof}

\begin{proof}[Solution~4 (sketch)]
First prove that
\[
\limsup_{n \to \I} x_n = \lim_{n \to \I} \sup_{k \geq n} x_k.
\]
(We will probably prove this result in class; otherwise, see
Problem~A in Homework~6.
This formula is closer to the usual definition of
$\limsup_{n \to \I} x_n$, which is
\[
\limsup_{n \to \I} x_n = \inf_{n \in \N} \sup_{k \geq n} x_k,
\]
using a limit instead of an infimum.)

Then prove that
\[
\sup_{k \geq n} (a_k + b_k)
 \leq \sup_{k \geq n} a_k + \sup_{k \geq n} b_k,
\]
provided that the right hand side is defined.
(For example, if both terms on the right are finite, then the right hand
side is clearly an upper bound for $\{ a_k + b_k \colon k \geq n \}$.)
Now take limits to get the result.
\end{proof}

\vspace{1ex}

\noindent
{\emph{Remark:}}
It is quite possible to have
\[
\limsup_{n \to \I} (a_n + b_n)
   < \limsup_{n \to \I} a_n + \limsup_{n \to \I} b_n.
\]

\vspace{2ex}

\noindent
{\textbf{Problem 3.6:}}
Investigate the convergence or divergence of the following series.

(Note: I have supplied lower limits of summation, which I have chosen
for maximum convenience.
Of course, convergence is independent of the lower limit, provided none
of the individual terms is infinite.)

(a)
\[
\sum_{n = 0}^{\I} \left( \sqrt{n + 1} - \sqrt{n} \right).
\]

\begin{proof}[Solution]
The $n$-th partial sum is
\[
\left( \sqrt{n + 1} - \sqrt{n} \right)
  + \left( \sqrt{n} - \sqrt{n-1} \right) + \cdots
  + \left( \sqrt{1} - \sqrt{0} \right) = \sqrt{n + 1}.
\]
We have $\limi{n} \sqrt{n} = \I$, so $\limi{n} \sqrt{n + 1} = \I$.
Therefore the series diverges.
\end{proof}


\noindent
{\emph{Remark:}}
This sort of series is known as a telescoping series.
The more interesting cases of telescoping series
are the ones that converge.

\begin{proof}[Alternate solution]
We calculate:
\begin{align*}
\sqrt{n + 1} - \sqrt{n}
  & = \left( \sqrt{n + 1} - \sqrt{n} \right) \cdot
         \left( \frac{\sqrt{n + 1}
               + \sqrt{n}}{\sqrt{n + 1} + \sqrt{n}} \right)
\\
  & = \frac{1}{\sqrt{n + 1} + \sqrt{n}}
    \geq \frac{1}{\sqrt{n + 1} + \sqrt{n + 1}}
     = \frac{1}{2 \sqrt{n + 1}}.
\end{align*}
Now $\sum_{n = 1}^{\I} \frac{1}{\sqrt{n}}$ diverges
by Theorem~3.28 of Rudin's book.
Therefore $\sum_{n = 1}^{\I} \frac{1}{2 \sqrt{n}}$ diverges, and hence
so does $\sum_{n = 0}^{\I} \frac{1}{2 \sqrt{n + 1}}$.
So the Comparison Test implies that
\[
\sum_{n = 0}^{\I} \left( \sqrt{n + 1} - \sqrt{n} \right)
\]
diverges.
\end{proof}


(b) 
\[
\sum_{n = 1}^{\I} \frac{\sqrt{n + 1} - \sqrt{n}}{n}.
\]

\begin{proof}[Solution]
For $n \in \N$ we have
\[
0 \leq \frac{\sqrt{n + 1} - \sqrt{n}}{n}
  = \frac{1}{n \left( \sqrt{n + 1} + \sqrt{n} \right)}
   \leq \frac{1}{n^{3/2}}.
\]
Since $\sum_{n = 1}^{\I} 1 / n^{3/2}$ converges
(by Theorem~3.28 of Rudin's book),
our series converges by the Comparison Test.
\end{proof}


(c) 
\[
\sum_{n = 1}^{\I} \left( \sqrt[n]{n} - 1 \right)^n.
\]

\begin{proof}[Solution]
We use the Root Test (Theorem~3.33 of Rudin).
With $a_n = \left( \sqrt[n]{n} - 1 \right)^n$, we have
$\sqrt[n]{a_n} = \sqrt[n]{n} - 1$.
Theorem~3.20(c) of Rudin implies that $\limi{n} \sqrt[n]{n} = 1$.
Therefore $\limi{n} \sqrt[n]{a_n} = 0$.
Since $\limi{n} \sqrt[n]{a_n} < 1$, convergence follows.
\end{proof}

\begin{proof}[Alternate solution (sketch)]
Let $x_n = \sqrt[n]{n} - 1$,
so that
\[
\left( \sqrt[n]{n} - 1 \right)^n = x_n^n  \A  (1 + x_n)^n = n.
\]
The binomial formula implies that 
\[
n = (1 + x_n)^n = 1 + n x_n + \frac{n (n - 1)}{2} \cdot x_n^2 + \cdots
   \geq \frac{n (n - 1)}{2} \cdot x_n^2,
\]
from which it follows that
\[
0 \leq x_n \leq \sqrt{ \frac{2}{n - 1}}
\]
for $n \geq 2$.
Hence, for $n \geq 4$,
\[
\left( \sqrt[n]{n} - 1 \right)^n
  = x_n^n \leq \left( \frac{2}{n - 1} \right)^{n/2}
   \leq \left( \left( {\textstyle{\frac{2}{3} }} \right)^{1/2} \right)^n.
\]
Since $\left( {\textstyle{\frac{2}{3} }} \right)^{1/2} < 1$,
the series converges by the Comparison Test.
\end{proof}


(d) 
\[
\sum_{n = 0}^{\I} \frac{1}{1 + z^n},
\]
for $z \in \C$ arbitrary.

\begin{proof}[Solution]
We show that the series converges \ifo\  $| z | > 1$.

If there are an
odd integer $k$ and  an even integer $l$
such that $z = \exp (2 \pi i k / l)$,
then $z^n = -1$ for infinitely
many values of $n$, so that infinitely many of the terms of the
series are undefined.
Convergence is therefore clearly impossible.

In all other cases with $|z| \leq 1$, we have
\[
|1 + z^n| \leq 1 + |z^n| \leq 1 + 1 = 2,
\]
which implies that
\[
\left| \frac{1}{1 + z^n} \right| \geq \frac{1}{2}.
\]
The terms thus don't converge to $0$,
and again the series diverges.

Now let $|z| >1$.
Then $|1 + z^n| \geq | z^n | - 1 = |z|^n - 1$.
Choose $N$ such that if $n \geq N$ then $| z |^n > 2$.
For such $n$, we have
\[
\frac{| z |^n}{2} > 1,
\]
whence
\[
|1 + z^n| \geq |z|^n - 1 = \frac{| z |^n}{2}
   + \left( \frac{| z |^n}{2} - 1 \right) > \frac{| z |^n}{2}.
\]
So
\[
\left| \frac{1}{1 + z^n} \right| \leq 2 \cdot \frac{1}{| z |^n}
\]
for all $n \geq N$.
Since $| z | > 1$, the Comparison Test implies that
\[
\sum_{n = 0}^{\I} \frac{1}{1 + z^n}
\]
converges.
\end{proof}

\vspace{2ex}
\noindent
{\textbf{Problem 3.10:}}
Suppose the coefficients of the power series
$\sum_{n = 0}^{\I} a_n z^n$ are integers,
infinitely many of which are nonzero.
Prove that the radius of convergence is at most $1$.

\begin{proof}[Solution~1]
For infinitely many $n$, the numbers $a_n$ are nonzero integers,
and therefore satisfy $| a_n | \geq 1$.
So, if $|z| \geq 1$, then infinitely many of the terms $a_n z^n$
have absolute value $|a_n z^n| \geq | a_n | \geq 1$,
and the terms of the series
$\sum_{n = 0}^{\I} a_n z^n$ don't approach zero.
This shows that $\sum_{n = 0}^{\I} a_n z^n$ diverges for $|z| \geq 1$,
and therefore that its radius of convergence is at most $1$.
\end{proof}

\begin{proof}[Solution~2 (sketch)]
There is a subsequence $(a_{k (n)})_{n \in \N}$
of $(a_n)_{n \in \N}$ such that
$| a_{k (n)} | \geq 1$ for all $n$.
So $\sqrt[k (n)]{| a_{k (n)} |} \geq 1$ for all $n$,
whence $\limsup_{n \to \I} \sqrt[n]{| a_n |} \geq 1$.
\end{proof}


\noindent
{\emph{Remarks:}}
(1) It is quite possible that infinitely many of the $a_n$ are zero,
so that $\liminf_{n \to \I} \sqrt[n]{| a_n |}$ could be zero.
For example, we could have $a_n = 0$ for all odd $n$.

(2) In Solution~2, the expression $\sqrt[n]{| a_{k (n)} |}$, and
its possible limit as $n \to \I$, have no relation to the
radius of convergence.

\vspace{2ex}

\noindent
{\textbf{Problem 3.21:}}
Prove the following analog of Theorem~3.10(b):
If
\[
E_1 \supset E_2 \supset E_3 \supset \cdots
\]
are closed bounded nonempty subsets of a complete metric space $X$,
and if
\[
\limi{n} {\operatorname{diam}} (E_n) = 0,
\]
then $\bigcap_{n = 1}^{\I} E_n$ consists of exactly one point.

\vspace{1ex}

\begin{proof}[Solution (sketch)]
It is clear that $\bigcap_{n = 1}^{\I} E_n$ can contain no more
than one point, so we need to prove that
$\bigcap_{n = 1}^{\I} E_n \neq \E$.

For each $n$, choose some $x_n \in E_n$.
Then, for each $n$, we have
\[
\bigl\{ x_n, \, x_{n + 1}, \, \dots \bigr\} \S E_n,
\]
whence
\[
{\operatorname{diam}}
         \bigl( \bigl\{ x_n, \, x_{n + 1}, \, \dots \bigr\} \bigr)
      \leq {\operatorname{diam}} (E_n).
\]
Therefore $(x_n)_{n \in \N}$ is a Cauchy sequence.
Since $X$ is complete, $x = \limi{n} x_n$ exists in $X$.
For every $n \in \N$,
the set $E_n$ is closed, so $x \in E_n$.
Thus $x \in \bigcap_{n = 1}^{\I} E_n$.
\end{proof}

\vspace{2ex}

\noindent
{\textbf{Problem 3.22:}}
Prove the Baire Category Theorem:
If $X$ is a complete metric space,
and if $(U_n)_{n \in \N}$ is a sequence
of dense open subsets of $X$, then $\bigcap_{n = 1}^{\I} U_n$ is dense
in $X$.

\vspace{1ex}

In this formulation, the statement is true even if $X = \E$.

\begin{proof}[Solution (sketch)]
Let $x \in X$ and let $\ep > 0$.
We recursively construct points $x_n \in X$ and numbers $\ep_n > 0$
such that
\[
d (x, x_1) < \frac{\ep}{3}, \qquad \ep_1 < \frac{\ep}{3},
\qquad \ep_n \to 0,
\]
and
\[
\O{ N_{\ep_{n + 1}} (x_{n + 1} )} \S U_{n + 1} \cap \O{ N_{\ep_n} (x_n )}
\]
for all $n \in \N$.
Using ${\operatorname{diam}}
 \bigl( \O{ N_{\ep_n} (x_n )} \bigr) \leq 2 \ep_n$
and Problem~3.21, it will follow that
\[
\bigcap_{n = 1}^{\I} \O{ N_{\ep_n} (x_n ) } \neq \E.
\]
One easily checks that
\[
\bigcap_{n = 1}^{\I} \O{ N_{\ep_n} (x_n ) }
   \S N_{\ep} (x) \cap \bigcap_{n = 1}^{\I} U_n.
\]
Thus, we will have shown that $\bigcap_{n = 1}^{\I} U_n$ contains points
arbitrarily close to $x$, proving density.

Since $U_1$ is dense in $X$, there is $x_1 \in U_1$ such that
$d (x, x_1) < \frac{\ep}{3}$.
Choose $\ep_1 > 0$ so small that
\[
\ep_1 < 1,  \qquad \ep_1 < \frac{\ep}{3},
 \andeqn N_{2 \ep_1} (x_1 ) \S U_1.
\]
Then also
\[
\O{ N_{\ep_1} (x_1 )} \S U_1.
\]
Given $\ep_n$ and $x_n$,
use the density of $U_{n + 1}$ in $X$ to choose
\[
x_{n + 1} \in U_{n + 1} \cap N_{\ep_n / 2} (x_n ).
\]
Choose $\ep_{n + 1} > 0$ so small that
\[
\ep_{n + 1} < \frac{1}{n + 1},
 \qquad \ep_{n + 1} < \frac{\ep_n}{2},
 \andeqn N_{2 \ep_{n + 1}} (x_{n + 1} ) \S U_{n + 1}.
\]
Then also
\[
\O{ N_{\ep_{n + 1}} (x_{n + 1} )} \S U_{n + 1}.
\]
This gives all the required properties.
(We have $\ep_n \to 0$ since $\ep_n < \frac{1}{n}$ for all $n$.)
\end{proof}

We don't really need to use Problem~3.21 here.
If we always require $\ep_n < 2^{-n}$ in the argument above, we will
get $d (x_n, \, x_{n + 1}) < 2^{-n - 1}$ for all $n$.
This inequality implies that $(x_n)_{n \in \N}$ is a Cauchy sequence.

\end{document}