\section{}
\documentclass[leqno]{article}
%------------------------------------------------------------
\usepackage{amsmath,amssymb,amsthm}
%------------------------------------------------------------
\usepackage[utf8]{inputenc}
\usepackage[T1]{fontenc}
\usepackage[table,xcdraw]{xcolor}
\usepackage[colorlinks=true,pagebackref=true]{hyperref}
\hypersetup{urlcolor=blue, citecolor=red, linkcolor=blue}
\usepackage[capitalise,noabbrev,nameinlink]{cleveref}

\usepackage{graphicx}
\usepackage{tikz}
\usepackage{authblk}
\usepackage{titlesec}
\usepackage{amsthm}
\usepackage{amsfonts}
\usepackage{amssymb}
\usepackage{array}
\usepackage{booktabs}
\usepackage{ragged2e}
\usepackage{enumerate}
\usepackage{enumitem}
\usepackage{cleveref}
\usepackage{slashed}
\usepackage{commath}
\usepackage{lipsum}
\usepackage{colonequals}
\usepackage{addfont}
\usepackage{enumitem}
\usepackage{sectsty}
\usepackage{mathtools}
\usepackage{mathrsfs}
\usepackage{biblatex}

\addbibresource{Analysis.bib}

\hypersetup{
    colorlinks=true,
    linkcolor=blue,
    filecolor=magenta,      
    urlcolor=cyan,
}

\newtheorem{theorem}{Theorem}[section]
\newtheorem{corollary}{Corollary}[theorem]
\newtheorem{lemma}{Lemma}[theorem]
\theoremstyle{definition}
\newtheorem{prop}{Proposition}[section]
\newtheorem{definition}{Definition}[section]
\theoremstyle{remark}
\newtheorem*{remark}{Remark}

\let\oldproofname=\proofname
\renewcommand{\proofname}{\bf{\textit{\oldproofname}}}

\newcommand{\closure}[2][3]{%
  {}\mkern#1mu\overline{\mkern-#1mu#2}}


\newtheorem{example}{Example}[section]

\newtheorem*{discussion}{Discussion}

\makeatletter
\renewenvironment{proof}[1][\proofname]{\par
  \pushQED{\qed}%
  \normalfont \topsep6\p@\@plus6\p@\relax
  \list{}{\leftmargin=0mm
          \rightmargin=0mm
          \settowidth{\itemindent}{\itshape#1}%
          \labelwidth=4mm
          \parsep=0pt \listparindent=0mm%\parindent 
  }
  \item[\hskip\labelsep
        \itshape
    #1\@addpunct{.}]\ignorespaces
}{%
  \popQED\endlist\@endpefalse
}

\makeatletter
\renewenvironment{proof*}[1][\proofname]{\par
  \pushQED{\qed}%
  \normalfont \topsep6\p@\@plus6\p@\relax
  \list{}{\leftmargin=0mm
          \rightmargin=0mm
          \settowidth{\itemindent}{\itshape#1}%
          \labelwidth=\itemindent
          \parsep=0pt \listparindent=0mm%\parindent 
  }
  \item[\hskip\labelsep
        \itshape
    #1\@addpunct{.}]\ignorespaces
}{%
  \popQED\endlist\@endpefalse
}

\newenvironment{solution}[1][\bf{\textit{Solution}}]{\par
  
  \normalfont \topsep6\p@\@plus6\p@\relax
  \list{}{\leftmargin=0mm
          \rightmargin=0mm
          \settowidth{\itemindent}{\itshape#1}%
          \labelwidth=\itemindent
          \parsep=0pt \listparindent=\parindent 
  }
  \item[\hskip\labelsep
        \itshape
    #1\@addpunct{.}]\ignorespaces
}{%
  \popQED\endlist\@endpefalse
}


\begin{document}
\title{Powers of $m$-length cycles}
\author{}
\date{13, Dec 2020}
\maketitle
    
    \section{$m$ is even}
    Let $\alpha=(\alpha_1\alpha_2\cdots\alpha_{m})$ be a $m$-length cycle, where $m$ is an positive even integer. In taking successive powers of $\alpha$ we find the following results
        \begin{align*}
            \alpha^1&=(\alpha_1\alpha_2\cdots\alpha_m)=\alpha \\
            \alpha^2&=(\alpha_1\alpha_3\alpha_5\cdots\alpha_{m-1})(\alpha_2\alpha_4\alpha_6\cdots\alpha_m) \\
            \alpha^3&=(\alpha_1\alpha_4\alpha_7\alpha_{10}\cdots\alpha_i\alpha_{i+3}\cdots)\cdots\quad\quad (*) \\
            &\hspace{2mm}\vdots \\
            \alpha^{m/2}&=(\alpha_1\alpha_{1+\frac{m}{2}})(\alpha_2\alpha_{2+\frac{m}{2}})\cdots(\alpha_{\frac{m}{2}}\alpha_m) \\
            \alpha^{m/2+1}&=\big(\alpha^{m/2-1}\big)^{-1} \\
            \alpha^{m/2+2}&=\big(\alpha^{m/2-2}\big)^{-1} \\
            &\hspace{2mm}\vdots \\
            \alpha^{m/2+k}&=\big(\alpha^{m/2-k}\big)^{-1} \\
            &\hspace{2mm}\vdots \\
            \alpha^{m-1}&=\alpha^{-1} \\
            \alpha^m&=(1)
        \end{align*}
    In general, for any $1\leq k\leq \frac{m}{2}$, if $k\mid m$, then $\alpha^k$ will contain $k$ many disjoint cycles of length $\frac{m}{k}$, where the $i$th cycle takes the form $(\alpha_i\alpha_{i+k}\alpha_{i+2k}\cdots)$. If $k\nmid m$, then $\alpha^k$ is 1 cycle of length $m$ of the form $(\alpha_1\alpha_{1+k}\cdots\alpha_{i}\alpha_{i+k}\cdots\alpha_{m-k}\alpha_{k})$. Finally, for powers $k'>\frac{m}{2}$, we find that $\alpha^{k'}=(\alpha^{k})^{-1}$, where $k'\equiv k\pmod{\frac{m}{2}}$.
    
    
\end{document}