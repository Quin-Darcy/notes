\section{}
\documentclass{article}

%------------------------------------------------------------
\usepackage{amsmath,amssymb,amsthm}
%------------------------------------------------------------
\usepackage[utf8]{inputenc}
\usepackage[T1]{fontenc}
\usepackage{xcolor}
\usepackage[colorlinks=true,pagebackref=true]{hyperref}
\hypersetup{urlcolor=blue, citecolor=red, linkcolor=blue}
\usepackage[capitalise,noabbrev,nameinlink]{cleveref}

\usepackage{graphicx}
\usepackage{tikz}
\usepackage{authblk}
\usepackage{titlesec}
\usepackage{amsthm}
\usepackage{amsfonts}
\usepackage{amssymb}
\usepackage{array}
\usepackage{booktabs}
\usepackage{ragged2e}
\usepackage{enumerate}
\usepackage{enumitem}
\usepackage{cleveref}
\usepackage{slashed}
\usepackage{commath}
\usepackage{lipsum}
\usepackage{colonequals}
\usepackage{addfont}
\usepackage{enumitem}
\usepackage{sectsty}
\usepackage{mathtools}
\usepackage{mathrsfs}


\hypersetup{
    colorlinks=true,
    linkcolor=blue,
    filecolor=magenta,      
    urlcolor=cyan,
}

\usetikzlibrary{decorations.pathreplacing}
\usetikzlibrary{arrows.meta}


%\subsectionfont{\itshape}

\newtheorem{theorem}{Theorem}
\newtheorem{corollary}{Corollary}
\newtheorem{lemma}{Lemma}
\theoremstyle{definition}
\newtheorem{prop}{Proposition}
\newtheorem{definition}{Definition}
\theoremstyle{remark}
\newtheorem*{remark}{Remark}

\let\oldproofname=\proofname
\renewcommand{\proofname}{\textit{\oldproofname}}

\newcommand{\closure}[2][3]{%
  {}\mkern#1mu\overline{\mkern-#1mu#2}}

\theoremstyle{definition}
\newtheorem{example}{Example}

\makeatletter
\renewenvironment{proof*}[1][\proofname]{\par
  \pushQED{\qed}%
  \normalfont \topsep6\p@\@plus6\p@\relax
  \list{}{\leftmargin=0mm
          \rightmargin=0mm
          \settowidth{\itemindent}{\itshape#1}%
          \labelwidth=4mm
          \parsep=0pt \listparindent=0mm%\parindent 
  }
  \item[\hskip\labelsep
        \itshape
    #1\@addpunct{.}]\ignorespaces
}{%
  \popQED\endlist\@endpefalse
}

\makeatletter
\renewenvironment{proof}[1][\proofname]{\par
  \pushQED{\qed}%
  \normalfont \topsep6\p@\@plus6\p@\relax
  \list{}{\leftmargin=0mm
          \rightmargin=0mm
          \settowidth{\itemindent}{\itshape#1}%
          \labelwidth=\itemindent
          \parsep=0pt \listparindent=0mm%\parindent 
  }
  \item[\hskip\labelsep
        \itshape
    #1\@addpunct{.}]\ignorespaces
}{%
  \popQED\endlist\@endpefalse
}

\newenvironment{solution}[1][\bf{\textit{Solution}}]{\par
  
  \normalfont \topsep6\p@\@plus6\p@\relax
  \list{}{\leftmargin=0mm
          \rightmargin=0mm
          \settowidth{\itemindent}{\itshape#1}%
          \labelwidth=\itemindent
          \parsep=0pt \listparindent=\parindent 
  }
  \item[\hskip\labelsep
        \itshape
    #1\@addpunct{.}]\ignorespaces
}{%
  \popQED\endlist\@endpefalse
}

\begin{document}
    \begin{prop}[Pg. 1]
        If $E$ is a field, then $(\text{Aut}(E),\circ)$ is a group.
    \end{prop}
    \begin{prop}[Pg. 1]
        If $E/F$, $g(x)\in F[x]$ is irreducible over $F$, $\varphi$ is an automorphism on $E$, and $\varphi$ is the identity on $F$, and $c$ is a root of $g(x)$, then $\varphi(c)$ is a root of $g(x)$.
    \end{prop}
    \begin{prop}[Pg. 1]
        If $E/F$, $g(x)\in F[x]$ is irreducible over $F$, and $c$ and $d$ are each roots of $g(x)$, then there is an automorphism $\varphi$ of $E$ such that $\varphi(c)=d$ and $\varphi$ is the identity on $F$.
    \end{prop}
    \begin{theorem}[Pg. 1]
        If $K$ is a field, $\alpha_1,\dots,\alpha_n$ are distinct automorphisms of $K$, then $\{\alpha_1,\dots,\alpha_n\}$ is linearly independent over $K$.
    \end{theorem}
    \begin{prop}[HW6]
        Assume that $\varphi$ is an automorphism of $E$. $F_{\varphi}=\{a\in E\mid \varphi(a)=a\}$ is a subfield of $E$. 
    \end{prop}
    \begin{prop}[Pg. 2]
        If $G=\{g_1,\dots,g_n\}$ is a group and $g\in G$, then $G=\{gg_1,\dots,gg_n\}$.
    \end{prop}
    \begin{theorem}[Pg. 2]
        If $E/F$, and $S$ is a finite group of automorphisms on $E$, then $[E\colon F_S]=\abs{S}$.
    \end{theorem}
    \begin{prop}[Pg. 4]
        Assume that $E/K$ is finite, and that $H$ is a subgroup of $G(E/K)$. Then $[E\colon F_H]=\abs{H}$. Also since $E/F_H/K$, then 
            \begin{equation*}
                [F_H\colon K]=\frac{[E\colon F_H]}{\abs{H}}.
            \end{equation*}
        If $H=G(E/K)$, then $[E\colon K]=\abs{G(E/K)}[F_{G(E/K)}\colon K]$. Thus $\abs{G(E/K)}\leq[E\colon K]$.
    \end{prop}
    \begin{prop}[Pg. 5]\hfill\par
        \begin{itemize}
            \item If $E/K/M$, then $G(E/K)\subseteq G(E/M)$.
            \item If $G$ and $H$ are subgroups of $\text{Aut}(E)$, $G\subseteq H$, then $F_H\subseteq F_G$.
            \item If $G$ is a subgroup of $\text{Aut}(E)$, the $[E\colon F_G]=\abs{G}$.
            \item $\abs{G(E/K)}\leq[E\colon K]$ with equality iff $F_{G(E/K)}=K$.
        \end{itemize}
    \end{prop}\newpage
    
    \begin{theorem}
        If $L$ is the splitting field of a separable polynomial in $F[x]$ and $\text{char}(F)=0$, then $\abs{G(L/F)}=[L\colon F]$.
    \end{theorem}
        \begin{proof}
            Assume that $L$ is the splitting field of a separable polynomial $f\in F[x]$ and that $F$ has characteristic 0. Then $L=F(\alpha_1,\dots,\alpha_n)$, where $\alpha_1,\dots,\alpha_n$ are the roots of $f$. By Pg. 8 of Galois Theory, $L$ is a finite extension of $F$ and $\text{char}(F)=0$ and thus $L/F$ is separable. By Theorem 4, there exists $c\in L$ such that $L=F(c)$. Let $h\in F[x]$ be the minimal polynomial of $c$. Note that $h$ is separable since $c$ is separable. Assume that $\text{deg}(h)=m$. Then $[L\colon F]=m$.\par\hspace{4mm} Since $L$ is the splitting field of $f\in F[x]$, then by Theorem 3, $L/F$ is normal
        \end{proof}
\end{document}