\documentclass{article}
\usepackage{graphicx}
\usepackage{tikz}
\usepackage{amsmath}
\usepackage{authblk}
\usepackage{titlesec}
\usepackage{amsthm}
\usepackage{amsfonts}
\usepackage{amssymb}
\usepackage{array}
\usepackage{booktabs}
\usepackage{ragged2e}
\usepackage{enumerate}
\usepackage{enumitem}
\usepackage{cleveref}
\usepackage{slashed}
\usepackage{commath}
\usepackage{lipsum}
\usepackage{colonequals}
\usepackage{addfont}
\usepackage{enumitem}
\usepackage{sectsty}
\usepackage{mathtools}
\usepackage{mathrsfs}
\usepackage{xcolor}

\usepackage{hyperref}
\hypersetup{
    colorlinks=true,
    linkcolor=blue,
    filecolor=magenta,      
    urlcolor=cyan,
}

\usetikzlibrary{decorations.pathreplacing}
\usetikzlibrary{arrows.meta}


%\subsectionfont{\itshape}

\newtheorem{theorem}{Theorem}[section]
\newtheorem{corollary}{Corollary}[theorem]
\newtheorem{lemma}{Lemma}[theorem]
\theoremstyle{definition}
\newtheorem{prop}{Proposition}
\newtheorem{definition}{Definition}
\theoremstyle{remark}
\newtheorem*{remark}{Remark}

\let\oldproofname=\proofname
\renewcommand{\proofname}{\bf{\textit{\oldproofname}}}

\newcommand{\closure}[2][3]{%
  {}\mkern#1mu\overline{\mkern-#1mu#2}}

\theoremstyle{definition}
\newtheorem{example}{Example}[section]

\newtheorem*{discussion}{Discussion}

\makeatletter
\renewenvironment{proof}[1][\proofname]{\par
  \pushQED{\qed}%
  \normalfont \topsep6\p@\@plus6\p@\relax
  \list{}{\leftmargin=0mm
          \rightmargin=0mm
          \settowidth{\itemindent}{\itshape#1}%
          \labelwidth=4mm
          \parsep=0pt \listparindent=0mm%\parindent 
  }
  \item[\hskip\labelsep
        \itshape
    #1\@addpunct{.}]\ignorespaces
}{%
  \popQED\endlist\@endpefalse
}

\makeatletter
\renewenvironment{proof*}[1][\proofname]{\par
  \pushQED{\qed}%
  \normalfont \topsep6\p@\@plus6\p@\relax
  \list{}{\leftmargin=0mm
          \rightmargin=0mm
          \settowidth{\itemindent}{\itshape#1}%
          \labelwidth=\itemindent
          \parsep=0pt \listparindent=0mm%\parindent 
  }
  \item[\hskip\labelsep
        \itshape
    #1\@addpunct{.}]\ignorespaces
}{%
  \popQED\endlist\@endpefalse
}

\newenvironment{solution}[1][\bf{\textit{Solution}}]{\par
  
  \normalfont \topsep6\p@\@plus6\p@\relax
  \list{}{\leftmargin=0mm
          \rightmargin=0mm
          \settowidth{\itemindent}{\itshape#1}%
          \labelwidth=\itemindent
          \parsep=0pt \listparindent=\parindent 
  }
  \item[\hskip\labelsep
        \itshape
    #1\@addpunct{.}]\ignorespaces
}{%
  \popQED\endlist\@endpefalse
}

\begin{document}
\section{An Assortment of Results and Definitions}\hline\hfill\par
    \begin{definition}\label{def:1}
        Let $f\in F[x]$ have degree $n>0$. Then an extension $L/F$ is a \textbf{\textit{splitting field}} of $f$ over $F$ if
            \begin{enumerate}
                \item $f=c(x-\alpha_1)\cdots(x-\alpha_n)$, where $c\in F$ and $\alpha_i\in L$, and
                \item $L=F(\alpha_1,\dots,\alpha_n)$.
            \end{enumerate}
    \end{definition}
    \begin{prop}\label{prop:1}
        Let $f\in F[x]$ be a polynomial of degree $n>0$, and let $L$ be a splitting field of $f$ over $F$. Then $[L\colon F]\leq n!$.
    \end{prop}
        \begin{proof}
            \emph{By induction on $n$. When $n=1$, $f=ax+b$ has the root $-b/a\in F$, since $a\neq 0$. Thus, $L=F$ in this case, and $[L\colon F]\leq 1!$.}\par\hspace{4mm}\emph{Now suppose that $f$ has degree $n>1$, and let $L=F(\alpha_1,\dots, \alpha_n)$ be a splitting field of $f$ over $F$. If we write $f=(x-\alpha_1)g$, then the division algorithm implies that $g\in F(\alpha_1)[x]$. Furthermore, the roots of $g$ are obviously $\alpha_2,\dots,\alpha_n$, so that a splitting field of $g$ over $F(\alpha_1)$ is given by}
                \begin{equation*}
                    F(\alpha_1)(\alpha_2,\dots,\alpha_n)=F(\alpha_1,\dots,\alpha_n)=L.
                \end{equation*}
            \emph{Since $g\in F(\alpha_1)[x]$ has degree $n-1$, our inductive hypothesis implies that}
                \begin{equation*}
                    [L\colon F(\alpha_1)]\leq (n-1)!.
                \end{equation*}
            \emph{By the Tower Theorem, we have}
                \begin{equation*}
                    [L\colon F]=[L\colon F(\alpha_1)][F(\alpha_1)\colon F]\leq (n-1)![F(\alpha_1)\colon F].
                \end{equation*}
            \emph{However, we know that $[F(\alpha_1)\colon F]$ is the degree of the minimal polynomial of $\alpha_1$ over $F$. Since $f(\alpha_1)=0$, then $[F(\alpha_1)\colon F]\leq n$, and then $[L\colon F]\leq n!$ follows.} 
        \end{proof}
    \begin{definition}\label{def:2}
        An element $\alpha$ of an extension field $E$ of a field $F$ is \textbf{\textit{algebraic over}} $F$ if $f(\alpha)=0$ for some nonzero $f(x)\in F[x]$.
    \end{definition}
    \begin{definition}\label{def:3}
        An extension field $E$ of a field $F$ is an \textbf{\textit{algebraic extension of}} $F$ if every element in $E$ is algebraic over $F$.
    \end{definition}
    \begin{definition}\label{def:4}
        Let $E$ be an extension field of $F$. Then
            \begin{equation*}
                \overline{F}_E=\{\alpha\in E\mid \alpha\text{ if algebraic over } F\}
            \end{equation*}
        is a subfield of $E$, the \textbf{\textit{algebraic closure of}} $F$ \textbf{\textit{in}} $E$.
    \end{definition}
    \begin{definition}\label{def:5}
        A field $F$ is \textbf{\textit{algebraically closed}} if every nonconstant polynomial in $F[x]$ has a zero in $F$.
    \end{definition}
    \begin{definition}\label{def:6}
        Let $E$ be an algebraic extension of a field $F$. Two elements $\alpha,\beta\in E$ are \textbf{\textit{conjugate}} over $F$ if irr$(\alpha,F)=$irr$(\beta,F)$, that is, if $\alpha$ and $\beta$ are zeros of the same irreducible polynomial.
    \end{definition}
    \begin{prop}\label{prop:2}
        Let $L$ be the splitting field of $f\in F[x]$, and let $g\in F[x]$ be irreducible. If $g$ has one root in $L$, then $g$ splits completely over $L$.
    \end{prop}
        \begin{proof}
            \emph{We can assume $f$ and $g$ are monic. Then $L=F(\alpha_1,\dots, \alpha_n)$, where $f=(x-\alpha_1)\cdots(x-\alpha_n)$. If $\beta\in L$ is a root of $g$, then $g$ is the minimal polynomial of $\beta$ since $g$ is irreducible and monic over $F$. We need to prove that all the roots of $g$ lie in $L$. }TBC...
        \end{proof}
    \begin{prop}\label{prop:3}
        Let $F$ be a field, $f(x)\in F[x]$, and $E/F$ be the splitting field for $f(x)$. Then $f(x)$ has a multiple root in E \textit{iff} $f(x)$ and $f'(x)$ share a common factor $d(x)\in E[x]$, where $\text{deg}(d(x))\geq 1$. 
    \end{prop}
        \begin{proof}
            \emph{Assume that $f(x)$ has a multiple root. Then for some $a\in E$ and $k>1$, $f(x)=(x-a)^kg(x)$, where $g(x)\in F[x]$. Taking the derivative of $f(x)$ gives:
                \begin{equation*}
                    \begin{split}
                        f'(x)&=k(x-a)^{k-1}+(x-a)g'(x) \\
                        &= (x-a)(k(x-a)^{k-2}+(x-a)^{k-1}g'(x).
                    \end{split}
                \end{equation*}
                Thus $f(x)$ and $f'(x)$ share a common factor $(x-a)^k$ where $k\geq 1$.}\par\hspace{4mm}
            \emph{Conversely, if $f(x)$ has no multiple root, then for all $a\in E$ with $f(a)=0$, there exists $g_a(x)\in F[x]$ such that $f(x)=(x-a)g_a(x)$ and $(x-a)\nmid g_a(x)$. Note that if $(x-a)\mid g(x)$, then for some $k>1$, $(x-a)^k\mid f(x)$ which contradicts our converse assumption. Taking the derivative of $f(x)$ gives
                \begin{equation*}
                    f(x)=g(x)+(x-a)g'(x),
                \end{equation*}
            which is not a multiple of $x-a$. Since $a\in E$ was an arbitrary root of $f(x)$, then $f(x)$ and $f'(x)$ share no common factors.
                }
        \end{proof}
    \begin{prop}\label{prop:4}
        Let $F$ be a field with char$(F)=0$ and $f\in F[x]$ irreducible over $F$. Then $f$ has no roots of multiplicity greater than 1.
    \end{prop}
        \begin{proof}
            \emph{For contradiction, assume that $f$ has a multiple root. Then by} Proposition \ref{prop:2}, \emph{there exists $d\in F[x]$ such that $\text{deg}(d)\geq 1$, $d\mid f$, and $d\mid f'$. Since $d\mid f$, then for some $h\in F[x]$, $f=dh$. Taking the derivative then gives $f'=d'h+dh'$.}\par\hspace{4mm}
            \emph{Since $d\mid f'$ and $d\mid dh'$, then $d\mid d'h$. However, since $f$ is irreducible and $\text{deg}(d)\geq 1$, then $\text{deg}(h)=0$. Therefore, $h$ is a constant and so $(d,h)=1$. Then by Euclid's Lemma, $d\mid d'$ which implies $\text{deg}(d)\leq\text{deg}(d')$. However, note that for all $g\in F[x]$, $\text{deg}(g')\leq\text{deg}(g)$. Thus, $\text{deg}(d')\leq\text{deg}(d)$. Hence, $\text{deg}(d)=\text{deg}(d')$. This is only true if $\text{deg}(d)=0$. Thus, $d$ is a constant. This contradicts our assumption that $f$ has multiple roots.}
        \end{proof}
    \begin{prop}\label{prop:5}
        Let $F$ be a field such that $\text{char}(F)=p$ and let $f(x)\in F[x]$ be irreducible over $F$. Then if $f(x)$ has a multiple root, there exists $g(x)\in F[x]$ such that $f(x)=g(x^p)$. 
    \end{prop}
        \begin{proof}
            \emph{Assume that $f$ has a multiple root. Then by }Proposition \ref{prop:2}\emph{ there exists $d(x)\in F[x]$ with $\text{deg}(d(x))\geq 1$ such that $d(x)\mid f(x)$ and $d(x)\mid f'(x)$. If $f'(x)\neq 0$, then by the same argument used in} Proposition \ref{prop:3}\emph{, it would follow that $\text{deg}(f(x))=\text{deg}(f'(x))$ and so $f(x)=f'(x)=0$ which is a contradiction. Thus, $f'(x)=0$. Let}
                \begin{equation*}
                    f(x)=\sum\limits_{i=0}^na_ix^i,\quad\text{then}\quad f'(x)=\sum\limits_{i=1}^n ia_ix^{i-1}.
                \end{equation*}
            \emph{Since $f'(x)=0$, then $ia_i=0$ for all $i$. Thus, either $a_i=0$ or $p\mid i$ for all $i$. It cannot be the case that $a_i=0$ since that would imply that $f(x)=a_0$. Thus, $p\mid i$ for all $1\leq i\leq n$. Hence, $f(x)=\sum_{j=0}^na_j(x^p)^j=g(x^p)$. }
        \end{proof}
    \begin{prop}\label{prop:6}
        If $p$ is prime and $n\in\mathbb{N}$ such that $p\nmid n$, then $x^n-1$ has $n$ distinct solutions in $\mathbb{Z}_p$.
    \end{prop}
        \begin{proof}
            \emph{Let $f(x)=x^n-1$. Then $f'(x)=nx^{n-1}\neq 0$ since $p\nmid n$. Thus, $f(x)$ and $f'(x)$ share no common factors greater than or equal to 1. By} Proposition \ref{prop:2},\emph{ $f(x)$ has no multiple roots  in its splitting field. Thus, $f(x)$ has $n$ distinct roots.}
        \end{proof}
    \begin{prop}\label{prop:7}
        Let $t\in\mathbb{Z}$ and let $t=p_1^{r_1}p_2^{r_2}\cdots p_m^{r_m}$ be the prime factorization of $t$. The M\"obius function, $\mu$ is defined by: 
            \begin{equation*}
                \mu(t)=\begin{cases} 
                    1 &\text{if } t=1 \\
                    (-1)^m &\text{if } r_i=1 \text{ for all } i,\; 1\leq i\leq m \\
                    0 &\text{if } r_i>1 \text{ for some } i.
                \end{cases}
            \end{equation*}
        Now define
            \begin{equation*}
                \Phi_n(x)=\prod_{d\mid n}(x^{n/d}-1)^{\mu(d)}.
            \end{equation*}
        Then $\Phi_n(x)$ is irreducible over $\mathbb{Z}[x]$.
    \end{prop}
        \begin{proof}
            
        \end{proof}
    \begin{definition}
        An algebraic extension $L/F$ is \textbf{\textit{normal}} if every irreducible polynomial in $F[x]$ that has a root in $L$ splits completely over $L$.
    \end{definition}
    \begin{definition}
        A polynomial $f\in F[x]$ is \textbf{\textit{separable}} if it is nonconstant and its roots in a splitting field are simple.
    \end{definition}
    \begin{definition}
        Let $L/F$ be an algebraic extension.
            \begin{enumerate}
                \item $\alpha\in L$ is \textbf{\textit{separable}} over $F$ if its minimal polynomial over $F$ is separable.
                \item $L/F$ is a \textbf{\textit{separable extension}} is every $\alpha\in F$ is separable over $F$.
            \end{enumerate}
    \end{definition}
\end{document}