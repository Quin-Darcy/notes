\documentclass{article}
\usepackage{graphicx}
\usepackage{tikz}
\usepackage{amsmath}
\usepackage{authblk}
\usepackage{titlesec}
\usepackage{amsthm}
\usepackage{amsfonts}
\usepackage{amssymb}
\usepackage{array}
\usepackage{booktabs}
\usepackage{ragged2e}
\usepackage{enumerate}
\usepackage{enumitem}
\usepackage{cleveref}
\usepackage{slashed}
\usepackage{commath}
\usepackage{lipsum}
\usepackage{colonequals}
\usepackage{addfont}
\usepackage{enumitem}
\usepackage{sectsty}
\usepackage{mathtools}
\usepackage{mathrsfs}

\usepackage{hyperref}
\hypersetup{
    colorlinks=true,
    linkcolor=blue,
    filecolor=magenta,      
    urlcolor=cyan,
}

\usetikzlibrary{decorations.pathreplacing}
\usetikzlibrary{arrows.meta}


%\subsectionfont{\itshape}

\newtheorem{theorem}{Theorem}[section]
\newtheorem{corollary}{Corollary}[theorem]
\newtheorem{lemma}{Lemma}[theorem]
\theoremstyle{definition}
\newtheorem{prop}{Proposition}[section]
\newtheorem{definition}{Definition}[section]
\theoremstyle{remark}
\newtheorem*{remark}{Remark}

\let\oldproofname=\proofname
\renewcommand{\proofname}{\bf{\textit{\oldproofname}}}

\newcommand{\closure}[2][3]{%
  {}\mkern#1mu\overline{\mkern-#1mu#2}}

\theoremstyle{definition}
\newtheorem{example}{Example}[section]

\newtheorem*{discussion}{Discussion}

\makeatletter
\renewenvironment{proof}[1][\proofname]{\par
  \pushQED{\qed}%
  \normalfont \topsep6\p@\@plus6\p@\relax
  \list{}{\leftmargin=0mm
          \rightmargin=0mm
          \settowidth{\itemindent}{\itshape#1}%
          \labelwidth=4mm
          \parsep=0pt \listparindent=0mm%\parindent 
  }
  \item[\hskip\labelsep
        \itshape
    #1\@addpunct{.}]\ignorespaces
}{%
  \popQED\endlist\@endpefalse
}

\makeatletter
\renewenvironment{proof*}[1][\proofname]{\par
  \pushQED{\qed}%
  \normalfont \topsep6\p@\@plus6\p@\relax
  \list{}{\leftmargin=0mm
          \rightmargin=0mm
          \settowidth{\itemindent}{\itshape#1}%
          \labelwidth=\itemindent
          \parsep=0pt \listparindent=0mm%\parindent 
  }
  \item[\hskip\labelsep
        \itshape
    #1\@addpunct{.}]\ignorespaces
}{%
  \popQED\endlist\@endpefalse
}

\newenvironment{solution}[1][\bf{\textit{Solution}}]{\par
  
  \normalfont \topsep6\p@\@plus6\p@\relax
  \list{}{\leftmargin=0mm
          \rightmargin=0mm
          \settowidth{\itemindent}{\itshape#1}%
          \labelwidth=\itemindent
          \parsep=0pt \listparindent=\parindent 
  }
  \item[\hskip\labelsep
        \itshape
    #1\@addpunct{.}]\ignorespaces
}{%
  \popQED\endlist\@endpefalse
}


\begin{document}

\title{Notes for MATH 210B}
\author{Quin Darcy}
\date{Jan 23, 2020}
\affil{\small{California State University, Sacramento}}
\maketitle

\section{Definitions}

    \begin{definition}\label{df:1.1} 
        $(R,+,\cdot)$ is a \textbf{ring} iff $+$ and $\cdot$ are operations on $R$ such that $(R,+)$ is an Abelian group, $\cdot$ is associative, $\cdot$ distributes over $+$ on the left and on the right.
    \end{definition}
    
    \begin{definition}\label{df:1.2}
        If $(R,+,\cdot)$ is a ring and $\cdot$ is commutative on $R$, then $(R,+,\cdot)$ is called a \textbf{commutative ring}.
    \end{definition}
    
    \begin{definition}\label{df:1.3}
        If $(R,+,\cdot)$ is a ring and there exists an identity element, 1, in $R$ for $\cdot$, then $R$ is called a \textbf{ring with identity}.
    \end{definition}
    
    \begin{definition}\label{df:1.4}
        If $(R,+,\cdot)$ is a ring, $S\subseteq R$ such that $S\neq\varnothing$, then $S$ is a \textbf{subring} of $R$ iff $S$ is closed under $+,\cdot,-$. 
    \end{definition}
    
    \begin{definition}\label{df:1.5}
        A nonzero element in a ring $R$ is said to be a \textbf{left zero divisor} is there exists a nonzero $b\in R$ such that $ab=0$. A \textbf{zero divisor} is an element of $R$ which is both a left zero divisor and a right zero divisor.
    \end{definition}
    
    \begin{definition}\label{df:1.6}
        If $(R,+,\cdot)$ is commutative ring with identity, and $R$ has no zero divisors, then $(R,+,\cdot)$ is called an \textbf{integral domain}.
    \end{definition}
    
    \begin{definition}\label{df:1.7}
        Given a ring $R$, the \textbf{characteristic} of $R$ (denoted char$(R)$) is the smallest positive integer $N$ such that for all $r\in R$, $nr=0$. If no such integer exists, then $\text{char}(R)=0$. If $R$ has an identity $1_R$, then char$(R)$ is the smallest positive integer $n$ such that $n1_R=0$.
    \end{definition}
    
    \begin{definition}\label{df:1.8}
        If $I$ is a subring (see Definition \ref{df:1.4}) of $R$, then $I$ is an \textbf{ideal} of $R$ iff $(I,+)$ is a subgroup of $R$, and for all $r\in R$ and all $a\in I$, $ar\in I$ and $ra\in I$.
    \end{definition}
    
    \begin{definition}\label{df:1.9}
        If $(R,+,\cdot)$ is a commutative ring and $I$ is an ideal of $R$, then $I$ is called a \textbf{prime ideal} iff $I\neq R$, and whenever $ab\in I$ (where $a,b\in R$) then $a\in I$ or $b\in I$.
    \end{definition}
    
    \begin{definition}\label{df:1.10}
        If $(R,+\cdot)$ is a ring, and $I$ ia an ideal of $R$, then $I$ is called a \textbf{maximal ideal} of $R$ iff $I\neq R$ and there does not exist an ideal $J$ of $R$ such that $I\subset J\subset R$. That is, $I$ is a maximal ideal of $R$ iff $I\neq R$, and if $I\subseteq J\subseteq R$, then $I=J$ or $J=R$.
    \end{definition}
    
    \begin{definition}\label{df:1.11}
        An ideal $P$ in a ring $R$ is said to be \textbf{prime} if $P\neq R$ and for any ideals $A,B$ of $R$
            \begin{equation*}
                AB\subseteq P\Rightarrow A\subseteq P\quad\text{or}\quad B\subseteq P.
            \end{equation*}
    \end{definition}
    
    \begin{definition}\label{df:1.12}
        An integral domain $R$ is called a \textbf{Euclidean domain} iff there exists a function $v\colon R-\{0\}\rightarrow\mathbb{Z}^{+}$ such that for all $a,b\in R-\{0\}$, $v(a)\leq v(ab)$, and for all $a,b\in R-\{0\}$ there exists $q,r\in R$ such that $a=qb+r$ and either $r=0$ or $v(r)<v(b)$.
    \end{definition}
    
    \begin{definition}\label{df:1.13}
        If $R$ is a Euclidean domain and $a,b\in R$, then $a$ is a \textbf{divisor} of $b$, denoted $a\mid b$, iff there exists $c\in R$ such that $b=ac$.
    \end{definition}
    
    \begin{definition}\label{df:1.14}
        An element $a$ in a ring $R$ is called a \textbf{unit} iff $a\mid 1$.
    \end{definition}
    
    \begin{definition}\label{df:1.15}
        Elements $a,b\in R$ are called \textbf{associates} iff there exists a unit $u\in R$ such that $a=ub$. 
    \end{definition}
    
    \begin{definition}\label{df:1.16}
        If $a,b\in R$, then $d$ is the greatest common divisor of $a$ and $b$ iff $d\mid a$ and $d\mid b$, and if $g\mid a$ and $g\mid b$, then $g\mid d$.
    \end{definition}
    
    \begin{definition}\label{df:1.17}
        Let $R$ be a commutative ring with identity. An element $c\in R$ is \textbf{irreducible} iff $c$ is a nonzero nonunit and if $c=ab$, then $a$ or $b$ is a unit. 
    \end{definition}
    
    \begin{definition}\label{df:1.18}
        Let $R$ be a commutative ring with identity. An element $p\in R$ is \textbf{prime} iff $p$ is a nonzero nonunit, and if $p\mid ab$, then $\mid a$ or $p\mid b$.
    \end{definition}
    
    \begin{definition}\label{df:1.19}
        Let $R$ and $S$ be rings. A function $f\colon R\rightarrow S$ is a \textbf{homomorphism of rings} provided that for all $a,b\in R$:
            \begin{equation*}
                f(a+b)=f(a)+f(b)\quad\text{and}\quad f(ab)=f(a)f(b).
            \end{equation*}
    \end{definition}
    
    \begin{definition}\label{df:1.20}
        Let $X$ be a nonempty subset of a commutative ring $R$> An element $d\in R$ is a \textbf{greatest common divisor} of $X$ provided:
            \begin{enumerate}[label=(\roman*)]
                \item $d\mid a$ for all $a\in X$;
                \item If $c\mid a$ for all $a\in X$, then $c\mid d$.
            \end{enumerate}
    \end{definition}
    
    \begin{definition}\label{df:1.21}
        An ideal generated by a single element is called a \textbf{principle ideal}. An integral domain in which every ideal is principle is called a \textbf{principle ideal domain}.
    \end{definition}
    
    \begin{definition}\label{df:1.22}
        An integral domain $R$ is a \textbf{unique factorization domain} provided that:
            \begin{enumerate}[label=(\roman*)]
                \item every nonzero nonunit element $a$ of $R$ can be written $a=c_1c_2\cdots c_n$, with $c_1,\dots,c_n$ irreducible.
                \item If $a=c_1\cdots c_n$ and $a=d_1\cdots d_m$ ($c_i,d_i$ irreducible), then $n=m$ and for some permutation $\sigma$ of $\{1,2,\dots,n\}$, $c_i$ and $d_{\sigma(i)}$ are associates.
            \end{enumerate}
    \end{definition}\newpage

\section{Results}
    \begin{theorem}\label{thm:2.1}
        Let $R$ be a ring, then $(na)b=a(nb)=n(ab)$ for all $n\in\mathbb{Z}$ and all $a,b\in R$.
    \end{theorem}
        \begin{proof}
            By induction on $n$. If $n=1$, then $na=a$ and $nb=b$. Thus, $(na)b=(a)b=a(b)=a(nb)=n(ab)$. Assume the result holds for $n=k$. Then $((k+1)a)b=(a+\cdots+a)b=ab+\cdots+ab=a(b+\cdots+b)=a((k+1)b)=(k+1)(ab)$.
        \end{proof}

    \begin{theorem}\label{thm:2.2} Let $R$ be a ring with identity $1_R$ and characteristic $n>0$.
        \begin{enumerate}[label=(\roman*)]
            \item If $\varphi\colon \mathbb{Z}\rightarrow R$ is the map given by $m\mapsto m1_R$, then $\varphi$ is a homomorphism of rings with kernel $\langle n\rangle=\{kn\mid k\in\mathbb{Z}\}$.
            
            \item $n$ is the least positive integer such that $n1_R=0$.
            
            \item If $R$ has no zero divisors, then $n$ is prime.
        \end{enumerate}
    \end{theorem}
        \begin{proof}
            (i) Since for all $m\in\mathbb{Z}$, $m1_R\in R$, then clearly $\varphi(\mathbb{Z})\subseteq R$. Let $a,b\in R$ and assume $a=b$. Then $\varphi(a)=a1_R=b1_R=\varphi(b)$. Let $a,b\in R$. Then $\varphi(a+b)=(a+b)1_R=a1_R+b1_R$. Additionally, by Theorem \ref{thm:2.1}, it follows that $\varphi(ab)=(ab)1_R=(a1_R)(b1_R)=\varphi(a)\varphi(b)$. Thus, $\varphi$ is a homomorphism. Lastly, we let $x\in\ker\varphi$. Then $\varphi(x)=x1_R=0$. Thus, $n\mid x$. Hence, $x$ is an integer multiple of $n$. Therefore, $x\in\langle n\rangle$. Thus, $\ker\varphi\subseteq\langle n\rangle$. Inclusion in the other direction follows immediately. (ii) If $k$ is the least positive integer such that $k1_R=0$, then for all $a\in R$: $ka=k(1_Ra)=(k1_R)a=0\cdot a=0$ by Theorem \ref{thm:2.1}. (iii) If $n=kr$ with $1<k<n$, $1<r<n$, then $0=n1_R=(kr)1_R1_R=(k1_R)(r1_R)$ implies that $k1_R=0$ or $r1_R=0$, which contradicts (ii). 
        \end{proof}
    
    \begin{theorem}\label{thm:2.3}
        Let $a,b$ and $u$ be elements of a commutative ring $R$ with identity.
            \begin{enumerate}[label=(\roman*)]
                \item $a\mid b$ if and only if $(b)\subseteq (a)$.
                \item $a$ and $b$ are associates if and only if $(a)=(b)$.
                \item $u$ is a unit if and only if $u\mid r$ for all $r\in R$.
                \item $u$ is a unit if and only if $(u)=R$.
                \item The relation ``$a$ is an associate of $b$'' is an equivalence relation on $R$.
                \item If $a=br$, with $r\in R$ a unit, then $a$ and $b$ are associates. If $R$ is an integral domain, then the converse is true.
            \end{enumerate}
    \end{theorem}
        \begin{proof}
            (i) Assume $a\mid b$. Then there exists $x\in R$ such that $b=ax$. Thus, $b\in (a)$. Let $r\in R$, then since $br=axr$ and $axr\in(a)$, then $br\in(a)$ for all $r\in R$. Thus, $(b)\subseteq (a)$. Assume $(b)\subseteq (a)$. Then since $b\in(b)$, then $b\in(a)$. Thus, there exists $x\in R$ such that $b=ax$. Thus, $a\mid b$. (ii) Assume $a$ and $b$ are associates. Then there exists a unit $u\in R$ such that $a=ub$. Clearly, $b\mid a$ and since $u$ is a unit, then $u^{-1}a=b$ and thus $a\mid b$. By part (i) it follows that $(a)=(b)$. Now assume $(a)=(b)$. Then $a\in (b)$ and $b\in (a)$. Thus, there exists $x_1,x_2\in R$ such that $a=bx_1$ and $b=ax_2$. Hence, $a=(ax_2)x_1$ and so $a-ax_2x_1=0$, thus $a(1-x_2x_1)=0$. TBC ...
        \end{proof}
        
    \begin{theorem}\label{thm:2.4}
        If $R$ is a principle ideal ring and $(a_1)\subseteq(a_2)\subseteq\cdots$ is a chain of ideals in $R$, then for some positive integer $n$, $(a_j)=(a_n)$, for all $j\geq n$.
    \end{theorem}
        \begin{proof}
            Let $A=\bigcup\limits_{i\geq 1} (a_i)$. We claim that $A$ is an ideal. If $b,c\in A$, then for some $i,j\geq 1$, $b\in (a_i)$ and $c\in (a_j)$. Either $i\leq j$ or $i\geq j$; say $i\geq j$. Consequently, $(a_j)\subseteq (a_i)$ and $b,c\in (a_i)$. Since $(a_i)$ is an ideal, then by Definition \ref{df:1.8}, $((a_i),+)$ is a subgroup of $R$ and so $b-c\in (a_i)\subseteq A$. Similarly, if $r\in R$ and $b\in A$, then for some $i\geq 1$, $b\in (a_i)$ and since $(a_i)$ is an ideal then $rb\in(a_i)\subseteq A$ and $br\in(a_i)\subseteq A$. Thus, for all $b,c\in A$ we have that $b-c\in A$ and so $A$ is a subring of $R$, and for all $r\in R$ and $b\in A$ we have that $rb\in A$ and $br\in A$. Thus, $A$ is an ideal of $R$. By hypothesis, $A$ is principle, say $A=(a)$. Since $a\in A$, then $a\in(a_n)$ for some $n\geq 1$. Thus, if $x\in (a)$, then $x=ad$ for some $d\in R$ and since $a\in (a_n)$, then $a=a_nt$ for $t\in R$, thus $a=a_n(dt)$ and hence, $d\in (a_n)$. Thus, $(a)\subseteq (a_n)$. Therefore, for every $j\geq n$, it follows that $(a)\subseteq (a_n)\subseteq (a_j)\subseteq A=(a)$. Thus, $(a_n)=(a_j)$.
        \end{proof}
        
    \begin{theorem}\label{thm:2.5}
        If $R$ is an integral domain, then every prime element is irreducible.
    \end{theorem}
        \begin{proof}
            Let $p\in R$ be a prime element. Then by Definition \ref{df:1.18} $p$ is a nonzero nonunit, and if $p\mid ab$ then $p\mid a$ or $p\mid b$. Suppose for some $a,b\in R$ that $p=ab$. Then since $1p=ab$ it follows that $p\mid ab$. Thus, $p\mid a$ or $p\mid b$, say $p\mid a$. Then $a=px$ for some $x\in R$. Thus, $p=ab=pxb$ and so $p(1-xb)=0$. Since $p$ is prime then $p\neq 0$. Thus, $1-xb=0$. Hence, $xb=1$ and $b\mid 1$. Thus, $b$ is a unit. Therefore, $p$ is irreducible.
        \end{proof}
        
    \begin{theorem}\label{thm:2.6}
        If $R$ is a principle ideal domain, then every element $p\in R$ is prime iff $p$ is irreducible.
    \end{theorem}
        \begin{proof}
        
        \end{proof}
        
    \begin{theorem}\label{thm:2.7}
        Every principle ideal domain is a unique factorization domain.
    \end{theorem}
        \begin{proof}
            Assume that $R$ is a PID, $a\in R$, $a\neq 0$, and suppose that $a$ cannot be written as a product of irreducible elements of $R$. Then for all $i\in\mathbb{N}$, there exists $a_i,c_i\in R$ such that $a=a_1c_1$, $a_1=a_2c_2$, $\dots$, not units. In general $a_n=a_{a+1}c_{n+1}$. Note that since $a_1$ is not irreducible, then $a_2$ and $c_2$ are not units. Since $a_n=a_{n+1}c_{n+1}$, then $a_n\in (a_{n+1})$ and thus $(a_n)\subseteq (a_{n+1})$. Now assume $(a_{n+1})\subseteq (a_n)$. Then $a_{n+1}\in (a_n)$ and so $a_{n+1}=a_nd=(a_{n+1}c_{n+1})d$ implies $1=c_{n+1}d$ and so $c_{n+1}\mid 1$. Thus, by Definition \ref{df:1.14}, $c_{n+1}$ is a unit. This is a contradiction. Thus, $(a_{n+1})$ is not contained in $(a_n)$. Hence, $(a_n)\subsetneq (a_{n+1})$. Thus, for all $i\in\mathbb{N}$ we have that $(a_1)\subsetneq(a_2)\subsetneq\cdots$. Thus, by Theorem \ref{thm:2.4}, $R$ is not a PID, which is a contradiction.\par\hspace{4mm} Now assume that $a=p_1p_2\cdots p_n=q_1q_2\cdots q_m$ where for all $i$, $p_i$ and $q_i$ are irreducible. We need to prove that $n=m$ and that (when reordered) for all $i$, there exists a unit $u_i$ such that $p_i=u_iq_i$. $R$ is a PID, thus for all $i$, $p_i,q_i$ are prime.\par\hspace{4mm} By induction on $n$ (WLOG assume $m\geq n$). Assume $n=1$. Then $a=p_1=q_1\cdots q_m$, thus $p_1\mid q_1\cdots q_m$. Since $p_1$ is prime, then (WLOG), $p_1\mid q_1$. Thus, there exists $c_1$ such that $q_1=c_1p_1$. Since $q_1$ is irreducible, then $c_1$ is a unit. Then $p_1=u_1p_1\cdots q_m$. Thus, $1=uq_2\cdots q_m$ which implies $q_2,\dots,q_m$ are units and hence not primes. Thus, if $n=1$, then $m=1$ and $p_1=u_1q_1$.\par\hspace{4mm} Assume true for $k\geq 1$ and assume that $a=p_1\cdots p_kp_{k+1}=q_1\cdots q_m$ (where $m\geq k+1$ thus $m-1\geq 1$). We have that $p_{k+1}\mid q_1\cdots q_m$, and since $p_{k+1}$ is prime then (WLOG) $p_{k+1}\mid q_m$. But $q_m$ is irreducible, thus there exists a unit $u_{k+1}$ such that $q_m=u_{k+1}p_{k+1}$. Thus, $p_1\cdots p_{k+1}=q_1\cdots u_{k+1}p_{k+1}$ and so $p_1\cdots p_k=q_1\cdots q_{m-1}u_{k+1}$. By the inductive assumption, $k=m-1$ and for all $i$ there exists a unit $u_i$ such that $q_i=u_ip_i$. Thus $k+1=m$ and for all $i$ there exists a unit such that $q_i=u_ip_i$ and thus the statement is true for $k+1$.
        \end{proof}
        
    \begin{theorem}
        An ideal $(p(x))\neq \{0\}$ of $F[x]$ is maximal if and only if $p(x)$ is irreducible over $F$.
    \end{theorem}
        \begin{proof}
            Suppose $(p(x))\neq \{0\}$ and is a maximal ideal of $F[x]$. Then $(p(x))\neq F[x]$, so $p(x)\notin F$. This is because assuming $(p(x))\neq \{0\}$ means that $p(x)\neq 0$ and so if $p(x)\in F$, then $p(x)$ would be a unit. This would imply that $1\in (p(x))$ and so $1=p(x)q(x)$  (note that $f(x)=p(x)q(x)f(x)\in (p(x))$ for all $f(x)\in F[x]$) and so $(p(x))=F[x]$ which contradicts our assumption that $(p(x))$ is maximal.\par\hspace{4mm} Now let $p(x)=f(x)g(x)$ be a factorization of $p(x)$ in $F[x]$. Since $(p(x))$ is a maximal ideal, then by (??) it is a prime ideal. And so for $p(x)=f(x)g(x)\in (p(x))$ means that either $f(x)\in (p(x))$ or $g(x)\in(p(x))$; that is, either $f(x)$ or $g(x)$ has $p(x)$ as a factor. WLOG assume that $f(x)=p(x)q(x)$, then
                \begin{equation*}
                    p(x)=(p(x)q(x))g(x)\rightarrow p(x)(1-p(x)q(x))=0.        
                \end{equation*}
            By assumption $p(x)\neq 0$ and so $1-p(x)q(x)=0$. Thus, either $p(x)$ or $q(x)$ is a unit. As detailed earlier, $p(x)$ is not a unit and so $q(x)$ is a unit. Thus, $f(x)=\mu p(x)$ for some nonzero $\mu\in F$. Thus, $f(x)$ is irreducible and $\text{deg}(f(x))=\text{deg}(p(x))$. This implies that
                \begin{equation*}
                    \text{deg}(p(x))=\text{deg}(f(x)g(x))=\text{deg}(f(x))+\text{deg}(g(x))=\text{deg}(p(x))+\text{deg}(g(x))
                \end{equation*}
             and so $\text{deg}(g(x))=0$. Thus, $g(x)$ is a unit and $p(x)$ is irreducible over $F$.\par\hspace{4mm} Conversely, if $p(x)$ is irreducible, then supposing $N$ is an ideal of $F[x]$ such that $(p(x))\subseteq N\subseteq F[x]$ it follows that $N$ is principle and so let $(h(x))=N$. Then we have that $p(x)\in (h(x))$ and so for some $q(x)\in F[x]$, $p(x)=h(x)q(x)$. However, sine $p(x)$ is irreducible, then either $h(x)$ or $q(x)$ is a unit. If $h(x)$ is a unit, then $(h(x))=N=F[x]$. If $q(x)$ is a unit, then $p(x)=\mu h(x)$ and so $\mu^{-1}p(x)=h(x)$. Thus, $h(x)\in(p(x))$. Hence, $(p(x))=(h(x))=N$. Therefore, $(p(x))$ is maximal.  
        \end{proof}
    Recall that for a given field $E$ and a field $F\subseteq E$ and $c\in E$, then $F[c]$ is defined to be the smallest subring of $E$ which contains $F$ and $c$ and we proved (??) that $F[c]=\{f(c)\colon f(x)\in F[x]\}$. Let $F(c)$ denote the field of quotients of $F[c]$ and this is the smallest field that contains $F$ and $c$. We will now determine how $F(c)$ relates to
    \newpage
    \begin{prop}\label{prop:2.1}
        Let $F$ be a field, $f(x)\in F[x]$, and $E/F$ be the splitting field for $f(x)$. Then $f(x)$ has a multiple root in E \textit{iff} $f(x)$ and $f'(x)$ share a common factor $d(x)\in E[x]$, where $\text{deg}(d(x))\geq 1$. 
    \end{prop}
        \begin{proof*}
            \emph{Assume that $f(x)$ has a multiple root. Then for some $a\in E$ and $k>1$, $f(x)=(x-a)^kg(x)$, where $g(x)\in F[x]$. Taking the derivative of $f(x)$ gives:
                \begin{equation*}
                    \begin{split}
                        f'(x)&=k(x-a)^{k-1}+(x-a)g'(x) \\
                        &= (x-a)(k(x-a)^{k-2}+(x-a)^{k-1}g'(x).
                    \end{split}
                \end{equation*}
                Thus $f(x)$ and $f'(x)$ share a common factor $(x-a)^k$ where $k\geq 1$.}\par\hspace{4mm}
            \emph{Conversely, if $f(x)$ has no multiple root, then for all $a\in E$ with $f(a)=0$, there exists $g_a(x)\in F[x]$ such that $f(x)=(x-a)g_a(x)$ and $(x-a)\nmid g_a(x)$. Note that if $(x-a)\mid g(x)$, then for some $k>1$, $(x-a)^k\mid f(x)$ which contradicts our converse assumption. Taking the derivative of $f(x)$ gives
                \begin{equation*}
                    f(x)=g(x)+(x-a)g'(x),
                \end{equation*}
            which is not a multiple of $x-a$. Since $a\in E$ was an arbitrary root of $f(x)$, then $f(x)$ and $f'(x)$ share no common factors.
                }
        \end{proof*}
    \begin{prop}
        Let $F$ be a field with char$(F)=0$ and $f(x)\in F[x]$ irreducible over $F$. Then $f(x)$ has no roots of multiplicity greater than 1.
    \end{prop}
        \begin{proof*}
            \emph{For contradiction, assume that $f(x)$ has a root of multiplicity greater than 1. Then by} Proposition \ref{prop:2.1}\emph{, $f(x)$ and $f'(x)$ share a common factor $d(x)$ with degree $\geq 1$. Thus, for $h(x)\in F[x]$, we have that $f(x)=d(x)h(x)$ and $f'(x)=d'(x)h(x)+d(x)h'(x)$. However, since $d(x)\mid f'(x)$, then $d(x)$ divides $d'(x)h(x)$. But $f(x)$ is irreducible and so $h(x)=\pm 1$. Thus, $\text{gcd}(d(x),h(x))=1$ and by Euclid's Lemma, $d(x)\mid d'(x)$. Since deg$(d'(x))\leq\text{deg}(d(x))$, then this this is only possible if $d(x)\in F$ and since $h(x)=\pm 1$, then $f(x)=\pm d(x)\in F$. This contradicts our assumption since for any root, $a$, of $f(x)$, $(x-a)\nmid f(x)$ over $F(a)$ and thus $f(x)$ cannot have a root of multiplicity greater than 1.
            }
        \end{proof*}
\end{document}