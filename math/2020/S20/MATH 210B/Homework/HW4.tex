\documentclass[12pt]{article}
\usepackage[margin=1in]{geometry} 
\usepackage{graphicx}
\usepackage{amsmath}
\usepackage{authblk}
\usepackage{titlesec}
\usepackage{amsthm}
\usepackage{amsfonts}
\usepackage{amssymb}
\usepackage{array}
\usepackage{booktabs}
\usepackage{ragged2e}
\usepackage{enumerate}
\usepackage{enumitem}
\usepackage{cleveref}
\usepackage{slashed}
\usepackage{commath}
\usepackage{lipsum}
\usepackage{colonequals}
\usepackage{addfont}
\usepackage{enumitem}
\usepackage{sectsty}
\usepackage{lastpage}
\usepackage{fancyhdr}
\usepackage{accents}
\usepackage{xcolor}
\usepackage[inline]{enumitem}
\pagestyle{fancy}

\fancyhf{}
\rhead{Darcy}
\lhead{MATH 210B}
\rfoot{\thepage}
\setlength{\headheight}{10pt}

\subsectionfont{\itshape}

\newtheorem{theorem}{Theorem}[section]
\newtheorem{corollary}{Corollary}[theorem]
\newtheorem{prop}{Proposition}[section]
\newtheorem{lemma}[theorem]{Lemma}
\theoremstyle{definition}
\newtheorem{definition}{Definition}[section]
\theoremstyle{remark}
\newtheorem*{remark}{Remark}
 
\makeatletter
\renewenvironment{proof}[1][\proofname]{\par
  \pushQED{\qed}%
  \normalfont \topsep6\p@\@plus6\p@\relax
  \list{}{\leftmargin=0mm
          \rightmargin=4mm
          \settowidth{\itemindent}{\itshape#1}%
          \labelwidth=\itemindent
          \parsep=0pt \listparindent=\parindent 
  }
  \item[\hskip\labelsep
        \itshape
    #1\@addpunct{.}]\ignorespaces
}{%
  \popQED\endlist\@endpefalse
}

\newenvironment{solution}[1][\bf{\textit{Solution}}]{\par
  
  \normalfont \topsep6\p@\@plus6\p@\relax
  \list{}{\leftmargin=0mm
          \rightmargin=4mm
          \settowidth{\itemindent}{\itshape#1}%
          \labelwidth=\itemindent
          \parsep=0pt \listparindent=\parindent 
  }
  \item[\hskip\labelsep
        \itshape
    #1\@addpunct{.}]\ignorespaces
}{%
  \popQED\endlist\@endpefalse
}

\let\oldproofname=\proofname
\renewcommand{\proofname}{\bf{\textit{\oldproofname}}}


\newlist{mylist}{enumerate*}{1}
\setlist[mylist]{label=(\alph*)}

\begin{document}\thispagestyle{empty}\hline

\begin{center}
	\vspace{.4cm} {\textbf { \large MATH 210B}}
\end{center}
{\textbf{Name:}\ Quin Darcy \hspace{\fill} \textbf{Due Date:} 2/19/20   \\
{ \textbf{Instructor:}}\ Dr. Shannon \hspace{\fill} \textbf{Assignment:} Homework 4 \\ \hrule}

\justifying

    \begin{enumerate}[leftmargin=*]
        \item Prove that $\mathbb{Q}(i,\sqrt{2})=\mathbb{Q}(\sqrt{i})$.
            \begin{proof}
                We begin by noting that a basis for $\mathbb{Q}(i,\sqrt{2})$ is $B_1=\{1,i,\sqrt{2},\sqrt{2}i\}$ and a basis for $\mathbb{Q}(\sqrt{i})$ is $B_2=\{1,\sqrt{i},i\}$. Now since $\mathbb{Q}(i,\sqrt{2})$ is a field, then both $1/\sqrt{2}$ and $i/\sqrt{2}$ are elements. Moreover, it is closed under addition and multiplication. With this in mind consider
                    \begin{equation*}
                        \bigg(\frac{1}{\sqrt{2}}+\frac{i}{\sqrt{2}}\bigg)^2=\bigg(\frac{1+2i-1}{2}\bigg)=i.
                    \end{equation*}
                From this is follows that 
                    \begin{equation*}
                        \sqrt{i}=\frac{1}{\sqrt{2}}+\frac{i}{\sqrt{2}}\in\mathbb{Q}(i,\sqrt{2}).
                    \end{equation*}
                Thus, $B_2\subseteq\mathbb{Q}(i,\sqrt{2})$ and hence $\mathbb{Q}(\sqrt{i})\subseteq\mathbb{Q}(i,\sqrt{2})$. Similarly, we have shown that
                    \begin{equation*}
                        \sqrt{i}-\frac{i}{\sqrt{2}}=\frac{1}{\sqrt{2}}\in\mathbb{Q}(\sqrt{i}),
                    \end{equation*}
                and so this element has a multiplicative inverse, namely, $\sqrt{2}\in\mathbb{Q}(i)$. Lastly, since $\sqrt{2}(\sqrt{i})^2=\sqrt{2}i\in\mathbb{Q}(\sqrt{i})$, then $B_1\subseteq\mathbb{Q}(\sqrt{i})$ and thus $\mathbb{Q}(i,\sqrt{2})\subseteq\mathbb{Q}(\sqrt{i})$. Therefore, $\mathbb{Q}(i,\sqrt{2})=\mathbb{Q}(\sqrt{i})$.
            \end{proof}
        \item[3.] The M\"obious function, $\mu$, is defined by the following: $\mu(1)=1$; if $t>1$, then let $t=p_1^{r_1}\cdots p_m^{r_m}$ be the prime factorization of $t$. $\mu(t)=(-1)^m$, if $r_i=1$ for all $i$, $1\leq i\leq m$, and $\mu(t)=0$, if $r_i>1$ for some $i$. Then
            \begin{equation*}
                \Phi_n(x)=\prod_{d\mid n}(x^{n/d}-1)^{\mu(d)}.
            \end{equation*}
        Using this result, find $\Phi_3$, $\Phi_4$, $\Phi_5$, $\Phi_6$.
            \begin{solution}
                Using the given formula, we get:
                    \begin{enumerate}[label=]
                        \item $\Phi_3(x)=\prod\limits_{d\mid 3}(x^{3/d}-1)^{\mu(d)}=(x^3-1)(x-1)^{-1}=x^2+x+1$.
                        \item $\Phi_3(x)=\prod\limits_{d\mid 4}(x^{4/d}-1)^{\mu(d)}=(x^4-1)(x^2-1)^{-1}=x^2+1$.
                        \item $\Phi_5(x)=\prod\limits_{d\mid 5}(x^{5/d}-1)^{\mu(d)}=(x^5-1)(x-1)^{-1}=x^4+x^3+x^2+x+1$.
                        \item $\Phi_6(x)=\prod\limits_{d\mid 6}(x^{6/d}-1)^{\mu(d)}=\frac{x^7-x^6-x+1}{x^5-x^3-x^2+1}$.
                    \end{enumerate}
            \end{solution}\newpage
        \item[4.] Assume that $F$ is a finite field.
            \begin{enumerate}[label=(\alph*)]
                \item Explain (one sentence) why $\text{char}(F)$ is a prime number, $p$, and why $F$ contains a subfield that is isomorphic to $\mathbb{Z}_p$.
                    \begin{solution}
                        Since $F$ is a field, then it has no zero divisors and so if the characteristic were composite, say $ab$, then either $a1_F$ or $b1_F$ is zero, contradicting the definition of characteristic. Moreover, the set $\{a1_F\colon a\in\mathbb{Z}\}$ is a subfield of $F$ with order $p$, and thus isomorphic to $\mathbb{Z}_p$. 
                    \end{solution}
                \item By $(a)$, $F$ is a vector space over $\mathbb{Z}_p$. Assume that $[F\colon\mathbb{Z}_p]=n$. Determine (with proof) $\abs{F}$.
                    \begin{proof}
                        Let $\{r_0,\dots, r_{n-1}\}$ be a basis for $F$ over $\mathbb{Z}_p$ and consider any $w\in F$. Then for $a_0,\dots,a_{n-1}\in\mathbb{Z}_p$, we have 
                            \begin{equation*}
                                w=a_or_0+\cdots+a_{n-1}r_{n-1}.
                            \end{equation*}
                        since there are $n$ many basis elements and $p$ many choices for each coefficient, then it follows that there are $p^n$ many elements in $F$. Thus, $\abs{F}=p^n$. 
                    \end{proof}
                \item Give a field with 125 elements. 
                    \begin{solution}
                        By HW3, $x^3+x+1$ is irreducible over $\mathbb{Z}_5$. Thus, $\mathbb{Z}_5/(x^3+x+1)_i$ is a field. We know that $\{I+1,I+x,I+x^2\}$ is a basis and so $I+(a+bx+cx^2)$ represents any elements of this field. There are 5 choices for $a,b$ and $c$. Thus, there are $5^3=125$ elements in this field.
                    \end{solution}
            \end{enumerate}
        \item[5.] Factor $x^3-2$ into irreducible factors over $\mathbb{Q}$, over $\mathbb{R}$, over $\mathbb{C}$, over $\mathbb{Z}_3$, and over $\mathbb{Z}_5/(x^2+3x+4)_i$.
            \begin{solution}\hfill\par
                \begin{enumerate}[label=]
                    \item $\mathbb{Q}$: $x^3-2$;
                    \item $\mathbb{R}$: $(x-2^{1/3})(x^2+2^{1/3}x+2^{2/3})$;
                    \item $\mathbb{C}$: $(x-2^{1/3})(x+\frac{1-i\sqrt{3}}{2^{2/3}})(x+\frac{1+i\sqrt{3}}{2^{2/3}})$;
                    \item $\mathbb{Z}_3$: $(x-2)^2(x+1)$; 
                    \item $\mathbb{Z}_5/(x^2+3x+4)_i$: $(I+3)(I+x)(I+4x+2)$.
                \end{enumerate}
            \end{solution}
        \item[6.] Assume that $E$ is a field and $F$ is a subfield of $E$. Let $K=\{a\in E\colon a$ is algebraic over $F\}$. Prove that $K$ is a subfield of $E$ that contains $F$.
            \begin{proof}
                To begin we note that for any $a\in F$, $a$ is algebraic over $F$ since $a$ is a root of $x-a\in F[x]$. Thus, $F\subseteq K$. Now let $a,b\in K$. Then $x-a\mid f(x)$ and $x-b\mid g(x)$ for some $f(x),g(x)\in F[x]$. Thus, for $q_1(x),q_2(x)\in F[x]$, we have that 
                    \begin{equation*}
                        f(x)=q_1(x)(x-a)\quad\text{and}\quad g(x)=q_2(x)(x-b).
                    \end{equation*}
                Since $F[x]$ is a ring, then the addition and multiplication of any of the nonzero polynomials in $F[x]$ will result in another element of $F[x]$. From this we can first show that $-g(x)+2xq_2(x)=q_2(x)(x+b)$ and so $-b\in K$. Furthermore, it follows that 
                    \begin{equation*}
                        f(x)+g(x)+q_1(x)a-q_2(x)b+(q_1(x)+q_2(x))(-a+b)=(q_1(x)+q_2(x))(x-(a-b)).
                    \end{equation*}
                Thus, there is some polynomial in $F[x]$ for which $(x-(a-b))$ is a factor. Hence, $a-b\in K$. Now we must show that $ab^{-1}\in K$. First, we observe that if $g(x)=c_0+c_1x+\dots+c_nx^n$, then $g(b)=c_0+c_1b+\dots+c_nb^n=0$. Multiplying both sides by $b^{-n}$, we get $c_0b^{-n}+c_1b^{-n+1}+\cdots+c_n=0$. Hence, for the polynomial $g(x)b^n\in F[x]$, $b^{-1}$ is a root and thus $b^{-1}\in K$. Finally, since $ab^{-1}$ is a root of $bx-a\in F[x]$, then $ab^{-1}\in K$. It also follows that since $E$ is a field and $K\subseteq E$, then $K$ has no zero divisors. Thus, $K$ is a commutative ring with identity, no zero divisors and every nonzero element has a multiplicative inverse. Therefore, $K$ is a subfield of $E$.
            \end{proof}
        \item [7.] Prove that if $[E\colon F]$ is finite, then every element of $E$ is algebraic over $F$.
            \begin{proof}
                By Theorem 2, for all $c\in E$, $c$ is algebraic over $F$ iff $[F(c)\colon F]$ is finite. Thus, since $F(c)\subseteq E$ for all $c\in E$, then $[F(c)\colon F]$ is finite for all $c\in E$. Therefore, for all $c\in E$, $c$ is algebraic over $F$.  
            \end{proof}
        \item[8.] Find the minimal polynomial over $\mathbb{Q}$ of each of the following:
            \begin{enumerate}
                \item $3+\sqrt{2}$.
                    \begin{solution}
                        To begin we let $x=3+\sqrt{2}$ and so $(x-3)^2=2$. Thus, $x^2-6x+7=0$. Letting $p=3$, then by the $\mathbb{Z}_p$ test, we see that $\varphi(x^2-6x+7)=x^2+[1]$. This does not have a root in $\mathbb{Z}_3$ and is therefore irreducible over $\mathbb{Z}_3$ and therefore over $\mathbb{Q}$. Thus, $x^2-6x+7$ is the minimal polynomial. 
                    \end{solution}
                \item $\sqrt{-1+\sqrt{2}}$.
                    \begin{solution}
                        Letting $x=\sqrt{-1+\sqrt{2}}$, we get that $x^4+2x-1=0$. Using the $\mathbb{Z}_p$ test, with $p=3$ we get that $\varphi(x^4+2x-1)=x^4+2x+2$. It is clear that this has no linear factors since no element of $\mathbb{Z}_3$ is a root for the polynomial. Thus, if it is reducible, then for some $a,b,c,d,e,f\in\mathbb{Z}_3$ we have 
                            \begin{equation*}
                                x^4+2x+2=(ax^2+bx+c)(dx^2+ex+f). 
                            \end{equation*}
                        This yields the following conditions
                            \begin{enumerate}[label=\roman*.]
                                \item $ad=1$
                                \item $ae+bd=0$
                                \item $af+be+cd=2$
                                \item $bf+ce=0$
                                \item $cf=2$
                            \end{enumerate}
                        With some calculation we can determine that there are 8 possible polynomials in $\mathbb{Z}_3$ that satisfy the first three conditions, namely (when written as an ordered 6-tuples e.g $2x+1$ is $(2,1)$), are $(1,2,2,1,1,1),(1,2,0,1,1,0),(1,0,0,1,0,2)$\par,$(1,0,1,1,0,1),(2,2,2,2,1,1),(2,2,0,2,1,0),(2,0,1,2,0,0),(2,0,2,2,0,2)$. However, all of these polynomials either fail conditions iv. or v. Therefore, there is no factors of $x^4+2x+2$ in $\mathbb{Z}_3$. Thus, the polynomial is irreducible in $\mathbb{Z}_3$ and therefore irreducible in $\mathbb{Q}$. Thus, $x^4+2x-1$ is the minimal polynomial of $\sqrt{-1+\sqrt{2}}$.
                    \end{solution}
            \end{enumerate}
        \item[10.] Prove that $\mathbb{Q}(\sqrt{3},\sqrt{5})=\mathbb{Q}(\sqrt{3}+\sqrt{5})$.
            \begin{proof}
                It follows immediately that $\mathbb{Q}(\sqrt{3}+\sqrt{5})\subseteq\mathbb{Q}(\sqrt{3},\sqrt{5})$. Thus we must show that $\sqrt{3},\sqrt{5}\in\mathbb{Q}(\sqrt{3}+\sqrt{5})$. Since $\mathbb{Q}(\sqrt{3}+\sqrt{5})$ is a field, then $(\sqrt{3}+\sqrt{5})^{-1}$ is an element of this field. Thus, 
                    \begin{equation*}
                        \begin{split}
                            (\sqrt{3}+\sqrt{5})^{-1} &= \frac{1}{\sqrt{3}+\sqrt{5}} \\
                            &= \frac{1}{\sqrt{3}+\sqrt{5}}\frac{(\sqrt{3}-\sqrt{5})}{(\sqrt{3}-\sqrt{5})} \\
                            &= -\frac{1}{2}\sqrt{3}+\frac{1}{2}\sqrt{5}.
                        \end{split}
                    \end{equation*}
                And so $-\frac{1}{2}(\sqrt{3}-\sqrt{5})\in\mathbb{Q}(\sqrt{3}+\sqrt{5})$. $(-2)(-\frac{1}{2}(\sqrt{3}-\sqrt{5}))=\sqrt{3}-\sqrt{5}\in\mathbb{Q}(\sqrt{3}+\sqrt{5})$. This implies $\sqrt{3}-\sqrt{5}+\sqrt{3}+\sqrt{5}=2\sqrt{3}\in\mathbb{Q}(\sqrt{3}+\sqrt{5})$ and so $\sqrt{3}\in\mathbb{Q}(\sqrt{3}+\sqrt{5})$. A similar argument shows that $\sqrt{5}\in\mathbb{Q}(\sqrt{3}+\sqrt{5})$. Therefore, $\mathbb{Q}(\sqrt{3},\sqrt{5})=\mathbb{Q}(\sqrt{3}+\sqrt{5})$. 
            \end{proof}
    \end{enumerate}
\end{document}