\documentclass[12pt]{article}
\usepackage[margin=1in]{geometry} 
\usepackage{graphicx}
\usepackage{amsmath}
\usepackage{authblk}
\usepackage{titlesec}
\usepackage{amsthm}
\usepackage{amsfonts}
\usepackage{amssymb}
\usepackage{array}
\usepackage{booktabs}
\usepackage{ragged2e}
\usepackage{enumerate}
\usepackage{enumitem}
\usepackage{cleveref}
\usepackage{slashed}
\usepackage{commath}
\usepackage{lipsum}
\usepackage{colonequals}
\usepackage{addfont}
\usepackage{enumitem}
\usepackage{sectsty}
\usepackage{lastpage}
\usepackage{fancyhdr}
\usepackage{accents}
\usepackage{xcolor}
\usepackage[inline]{enumitem}
\pagestyle{fancy}

\fancyhf{}
\rhead{Darcy}
\lhead{MATH 210B}
\rfoot{\thepage}
\setlength{\headheight}{10pt}

\subsectionfont{\itshape}

\newtheorem{theorem}{Theorem}[section]
\newtheorem{corollary}{Corollary}[theorem]
\newtheorem{prop}{Proposition}[section]
\newtheorem{lemma}[theorem]{Lemma}
\theoremstyle{definition}
\newtheorem{definition}{Definition}[section]
\theoremstyle{remark}
\newtheorem*{remark}{Remark}
 
\makeatletter
\renewenvironment{proof}[1][\proofname]{\par
  \pushQED{\qed}%
  \normalfont \topsep6\p@\@plus6\p@\relax
  \list{}{\leftmargin=0mm
          \rightmargin=4mm
          \settowidth{\itemindent}{\itshape#1}%
          \labelwidth=\itemindent
          \parsep=0pt \listparindent=\parindent 
  }
  \item[\hskip\labelsep
        \itshape
    #1\@addpunct{.}]\ignorespaces
}{%
  \popQED\endlist\@endpefalse
}

\newenvironment{solution}[1][\bf{\textit{Solution}}]{\par
  
  \normalfont \topsep6\p@\@plus6\p@\relax
  \list{}{\leftmargin=0mm
          \rightmargin=4mm
          \settowidth{\itemindent}{\itshape#1}%
          \labelwidth=\itemindent
          \parsep=0pt \listparindent=\parindent 
  }
  \item[\hskip\labelsep
        \itshape
    #1\@addpunct{.}]\ignorespaces
}{%
  \popQED\endlist\@endpefalse
}

\let\oldproofname=\proofname
\renewcommand{\proofname}{\bf{\textit{\oldproofname}}}


\newlist{mylist}{enumerate*}{1}
\setlist[mylist]{label=(\alph*)}

\begin{document}\thispagestyle{empty}\hline

\begin{center}
	\vspace{.4cm} {\textbf { \large MATH 210B}}
\end{center}
{\textbf{Name:}\ Quin Darcy \hspace{\fill} \textbf{Due Date:} 3/4/20   \\
{ \textbf{Instructor:}}\ Dr. Shannon \hspace{\fill} \textbf{Assignment:} Homework 5 \\ \hrule}

\justifying

    \begin{enumerate}[leftmargin=*]
        \item[4.] Assume that in the equation of a line, $y=mx+b$, $m,b\in\mathbb{Q}$, and that in the equation of a circle $(x-h)^2+(y-k)^2=r^2$, $h,k,r\in\mathbb{Q}$. Assuming that the line and the circle intersect, explain why the coordinates of the points of intersection are elements of an extension, $\mathbb{Q}(\sqrt{g})$, of $\mathbb{Q}$, where $[\mathbb{Q}(\sqrt{g})\colon\mathbb{Q}]=2$ or $=1$.
            \begin{solution}
                Given that $y=mx+b$, we can substitute this in for the equation of the circle. Obtaining,
                    \begin{equation*}
                        \begin{split}
                            (x-h)^2+(y-k)^2-r^2 &= \big[x^2-2hx+h^2\big]+\big[y^2-2ky+k^2\big]-r^2\\
                            &=\big[x^2-2hx+h^2\big]+\big[(mx+b)^2-2k(mx+b)+k^2\big]-r^2 \\
                            &=(1+m^2)x^2+(2mb-2mk-2h)x+(h^2-2bk+k^2-r^2) \\
                            &= 0.
                        \end{split}
                    \end{equation*}
                Using the quadratic formula to solve for $x$ we get
                    \begin{equation*}
                        x=\frac{-(2mb-2mk-2h)\pm\sqrt{(2mb-2mk-2h)^2-4(1+m^2)(h^2-2bk+k^2-r^2)}}{2(1+m^2)}.
                    \end{equation*}
                 Let $g=(2mb-2mk-2h)^2-4(1+m^2)(h^2-2bk+k^2-r^2)$. Then if $\sqrt{g}\in\mathbb{Q}$, we have that $x\in\mathbb{Q}$ and $[\mathbb{Q}(\sqrt{g})\colon\mathbb{Q}]=1$. Otherwise, we have that $\sqrt{g}\in\mathbb{Q}(\sqrt{g})$ and since 
                    \begin{equation*}
                        \frac{-(2mb-2mk-2h)}{2(1+m^2)}=\frac{-mb+mk+h}{1+m^2}\in\mathbb{Q}
                    \end{equation*}
                and $2(1+m^2)\in\mathbb{Q}$, then both of these rational numbers are elements of $\mathbb{Q}(\sqrt{g})$. Hence, $x\in\mathbb{Q}(\sqrt{g})$. Moreover, if $\sqrt{g}\notin\mathbb{Q}$, then $[\mathbb{Q}(\sqrt{g})\colon\mathbb{Q}]=2$ since $\{1,\sqrt{g}\}$ is a basis for $\mathbb{Q}(\sqrt{g})$ over $\mathbb{Q}$. By the same argument we can show that $y\in\mathbb{Q}(\sqrt{g})$. Therefore, the coordinates of the two intersection points of the line with the circle are elements of $\mathbb{Q}(\sqrt{g})$.  
            \end{solution}\newpage
        \item[5.] Recall that an automorphism of $F$ is an isomorphism of $F$ onto $F$. Find all the automorphisms of $\mathbb{Q}(\sqrt[3]{2},\omega)$.
            \begin{solution}
                We have that $\sqrt[3]{2}$ is algebraic over $\mathbb{Q}$ since it is a root of $x^3-2$. Thus, $\mathbb{Q}(\sqrt[3]{2})\cong\mathbb{Q}[x]/(x^3-2)_i$. Now note that a basis for $\mathbb{Q}[x]/(x^3-2)_i$ is $\{(x^3-2)_i+1,(x^3-2)_i+x,(x^3-2)_i+x^2\}$. It follows from this that $[\mathbb{Q}(\sqrt[3]{2})\colon\mathbb{Q}]=3$. Additionally, a basis for this extension is $\{1,\sqrt[3]{2},\sqrt[3]{4}\}$. Similarly, $\omega$ is algebraic over $\mathbb{Q}$ since it is a root of $x^2+x+1$ and by the same reasoning as above, it follows that $[\mathbb{Q}(\omega)\colon\mathbb{Q}]=2$ and a basis for this extension is $\{1,\omega\}$. Thus, $[\mathbb{Q}(\sqrt[3]{2},\omega)\colon\mathbb{Q}]=6$ and a basis for this extension is $\{1,\sqrt[3]{2},\sqrt[3]{4},\omega,\omega\sqrt[3]{2},\omega\sqrt[3]{4}\}$. Now consider some $\alpha\in\mathbb{Q}(\sqrt[3]{2},\omega)$. Given our basis, it can be expressed as 
                    \begin{equation*}
                        \alpha=a+b\sqrt[3]{2}+c\sqrt[3]{4}+d\omega+e\omega\sqrt[3]{2}+f\omega\sqrt[3]{4},
                    \end{equation*}
                for $a,b,c,d,e,f\in\mathbb{Q}$. Now let $\sigma\colon\mathbb{Q}(\sqrt[3]{2},\omega)\rightarrow\mathbb{Q}(\sqrt[3]{2},\omega)$ be an automorphism. Since $\mathbb{Q}$ is a subfield of $\mathbb{Q}(\sqrt[3]{2},\omega)$ and $\sigma$ is an automorphism, then it follows that $\mathbb{Q}\cong\sigma(\mathbb{Q})$. However, by HW1, $\sigma$ can only be the identity map on $\mathbb{Q}$. In other words, $\sigma(q)=q$ for all $q\in\mathbb{Q}$. From this it follows that 
                    \begin{equation*}
                        \begin{split}
                            \sigma(\alpha) &= \sigma(a+b\sqrt[3]{2}+c\sqrt[3]{4}+d\omega+e\omega\sqrt[3]{2}+f\omega\sqrt[3]{4})\\
                            &=a+b\sigma(\sqrt[3]{2})+c\sigma(\sqrt[3]{4})+d\sigma(\omega)+e\sigma(\omega\sqrt[3]{2})+f\sigma(\omega\sqrt[2]{4}).
                        \end{split}
                    \end{equation*}
                Thus, any such automorphism is uniquely determined by where it sends the basis elements. Now if we consider the polynomial $x^3-2=(x-\sqrt[3]{2})(x-\omega\sqrt[3]{2})(x-\omega^2\sqrt[3]{2})$, and let $\theta\colon \mathbb{Q}(\sqrt[3]{2},\omega)[x]\rightarrow\mathbb{Q}(\sqrt[3]{2},\omega)[x]$ be the isomorphism given on Exam 1, then given any root, call it $\alpha$, of $x^3-2$ must give that $\sigma(\alpha)$ is a root of $\theta(x^3-2)$. Similarly, each root, $\alpha$, of $x^2+x+1=(x-\omega)(x-\omega^2)$ must correspond to a root $\sigma(\alpha)$ of the same polynomial. With these restrictions, we can conclude that there are 6 possible automorphisms:
                    \begin{align*}
                        &\sigma_1:=\begin{cases} 
                            \sqrt[3]{2}\mapsto\sqrt[3]{2}\\
                            \omega\mapsto\omega
                        \end{cases} && \sigma_2:=\begin{cases} 
                            \sqrt[3]{2}\mapsto\omega\sqrt[3]{2}\\
                            \omega\mapsto\omega
                        \end{cases} && \sigma_3:=\begin{cases} 
                            \sqrt[3]{2}\mapsto\omega^2\sqrt[3]{2}\\
                            \omega\mapsto\omega
                        \end{cases} \\
                        &\sigma_4:=\begin{cases} 
                            \sqrt[3]{2}\mapsto\sqrt[3]{2}\\
                            \omega\mapsto\omega^2
                        \end{cases} && \sigma_5:=\begin{cases} 
                            \sqrt[3]{2}\mapsto\omega\sqrt[3]{2}\\
                            \omega\mapsto\omega^2
                        \end{cases} && \sigma_6 :=\begin{cases} 
                            \sqrt[3]{2}\mapsto\omega^2\sqrt[3]{2}\\
                            \omega\mapsto\omega^2.
                        \end{cases}
                    \end{align*}
            \end{solution}
        \item[6.] Find (with proof) $\gamma$ such that $\mathbb{Q}(\gamma)=\mathbb{Q}(\sqrt{2},\sqrt[3]{5})$.
            \begin{proof}
                Consider $\mathbb{Q}(\sqrt{2}\sqrt[3]{5})$. Since $\sqrt{2},\sqrt[3]{5}\in\mathbb{Q}(\sqrt{2},\sqrt[3]{5})$ and this set is closed under multiplication, then $\sqrt{2}\sqrt[3]{5}\in\mathbb{Q}(\sqrt{2},\sqrt[3]{5})$. Thus, $\mathbb{Q}(\sqrt{2}\sqrt[3]{5})\subseteq\mathbb{Q}(\sqrt{2},\sqrt[3]{5})$. Next, we note that $(\sqrt{2}\sqrt[3]{5})^3=10\sqrt{2}$ and so $\sqrt{2}\in\mathbb{Q}(\sqrt{2}\sqrt[3]{5})$. Similarly, $(\sqrt{2}\sqrt[3]{5})^4=20\sqrt[3]{5}$ and so $\sqrt[3]{5}\in\mathbb{Q}(\sqrt{2}\sqrt[3]{5})$. Thus, $\mathbb{Q}(\sqrt{2},\sqrt[3]{5})\subseteq\mathbb{Q}(\sqrt{2},\sqrt[3]{5})$. Therefore, $\mathbb{Q}(\sqrt{2}\sqrt[3]{5})=\mathbb{Q}(\sqrt{2},\sqrt[3]{5})$.
            \end{proof}\newpage
        \item[7.] Assume that $K/F$, $K/E$, $K/F'$. Assume that $F'/F$ is algebraic and $E/F$. Let $E'$ be the smallest subfield of $K$ which contains $E$ and $F'$. Prove that $E'/E$ is algebraic. 
            \begin{proof}
                Let $M=\cup\{E(a_1,\dots,a_n)\colon a_i\in F', n\in\mathbb{Z}^+\}$. Then since for every $n\in\mathbb{Z}^+$, $E(a_1,\dots,a_n)\subseteq E'$ for $a_i\in F'$. Thus, $M\subseteq E'$. Now let $c,d\in M$. Then for some $i,j\in\mathbb{Z}^+$, $c\in E(a_1,\dots,a_i)$ and $d\in E(b_1,\dots,b_j)$. If $c$ and $d$ are in the same extension, then $c-d$ and $cd$ are in that extension as well. Otherwise, we let $c=t_1a_1+\cdots+t_ia_i$ and $d=r_1b_1+\cdots +r_jb_j$, where $t_i,r_i\in E$. Then (assuming $j\geq i$) $c-d=(t_1a_1-r_1b_1)+\cdots+(t_ia_i-r_ib_i)-\cdots-r_jb_j$. It follows that $c-d\in E(a_1,\dots,a_i,b_1,\dots,b_j)$ and thus $c-d\in M$. Similarly, $cd\in E(a_1b_1,a_1b_2,\dots, a_1b_j,\dots,a_ib_j)$ and so $cd\in M$. Note commutativity of addition and multiplication, multiplicative inverses, and no zero divisors are all inherited from the fields over which $M$ is the union. Thus, $M$ is a field. Finally, since $E\subseteq M$ and $F'\subseteq M$, then $E'\subseteq M$ because $E'$ is the smallest subfield which contains $E$ and $F'$. Now let $s\in E'$, then for some $k\in\mathbb{Z}^+$, $s\in E(a_1,\dots,a_k)$. Then $(x-s)\in E(a_1,\dots,a_k)[x]$ and thus $s$ is algebraic over $E$ since $E\subseteq E(a_1,\dots,a_k)$. Therefore, $E'/E$ is algebraic.
            \end{proof}
        \item[8.] \hfill\par
            \begin{enumerate}
                \item Determine, with explanation, if the following are splitting fields for $x^3-2$ over $\mathbb{Q}$:
                    \begin{enumerate}[label=(\roman*)]
                        \item $\mathbb{Q}(\sqrt[3]{2},\omega)$
                            \begin{solution}
                                Note that $x^3-2=(x-\sqrt[3]{2})(x-\omega\sqrt[3]{2})(x-\omega^2\sqrt[3]{2})$ and so any splitting field must contain these roots. Clearly, $\sqrt[3]{2},\omega\in\mathbb{Q}(\sqrt[3]{2},\omega)$ and so $\omega\sqrt[3]{2},\omega^2\sqrt[3]{2}\in\mathbb{Q}(\sqrt[3]{2},\omega)$. Now note that $\mathbb{Q}(\sqrt[3]{2},\omega\sqrt[3]{2},\omega^2\sqrt[3]{2})$ is a splitting field for $x^3-2$ since its the smallest field containing all the roots. Then $[\mathbb{Q}(\sqrt[3]{2})\colon\mathbb{Q}]=3$ since $\sqrt[3]{2}$ is a root of the third degree irreducible polynomial $x^3-2$. And $[\mathbb{Q}(\sqrt[3]{2},\omega\sqrt[3]{2})\colon\mathbb{Q}(\sqrt[3]{2})]=2$ since $\omega$ is a root of the irreducible polynomial $x^2+x+1\in\mathbb{Q}(\sqrt[3]{2})[x]$. Finally, $[\mathbb{Q}(\sqrt[3]{2},\omega\sqrt[3]{2},\omega^2\sqrt[3]{2})\colon\mathbb{Q}(\sqrt[3]{2},\omega\sqrt[3]{2})]=1$ since $\omega^2\sqrt[3]{2}$ can be generated from $\sqrt[3]{2}$ and $\omega\sqrt[3]{2}$. Thus, the degree of the splitting field is $[\mathbb{Q}(\sqrt[3]{2},\omega\sqrt[3]{2},\omega^2\sqrt[3]{2})\colon\mathbb{Q}]=6$. From problem 5. we saw that $[\mathbb{Q}(\sqrt[3]{2},\omega)\colon\mathbb{Q}]=6$ and so $\mathbb{Q}(\sqrt[3]{2},\omega)$ is a splitting field for $x^3-2$ over $\mathbb{Q}$.
                            \end{solution}
                        \item $\mathbb{Q}(\omega\sqrt[3]{2},\omega)$
                            \begin{solution}
                                This is a splitting field. Since $\omega$ is an element, then $\omega^{-1}$ is an element and so $\sqrt[3]{2}$ is therefore an element. Thus, $\mathbb{Q}(\sqrt[3]{2},\omega)\subseteq\mathbb{Q}(\omega\sqrt[3]{2},\omega)$. Similarly, $\omega\sqrt[3]{2},\omega\in\mathbb{Q}(\sqrt[3]{2},\omega)$ and so $\mathbb{Q}(\omega\sqrt[3]{2},\omega)$. Therefore, $\mathbb{Q}(\omega\sqrt[3]{2},\omega)=\mathbb{Q}(\sqrt[3]{2},\omega)$.
                            \end{solution}
                        \item $\mathbb{Q}(\sqrt[3]{2},\omega\sqrt[3]{2},\omega^2\sqrt[3]{2})$.
                            \begin{solution}
                                This is a splitting field. As mentioned in part (i), this is, by definition, the smallest field which contains $\mathbb{Q}$ and all the roots of $x^3-2$ and is therefore the splitting field of $x^3-2$.
                            \end{solution}\newpage
                        \item $\mathbb{Q}(\sqrt[3]{2},i\sqrt{3})$.
                            \begin{solution}
                                This is a splitting field. Since $i\sqrt{3}$ is an element, then $-1/2+i\sqrt{3}/2$ is an element. Thus, $\omega\in\mathbb{Q}(\sqrt[3]{2},i\sqrt{3})$ and since $\sqrt[3]{2}$ is also an element, then $\mathbb{Q}(\sqrt[3]{2},\omega)\subseteq\mathbb{Q}(\sqrt[3]{2},i\sqrt{3})$. Similarly, from $\omega$ we can add $1/2$ and multiply by 2 to obtain $i\sqrt{3}$ and thus $i\sqrt{3}\in\mathbb{Q}\sqrt[3]{2},\omega)$ and since $\sqrt[3]{2}$ is also an element, then $\mathbb{Q}(\sqrt[3]{2},i\sqrt{3})\subseteq\mathbb{Q}(\sqrt[3]{2},\omega)$. Therefore, $\mathbb{Q}(\sqrt[3]{2},i\sqrt{3})=\mathbb{Q}(\sqrt[3]{2},\omega)$.
                            \end{solution}
                    \end{enumerate}
                \item Find, with explanation, the splitting field of $x^6+1$ over $\mathbb{Q}$, and find the degree of the splitting field over $\mathbb{Q}$.
                    \begin{solution}
                        Consider $x^{12}-1=(x^6+1)(x^6-1)$. This shows that $x^{12}-1$ contains the roots of $x^6+1$. Using De Moivere's Theorem, we can obtain the $12^{\text{th}}$ roots of unity. However, we will first look at the roots of $x^6-1$ since these roots are distinct from those in $x^6+1$. We have: $1, \frac{1}{2}+i\frac{\sqrt{3}}{2},\omega,-1,\omega^2$. Thus, the roots of $x^6+1$ are $e^{2k\pi/12}$, where $k$ is odd. However, letting $\zeta=\frac{\sqrt{3}}{2}+\frac{i}{2}$, then we observe that $\zeta$ generates all 12 roots. Thus, and extension of $\mathbb{Q}$ which contains $\zeta$ allows $x^12-1$ and thus $x^6+1$ to split completely. Now note that $x^6+1=(x^2+1)(x^4-x^2+1)$ and so the minimal polynomial associated with the desired splitting field is of degree $2$ or $4$. It cannot be degree 2 since $x^2+1$ has roots $i$ and $-i$, but $x^6+1$ does not split over $\mathbb{Q}(i)$. Thus, $x^4-x^2+1$ is the minimal polynomial. Moreover, since the splitting field must contain $\zeta$ and $\zeta$ generates all 12 roots, then the splitting field is $\mathbb{Q}(\zeta)$ and $[\mathbb{Q}(\zeta)\colon\mathbb{Q}]=4$. 
                    \end{solution}
            \end{enumerate}
    \end{enumerate}
\end{document}