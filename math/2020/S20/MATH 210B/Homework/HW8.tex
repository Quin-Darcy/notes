\documentclass[12pt]{article}
\usepackage[margin=1in]{geometry} 
\usepackage{graphicx}
\usepackage{amsmath}
\usepackage{authblk}
\usepackage{titlesec}
\usepackage{amsthm}
\usepackage{amsfonts}
\usepackage{amssymb}
\usepackage{array}
\usepackage{booktabs}
\usepackage{ragged2e}
\usepackage{enumerate}
\usepackage{enumitem}
\usepackage{cleveref}
\usepackage{slashed}
\usepackage{commath}
\usepackage{lipsum}
\usepackage{colonequals}
\usepackage{addfont}
\usepackage{enumitem}
\usepackage{sectsty}
\usepackage{lastpage}
\usepackage{fancyhdr}
\usepackage{accents}
\usepackage{xcolor}
\usepackage[inline]{enumitem}
\pagestyle{fancy}

\fancyhf{}
\rhead{Darcy}
\lhead{MATH 210B}
\rfoot{\thepage}
\setlength{\headheight}{10pt}

\subsectionfont{\itshape}

\newtheorem{theorem}{Theorem}[section]
\newtheorem{corollary}{Corollary}[theorem]
\newtheorem{prop}{Proposition}[section]
\newtheorem{lemma}[theorem]{Lemma}
\theoremstyle{definition}
\newtheorem{definition}{Definition}[section]
\theoremstyle{remark}
\newtheorem*{remark}{Remark}
 
\makeatletter
\renewenvironment{proof}[1][\proofname]{\par
  \pushQED{\qed}%
  \normalfont \topsep6\p@\@plus6\p@\relax
  \list{}{\leftmargin=0mm
          \rightmargin=4mm
          \settowidth{\itemindent}{\itshape#1}%
          \labelwidth=\itemindent
          \parsep=0pt \listparindent=\parindent 
  }
  \item[\hskip\labelsep
        \itshape
    #1\@addpunct{.}]\ignorespaces
}{%
  \popQED\endlist\@endpefalse
}

\newenvironment{solution}[1][\bf{\textit{Solution}}]{\par
  
  \normalfont \topsep6\p@\@plus6\p@\relax
  \list{}{\leftmargin=0mm
          \rightmargin=4mm
          \settowidth{\itemindent}{\itshape#1}%
          \labelwidth=\itemindent
          \parsep=0pt \listparindent=\parindent 
  }
  \item[\hskip\labelsep
        \itshape
    #1\@addpunct{.}]\ignorespaces
}{%
  \popQED\endlist\@endpefalse
}

\let\oldproofname=\proofname
\renewcommand{\proofname}{\bf{\textit{\oldproofname}}}


\newlist{mylist}{enumerate*}{1}
\setlist[mylist]{label=(\alph*)}

\begin{document}\thispagestyle{empty}\hline

\begin{center}
	\vspace{.4cm} {\textbf { \large MATH 210B}}
\end{center}
{\textbf{Name:}\ Quin Darcy \hspace{\fill} \textbf{Due Date:} 4/8/20   \\
{ \textbf{Instructor:}}\ Dr. Shannon \hspace{\fill} \textbf{Assignment:} Homework 8 \\ \hrule}

\justifying

    \begin{enumerate}[leftmargin=*]
        \item[13.] For each of the following extensions, $E$, of $F$, determine if $E/F$ is a Galois extension, find the elements of $G(E/F)$, and determine what group $G(E/F)$ is isomorphic to.
            \begin{enumerate}
                \item $E=\mathbb{Q}(\omega,\sqrt[3]{2})$, $F=\mathbb{Q}$.
                    \begin{solution}
                        Note that by HW 5, $E$ is the splitting field for $x^3-2\in F[x]$, which factors as $(x-\sqrt[3]{2})(x-\omega\sqrt[3]{2})(\omega^2i^2\sqrt[3]{2})$. Since the roots of $x^3-2$, are all distinct, then $x^3-2$ is separable over $F$, so that $E$ is the splitting field of an irreducible polynomial in $F[x]$. By Theorem (??), $E/F$ is Galois.\par\hspace{4mm} Let $\varphi\in G(E/F)$. Then $\varphi\in\text{Aut}(E)$ and for all $a\in F$, $\varphi(a)=a$. By HW/Thm (??), the minimal polynomials for $\sqrt[3]{2},\omega$ over $F$, $p(x)=x^3-2$ and $x^2+x+1$, have roots $\alpha,\beta$, respectively, such that $\varphi(\alpha)\in\{\sqrt[3]{2}, \omega\sqrt[3]{2},\omega^2\sqrt[3]{2}\}$ and $\varphi(\beta)\in\{\omega, \omega^2\}$. This gives us $6=[E\colon F]=\abs{G(E/F)}$ automorphisms. These are defined as follows:
                            \begin{align*}
                                &\varphi_1:=
                                    \begin{cases}
                                        a\in F\mapsto a \\
                                        \sqrt[3]{2}\mapsto\sqrt[3]{2}\\
                                        \omega\mapsto\omega
                                    \end{cases} 
                                    && 
                                \varphi_2:=
                                    \begin{cases} 
                                        a\in F\mapsto a \\
                                        \sqrt[3]{2}\mapsto\omega\sqrt[3]{2}\\
                                        \omega\mapsto\omega
                                    \end{cases} 
                                    && 
                                \varphi_3:=
                                    \begin{cases} 
                                        a\in F\mapsto a \\
                                        \sqrt[3]{2}\mapsto\omega^2\sqrt[3]{2}\\
                                        \omega\mapsto\omega
                                    \end{cases} \\
                                &\varphi_4:=
                                    \begin{cases} 
                                        a\in F\mapsto a \\
                                        \sqrt[3]{2}\mapsto\sqrt[3]{2}\\
                                        \omega\mapsto\omega^2
                                    \end{cases} 
                                    && 
                                \varphi_5:=
                                    \begin{cases} 
                                        a\in F\mapsto a \\
                                        \sqrt[3]{2}\mapsto\omega\sqrt[3]{2}\\
                                        \omega\mapsto\omega^2
                                    \end{cases} 
                                    && 
                                \varphi_6 :=
                                    \begin{cases} 
                                        a\in F\mapsto a \\
                                        \sqrt[3]{2}\mapsto\omega^2\sqrt[3]{2}\\
                                        \omega\mapsto\omega^2.
                                    \end{cases}
                            \end{align*}
                        By Thm/HW (??) $G(E/F)$ is a group. The cayley table for this group is
                            \begin{table}[htp]
                                \centering
                                \begin{tabular}{|l|llllll}
                                    \hline
                                    & \multicolumn{1}{l|}{$\varphi_1$} & \multicolumn{1}{l|}{$\varphi_2$} & \multicolumn{1}{l|}{$\varphi_3$} & \multicolumn{1}{l|}{$\varphi_4$} & \multicolumn{1}{l|}{$\varphi_5$} & \multicolumn{1}{l|}{$\varphi_6$} \\ \hline
                                        $\varphi_1$ & $\varphi_1$ & $\varphi_2$ & $\varphi_3$ & $\varphi_4$ & $\varphi_5$ & $\varphi_6$ \\ \cline{1-1}
                                        $\varphi_2$ & $\varphi_2$ & $\varphi_3$ & $\varphi_1$ & $\varphi_$ & $\varphi_$ & $\varphi_$ \\ \cline{1-1}
                                        $\varphi_3$ & $\varphi_3$ & $\varphi_1$ & $\varphi_2$ & $\varphi_$ & $\varphi_$ & $\varphi_$ \\ \cline{1-1}
                                        $\varphi_4$ & $\varphi_4$ & $\varphi_6$ & $\varphi_5$ & $\varphi_$ & $\varphi_$ & $\varphi_$ \\ \cline{1-1}
                                        $\varphi_5$ & $\varphi_5$ & $\varphi_4$ & $\varphi_6$ & $\varphi_$ & $\varphi_$ & $\varphi_$ \\ \cline{1-1}
                                        $\varphi_6$ & $\varphi_6$  & $\varphi_5$ & $\varphi_4$ & $\varphi_$ & $\varphi_$ & $\varphi_$ \\ \cline{1-1}
                                \end{tabular}$\longrightarrow$         
                                \centering
                                \begin{tabular}{|l|llllll}
                                    \hline
                                    & \multicolumn{1}{l|}{$(1)$} & \multicolumn{1}{l|}{$(123)$} & \multicolumn{1}{l|}{$(132)$} & \multicolumn{1}{l|}{$(23)$} & \multicolumn{1}{l|}{$(132)$} & \multicolumn{1}{l|}{$(123)$} \\ \hline
                                        $(1)$ & $(1)$ & $(123)$ & $(132)$ & $(23)$ & $(132)$ & $(123)$ \\ \cline{1-1}
                                        $(123)$ & $(123)$ & $(132)$ & $(1)$ & $\varphi_$ & $\varphi_$ & $\varphi_$ \\ \cline{1-1}
                                        $(132)$ & $(13)$ & $(1)$ & $\varphi_2$ & $\varphi_$ & $\varphi_$ & $\varphi_$ \\ \cline{1-1}
                                        $(23)$ & $(23)$ & $\varphi_6$ & $\varphi_5$ & $\varphi_$ & $\varphi_$ & $\varphi_$ \\ \cline{1-1}
                                        $(132)$ & $(132)$ & $\varphi_4$ & $\varphi_6$ & $\varphi_$ & $\varphi_$ & $\varphi_$ \\ \cline{1-1}
                                        $(123)$ & $(123)$  & $\varphi_5$ & $\varphi_4$ & $\varphi_$ & $\varphi_$ & $\varphi_$ \\ \cline{1-1}
                                \end{tabular}
                            \end{table}
                    \end{solution}
                \item $E=\mathbb{Q}(\sqrt[4]{5},i)$, $F=\mathbb{Q}$.
                    \begin{solution}
                        The minimal polynomial of $\sqrt[4]{5}$ over $F$ is $x^4-5$. The roots of this polynomial are $\sqrt[4]{5},-\sqrt[4]{5},i\sqrt[4]{5},-i\sqrt[4]{5}$ and so $x^4-5$ splits over $E$. Additionally, $E$ is the smallest field containing $F$ and the roots of $x^4-5$ and as such, $E$ is the splitting field for $x^4-5$. Since $\sqrt[4]{5}\notin F(i)$, then $x^2-\sqrt[4]{25}$ is irreducible over $F(i)$ and so $[E\colon F(i)]=2$. Similarly, since $x^2+1$ is irreducible over $F$, then $[F(i)\colon F]=2$. Thus, $[E\colon F]=[E\colon F(i)][F(i)\colon F]=4$. Now note that any automorphism of $E$ fixes $F$ by HW 1 and so $G(E/F)=\text{Aut}(E)$. Moreover, for any $\varphi\in\text{Aut}(E)$, we have that 
                            \begin{equation*}
                                \varphi(\sqrt[4]{5})=\pm\sqrt[4]{5}\quad\text{and}\quad\varphi(i)=\pm i.
                            \end{equation*}
                        Thus, $\abs{\text{Aut}(E)}=\abs{G(E/F)}=4$ and so $\abs{G(E/F)}=[E\colon F]$. Thus, by Theorem 5, $E/F$ is Galois. Finally, let $\varphi,\tau\in G(E/F)$ such that 
                            \begin{equation*}
                                \varphi(\sqrt[4]{5})=-\sqrt[4]{5},\;\varphi(i)=i\quad\text{and}\quad\tau(\sqrt[4]{5})=\sqrt[4]{5},\;\tau(i)=-i.
                            \end{equation*}\newpage
                        Letting $\alpha_1=\sqrt[4]{5},\alpha_2=-\sqrt[4]{5},\alpha_3=i\sqrt[4]{5},$ and $\alpha_4=-i\sqrt[4]{5}$, then\par 
                            \begin{table}[htp]
                                \centering
                                \begin{tabular}{|l|l|l|l|l|l|}
                                \hline
                                    & $\alpha_1$ & $\alpha_2$ & $\alpha_3$ & $\alpha_4$ &          \\ \hline
                                    $1_E$        & $\alpha_1$ & $\alpha_2$ & $\alpha_3$ & $\alpha_4$ & (1)      \\ \hline
                                    $\varphi$     & $\alpha_2$ & $\alpha_1$ & $\alpha_3$ & $\alpha_4$ & (12)     \\ \hline
                                    $\tau$       & $\alpha_1$ & $\alpha_2$ & $\alpha_4$ & $\alpha_3$ & (34)     \\ \hline
                                    $\varphi\tau$ & $\alpha_2$ & $\alpha_1$ & $\alpha_4$ & $\alpha_3$ & (12)(34) \\ \hline
                                \end{tabular}
                            \end{table}\par 
                        From this table we can see that $G(E/F)\cong\{(1),(12),(34),(12)(34)\}\cong\mathbb{Z}_2\times\mathbb{Z}_2$.
                    \end{solution}
                \item $E=\mathbb{Q}(\sqrt[4]{5},i)$, $F=\mathbb{Q}(i)$.
                    \begin{solution}
                        By part (b), we know that $\abs{\text{Aut}(E)}=4$ and two of these automorphisms map $i$ to $-i$. Thus, $G(E/F)=\{1_E,\varphi\}$, where $1_E$ is the identity on $E$ and $\varphi(\sqrt[4]{5})=-\sqrt[4]{5}$ and $\varphi(i)=-i$. Thus, $\abs{G(E/F)}=2$. Additionally, the minimal polynomial of $\sqrt[4]{5}$ over $F$ is $x^2-\sqrt[4]{25}$ and so $[E\colon F]=2$. Thus, $\abs{G(E/F)}=[E\colon F]$ and by Theorem 5, $E/F$ is Galois. Finally, by part (b), it follows that $G(E/F)=\{1_E,\varphi\}\cong\{(1),(12)\}\cong\mathbb{Z}_2$.
                    \end{solution}
                \item $E=\mathbb{Q}(\sqrt[4]{5},i)$, $F=\mathbb{Q}(\sqrt[4]{5})$.
                    \begin{solution}
                        By parts (b) and (c) it follows that $G(E/F)=\{1_E,\tau\}$. Additionally, since the minimal polynomial of $i$ over $F$ is $x^2+1$, then $[E\colon F]=2$. Thus, $\abs{G(E/F)}=[E\colon F]$ and by Theorem 5, $E/F$ is Galois. Finally, $G(E/F)\cong\{(1),(34)\}\cong\mathbb{Z}_2$.
                    \end{solution}
                \item $E=\mathbb{Q}(\sqrt{2},\sqrt{3})$, $F=\mathbb{Q}$.
                    \begin{solution}
                        Since $E$ is the splitting field of the separable polynomial $(x^2-2)(x^2-3)$, then $\abs{G(E/F)}=[E\colon F]=4$ and so $E/F$ is Galois. Thus, for any $\varphi\in G(E/F)$ we'll have that $\varphi(\sqrt{2})=\pm\sqrt{2}$ and $\varphi(\sqrt{3})=\pm \sqrt{3}$. In particular, we can find $\varphi_1,\varphi_2\in G(E/F)$ such that 
                            \begin{align*}
                                &\varphi_1(\sqrt{2})=\sqrt{2}, &\varphi_1(\sqrt{3})=-\sqrt{3} \\
                                &\varphi_2(\sqrt{2})=-\sqrt{2} &\varphi_2(\sqrt{3})=\sqrt{3}.
                            \end{align*}
                        This results in $G(E/F)=\{\varphi_0,\varphi_1,\varphi_2,\varphi_1\varphi_2\}\cong\mathbb{Z}_2\times\mathbb{Z}_2$.
                    \end{solution}
            \end{enumerate}
    \end{enumerate}
\end{document}