\documentclass[12pt]{article}
\usepackage[margin=1in]{geometry} 
\usepackage{graphicx}
\usepackage{amsmath}
\usepackage{authblk}
\usepackage{titlesec}
\usepackage{amsthm}
\usepackage{amsfonts}
\usepackage{amssymb}
\usepackage{array}
\usepackage{booktabs}
\usepackage{ragged2e}
\usepackage{enumerate}
\usepackage{enumitem}
\usepackage{cleveref}
\usepackage{slashed}
\usepackage{commath}
\usepackage{lipsum}
\usepackage{colonequals}
\usepackage{addfont}
\usepackage{enumitem}
\usepackage{sectsty}
\usepackage{lastpage}
\usepackage{fancyhdr}
\usepackage{accents}
\usepackage{xcolor}
\usepackage[inline]{enumitem}
\pagestyle{fancy}

\fancyhf{}
\rhead{Darcy}
\lhead{MATH 210B}
\rfoot{\thepage}
\setlength{\headheight}{10pt}

\subsectionfont{\itshape}

\newtheorem{theorem}{Theorem}[section]
\newtheorem{corollary}{Corollary}[theorem]
\newtheorem{prop}{Proposition}[section]
\newtheorem{lemma}[theorem]{Lemma}
\theoremstyle{definition}
\newtheorem{definition}{Definition}[section]
\theoremstyle{remark}
\newtheorem*{remark}{Remark}
 
\makeatletter
\renewenvironment{proof}[1][\proofname]{\par
  \pushQED{\qed}%
  \normalfont \topsep6\p@\@plus6\p@\relax
  \list{}{\leftmargin=0mm
          \rightmargin=4mm
          \settowidth{\itemindent}{\itshape#1}%
          \labelwidth=\itemindent
          \parsep=0pt \listparindent=\parindent 
  }
  \item[\hskip\labelsep
        \itshape
    #1\@addpunct{.}]\ignorespaces
}{%
  \popQED\endlist\@endpefalse
}

\newenvironment{solution}[1][\bf{\textit{Solution}}]{\par
  
  \normalfont \topsep6\p@\@plus6\p@\relax
  \list{}{\leftmargin=0mm
          \rightmargin=4mm
          \settowidth{\itemindent}{\itshape#1}%
          \labelwidth=\itemindent
          \parsep=0pt \listparindent=\parindent 
  }
  \item[\hskip\labelsep
        \itshape
    #1\@addpunct{.}]\ignorespaces
}{%
  \popQED\endlist\@endpefalse
}

\let\oldproofname=\proofname
\renewcommand{\proofname}{\bf{\textit{\oldproofname}}}

\newcommand\defeq{\stackrel{\smash{\scriptscriptstyle\mathrm{def}}}{=}}


\newlist{mylist}{enumerate*}{1}
\setlist[mylist]{label=(\alph*)}

\begin{document}\thispagestyle{empty}\hline

\begin{center}
	\vspace{.4cm} {\textbf { \large MATH 210B}}
\end{center}
{\textbf{Name:}\ Quin Darcy \hspace{\fill} \textbf{Due Date:} 2/12/20   \\
{ \textbf{Instructor:}}\ Dr. Shannon \hspace{\fill} \textbf{Assignment:} Homework 3 \\ \hrule}

\justifying

    \begin{enumerate}[leftmargin=*]
        \item[3.] Assume that $F$ is a field, $f(x)\in F[x]$, and that $a\in F$. Prove that $a$ is a root of $f(x)$ iff $(x-a)\mid f(x)$.
            \begin{proof}
                Assume that $a$ is a root of $f(x)$. Then $f(a)=0$. Since $F$ is a field and $f(x),(x-a)\in F[x]$ where $(x-a)\neq 0$, then by the proof on pg.5 there exists unique polynomials $q(x),r(x)\in F[x]$ such that $f(x)=q(x)(x-2)+r(x)$ and $r(x)=0$ or $\text{deg}(r(x))<\text{deg}(x-a)$. Since $\text{deg}((x-a))=1$, then $\text{deg}(r(x))=0$. Thus, for some $t\in F$, $r(x)=t$. By assumption, $f(a)=0$ and so $q(a)(a-a)+t=t=0$. Thus, $f(x)=q(x)(x-a)$. Therefore, $(x-a)\mid f(x)$. Now assume that $(x-a)\mid f(x)$. Then for some $q(x)\in F[x]$, we have that $f(x)=q(x)(x-a)$. Thus, $f(a)=q(a)(a-a)=0$. Therefore, $a$ is a root of $f(x)$.
            \end{proof}
        \item[4.] Assume that $F$ is a field. Then $F[x]$ is a Euclidean domain, thus any two polynomials have a gcd and the gcd can be expressed as a linear combination of the two polynomials. Over $\mathbb{Z}_7[x]$, find the gcd of $f(x)=3x^3+5x^2+6x$ and $g(x)=4x^4+2x^3+6x^2+4x+5$ and express the gcd as a linear combination of $f(x)$ and $g(x)$.
            \begin{solution}
                We begin by finding the gcd of the two given polynomials and this requires repeated applications of the division algorithm. In doing this we obtain
                    \begin{equation*}
                        \begin{split}
                            g(x) &=(6x)f(x)+(5x^2+4x+5)=q_1(x)f(x)+r_1(x) \\
                            f(x) &=(2x+5)r_1(x)+(4x+3).
                        \end{split}
                    \end{equation*}
                Thus, the gcd of the two given polynomials is $4x+3$. Using back substitution, we get that 
                    \begin{equation*}
                        \begin{split}
                            (1+q_2(x)q_1(x))f(x)-(q_2(x))g(x)=4x+3.
                        \end{split}
                    \end{equation*}
            \end{solution}
        \item[5.] Find the elementary symmetric functions for $x^3+bx^2+cx+d=(x-r)(x-s)(x-t)$, and for $x^4-2x^2-3$.
            \begin{solution}
                To begin, we compute the elementary symmetric functions for $x^3+bx^2+cx+d$ and obtain $\sigma_1=r+s+t$, $\sigma_2=rs+rt+st$, and $\sigma_3=rst$. For $x^4-2x^2-3$ we get that 
                    \begin{equation*}
                        \begin{split}
                            \sigma_1 &= i+(-i)+\sqrt{3}+(-\sqrt{3}) =0 \\
                            \sigma_2 &= (i)(-i)+(i)(\sqrt{3})+(i)(-\sqrt{3})+(-i)(\sqrt{3})+(-\sqrt{3})(\sqrt{3}) =-2 \\
                            \sigma_3 &= (i)(-i)(\sqrt{3})+(i)(-i)(-\sqrt{3})+(i)(\sqrt{3})(-\sqrt{3})+(-i)(\sqrt{3})(-\sqrt{3})=0 \\
                            \sigma_4 &= (i)(-i)(\sqrt{3})(-\sqrt{3})=-3.
                        \end{split}
                    \end{equation*}
            \end{solution}\newpage
        \item[6.] Assume that $F$ is a field, that $p(x)\in F[x]$ is irreducible over $F[x]$, and that $\text{deg}(p(x))=n$. Let $I=(p(x))_i$.
            \begin{enumerate}
                \item Prove that every element of $F[x]/I$ can be written in the form $I+h(x)$ where $h(x)=0$ or $\text{deg}(h(x))<n$, and that this representation is unique. 
                    \begin{proof}
                        Let $h(x)+I\in F[x]/I$. If $\text{deg}(h(x))< n$, then we are done. Otherwise, if $\text{deg}(h(x))\geq n$, then by the division algorithm, there exists $q(x),r(x)\in F[x]/I$ such that $h(x)=q(x)p(x)+r(x)$, where $r(x)=0$ or $\text{deg}(r(x))<n$. If $r(x)=0$, then $h(x)\in I$. If $r(x)\neq 0$, then since $q(x)p(x)\in I$, then $I+h(x)=I+r(x)$.\par\hspace{4mm} Now suppose $I+h(x)=I+f_1(x)=I+f_2(x)$ where $\text{deg}(f_1(x))<n$ and $\text{deg}(f_2(x))<n$. Then $I+(f_1(x)-f_2(x))=I$ and it follows that $p(x)\mid(f_1(x)-f_2(x))$. Thus, for some $q(x)\in F[x]$, we have $f_1(x)=q(x)p(x)+f_2(x)$. However, since $\text{deg}(f_1(x))<n$ and $\text{deg}(q(x)p(x)+f_2(x))\geq n$ for $q(x)\neq 0$, but this is impossible. Thus $q(x)=0$ and $f_1(x)=f_2(x)$. 
                    \end{proof}
                \item Prove that $\{I+1,I+x,\dots,I+x^{n-1}\}$ is a basis for $F[x]/I$ over $F$.
                    \begin{proof}
                        Suppose that for some $a_0,\dots,a_{n-1}\in F$
                            \begin{equation*}
                                \sum_{i=0}^{n-1}(I+a_ix^i)=I+\sum_{i=0}^{n-1}a_ix^i=I.
                            \end{equation*}
                        By the unique representation component of part (a), it follows that 
                            \begin{equation*}
                                \sum_{i=0}^{n-1}a_ix^i=0.
                            \end{equation*}
                        Moving the constant term over we see that $a_1x+\cdots+a_{n-1}x^{n-1}=-a_0$.\par\hspace{4mm}With indeterminates on the left and none on the right, this equality only holds for $a_i=0$ for all $i$. Thus, the set is linearly independent. Since $F[x]/I$ is a field, then clearly $\langle\{I+1,\dots,I+x^{n-1}\}\rangle\subseteq F[x]/I$ and so it suffices to show that the inclusion goes the other way. Let $I+h(x)\in F[x]/I$. By part (a), we can assume that $\text{deg}(h(x))<n$, say $\text{deg}(h(x))=j$. Then $h(x)=a_0+a_1x+\cdots+a_jx^j$ and so 
                            \begin{equation*}
                                I+h(x)=I+\sum_{i=0}^ja_ix^i\in\langle\{I+1,\dots,I+x^{n-1}\}\rangle.
                            \end{equation*}
                        Thus, $\langle\{I+1,\dots,I+x^{n-1}\}\rangle\subseteq F[x]/I$ and so $\{I+1,\dots,+I+x^{n-1}\}$ is a basis for $F[x]/I$.
                    \end{proof}
                \item Find a basis for $\mathbb{Q}[x]/(x^3-2)_i$ over $\mathbb{Q}$. 
                    \begin{solution}
                        Let $I=(x^3-2)_i$ and $S=\{I+1,I+x,I+x^2\}$. Assume for $a_0,a_1,a_2\in\mathbb{Q}$ that 
                            \begin{equation*}
                                (I+a_0)+(I+a_1x)+(I+a_2x^2)=I+(a_0+a_1x+a_2x^2)=I.
                            \end{equation*}
                        By 6.a, this representation is unique and so $a_0+a_1x+a_2x^2=0$ which only holds for $a_i=0$ for all $i$. Thus, $S$ is linearly independent. Now take $I+h(x)\in\mathbb{Q}[x]/I$. Then by 6.a, we can assume that $\text{deg}(h(x))<3$ and so let $h(x)=b_0+b_1x+b_2x^2$, then 
                            \begin{equation*}
                                I+h(x)=I+(b_0+b_1x+b_2x^2)=(I+b_0)+(I+b_1x)+(I+b_2x^2)\in S.
                            \end{equation*}
                        Thus, $\langle S\rangle=\mathbb{Q}/(x^3-2)_i$. Therefore, $S$ is a basis.
                    \end{solution}
            \end{enumerate}
        \item[7.] Determine the number of elements of $\mathbb{Z}_5/(x^2-3)_i$, and determine whether or not $\mathbb{Z}_5[x]/(x^2-3)_i$ is a field.
            \begin{solution}
                By 6.a, we know every element of $\mathbb{Z}_5[x]/(x^2-3)_i$ can be written as $I+h(x)$ where $I=(x^2-3)_i$ and $h(x)=0$ or $\text{deg}(h(x))<2$. Thus, $h(x)=a_0+a_1x$ for $a_0,a_1\in\mathbb{Z}_5$. There are 5 elements in $\mathbb{Z}_5$ and so $a_0$ can take on any one of these 5 values, and the same for $a_1$. Thus, there are $5\cdot 5=25$ elements in $\mathbb{Z}_5[x]/(x^2-3)_i$. Lastly, we will check to see if $x^2-3$ is irreducible over $\mathbb{Z}_5$. Supposing it is, we would have 
                    \begin{equation*}
                        x^2-3=(ax+b)(cx+d)
                    \end{equation*}
                which gives us the relations $ac=1$, $ad+bc=0$, $bd=-3$. Checking every element, we find that there are no solutions. Thus, $x^2-3$ is irreducible, which implies $(x^2-3)_i$ is maximal and so $\mathbb{Z}_5[x]/(x^2-3)_i$ is a field.
            \end{solution}
        \item[8.]\hfill\par
            \begin{enumerate}
                \item Assume that $F$ is a field, $f(x),g(x),h(x)\in F[x]$, $f(x)=g(x)\cdot h(x)$, and for all $c\in F$, $g(c)\neq 0$ and $h(c)\neq 0$. Must $f(x)$ be irreducible over $F[x]$? Explain your answer.
                    \begin{solution}
                        No, as a counter example consider $f(x)\in\mathbb{Q}[x]$ where
                            \begin{equation*}
                                f(x)=x^4-x^2-2=(x^2-2)(x^2+1)
                            \end{equation*}
                        In this case, $f$ is reducible since neither $g(x)$ or $h(x)$ is a unit, and the roots of both factors are not elements of $\mathbb{Q}$. 
                    \end{solution}
                \item Assume that $F$ is a field, $a\in F$, $m\mid n$, $m\neq 1$, and there exists $d\in F$, $d\neq\pm a$, such that $d^m=a$. Prove that $x^n-a$ is reducible over $F$. Determine if the converse is true.
                    \begin{proof}
                        We begin by noting that for any $m\in\mathbb{N}$,
                            \begin{equation*}
                                x^m-t^m=(x-t)(x^{m-t}+x^{m-2}t+\cdots+t^{m-1})
                            \end{equation*}
                         Since $d^m=a$, then we may write $x^n-a=x^n-d^m$. Moreover, since $m\mid n$, then for some $k\in\mathbb{Z}$, we have $n=mk$. Thus,
                            \begin{equation*}
                                \begin{split}
                                    x^n-a &= x^n-d^m \\
                                    &= x^{mk}-d^m \\
                                    &= (x^k)^m-d^m \\
                                    &= (x^k-d)((x^k)^{m-1}+(x^k)^{m-2}d+\cdots+d^{m-1})
                                \end{split}
                            \end{equation*}
                        Since neither of the polynomials in the final product are a unit, then $x^n-a$ is irreducible.
                    \end{proof}
            \end{enumerate}
        \item[9.] Find (and explain) the multiplicative inverse of $(x^3+x+1)_i+x^2+2$ in $\mathbb{Z}_5[x]/(x^3+x+1)_i$.
            \begin{solution}
                We being by applying the division algorithm on $x^3+x+1$ and $x^2+2$ in order to find the gcd of the two polynomials. In doing so we get
                    \begin{enumerate}[label=\arabic*)]
                        \item $(x^3+x+1)=x(x^2+2)+(4x+1)$;
                        \item $(x^2+2)=(4x+4)(4x+1)+(3)$;
                        \item $(4x+1)=(3x+1)(3)$.
                    \end{enumerate}
                Thus, 3 is the gcd of the two polynomials. Letting $f(x)=x^2+2$, $g(x)=x^3+x+1$, $q_1(x)=x$, and $q_2(x)=4x+4$, we can use back substitution to obtain the following linear combination of $f(x)$ and $g(x)$:
                    \begin{equation*}
                        (1-q_2(x)q_1(x))f(x)-q_2(x)g(x)=3.
                    \end{equation*}
                Multiplying both sides by 2 and expanding, we get
                    \begin{equation*}
                        \begin{split}
                            2\big[1-(4x+4)(x)\big]f(x)-2q_2(x)g(x) &= \big[2-8x^2-8x]f(x)-2q_2(x)g(x) \\
                            &= [2+2x^2+2x]f(x)-2q_2(x)g(x) \\
                            &= [2x^2+2x+2]f(x)-2q_2(x)g(x) \\
                            &= 1.
                        \end{split}
                    \end{equation*}
                For space economy, let $I=(g(x))_i$. Since $-2q_2(x)g(x)\in I$, then $I-2q_2(x)g(x)=I$, from which it follows
                    \begin{equation*}
                        \begin{split}
                            \big[I+f(x)\big]\big[I+(2x^2+2x+2)\big] &= I+(2x^2+2x+2)f(x) \\
                            &= \big[I+(2x^2+2x+2)f(x)\big]+I \\
                            &= \big[I+(2x^2+2x+2)f(x)\big]+\big[I-2q_2(x)g(x)\big] \\
                            &= I+\big[(2x^2+2x+2)f(x)-2q_2(x)g(x)\big] \\
                            &= I+1.
                        \end{split}
                    \end{equation*}
                Hence, the multiplicative inverse of $(x^3+x+2)_i+x^2+2$ is $(x^3+x+1)_i+2x^2+2x+2$.
            \end{solution}
        \item[10.] Assume that $F$ is a field, $p(x)\in F[x]$ is irreducible over $F$ and $I=(p(x))_i$. Prove that $I+x$ is a root of $p(x)$ in $F[x]/I$.
            \begin{proof}
                Since $p(x)$ is irreducible, then $I$ is maximal and so $F[x]/I$ is a field. Now consider a map $\psi\colon F\rightarrow F[x]/I$ defined by $\psi(a)=I+a$. If we let $a,b\in F$ such that $a=b$, then $\psi(a)=I+a=I+b=\psi(b)$ and so $\psi$ is well defined. If $\psi(a)=\psi(b)$, then $I+a=I+b$ and so $I+(a-b)=I$. Thus, $a-b\in I$. Hence, $a-b$ is a multiple of $p(x)$ which has degree greater than or equal to 1. Since $a,b\in F$ then the only way in which $a-b$ is equal to a multiple of a polynomial of degree $\geq 1$ is if $a-b=0$ and so $a=b$. Therefore, $\psi$ is 1-1. Moreover, we have that 
                    \begin{equation*}
                        \psi(a+b)=I+a+b=(I+a)+(I+b)=\psi(a)+\psi(b)
                    \end{equation*}
                \noindent and
                    \begin{equation*}
                        \psi(ab)=I+ab=(I+a)(I+b)=\psi(a)\psi(b).
                    \end{equation*}
                Thus, $\psi\colon F\rightarrow \psi(F)$ is an isomorphism and so we can identify each element $a\in F$ with a coset $I+a\in F[x]/I$. Moreover, since $F$ is a field, $F[x]/I$ is a field, and $\psi$ is an isomorphism onto the image of $F$, then $\psi(F)$ is a subfield of $F[x]/I$. Now consider the homomorphism from HW2, problem 7 with $\theta\colon F[x]\rightarrow F[x]/I$ defined by $\theta(f(x))=f(I+x)$. Thus, if $p(x)=a_0+a_1x+\cdots a_nx^n$ with $a_i\in F$, then 
                    \begin{equation*}
                        \begin{split}
                            \theta(p(x)) &= p(I+x) \\
                            &= a_0(I+x)^0+a_1(I+x)^1+a_2(I+x)^2+\cdots+a_n(I+x)^n \\
                            &= (I+a_0) +(I+a_1x)+(I+a_2x^2)+\cdots+(I+a_nx^n) \\
                            &= I+(a_0+a_1x+a_2x^2+\cdots+a_nx^n) \\
                            &= I+p(x) \\
                            &= I.
                        \end{split}
                    \end{equation*}
                Thus, $p(I+x)=I=\psi(0)$. So since $F\cong\psi(F)\subseteq F[x]/I$, then it follows that $F[x]/I$ can be thought of as an extension of $F$ which contains a root of $p(x)$, namely $I+x$.
            \end{proof}
    \end{enumerate}
\end{document}