\documentclass[12pt]{article}
\usepackage[margin=1in]{geometry} 
\usepackage{graphicx}
\usepackage{amsmath}
\usepackage{authblk}
\usepackage{titlesec}
\usepackage{amsthm}
\usepackage{amsfonts}
\usepackage{amssymb}
\usepackage{array}
\usepackage{booktabs}
\usepackage{ragged2e}
\usepackage{enumerate}
\usepackage{enumitem}
\usepackage{cleveref}
\usepackage{slashed}
\usepackage{commath}
\usepackage{lipsum}
\usepackage{colonequals}
\usepackage{addfont}
\usepackage{enumitem}
\usepackage{sectsty}
\usepackage{lastpage}
\usepackage{fancyhdr}
\usepackage{accents}
\usepackage{xcolor}
\usepackage[inline]{enumitem}
\pagestyle{fancy}

\fancyhf{}
\rhead{Darcy}
\lhead{MATH 210B}
\rfoot{\thepage}
\setlength{\headheight}{10pt}

\subsectionfont{\itshape}

\newtheorem{theorem}{Theorem}[section]
\newtheorem{corollary}{Corollary}[theorem]
\newtheorem{prop}{Proposition}[section]
\newtheorem{lemma}[theorem]{Lemma}
\theoremstyle{definition}
\newtheorem{definition}{Definition}[section]
\theoremstyle{remark}
\newtheorem*{remark}{Remark}
 
\makeatletter
\renewenvironment{proof}[1][\proofname]{\par
  \pushQED{\qed}%
  \normalfont \topsep6\p@\@plus6\p@\relax
  \list{}{\leftmargin=0mm
          \rightmargin=4mm
          \settowidth{\itemindent}{\itshape#1}%
          \labelwidth=\itemindent
          \parsep=0pt \listparindent=\parindent 
  }
  \item[\hskip\labelsep
        \itshape
    #1\@addpunct{.}]\ignorespaces
}{%
  \popQED\endlist\@endpefalse
}

\newenvironment{solution}[1][\bf{\textit{Solution}}]{\par
  
  \normalfont \topsep6\p@\@plus6\p@\relax
  \list{}{\leftmargin=0mm
          \rightmargin=4mm
          \settowidth{\itemindent}{\itshape#1}%
          \labelwidth=\itemindent
          \parsep=0pt \listparindent=\parindent 
  }
  \item[\hskip\labelsep
        \itshape
    #1\@addpunct{.}]\ignorespaces
}{%
  \popQED\endlist\@endpefalse
}

\let\oldproofname=\proofname
\renewcommand{\proofname}{\bf{\textit{\oldproofname}}}


\newlist{mylist}{enumerate*}{1}
\setlist[mylist]{label=(\alph*)}

\begin{document}\thispagestyle{empty}\hline

\begin{center}
	\vspace{.4cm} {\textbf { \large MATH 210B}}
\end{center}
{\textbf{Name:}\ Quin Darcy \hspace{\fill} \textbf{Due Date:} 3/11/20   \\
{ \textbf{Instructor:}}\ Dr. Shannon \hspace{\fill} \textbf{Assignment:} Homework 6 \\ \hrule}

\justifying

    \begin{enumerate}[leftmargin=*]
        \item[4.] Let $\zeta$ be a primitive $6^{\text{th}}$ root of unity. Find (with explanation) all 1-1 homomorphisms of $\mathbb{Q}(\sqrt[3]{5},\zeta)$ to $\mathbb{C}$, and all 1-1 homomorphisms from $\mathbb{Q}(\sqrt{2},i)$ to $\mathbb{C}$.
            \begin{solution}
                \emph{Let $\zeta=\frac{1}{2}+i\frac{\sqrt{3}}{2}$. Then $\zeta$ is equal to one of the 2 primitive $6^{\text{th}}$ roots of unity. The minimal polynomial of $\sqrt[3]{5}$ over $\mathbb{Q}$ is $x^3-5$. The minimal polynomial of $\zeta$ over $\mathbb{Q}$ is $x^2-x+1$. Since $\sqrt[3]{5}\notin\mathbb{Q}(\zeta)$ and $\zeta\notin\mathbb{Q}(\sqrt[3]{5})$, then $[\mathbb{Q}(\sqrt[3]{5},\zeta)\colon\mathbb{Q}]=6$. Note that $x^3-5$ splits over $\mathbb{Q}(\sqrt[3]{5},\zeta)$ as
                    \begin{equation*}
                        (x-\sqrt[3]{5})(x-\zeta^2\sqrt[3]{5})(x-\zeta^4\sqrt[3]{5}).
                    \end{equation*}
                Similarly, $x^2-x+1$ splits over $\mathbb{Q}(\sqrt[3]{5},\zeta)$ as 
                    \begin{equation*}
                        (x-\zeta)(x-\zeta^5).
                    \end{equation*}
                Since $\mathbb{Q}(\sqrt[3]{5},\zeta)\subseteq\mathbb{C}$, then both polynomials split over $\mathbb{C}$. Thus, if $\psi\colon\mathbb{Q}(\sqrt[3]{5},\zeta)\rightarrow\mathbb{C}$ is a 1-1 homomorphism, then by Exam 1 and HW5 we have 6 possibilities
                    \begin{align*}
                        &\psi_1:=\begin{cases} 
                            \sqrt[3]{5}\mapsto\sqrt[3]{5}\\
                            \zeta\mapsto\zeta
                        \end{cases} && \psi_2:=\begin{cases} 
                            \sqrt[3]{5}\mapsto\zeta^2\sqrt[3]{5}\\
                            \zeta\mapsto\zeta
                        \end{cases} && \psi_3:=\begin{cases} 
                            \sqrt[3]{5}\mapsto\zeta^4\sqrt[3]{5}\\
                            \zeta\mapsto\zeta
                        \end{cases} \\
                        &\psi_4:=\begin{cases} 
                            \sqrt[3]{5}\mapsto\sqrt[3]{5}\\
                            \zeta\mapsto\zeta^5
                        \end{cases} && \psi_5:=\begin{cases} 
                            \sqrt[3]{5}\mapsto\zeta^2\sqrt[3]{5}\\
                            \zeta\mapsto\zeta^5
                        \end{cases} && \psi_6 :=\begin{cases} 
                            \sqrt[3]{5}\mapsto\zeta^4\sqrt[3]{2}\\
                            \zeta\mapsto\zeta^5.
                        \end{cases}
                    \end{align*}
                As above, we see that the minimal polynomial of $\sqrt{2}$ over $\mathbb{Q}$ is $x^2-2$ which factors as $(x-\sqrt{2})(x+\sqrt{2})$ over $\mathbb{Q}(\sqrt{2})$. The minimal polynomial of $i$ over $\mathbb{Q}$ is $x^2+1$, which factors as $(x-i)(x+i)$ over $\mathbb{Q}(i)$. Thus, there are 4 possible 1-1 homomorphisms from $\mathbb{Q}(\sqrt{2},i)$ to $\mathbb{C}$. We have:
                    \begin{align*}
                        &\psi_1:=\begin{cases}
                            \sqrt{2}\rightarrow\sqrt{2}\\
                            i\rightarrow i
                        \end{cases} && \psi_2:=\begin{cases}
                            \sqrt{2}\rightarrow-\sqrt{2}\\
                            i\rightarrow i
                        \end{cases} \\
                        &\psi_3:=\begin{cases}
                            \sqrt{2}\rightarrow\sqrt{2}\\
                            i\rightarrow-i
                        \end{cases} && \psi_4:=\begin{cases}
                            \sqrt{2}\rightarrow-\sqrt{2}\\
                            i\rightarrow-i.
                        \end{cases}
                    \end{align*}}
            \end{solution}
        \item[5.] For each of the following fields, and mappings, $\varphi$, determine if $\varphi$ is an automorphism of the field, and if so, then find $F_{\varphi}$, and find $[E\colon F_{\varphi}]$.
            \begin{enumerate}
                \item $\mathbb{Q}(i)$, $\varphi(i)=-i$.
                    \begin{solution}
                        This is an automorphism. Since $\mathbb{Q}(i)$ is the splitting field for $x^2+1=(x-i)(x+i)$, then all we need is that $i$ be mapped to itself or $-i$. We also know that $\varphi$ is the identity over $\mathbb{Q}$ and thus for any $a+bi\in\mathbb{Q}(i)$, we have $\varphi(a+bi)=a-bi$ and so $\mathbb{Q}\subseteq F_{\varphi}$. Moreover, since $i$ is the only element which does not map to itself, then $F_{\varphi}=\mathbb{Q}$ and so $[\mathbb{Q}(i)\colon\mathbb{Q}]=2$.
                    \end{solution}
                \item $\mathbb{Q}(\omega)$, $\varphi(\omega)=\omega^2$.
                    \begin{solution}
                        This is an automorphism. Since $x^2+x+1=(x-\omega)(x-\omega^2)$ is the minimal polynomial associated with $\omega$, then all we need is that $\varphi(\omega)\in\{\omega,\omega^2\}$, which it is. From the definition, it follows that $\varphi(\omega^2)=\omega$. Thus, the only fixed elements of this automorphism are those in $\mathbb{Q}$ and so $F_{\varphi}=\mathbb{Q}$ and $[\mathbb{Q}(\omega)\colon\mathbb{Q}]=2$.
                    \end{solution}
                \item $\mathbb{Q}(\omega)$, $\varphi(\omega)=-\omega$.
                    \begin{solution}
                        This is not an automorphism, since $-\omega$ is not a root of $x^2+x+1$.
                    \end{solution}
                \item $\mathbb{Q}(x)$, $\varphi(x)=1/x$.
                    \begin{solution}
                        $\varphi$ is an automorphism and so $\varphi(x+\frac{1}{x})=\varphi(x)+\varphi(\frac{1}{x})=\frac{1}{x}+x$. Thus, $\mathbb{Q}(x+\frac{1}{x})\subseteq F_{\varphi}$. Note that $x$ is a root of $z^-(x+\frac{1}{x})z+1$. And so the minimal polynomial for $x$ has degree $\leq 2$. However, if it had degree 1, then $[\mathbb{Q}(x)\colon\mathbb{Q}(x+\frac{1}{x})]=1$ which would imply that $\mathbb{Q}(x)=\mathbb{Q}(x+\frac{1}{x})$. This is not true since not every element of $\mathbb{Q}(x)$ is fixed. Thus, $[\mathbb{Q}(x)\colon\mathbb{Q}(x+\frac{1}{x})]=2$. Since for any $f(x)\in F_{\varphi}$ such that $f(x)\notin\mathbb{Q}(x+\frac{1}{x})$ we would have $[\mathbb{Q}(x+\frac{1}{x})\colon\mathbb{Q}(x+\frac{1}{x})]>1$ and so $[\mathbb{Q}(x)\colon\mathbb{Q}(x+\frac{1}{x},f(x))]=1$, but this is not possible. Therefore, $F_{\varphi}=\mathbb{Q}(x+\frac{1}{x})$ and $[\mathbb{Q}(x)\colon F_{\varphi}]=2$.
                    \end{solution}
                \item $GF(2^n)$, $\varphi(a)=a^2$.
                    \begin{solution}
                        $\varphi$ is an automorphism. Since it is 1-1, then for any $a\in GL(2^n)$ such that $\varphi(a)=a^2=a$, it follows that $a=1$ or $a=0$. However, $a\neq 1$ since if $1\in GF(2^n)$, then $\varphi(1+1)=1+1=2$, but we know $\varphi(2)=4$. Thus, $F_{\varphi}=\{0\}$ and $[GL(2^n)\colon F_{\varphi}]=2^n$.
                    \end{solution}
            \end{enumerate}
        \item[6.] For each of the fields, and subsets, $S$, of the automorphism group of the field, find $F_S$, and find $[E\colon F_S]$. For (b)-(e), the $\varphi$'s are defined in the solution to HW5.
            \begin{enumerate}
                \item $\mathbb{Q}(i)$, $S=\{\text{identity}, \varphi(i)=i\}$.
                    \begin{solution}
                        Clearly, with the identity automorphism we have that all of $\mathbb{Q}(i)$ is fixed, however, with $\varphi(i)=-i$, only $\mathbb{Q}$ is fixed. Thus, $F_S=\mathbb{Q}(i)\cap\mathbb{Q}=\mathbb{Q}$ and so $[\mathbb{Q}(i)\colon\mathbb{Q}]=2$. 
                    \end{solution}
                \item $\mathbb{Q}(\sqrt[3]{2},\omega)$, $S=\{\varphi_1,\varphi_2\}$.
                    \begin{solution}
                        Here we have that the fixed field of $\varphi_1$ is $\mathbb{Q}(\sqrt[3]{2},\omega)$ and for $\varphi_2$ it is just $\mathbb{Q}(\omega \sqrt[3]{2}$. Thus, $F_S=\mathbb{Q}(\sqrt[3]{2})$ and, by HW5, $[\mathbb{Q}(\sqrt[3]{2},\omega)\colon\mathbb{Q}]=2$.
                    \end{solution}
                \item $\mathbb{Q}(\sqrt[3]{2},\omega)$, $S=\{\varphi_1,\varphi_3\}$.
                    \begin{solution}
                        With $\varphi_1$, the fixed field is all of $\mathbb{Q}(\sqrt[3]{2},\omega)$ and since with $\varphi_3$, we have that $\varphi(\sqrt[3]{2})=\sqrt[3]{2}$, then only $\omega$ has changed. Thus, $F_S=\mathbb{Q}(\sqrt[3]{2})$. Additionally, since $\omega\notin\mathbb{Q}(\sqrt[3]{2})$, then the minimal polynomial of $\omega$ over $\mathbb{Q}(\sqrt[3]{2})$ is $x^2+x+1$ and thus $[\mathbb{Q}(\sqrt[3]{2},\omega)\colon\mathbb{Q}(\sqrt[3]{2})]=2$.
                    \end{solution}
                \item $\mathbb{Q}(\sqrt[3]{2},\omega)$, $S=\{\varphi_1, \varphi_4,\varphi_5\}$. 
                    \begin{solution}
                        The fixed field for the identity map is all of $\mathbb{Q}(\sqrt[3]{2},\omega)$. The fixed field for $\varphi_4$ is $\mathbb{Q}(\omega)$ since $\varphi(\sqrt[3]{2})=\omega\sqrt[3]{2}$ and $\varphi_4(\omega)=\omega$. Similarly, $\varphi_5$ leaves $\omega$ unchanged while $\varphi_5(\sqrt[3]{2})=\omega^2\sqrt[3]{2}$ and so the fixed field for $\varphi_5$ is $\mathbb{Q}(\omega)$. Thus, $F_S=\mathbb{Q}(\omega)$ and $[\mathbb{Q}(\sqrt[3]{2},\omega)\colon\mathbb{Q}(\omega)]=3$ since $x^3-2$ is the minimal polynomial of $\sqrt[3]{2}$ over $\mathbb{Q}(\omega)$.  
                    \end{solution}
                \item $\mathbb{Q}(\sqrt[3]{2},\omega)$, $S=\{\varphi_1,\varphi_2, \varphi_6\}$.
                    \begin{solution}
                        In this case both $\varphi_2$ and $\varphi_6$ map $\sqrt[3]{2}$ and $\omega$ to elements different from themselves. However, $\varphi_6(\omega^2\sqrt[3]{4})$. And so $F_S=\mathbb{Q}(\omega^2\sqrt[3]{4})$ and $[\mathbb{Q}(\sqrt[3]{2},\omega)\colon\mathbb{Q}(\omega^2\sqrt[3]{4})]=3$.
                    \end{solution}
            \end{enumerate}
        \item[7.] Prove that $f(x)$ has a multiple root in its splitting field iff $f$ and $f'$ have a common factor of degree $\geq 1$.
            \begin{proof}
                Let $E$ denote the splitting field of $f(x)$ and assume $\alpha\in E$ is a root of multiplicity $k>1$. Then by definition, $(x-\alpha)^k\mid f(x)$ and so we may write
                    \begin{equation*}
                        f(x)=(x-\alpha)^k\sum\limits_{i=0}^ma_ix^i.
                    \end{equation*}
                Taking the derivative of both sides we get
                    \begin{equation*}
                        \begin{split}
                            \big(f(x)\big)' &= \bigg((x-\alpha)^k\sum\limits_{i=0}^ma_ix^i\bigg)'\\
                            &=k(x-\alpha)^{k-1}\sum\limts_{i=0}^ma_ix^i+(x-\alpha)^k\sum\limits_{i=1}^mia_ix^{i-1} \\
                            &= (x-\alpha)^{k-1}\bigg(k\sum\limits_{i=0}^ma_ix^i+(x-\alpha)\sum\limits_{i=1}ia_ix^{i-1}\bigg).
                        \end{split}
                    \end{equation*}
                Thus, $(x-\alpha)^{k-1}\mid f'(x)$ and $(x-\alpha)^{k-1}\mid f(x)$. Since $k>1$, then $k-1\geq 1$. Hence, $f(x)$ and $f'(x)$ have a common factor of degree $\geq 1$. Now to argue the contrapositive, assume that $f(x)$ does not have a multiple root in its splitting field. Then letting $\alpha$ be a root of $f(x)$, we can write $f(x)=(x-\alpha)g(x)$, where $(x-\alpha)\nmid g(x)$. From here we see that the derivative of $f$ is $f'(x)=g(x)+(x-\alpha)g'(x)$. Thus, $(x-\alpha)\nmid f'(x)$ and since $\alpha$ was any root of $f(x)$, then it follows that $f(x)$ and $f'(x)$ do not share any common factors.
            \end{proof}
        \item[8.] \hfill\par
            \begin{enumerate}
                \item Assume that $\text{char}(F)=0$, $f(x)\in F[x]$, and $f(x)$ is irreducible over $F$. Prove that $f$ cannot have any roots of multiplicity greater than 1.
                    \begin{proof}
                        Assume for contradiction that $f(x)$ has a root of multiplicity greater than 1. Then by 7., $f(x)$ and $f'(x)$ have a common factor. Thus, if $a$ is the root of multiplicity, then the minimal polynomial of $a$ over $F$ is $f(x)$ since $f(x)$ is irreducible (assuming it is monic). However, since $a$ is also a root of $f'(x)$, then it must be the case that $f(x)\mid f'(x)$. This is a contradiction as $\text{deg}(f'(x))<\text{deg}(f(x))$.
                    \end{proof}
                \item Assume that $\text{char}(F)=p$, $f(x)\in F[x]$, $f(x)$ is irreducible over $F$. Prove that if $f$ has a multiple root then there exists $g(x)\in F[x]$ such that $f(x)=g(x^p)$.
                    \begin{proof}
                        Assume that $f$ has a multiple root. Then by 7., $f$ and $f'$ share a common factor. Since $f$ is irreducible, then $\text{gcd}(f,f')=f$. This implies that $f'(x)=0$. Thus, for some $g(x)\in F[x]$, $f(x)=g(x^p)$. In other words, if we let $f(x)=a_0+a_1x^p+\cdots a_nx^{np}$, then $f'(x)=pa_1x^{p-1}+\cdots npa_nx^{np-1}$ and since $\text{char}(F)=p$, then every term in $f'(x)$ vanishes.  
                    \end{proof}
            \end{enumerate}
        \item[9.] \hfill\par 
            \begin{enumerate}
                \item Prove that if $p$ is prime, $p\nmid n$, then $x^n-1$ has $n$ distinct roots over $\mathbb{Z}_p$.
                    \begin{proof}
                        Let $f(x)=x^n-1$, then $f'(x)=nx^{n-1}$. Since $p\nmid n$, then $f'(1)\neq 0$. Seeing that the only root of $f'(x)$ is 0 and $f(0)\neq 0$, then it follows that $f(x)$ and $f'(x)$ share no common roots and thus no common factors. Thus, $f(x)$ has no roots of multiplicity. Thus, $x^n-1$ has $n$ distinct roots over $\mathbb{Z}_p$.
                    \end{proof}
                \item[(c)] Assume that $\zeta$ is a primitive $n^{\text{th}}$ root of unity. Determine $[\mathbb{Q}(\zeta)\colon\mathbb{Q}]$.
                    \begin{solution}
                        By part (b) we know that if $f(x)\mid x^n-1$ and $\zeta$ is a root of $f(x)$ then $\zeta^p$, where $p$ prime and $p\nmid n$, is also a root. Thus, the total number of roots of such an $f(x)$ is given by $\phi(n)$. Thus, the minimal polynomial of $\zeta$ has degree $\phi(n)$ and therefore, $[\mathbb{Q}(\zeta)\colon\mathbb{Q}]=\phi(n)$.
                    \end{solution}
            \end{enumerate}
    \end{enumerate}
\end{document}