\documentclass[12pt]{article}
\usepackage[margin=1in]{geometry} 
\usepackage{graphicx}
\usepackage{amsmath}
\usepackage{authblk}
\usepackage{titlesec}
\usepackage{amsthm}
\usepackage{amsfonts}
\usepackage{amssymb}
\usepackage{array}
\usepackage{booktabs}
\usepackage{ragged2e}
\usepackage{enumerate}
\usepackage{enumitem}
\usepackage{cleveref}
\usepackage{slashed}
\usepackage{commath}
\usepackage{lipsum}
\usepackage{colonequals}
\usepackage{addfont}
\usepackage{enumitem}
\usepackage{sectsty}
\usepackage{lastpage}
\usepackage{fancyhdr}
\usepackage{accents}
\usepackage{xcolor}
\usepackage[inline]{enumitem}
\pagestyle{fancy}

\fancyhf{}
\rhead{Darcy}
\lhead{MATH 210B}
\rfoot{\thepage}
\setlength{\headheight}{10pt}

\subsectionfont{\itshape}

\newtheorem{theorem}{Theorem}[section]
\newtheorem{corollary}{Corollary}[theorem]
\newtheorem{prop}{Proposition}[section]
\newtheorem{lemma}[theorem]{Lemma}
\theoremstyle{definition}
\newtheorem{definition}{Definition}[section]
\theoremstyle{remark}
\newtheorem*{remark}{Remark}
 
\makeatletter
\renewenvironment{proof}[1][\proofname]{\par
  \pushQED{\qed}%
  \normalfont \topsep6\p@\@plus6\p@\relax
  \list{}{\leftmargin=0mm
          \rightmargin=4mm
          \settowidth{\itemindent}{\itshape#1}%
          \labelwidth=\itemindent
          \parsep=0pt \listparindent=\parindent 
  }
  \item[\hskip\labelsep
        \itshape
    #1\@addpunct{.}]\ignorespaces
}{%
  \popQED\endlist\@endpefalse
}

\newenvironment{solution}[1][\bf{\textit{Solution}}]{\par
  
  \normalfont \topsep6\p@\@plus6\p@\relax
  \list{}{\leftmargin=0mm
          \rightmargin=4mm
          \settowidth{\itemindent}{\itshape#1}%
          \labelwidth=\itemindent
          \parsep=0pt \listparindent=\parindent 
  }
  \item[\hskip\labelsep
        \itshape
    #1\@addpunct{.}]\ignorespaces
}{%
  \popQED\endlist\@endpefalse
}

\let\oldproofname=\proofname
\renewcommand{\proofname}{\bf{\textit{\oldproofname}}}


\newlist{mylist}{enumerate*}{1}
\setlist[mylist]{label=(\alph*)}

\begin{document}\thispagestyle{empty}\hline

\begin{center}
	\vspace{.4cm} {\textbf { \large MATH 210B}}
\end{center}
{\textbf{Name:}\ Quin Darcy \hspace{\fill} \textbf{Due Date:} 3/25/20   \\
{ \textbf{Instructor:}}\ Dr. Shannon \hspace{\fill} \textbf{Assignment:} Homework 7 \\ \hrule}

\justifying

    \begin{enumerate}[leftmargin=*]
        \item[10.] Let $E=\mathbb{Q}(\sqrt[3]{2},\sqrt[4]{5},i)$. For $0\leq j\leq 3$, define $\varphi_j\in\text{Aut}(E)$ by $\varphi_j(i)=(-1)^ji$ and $\varphi_j(\sqrt[4]{5})=i^j\sqrt[4]{5}$, and for $4\leq j\leq 7$, define $\varphi_j\in\text{Aut}(E)$ by $\varphi_j(i)=(-1)^{j+1}i$ and $\varphi_j(\sqrt[4]{5})=i^j\sqrt[4]{5}$.
            \begin{enumerate}
                \item Find $G(E/\mathbb{Q}(i))$, $G(E/\mathbb{Q}(\sqrt[3]{2}))$, $G(E/\mathbb{Q}(\sqrt[4]{5}))$, $G(E/\mathbb{Q}(\sqrt[4]{5},i))$.
                    \begin{solution}
                        Considering all the automorphisms which map $i$ to $i$ we get that $G(E/\mathbb{Q}(i))=\{\varphi_0,\varphi_2,\varphi_5,\varphi_7\}$. Note that $x^3-2$ is the minimal polynomial of $\sqrt[3]{2}$ over $E$ and so $E$ only contains one root of this polynomial. Thus, every automorphism must map $\sqrt[3]{2}$ to itself and so $G(E/\mathbb{Q}(\sqrt[3]{2}))=\text{Aut}(E)$. Considering all the automorphisms which map $\sqrt[4]{5}$ to itself, we get that $G(E/\mathbb{Q}(\sqrt[4]{5})=\{\varphi_0,\varphi_4\}$. Finally, there is only one automorphism which fixes both $i$ and $\sqrt[4]{5}$ and so $G(E/\mathbb{Q}(\sqrt[4]{5},i))=\{\varphi_0\}$.
                \end{solution} 
            \item Find $[E\colon\mathbb{Q}(i)]$, $[E\colon\mathbb{Q}(\sqrt[3]{2})]$, $[E\colon\mathbb{Q}(\sqrt[4]{5})]$, $[E\colon\mathbb{Q}(\sqrt[4]{5},i)]$.
                \begin{solution}
                    Since the minimal polynomial of $\sqrt[3]{2}$ over $\mathbb{Q}(i)$ is $x^3-2$ and the minimal polynomial of $\sqrt[4]{5}$ over $\mathbb{Q}(i)$ is $x^4-5$, then we have that $[E\colon\mathbb{Q}(i)]=3\cdot 4=12$. Similarly, since $x^4-5$ is the minimal polynomial of $\sqrt[4]{5}$ over $\mathbb{Q}(\sqrt[3]{2})$ and $x^2+1$ is the minimal polynomial of $i$ over $\mathbb{Q}(\sqrt[3]{2})$, then $[E\colon\mathbb{Q}(\sqrt[3]{2})]=4\cdot 2=8$. Again, we have that $x^3-2$ is the minimal polynomial of $\sqrt[3]{2}$ over $\mathbb{Q}(\sqrt[4]{5})$ and $x^2+1$ is the minimal polynomial of $i$ over $\mathbb{Q}(\sqrt[4]{5})$ and so $[E\colon\mathbb{Q}(\sqrt[4]{5})]=3\cdot 2=6$. Lastly, the minimal polynomial of $\sqrt[3]{2}$ over $\mathbb{Q}(\sqrt[4]{5},i)$ is $x^3-2$ and so $[E\colon\mathbb{Q}(\sqrt[4]{5},i)]=3$.
                \end{solution}
            \item Determine which of these extensions are normal.
                \begin{solution}
                    Note that $x^3-2$ is irreducible over $\mathbb{Q}(i)$, $\mathbb{Q}(\sqrt[4]{5})$, and $\mathbb{Q}(\sqrt[4]{5},i)$, and $E$ contains a root of $x^3-2$, yet $x^3-2$ does not split over $E$ and so none of these extensions are normal. Now note that $x^3-2$ is reducible over $\mathbb{Q}(\sqrt[3]{2})$, but its nonlinear irreducible factor is $x^2+\sqrt[3]{2}x+\sqrt[3]{4}$. $E$ does not contain a root of this polynomial and so all other irreducible polynomials over $\mathbb{Q}(\sqrt[3]{2})$ split over $E$. Thus, $E/\mathbb{Q}(\sqrt[3]{2})$ is a normal extension.
                \end{solution}
            \item Determine which of these extensions are separable.
                \begin{solution}
                    We have that both $\sqrt[3]{2}$ and $\sqrt[4]{5}$ are algebraic over $\mathbb{Q}(i)$ and neither $x^3-2$ and $x^4-5$ have multiple roots in any extension. Thus, $E/\mathbb{Q}(i)$ is separable. By the same reasoning we can see that $i,\sqrt[3]{2},$ and $\sqrt[4]{5}$ are all algebraic over each of the base fields and all of the respective minimal polynomials do not contain multiple roots. Thus, every extension is separable.
                \end{solution}\newpage
            \item For each $K$, determine $F_{G(E/K)}$, and determine if $F_{G(E/K)}=K$.
                \begin{solution}
                    Based on part (a), we have that
                        \begin{equation*}
                            \begin{split}
                                F_{G(E/\mathbb{Q}(i))} &= \mathbb{Q}(\sqrt[3]{2},i) \\
                                F_{G(E/\mathbb{Q}(\sqrt[3]{2}))}&=\mathbb{Q}(\sqrt[3]{2}) \\
                                F_{G(E/\mathbb{Q}(\sqrt[4]{5}))}&=\mathbb{Q}(\sqrt[3]{2},\sqrt[4]{5}) \\
                                F_{G(E/\mathbb{Q}(\sqrt[4]{5},i))} &= \mathbb{Q}(\sqrt[3]{2},\sqrt[4]{5},i).
                            \end{split}
                        \end{equation*}
                    $E/\mathbb{Q}(\sqrt[3]{2})$ satisfies $F_{G(E/K)}=K$.
                \end{solution}
        \end{enumerate}    
    \end{enumerate}
\end{document}