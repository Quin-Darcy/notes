\documentclass[12pt]{article}
\usepackage[margin=1in]{geometry} 
\usepackage{graphicx}
\usepackage{amsmath}
\usepackage{authblk}
\usepackage{titlesec}
\usepackage{amsthm}
\usepackage{amsfonts}
\usepackage{amssymb}
\usepackage{array}
\usepackage{booktabs}
\usepackage{ragged2e}
\usepackage{enumerate}
\usepackage{enumitem}
\usepackage{cleveref}
\usepackage{slashed}
\usepackage{commath}
\usepackage{lipsum}
\usepackage{colonequals}
\usepackage{addfont}
\usepackage{enumitem}
\usepackage{sectsty}
\usepackage{lastpage}
\usepackage{fancyhdr}
\usepackage{accents}
\usepackage{xcolor}
\usepackage[inline]{enumitem}
\pagestyle{fancy}

\fancyhf{}
\rhead{Darcy}
\lhead{MATH 210B}
\rfoot{\thepage}
\setlength{\headheight}{10pt}

\subsectionfont{\itshape}

\newtheorem{theorem}{Theorem}[section]
\newtheorem{corollary}{Corollary}[theorem]
\newtheorem{prop}{Proposition}[section]
\newtheorem{lemma}[theorem]{Lemma}
\theoremstyle{definition}
\newtheorem{definition}{Definition}[section]
\theoremstyle{remark}
\newtheorem*{remark}{Remark}
 
\makeatletter
\renewenvironment{proof}[1][\proofname]{\par
  \pushQED{\qed}%
  \normalfont \topsep6\p@\@plus6\p@\relax
  \list{}{\leftmargin=0mm
          \rightmargin=4mm
          \settowidth{\itemindent}{\itshape#1}%
          \labelwidth=\itemindent
          \parsep=0pt \listparindent=\parindent 
  }
  \item[\hskip\labelsep
        \itshape
    #1\@addpunct{.}]\ignorespaces
}{%
  \popQED\endlist\@endpefalse
}

\newenvironment{solution}[1][\bf{\textit{Solution}}]{\par
  
  \normalfont \topsep6\p@\@plus6\p@\relax
  \list{}{\leftmargin=0mm
          \rightmargin=4mm
          \settowidth{\itemindent}{\itshape#1}%
          \labelwidth=\itemindent
          \parsep=0pt \listparindent=\parindent 
  }
  \item[\hskip\labelsep
        \itshape
    #1\@addpunct{.}]\ignorespaces
}{%
  \popQED\endlist\@endpefalse
}

\let\oldproofname=\proofname
\renewcommand{\proofname}{\bf{\textit{\oldproofname}}}


\newlist{mylist}{enumerate*}{1}
\setlist[mylist]{label=(\alph*)}

\begin{document}\thispagestyle{empty}\hline

\begin{center}
	\vspace{.4cm} {\textbf { \large MATH 296B}}
\end{center}
{\textbf{Name:}\ Quin Darcy \hspace{\fill} \textbf{Due Date:} 4/28/20   \\
{ \textbf{Instructor:}}\ Dr. Pigno \hspace{\fill} \textbf{Assignment:} Homework 6 \\ \hrule}

\justifying

    \begin{enumerate}[leftmargin=*]
        \item[2.] Prove that if $(n,k)=1$, then $n\mid\binom{n}{k}$.
            \begin{proof}
                Assume that $(n,k)=1$. Then 
                    \[
                        \binom{n}{k}=\frac{n!}{k!(n-k)!}.
                    \]
                Additionally, since $(n,k)=1$, then there is no cancellation that occurs with $\frac{n!}{k!}$. Moreover, since $(n-k)!=(n-k)(n-k-1)\cdots2\cdot1$, then 
                    \begin{equation*}
                        \begin{split}
                            \binom{n}{k}&=\frac{n!}{k!(n-k)!}\\&=\frac{n\cdot(n-1)\cdots(n-k)(n-k-1)\cdots2\cdot1}{k!(n-k)(n-k-1)\cdots2\cdot1}\\&=\frac{n\cdot(n-1)\cdots(n-k+1)}{k!}. 
                        \end{split}
                    \end{equation*}
                Since $n\mid n(n-1)\cdots(n-k+1)$, then $n\mid\binom{n}{k}$.
            \end{proof}
        \item[3.]\hfill\par
            \begin{enumerate}
                \item Prove that $\sigma(p^e)\geq p^e(1+\frac{1}{p})$.
                    \begin{proof}
                        Given that $\sigma(n)$ denotes the sum of the positive divisors of $n$, then we have that for $p^e$, then divisors are:
                            \begin{equation*}
                                1, p, p^2,\dots, p^e,
                            \end{equation*}
                        and so
                            \begin{equation*}
                                \sigma(p^e)=1+p+p^2+\cdots+p^e=p^e(\frac{1}{p^e}+\frac{1}{p^{e-1}}+\cdots+\frac{1}{p}+1).
                            \end{equation*}
                        Thus, since
                            \begin{equation*}
                                (1+\frac{1}{p})\leq (1+\frac{1}{p}+\cdots+\frac{1}{p^{e-1}}+\frac{1}{p^e})
                            \end{equation*}
                        then $\sigma(p^e)\geq p^e(1+\frac{1}{p})$.
                    \end{proof}
                \item Prove that for any $n\in\mathbb{N}$
                    \begin{equation*}
                        \phi(n)+\sigma(n)\geq 2n.
                    \end{equation*}
                    \begin{proof}
                        Let $n\in\mathbb{N}$ with $n=p_1^{e_1}\cdots p_s^{e_s}$. Then 
                            \begin{equation*}
                                \begin{split}
                                    \phi(n)&=n\prod_{i=1}^s\left(1-\frac{1}{p_i}\right)=n\prod_{i=1}^s\left(\frac{p_i-1}{p_i}\right).
                                \end{split}
                            \end{equation*}
                        And 
                            \begin{equation*}
                                \sigma(n)=\prod_{i=1}^s\left(\frac{p_i^{e_i+1}-1}{p_i-1}\right)
                            \end{equation*}
                        Dividing both functions by $n$, we obtain
                            \begin{equation*}
                                \frac{\phi(n)}{n}=\prod_{i=1}^s\left(\frac{p_i-1}{p_i}\right)\quad,\frac{\sigma(n)}{n}=\prod_{i=1}^s\left(\frac{p_i-\frac{1}{p_i^{e_i}}}{p_i-1}\right).
                            \end{equation*}
                        It follows that the latter formula satisfies
                            \begin{equation*}
                                \prod_{i=1}^s\left(\frac{p_i-\frac{1}{p_i^{e_i}}}{p_i-1}\right)\geq \prod_{i=1}^s\left(\frac{p_i+1}{p_i}\right).
                            \end{equation*}
                        Thus,
                            \begin{equation*}
                                \frac{\phi(n)}{n}+\frac{\sigma(n)}{n}\geq\prod_{i=1}^s\left(\frac{p_i-1}{p_i}\right)+\prod_{i=1}^s\left(\frac{p_i+1}{p_i}\right).
                            \end{equation*}
                        We can also see that for all $n\geq 1$, $(p-1)^n+(p+1)^n=2p^np^{n-1}\cdots p$ since all the terms in between cancel out. And so 
                            \begin{equation*}
                                \prod_{i=1}^s\left(\frac{p_i-1}{p_i}\right)+\prod_{i=1}^s\left(\frac{p_i+1}{p_i}\right)=2\prod_{i=1}^s\frac{p_i}{p_i}=2.
                            \end{equation*}
                        Finally, multiplying through by $n$, we obtain $\phi(n)+\sigma(n)\geq 2n$.
                    \end{proof}
            \end{enumerate}
        \item[5.] Show that $\sigma(n)=\sum_{d\mid n}\phi(n)\tau(\frac{n}{d})$.
            \begin{proof}
                In class we showed that if $f$ and $g$ are multiplicative functions, then so is 
                    \begin{equation*}
                        \sum_{d\mid n}f(d)g(\frac{n}{d}).
                    \end{equation*}
                So since $\phi(n)$ and $\tau(n)$ are both multiplicative, then 
                    \begin{equation*}
                        F(n)=\sum_{d\mid n}\phi(d)\tau(\frac{n}{d})
                    \end{equation*}
                is multiplicative. Now let $n=p_1^{e_1}\cdots p_r^{e_r}$. Then we have that 
                    \begin{equation*}
                        \sigma(n)=\sigma(p_1^{e_1}\cdots p_r^{e_r})=\sigma(p_1^{e_1})\cdots\sigma(p_r^{e_r})=\prod_{i=1}^r\left(\frac{p_i^{e_i+1}-1}{p_i-1}\right).
                    \end{equation*}
                Next, we evaluate $F(n)$. Doing this we find that \newpage
                    \begin{equation*}
                        \begin{split}
                            F(n)&=F(p_1^{e_1}\cdots p_r^{e_r}) \\
                            &= F(p_1^{e_1})\cdots F(p_r^{e_r}) \\
                            &= \left(\sum_{d\mid p_1^{e_1}}\phi(d)\tau\left(\frac{p_1^{e_1}}{d}\right)\right)\cdots\left(\sum_{d\mid p_r^{e_r}}\phi(d)\tau\left(\frac{p_r^{e_r}}{d}\right)\right) \\
                            &=\prod_{i=1}^r\left(\sum_{d\mid p_i^{e_i}}\phi(d)\tau\left(\frac{p_i^{e_i}}{d}\right)\right).
                        \end{split}
                    \end{equation*}
                Noting that for any $i$, the divisors of $p_i^{e_i}$ are simply $1,p_i,p_i^2,\dots,p_i^{e_i}$, then we can rewrite the last equality as
                    \begin{equation*}
                        \prod_{i=1}^r\left(\sum_{j=0}^{e_i}\phi(p_i^j)\tau\left(\frac{p_i^{e_i}}{p_i^j}\right)\right)=\prod_{i=1}^r\left(\sum_{j=0}^{e_i}\phi(p_i^j)\tau(p_i^{e_i-j})\right).
                    \end{equation*}
                We can further simplify by noting that for each $i$ and each $j$ we have
                    \begin{equation*}
                        \phi(p_i^j)=p_i^j-p_i^{j-1}\quad\text{and}\quad\tau(p_i^{e_i-j})=e_i-j+1.
                    \end{equation*}
                It then follows that for all $i$
                    \begin{equation*}
                        \sum_{j=0}^{e_i}\left((p_i^j-p_i^{j-1})(e_i-j+1)\right)=\frac{p_i^{e+1}-1}{p_i-1}
                    \end{equation*}
                and so
                    \begin{equation*}
                        \sum_{d\mid n}\phi(d)\tau(\frac{n}{d})=F(n)=\prod_{i=1}^r\left(\frac{p_i^{e_i+1}-1}{p_i-1}\right)=\sigma(n).
                    \end{equation*}
            \end{proof}
        \item[6.] Let $\Omega(n)$ denote the number of prime factors of $n$ counted with multiplicity. Let $\lambda(n)=(-1)^{\Omega(n)}$. Show that $\lambda$ is totally multiplicative and 
            \begin{equation*}
                \sum_{d\mid n}\lambda(d)=\begin{cases} 1, & \text{if $n$ is a square} \\ 0, & \text{otherwise} \end{cases}.
            \end{equation*}
            \begin{proof}
                To begin we note that $\Omega(p)=1$, $\Omega(p^a)=a$. Now let
                    \begin{equation*}
                        n=ab=(p_1^{e_1}\cdots p_r^{e_r})(q_1^{m_1}\cdots q_s^{m_s}).
                    \end{equation*}
                Then 
                    \begin{equation*}
                        \begin{split}
                            \Omega(n) &= \Omega(ab) \\
                            &= \Omega(p_1^{e_1}\cdots p_r^{e_r}q_1^{m_1}\cdots q_s^{m_s}) \\
                            &= e_1+\cdots+ e_r+ m_1+\cdots +m_s.
                        \end{split}
                    \end{equation*}
                It follows that
                    \begin{equation*}
                        \begin{split}
                            \lambda(n)&=\lambda(ab) \\
                            &=(-1)^{\Omega(ab)} \\
                            &=(-1)^{\Omega(p_1^{e_1}\cdots p_r^{e_r}q_1^{m_1}\cdots q_s^{m_s})} \\
                            &=(-1)^{e_1+\cdots e_r+m_1+\cdots+m_s} \\
                            &=(-1)^{e_1+\cdots+e_r}(-1)^{m_1+\cdots+m_s} \\
                            &=(-1)^{\Omega(p_1^{e_1}\cdots p_r^{e_r})}(-1)^{\Omega(q_1^{m_1}\cdots q_s^{m_s})} \\
                            &=(-1)^{\Omega(a)}(-1)^{\Omega(b)} \\
                            &=\lambda(a)\lambda(b).
                        \end{split}
                    \end{equation*}
                Thus, $\lambda$ is totally multiplicative. Letting
                    \begin{equation*}
                        F(n)=\sum_{d\mid n}\lambda(d)
                    \end{equation*}
                then from results proven in class, we know $F(n)$ is multiplicative. Thus if we assume $n$ is a square, then $n=\sqrt{n}\sqrt{n}$ and
                    \begin{equation*}
                        F(n)=F(\sqrt{n}\sqrt{n})=F(\sqrt{n})F(\sqrt{n})=\sum_{d\mid \sqrt{n}}\lambda(d)\sum_{d\mid \sqrt{n}}\lambda(d)=\left(\sum_{d\mid \sqrt{n}}\lambda(d)\right)^2.
                    \end{equation*}
                And this is where I am stuck $\dots$ Not sure if I can further simplify and say
                    \begin{equation*}
                        F(n)=\sum_{d\mid\sqrt{n}}\lambda(d^2);
                    \end{equation*}
                which, if $d=p_1^{e_1}\cdots p_w^{e_w}$, then $d^2$ would place a factor of $2$ on each exponent and so $\lambda(d^2)=1$? But that would mean $F(n)>1$ which is not what we want. Sorry the homework is not only late, but also not even complete! I am going to turn it in now, but keep working on this problem.
            \end{proof}
    \end{enumerate}
\end{document}