\documentclass[12pt]{article}
\usepackage[margin=1in]{geometry} 
\usepackage{graphicx}
\usepackage{amsmath}
\usepackage{authblk}
\usepackage{titlesec}
\usepackage{amsthm}
\usepackage{amsfonts}
\usepackage{amssymb}
\usepackage{array}
\usepackage{booktabs}
\usepackage{ragged2e}
\usepackage{enumerate}
\usepackage{enumitem}
\usepackage{cleveref}
\usepackage{slashed}
\usepackage{commath}
\usepackage{lipsum}
\usepackage{colonequals}
\usepackage{addfont}
\usepackage{enumitem}
\usepackage{sectsty}
\usepackage{lastpage}
\usepackage{fancyhdr}
\usepackage{accents}
\usepackage{xcolor}
\usepackage[inline]{enumitem}
\pagestyle{fancy}

\fancyhf{}
\rhead{Darcy}
\lhead{MATH 210B}
\rfoot{\thepage}
\setlength{\headheight}{10pt}

\subsectionfont{\itshape}

\newtheorem{theorem}{Theorem}[section]
\newtheorem{corollary}{Corollary}[theorem]
\newtheorem{prop}{Proposition}[section]
\newtheorem{lemma}[theorem]{Lemma}
\theoremstyle{definition}
\newtheorem{definition}{Definition}[section]
\theoremstyle{remark}
\newtheorem*{remark}{Remark}
 
\makeatletter
\renewenvironment{proof}[1][\proofname]{\par
  \pushQED{\qed}%
  \normalfont \topsep6\p@\@plus6\p@\relax
  \list{}{\leftmargin=0mm
          \rightmargin=4mm
          \settowidth{\itemindent}{\itshape#1}%
          \labelwidth=\itemindent
          \parsep=0pt \listparindent=\parindent 
  }
  \item[\hskip\labelsep
        \itshape
    #1\@addpunct{.}]\ignorespaces
}{%
  \popQED\endlist\@endpefalse
}

\newenvironment{solution}[1][\bf{\textit{Solution}}]{\par
  
  \normalfont \topsep6\p@\@plus6\p@\relax
  \list{}{\leftmargin=0mm
          \rightmargin=4mm
          \settowidth{\itemindent}{\itshape#1}%
          \labelwidth=\itemindent
          \parsep=0pt \listparindent=\parindent 
  }
  \item[\hskip\labelsep
        \itshape
    #1\@addpunct{.}]\ignorespaces
}{%
  \popQED\endlist\@endpefalse
}

\let\oldproofname=\proofname
\renewcommand{\proofname}{\bf{\textit{\oldproofname}}}


\newlist{mylist}{enumerate*}{1}
\setlist[mylist]{label=(\alph*)}

\begin{document}\thispagestyle{empty}\hline

\begin{center}
	\vspace{.4cm} {\textbf { \large MATH 296B}}
\end{center}
{\textbf{Name:}\ Quin Darcy \hspace{\fill} \textbf{Due Date:} 2/17/20   \\
{ \textbf{Instructor:}}\ Dr. Pigno \hspace{\fill} \textbf{Assignment:} Homework 2 \\ \hrule}

\justifying

    \begin{enumerate}[leftmargin=*]
        \item Prove that if $(b,c)=1$ then for any integer $a$, $(a,bc)=(a,b)(a,c)$.
            \begin{proof}
                We begin by noting that $(a,c)\mid a$, $(a,c)\mid c$, $c\mid bc$ and so $(a,c)\mid(a,bc)$. Similarly, we have that $(a,b)\mid a$, $(a,b)\mid b$, $b\mid bc$ and thus $(a,b)\mid (a,bc)$. Since $(a,c)\mid (a,bc)$ and $(a,b)\mid(a,bc)$, then $(a,b)(a,c)\mid(a,bc)$. Lastly, since $(a,bc)\mid(a,c)$, then $(a,bc)\mid(a,b)(a,c)$. Therefore, $(a,bc)=(a,b)(a,c)$.
            \end{proof}
        \item Prove that for any positive integers $a,b>1$ and $(a,b)=1$ we have that $\log_b(a)$ is irrational.
            \begin{proof}
                Assume for contradiction that $\log_b(a)$ is rational. Then for $p,q\in\mathbb{Z}$ with $q\neq 0$ we have that $\log_b(a)=\frac{p}{q}$. Taking $b$ as the base of both sides we get that $a=b^{p/q}$ which implies that $a^q=b^p$. Hence, $a\mid b^q$. However, $(a,b)=1$ and so $a\nmid b$ and $b\nmid a$ which implies that $a\nmid b^q$ which is a contradiction. Thus, $\log_b(a)$ is irrational.
            \end{proof}
        \item Show that if $n$ is a positive integer having no prime factor $p\leq\sqrt{n}$, then $n$ is prime. 
            \begin{proof}
                Assume $n$ is a positive integer having no prime factor $p\leq \sqrt{n}$. Then if $n=p_1p_2\dots p_m$ is the prime factorization of $n$, then $p_i>\sqrt{n}$ for all $i$. Thus, for any two $p_i,p_{i+1}$ from the factorization of $n$, then it follows that $\sqrt{n}\sqrt{n}=n<p_ip_{i+1}$ and so $n<p_1\cdots p_m=n$, which is a contradiction. Hence, either $n$ must have a prime factor $p\leq\sqrt{n}$ or $n$ has no factors other than itself and 1, i.e. $n$ is prime.
            \end{proof}
        \item Prove that there are an infinite number of primes of the form $4n-1$ and of the form $6n-1$. 
            \begin{proof}
                Assume there are only finitely many primes of the form $4n-1$, say $p_1p_2\cdots P$, where $P$ is the largest prime of this form. Now consider some $N=4(p_1p_2\cdots P)-1$, it follows that each $p_1,\dots, P$ is a divisor of $N+1$ and so $N$ has no prime divisors of the form $4n-1$. However, since $N$ cannot be prime, then one of its prime divisors must be less than or equal to $P$ which is a contradiction. Thus, there are infinitely many primes of the form $4n-1$. A similar argument shows this is true for $6n-1$. 
            \end{proof}
        \item[8.] Let $S$ be the set of integers $1,2,3,\dots,n$. Let $2^k$ be the largest power of two that is in $S$. Prove that $2^k$ is not a divisor of any other integer in $S$. Further show that for any positive integer $n\geq 2$, $\sum_{k=1}^n\frac{1}{k}$ is not a natural number.
            \begin{proof}
                Let $2^k$ be the highest power of $2$ in $S$. Then the next number in $S$, greater than $2^k$, which has $2^k$ as a factor must be a distance of $2^k$ away which would imply that $2^{k+1}$ is the next closest multiple of $2^k$. However, $2^k$ is the highest power of $2$ in $S$ and so no other number in $S$ can have $2^k$ as a factor in $S$.\hfill\par\hspace{4mm} Now assume that for some $n\geq 2$, $\sum_{k=1}^n \frac{1}{k}=z$ is an integer. Then let $2^k$ be the largest power of 2 between $1$ and $n$. Now define $A$ to be equal to $\frac{2^{k-1}n!}{2^a}$, where $a$ is the exponent on $2$ in the prime factorization of $n!$. Then multiplying both sides of this equation by $A$ we get 
                    \begin{equation*}
                        A+\frac{A}{2}+\frac{A}{3}+\cdots+\frac{A}{2^k}+\cdots+\frac{A}{n}=Az.
                    \end{equation*}
                We see that the factors of $A$ cancel out with every denominator on the LHS except for $\frac{1}{2^k}$. Thus, with the left hand side being a non integer and the right hand side being an integer, we have arrived at a contradiction.
            \end{proof}
    \end{enumerate}
\end{document}