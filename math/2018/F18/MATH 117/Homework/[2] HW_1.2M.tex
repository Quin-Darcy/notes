\documentclass[12pt, a4paper]{article}
\usepackage[margin=1in]{geometry}
\usepackage[latin1]{inputenc}
\usepackage{titlesec}
\usepackage{amsmath}
\usepackage{amsthm}
\usepackage{amsfonts}
\usepackage{amssymb}
\usepackage{array}
\usepackage{booktabs}
\usepackage{ragged2e}
\usepackage{enumerate}
\usepackage{enumitem}
\usepackage{cleveref}
\usepackage{slashed}
\usepackage{commath}
\usepackage{lipsum}
\usepackage{colonequals}
\usepackage{addfont}
\addfont{OT1}{rsfs10}{\rsfs}
\renewcommand{\baselinestretch}{1.1}
\usepackage[mathscr]{euscript}
\let\euscr\mathscr \let\mathscr\relax
\usepackage[scr]{rsfso}
\newcommand{\powerset}{\raisebox{.15\baselineskip}{\Large\ensuremath{\wp}}}
\usepackage{longtable}
\usepackage{multirow}
\usepackage{multicol}
\usepackage{calligra}
\usepackage[T1]{fontenc}
\newcounter{proofc}
\renewcommand\theproofc{(\arabic{proofc})}
\DeclareRobustCommand\stepproofc{\refstepcounter{proofc}\theproofc}
\newenvironment{twoproof}{\tabular{@{\stepproofc}c|l}}{\endtabular}
\newcolumntype{C}{>$c<$}
\usepackage{fancyhdr}
\pagestyle{fancy}
\fancyhf{}
\renewcommand{\headrulewidth}{0pt}
\fancyhead[R]{\thepage}
\usepackage{enumitem}
\usepackage{tikz}
\usepackage{commath}
\usepackage{colonequals}
\usepackage{bm}
\usepackage{tikz-cd}
\renewcommand{\baselinestretch}{1.1}
\usepackage[mathscr]{euscript}
\let\euscr\mathscr \let\mathscr\relax
\usepackage[scr]{rsfso}
\usepackage{titlesec}

\newcommand*{\logeq}{\ratio\Leftrightarrow}

\setlist[description]{leftmargin=10mm,labelindent=10mm}

\begin{document}

Quin Darcy\par
MATH 117\par
Dr.Ebrahimzadeh\par
8/27/18\par

\vspace{4mm}

\centerline{Textbook problems: 17, 19, 21, 22}

\vspace{4mm}

\begin{description}


\item\textbf{17. } Let $V=\{(a_1,a_2)|a_1,a_2\in F\}$, where $F$ is a field. Define addition of elements of $V$\ coordinatewise, and for $c\in F$ and $(a_1,a_2)\in V$, define

\vspace{2mm}

\centerline{$c(a_1,a_2)=(a_1,0)$.}

\vspace{2mm}

\par Is $V$ a vector space over $F$?

\vspace{4mm}

\textbf{Solution}

\vspace{4mm}

(VS1) Let $u,v\in V$. Then $u+v=(a_1,a_2)+(b_1,b_2)$. Since addition is defined coordinatewise, we have the final sum as $(a_1+b_1,a_2+b_2)$. Because $a_1,a_2,b_1,b_2\in F$ and $F$ is a field, then the addition of its elements are commutative. Thus, $(a_1+b_1,a_2+b_2)=(b_1+a_1,b_2+a_2)$. The right-hand side of this equality can be written as $(b_1+a_1,b_2+a_2)=(b_1,b_2)+(a_1,a_2)$. Thus, $(a_1,a_2)+(b_1,b_2)=(b_1,b_2)+(a_1,a_2)$. Therefore, vector addition is commutative.

\vspace{4mm}

(VS2) Let $u,v,w\in V$. Now consider the following\par

\vspace{4mm}

\centerline{$(u+v)+w=((a_1,a_2)+(b_1,b_2))+(c_1,c_2)$.}

\vspace{4mm}

Continuing through with the addition yields\par

\vspace{4mm}

\centerline{$(u+v)+w=(a_1+b_1,a_2+b_2)+(c_1+c_2)=(a_1+b_1+c_1,a_2+b_2+c_2)$.}

\vspace{4mm}

Since the elements of the field are associative with respect to their addition, we may regroup in the following way\par

\vspace{4mm}

\centerline{$(u+v)+w=(a_1+(b_1+c_1),(a_2+(b_2+c_2))=(a_1,a_2)+(b_1+c_1,b_2+c_2)$.}

\vspace{4mm}

Further simplification gives\par

\vspace{4mm}

\centerline{$(u+v)+w=(a_1,a_2)+((b_1,b_2)+(c_1,c_2))=u+(v+w)$.}

\vspace{4mm}

Therefore, vector addition is associative.

\vspace{4mm}



\end{description}



\end{document}