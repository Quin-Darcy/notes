\documentclass[12pt, a4paper]{article}
\usepackage[margin=1in]{geometry}
\usepackage[latin1]{inputenc}
\usepackage{titlesec}
\usepackage{amsmath}
\usepackage{amsthm}
\usepackage{amsfonts}
\usepackage{amssymb}
\usepackage{array}
\usepackage{booktabs}
\usepackage{ragged2e}
\usepackage{enumerate}
\usepackage{enumitem}
\usepackage{cleveref}
\usepackage{slashed}
\usepackage{commath}
\usepackage{lipsum}
\usepackage{colonequals}
\usepackage{addfont}
\addfont{OT1}{rsfs10}{\rsfs}
\renewcommand{\baselinestretch}{1.1}
\usepackage[mathscr]{euscript}
\let\euscr\mathscr \let\mathscr\relax
\usepackage[scr]{rsfso}
\newcommand{\powerset}{\raisebox{.15\baselineskip}{\Large\ensuremath{\wp}}}
\usepackage{longtable}
\usepackage{multirow}
\usepackage{multicol}
\usepackage{calligra}
\usepackage[T1]{fontenc}
\newcounter{proofc}
\renewcommand\theproofc{(\arabic{proofc})}
\DeclareRobustCommand\stepproofc{\refstepcounter{proofc}\theproofc}
\usepackage{fancyhdr}
\pagestyle{fancy}

\renewcommand{\headrulewidth}{0pt}
\fancyhead[R]{}
\usepackage{enumitem}
\usepackage{tikz}
\usepackage{commath}
\usepackage{colonequals}
\usepackage{bm}
\usepackage{tikz-cd}
\renewcommand{\baselinestretch}{1.1}
\usepackage[mathscr]{euscript}
\let\euscr\mathscr \let\mathscr\relax
\usepackage[scr]{rsfso}
\usepackage{titlesec}

\usepackage{lscape}



\newcommand*{\logeq}{\ratio\Leftrightarrow}

\titleformat{\section}
  {\normalfont\Large\bfseries}{\thesection}{1em}{}[{\titlerule[0.8pt]}]
  
  \setlist[description]{leftmargin=4mm,labelindent=4mm}
  
  
  
  
 \begin{document}
 
 \begin{flushleft}
 
 Quin Darcy\par
 Dr. Ebrahimzadeh\par
 MATH 117\par
 8/31/18
 
 \end{flushleft}
 
 \centerline{Homework: 1F}
 
 \vspace{4mm}
 
 \noindent\textsc{Section 1.3}\par
 
 \justifying
 
 \vspace{1mm}
 
 \hline
 
 \vspace{4mm}
 
 \noindent\textbf{ 2.(d)}\par
 
 \vspace{4mm}
 
 The transpose of this matrix is
 
 \vspace{8mm}
 
 \centerline{${\left( \begin{array}{ccc} 10 & 0 & -8\\ 2 & -4 & 3\\ -5 & 7 & 6\end{array}\right)}^{t}=\left( \begin{array}{ccc} 10 & 2 & -5\\ 0 & -4 & 7\\ -8 & 3 & 6\end{array}\right)$.}
 
 \vspace{6mm}
 
 \noindent\textbf{5.}\par Let $M_{2\times 2}(F)$ be the set of all $2\times2$ matrices with entries $a_{ij}\in F$. Then for all\par $A\in M_{2\times 2}(F)$, it is true that $A+A^t$ is symmetric.\par
 
 \vspace{4mm}
 
 \textbf{Proof}\par 
 
 \vspace{2mm}
 
 Assume $A\in M_{2\times 2}(F)$. This is equivalent to saying that for $a_{ij}\in F$, \par where $1\leq i\leq2$ and $1\leq j\leq 2$,\par
 
 
 \begin{math}
 
 \centerline{A=\left(\begin{array}{cc} a_{11} & a_{12}\\
 a_{21} & a_{22}\end{array}\right).}
 
 
 \end{math}
 
 \vspace{6mm}
 
\par By the taking the transpose---which states $(A^t)_{ij}=A_{ji}$ for $1\leq i\leq 2$ and $ 1\leq j\leq 2$---\par of $A$ and adding it to $A$, we obtain\par

\vspace{2mm}

\begin{math}
 
 \centerline{A+A^t=\left(\begin{array}{cc} a_{11} & a_{12}\\
 a_{21} & a_{22}\end{array}\right) + \left(\begin{array}{cc} a_{11} & a_{21}\\
 a_{12} & a_{22}\end{array}\right)=\left(\begin{array}{cc} a_{11}+a_{11} & a_{12}+a_{21}\\
 a_{21}+a_{12} & a_{22}+a_{22}\end{array}\right).}
 
\end{math}

\vspace{6mm}

\par Using the properties of transpose, namely $(A+B)^t=B^t+A^t$, then the transpose of \par the previous matrix becomes\par

\vspace{2mm}

\begin{math}
 
 \centerline{(A+A^t)^t=\left(\begin{array}{cc} a_{11}+a_{11} & a_{21}+a_{12}\\
 a_{12}+a_{21} & a_{22}+a_{22}\end{array}\right)}.
 
\end{math}

\vspace{6mm}

\par However, since each entry in the matrix is an element of the field $F$, then by the\par definition of a field, it follows $a_{ij}+a_{ji}=a_{ji}+a_{ij}$ for $1\leq i\leq 2$ and $1\leq j\leq 2$. Thus, \par $(A+A^t)^t=A+A^t$. Therefore, for any square $2\times 2$ matrix $A$, $A+A^t$ is symmetric.\par

\hspace{150mm}\blacksquare

\newpage

\noindent\textbf{8.}

\vspace{2mm}

\par\textbf{a)} Let $W_1=\{(a_1,a_2,a_3)\in\mathbb{R}^3\colon a_1=3a_2\wedge a_3=-a_2\}$. To determine if $W_1$ is a \par subspace, we will first verify if $(0,0,0)\in W_1$. In this case, it is true that $a_1=3a_2$\par since $0=3\cdot 0$, and $a_3=-a_2$ since $0=-0$. Thus, $W_1$ contains the additive identity.\par Let $u,v\in W_1$. Then $u=(3a_2, a_2, -a_2)$ and $v=(3b_2,b_2,-b_2)$. Now consider their \par sum

\vspace{2mm}

 
 \centerline{$u+v = (3(a_2+b_2), a_2+b_2,-(a_2+b_2))$}
 
 \vspace{4mm}
 
 \par Thus, $u+v\in W_1$. Finally, let $c\in F$ and $u\in W_1$. Then $cu = (3ca_2,ca_2, -ca_2)$. \par The product of this scalar multiplication has resulted in a vector satisfying the \par definition of $W_1$. Therefore, $W_1$ is a subspace of $\mathbb{R}^3$.
 
 \vspace{4mm}
 
 \par\textbf{b)} Let $W_2=\{(a_1,a_2,a_3)\in\mathbb{R}^3\colon a_1=a_3+2\}$. Assume $(0,0,0)\in W_2$. Then since\par $a_1=a_3+2$, it follows that $0=0+2=2$. This is false, and we can conclude that $W_2$ \par does not contain the additive identity and is therefore not a subspace.
 
 \vspace{6mm}
 
 \noindent\textbf{23.}\par
 
 \vspace{4mm}
 
 Let $W_1$ and $W_2$ be subspaces of a vector space $V$.
 
 \vspace{4mm}
 
 \par \textbf{a)} The set $W_1+W_2$ is a subspace of $V$ which contains both $W_1$ and $W_2$.\par
 
 \vspace{4mm}
 
 \textbf{Proof}\par
 
 \vspace{2mm}
 
\hspace{4mm}\textbf{i.} Since $W_1$ and $W_2$ are subspaces by assumption, then they both contain the 
\par additive identity vector. Hence, $e_{+}\in W_1$ and $e_{+}\in W_2$. Now consider the sum of our\par two subspaces, $W_1+W_2=\{u+v\colon u\in W_1\wedge v\in W_2\}$. From the two previous lines, it\par follows that $e_{+}\in W_1\wedge e_{+}\in W_2\rightarrow e_{+}+e_{+}\in W_1+W_2$. Thus, $e_{+}\in W_1+W_2$.\par
 
 \vspace{4mm}
 
 \hspace{4mm}\textbf{ii.} Now assume $a,b\in W_1+W_2$. Then for some $u_1,u_2\in W_1$ and $v_1,v_2\in W_2$ we\par have\par
 
 \vspace{2mm}
 
 \centerline{$a=u_1+v_1$ and $b=u_2+v_2$.}
 
 \vspace{2mm}
 
 \par The sum of $a$ and $b$ is therefore
 
\begin{equation} 
\begin{split}
a+b & = (u_1+v_1)+(u_2+v_2) \\
 & = u_1+(v_1+u_2)+v_2 \\ \nonumber
 & = u_1+(u_2+v_1)+v_2 \\
 & = (u_1+u_2)+(v_1+v_2).
\end{split}
\end{equation}

\vspace{2mm}

\par Since $u_1,u_2\in W_1$ and $v_1,v_2\in W_2$, then $(u_1+u_2)\in W_1$ and $(v_1+v_2)\in W_2$. Therefore, \par $(u_1+u_2)+(v_1+v_2)\in W_1+W_2$. Thus, $a+b\in W_1+W_2$.

\newpage

\par\hspace{4mm}\textbf{iii.} Let $c$ be an element of the field that $V$ is a vector space over, and let\par $w\in W_1+W_2$. Then for some $u\in W_1$ and $v\in W_2$, we have $w=u+v$ and $cw=cu+cv$.\par Since $W_1$ and $W_2$ are subspaces, they are closed under scalar multiplication. Thus,\par $cu\in W_1$ and $cv\in W_2$. Therefore, $cu+cv\in W_1+W_2$ and $cw\in W_1+W_2$.\par

\vspace{2mm}

\hspace{4mm}\textbf{iv.} Let $u\in W_1$. Then since $e_{+}\in W_2$, we have $u\in W_1\wedge e_{+}\in W_2$. Thus,\par $u+e_{+}\in W_1+W_2$ and since $u+e_{+}=u$, we have $u\in W_1+W_2$. Hence, for all $u\in W_1$,\par $u\in W_1+W_2$. Therefore, $W_1\subseteq W_1+W_2$. It can be shown in a similar way that\par $W_2\subseteq W_1+W_2$. Therefore, $W_1+W_2$ contains both $W_1$ and $W_2$.\hspace{35mm}\blacksquare\par

\vspace{4mm}

\par\textbf{b)} If $Z$ is any subspace of the vector space $V$ such that $W_1\subseteq Z$ and $W_2\subseteq Z$, then\par $W_1+W_2\subseteq Z$.\par

\vspace{4mm}

\par\textbf{Proof}\par

\vspace{2mm}

Assume $Z$ is a subspace of $V$. The rest of the proof will be in two-column style.

\begin{table}[h!]
    \begin{center}
        \begin{tabular}{l l l}
        \hline
        
        1. & $W_1\subseteq Z\wedge W_2\subseteq Z$ & Assumption\\
        
        2. & \hspace{10mm}$(u+v)\in W_1+W_2$ & Assumption\\
        
        3. & \hspace{20mm}$u\in W_1\wedge v\in W_2$ & Definition 2\\
        
        4. & \hspace{20mm}$u\in W_1$ & Simplification 3\\
        
        5. & \hspace{20mm}$W_1\subseteq Z$ & Simplification 1\\
        
        6. & \hspace{20mm}$\forall t(t\in W_1\rightarrow t\in Z)$ & Definition 5\\
        
        7. & \hspace{20mm}$t\in W_1\rightarrow t\in Z$ & Universal Instantiation 6\\
        
        8. & \hspace{20mm}$u\in Z$ & Modus Ponens 7,4\\
        
        9. & \hspace{20mm}$W_2\subseteq Z$ & Simplification 1\\
        
        10.& \hspace{20mm}$\forall t(t\in W_2\rightarrow t\in Z)$ & Definition 9\\
        
        11.& \hspace{20mm}$t\in W_2\rightarrow t\in Z$ & Universal Intsantiation 10\\
        
        12.& \hspace{20mm}$v\in W_2$ & Simplification 3\\
        
        13.& \hspace{20mm}$v\in Z$ & Modus Ponens 11,12\\
        
        14.& \hspace{20mm}$u\in Z\wedge v\in Z$ & Conjunction 8,13\\
        
        15.& \hspace{20mm}$(u+v)\in Z$ & Closure of Addition 14\\
        
        16.& \hspace{10mm}$(u+v)\in W_1+W_2\rightarrow(u+v)\in Z$ & Direct Proof 2-15\\
        
        17.& \hspace{10mm}$x\in W_1+W_2\rightarrow x\in Z$ & Substitution 16\\
        
        18.& \hspace{10mm}$\forall x(x\in W_1+W_2\rightarrow x\in Z)$ & Universal Generalization 17\\
        
        19.& \hspace{10mm}$W_1+W_2\subseteq Z$ & Definition 18\\
        
        20.& $W_1\subseteq Z\wedge W_2\subseteq Z\rightarrow W_1+W_2\subseteq Z$ & Direct Proof 1-19\\
        
        \hline
        \end{tabular}
    \end{center}
\end{table}

\hspace{150mm}\blacksquare

\vspace{4mm}

\noindent\textbf{25.}\par

\vspace{4mm}

Let $W_1$ denote the set of all polynomials $f(x)$ in $P(F)$ such that the coefficients $a_i=0$\par whenever $i$ is odd. Additionally, let $W_2$ denote the set of all polynomials $g(x)$ in $P(F)$\par such that the coefficients $b_i=0$ whenever $i$ is even.

\newpage

\par These two sets can be defined in the following way\par

\vspace{6mm}

\centerline{$W_1=\{f(x)\in P(x)\colon f(x)=\sum\limits_{i=0}^n a_{2i+1}x^{2i+1}\}$}\par

\vspace{4mm}

\centerline{$W_2=\{g(x)\in P(x)\colon g(x)=\sum\limits_{i=0}^n b_{2i}x^{2i}\}$.}

\vspace{4mm}

\par Given these two sets, $P(F)=W_1\oplus W_2$.

\vspace{4mm}

\par\textbf{Proof}\par

\vspace{2mm}

In order to prove this claim, we must first prove that $W_1\cap W_2=\{0\}$. Assume\par  $h(x)\in W_1\cap W_2$. Then $h(x)\in W_1$ and $h(x)\in W_2$. If $h(x)\in W_1$, then for some $m\in\mathbb{Z}$\par where $2m+1\leq n$ we have\par

\vspace{4mm}

\centerline{$h(x)=a_1x^1+\dots +a_{2m+1}x^{2m+1}$.}

\vspace{4mm}

\par Similarly, if $h(x)\in W_2$ then for some $k\in\mathbb{Z}$ where $2k\leq n$, we have\par

\vspace{4mm}

\centerline{$h(x)=b_0+\dots +b_{2k}x^{2k}$.}

\vspace{4mm}

\par Since $h(x)=h(x)$, then it must be the case that $a_i=b_i=0$ for $0\leq i\leq n$. Thus,\par if $h(x)\in W_1\cap W_2$, then $h(x)=\{0\}$. Therefore, $W_1\cap W_2=\{0\}$. The next condition\par we must prove is that $W_1+W_2=P(F)$.\par

\vspace{2mm}

Assume $(f(x)+g(x))\in W_1+W_2$. Then since $f(x)\in W_1$ and $g(x)\in W_2$, it follows\par that for some $k,m\in\mathbb{Z}$ such that $2k\leq n$ and $2m+1\leq n$ and $k\leq m$\par

\vspace{4mm}

\begin{equation} 
\begin{split}
 f(x)+g(x)  & = (a_1+\dots +a_{2m+1}x^{2m+1})+(b_0+\dots +b_{2k}x^{2k})\\
 & = (0+b_0)+(a_1+0)x^1+\dots + (0+b_{2k})x^{2k} +\dots + (a_{2m+1}+0)x^{2m+1}\\ \nonumber
 & = u_1+(u_2+v_1)+v_2 \\
& = b_0+a_1x^1+\dots b_{2k}x^{2k}+\dots a_{2m+1}x^{2m+1}.
\end{split}
\end{equation}

\vspace{4mm}

\par The sum of the two polynomials resulted in a polynomial of degree $2m+1\leq n$. Thus,\par $f(x)+g(x)\in P(F)$ and $W_1+W_2\subseteq P(F)$. It should be noted that this resulted also\par could have been deduced from result of part (a). Now assume $s(x)\in P(F)$. Then for\par some $t\leq n$, $s(x)=c_0+\dots +c_tx^t$. Consider the polynomial $f(x)=c_1x^1+\dots +c_tx^t$.\par If $t$ is odd, then $f(x)\in W_1$. Now consider $g(x)=c+0+\dots +c_{t-1}x^{t-1}$. Again, if $t$ is\par odd, then $g(x)\in W_2$. We will assume $t$ is odd arbitrarily since it would be equivalent\par of we assumed it were even. Hence, with $f(x)\in W_1$ and $g(x)\in W_2$, it follows that\par $f(x)+g(x)\in W_1+W_2$. Furthermore, by construction we have $f(x)+g(x)=s(x)$.\par Thus, $s(x)\in W_1+W_2$. Therefore, $P(F)\subseteq W_1+W_2$. By definition then, $W_1+W_2=$\par $P(F)$ and we may conclude $P(F)=W_1\oplus W_2$.\hspace{67mm}\blacksquare

\newpage

\noindent\textbf{30.}

\vspace{4mm}

\par Let $W_1$ and $W_2$ be subspaces of a vector space $V$. Then $V$ is the direct sum of $W_1$\par and $W_2$ if and only if each vector in $V$ can be uniquely written as $x_1+x_2$, where\par $x_1\in W_1$ and $x_2\in W_2$.\par

\vspace{4mm}

\textbf{Proof}\par

\vspace{2mm}

\par\hspace{4mm}\textbf{i.} Assume $V=W_1\oplus W_2$ and assume there exists a vector $u\in V$ such that for\par $x_1,x_3\in W_1$ and $x_2,x_2\in W_2$, $u=x_1+x_2$ and $u=x_3+x_4$. Then by transitivity of\par equality, we have $x_1+x_2=x_3+x_4$, which by using respective inverses we can rearrange\par as $x_1-x_3=x_4-x_2$. Since $x_1-x_3\in W_1$ and $x_4-x_2\in W_2$, then $W_1\cap W_2\neq\{0\}$.\par This is a contradiction since we assumed $V=W_1\oplus W_2$ which implies $W_1\cap W_2=\{0\}$.\par Thus, if $V=W_1\oplus W_2$, then each vector in $V$ can be written uniquely as $x_1+x_2$, for\par $x_1\in W_1$ and $x_2\in W_2$.

\vspace{4mm}

\par\hspace{4mm}\textbf{ii.} Assume each vector in $V$ can be written uniquely as $x_1+x_2$ for $x_1\in W_1$ and\par $x_2\in W_2$. Now let $v\in W_1\cap W_2$. Then $v\in W_1$ and $v\in W_2$. Thus, there is some\par $w_1\in W_1$ and $w_2\in W_2$, such that $v=w_1$ and $v=w_2$. Additionally, since $v\in W_1+W_2$,\par there is $x_1\in W_1$ and $x_2\in W_2$, such that $v=x_1+x_2$. However, since we have\par assumed uniqueness of each element, then $v$ must be equal to the zero vector. Thus,\par $W_1\cap W_2=\{0\}$. Finally, since $V\subseteq V$, $W_1\subseteq V$, and $W_2\subseteq V$, then $W_1+W_2\subseteq V$ by\par problem 23.b. Now assume $v\in V$, then there exists $x_1\in W_1$ and $x_2\in W_2$, such that\par $v=x_1+x_2$. Thus, $v\in W_1+W_2$. Hence, $V\subseteq W_1+W_2$. By double inclusion, we have\par $V=W_1+W_2$. Therefore, $V=W_1\oplus W_2$.\hspace{76mm}\blacksquare

\vspace{4mm}

\noindent\textbf{31.}\par

\vspace{4mm}

Let $W$ be a subspace of a vector space $V$.\par

\vspace{4mm}

\par\textbf{a)} The set $v+W=\{v+w\colon w\in W\}$ is a subspace of $V$ if and only if $v\in W$.\par

\vspace{4mm}

\textbf{Proof}

\vspace{2mm}\par

\hspace{4mm}\textbf{i.} Assume $v+W$ is a subspace of $V$ and $v\notin W$. Now consider some $w_1\in v+W$.\par It follows that for some $w_2\in W$, $w_1=v+w_2$. Thus, $v=w_1-w_2$. However, $v\notin W$\par so $w_1-w_2\notin W$. This means $W$ is not closed and is therefore not a subspace. This \par is a contradiction. Thus, if $v+W$ is a subspace of $V$, then $v\in W$.

\vspace{4mm}

\par\hspace{4mm}\textbf{ii.} Assume $v\in W$. Then let $u\in v+W$. This means that for some $w_1$, $u=v+w_1$.\par However, since $v\in W$, then $v+w_1\in W$. Hence, $u\in W$ and $v+W\subseteq W$. Let $u\in W$\par and suppose $u=w_1+w_2$, for some $w_1,w_2\in W$. Then $u+v=(w_1+w_2)+v$.\par Furthermore, $u+v=v+(w_1+w_2)$ by commutativity. Adding the inverse of $v$ to\par both sides we obtain $u=v+(w_1+w_2)-v=v+(w_1+w_2-v)$. Since $W$ is closed, \par it follows that $(w_1+w_2-v)\in W$. Thus, $v+(w_1+w_2-v)\in v+W$ which means\par $u\in v+W$. Thus, $W\subeteq v+W$ and by double inclusion $W=v+W$. Since $W$ is a\par subspace by assumption, then $v+W$ is a subspace. \hspace{55mm}\blacksquare

\newpage

\par\textbf{b)} $v_1+W=v_2+W$ if and only if $v_1-v_2\in W$.\par

\vspace{4mm}

\textbf{Proof}\par

\vspace{2mm}

\hspace{4mm}\textbf{i.} Assume $v_1+W=v_2+W$. It follows that since $v_1\in v_1+W$, then $v_1\in v_2+W$.\par Thus, for some $w\in W$ $v_1=v_2+w$. Hence, $w=v_1-v_2$. Thus, $v_1-v_2\in W$.\par Therefore, if $v_1+W=v_2+W$, then $v_1-v_2\in W$.

\vspace{4mm}

\par\hspace{4mm}\textbf{ii.} Assume $v_1-v_2\in W$. Let $u\in(v_1+W)\cap(v_2+W)$. Then $u\in v_1+W$ and\par $u\in v_2+W$. Thus, for some $w_1,w_2\in W$ it follows $u=v_1+w_1$ and $u=v_2+w_2$.\par
Thus, $v_1+w_1=v_2+w_2$ which means $v_1=v_2+(w_2-v_1)$.  Therefore, $v_1+W$ can be\par written as $v_1+W=\{(v_2+(w_2-v_1))+w_3\colon w_3\in W\}$. However, since $(w_2-v_1+w_3)\in W$\par we may substitute $w_4=w_2-v_1+w_3$. Therefore, $v_1+W=\{v_2+w_4\colon w_4\in W\}$. It\par then follows that since $\{v_2+w_4\colon w_4\in W\}=v_2+W$, then $v_1+W=v_2+W$.\hspace{9mm}\blacksquare







 





 
 
 
 
 
 
 
 
 \end{document}