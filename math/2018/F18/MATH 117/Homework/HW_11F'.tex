\documentclass[12pt, a4paper]{article}
\usepackage[margin=1in]{geometry}
\usepackage[latin1]{inputenc}
\usepackage{titlesec}
\usepackage{amsmath}
\usepackage{amsthm}
\usepackage{amsfonts}
\usepackage{amssymb}
\usepackage{array}
\usepackage{booktabs}
\usepackage{ragged2e}
\usepackage{enumerate}
\usepackage{enumitem}
\usepackage{cleveref}
\usepackage{slashed}
\usepackage{commath}
\usepackage{lipsum}
\usepackage{colonequals}
\usepackage{addfont}
\addfont{OT1}{rsfs10}{\rsfs}
\renewcommand{\baselinestretch}{1.1}
\usepackage[mathscr]{euscript}
\let\euscr\mathscr \let\mathscr\relax
\usepackage[scr]{rsfso}
\newcommand{\powerset}{\raisebox{.15\baselineskip}{\Large\ensuremath{\wp}}}
\usepackage{longtable}
\usepackage{multirow}
\usepackage{multicol}
\usepackage{calligra}
\usepackage[T1]{fontenc}
\newcounter{proofc}
\renewcommand\theproofc{(\arabic{proofc})}
\DeclareRobustCommand\stepproofc{\refstepcounter{proofc}\theproofc}
\usepackage{fancyhdr}
\pagestyle{fancy}

\renewcommand{\headrulewidth}{0pt}
\fancyhead[R]{}
\usepackage{enumitem}
\usepackage{tikz}
\usepackage{commath}
\usepackage{colonequals}
\usepackage{bm}
\usepackage{tikz-cd}
\renewcommand{\baselinestretch}{1.1}
\usepackage[mathscr]{euscript}
\let\euscr\mathscr \let\mathscr\relax
\usepackage[scr]{rsfso}
\usepackage{titlesec}
\usepackage{scrextend}
\usepackage{lscape}
\usepackage{relsize}

\usepackage[english]{babel}
\usepackage{blindtext}
\usepackage{polynom}



\newcommand*{\logeq}{\ratio\Leftrightarrow}

\titleformat{\section}
  {\normalfont\Large\bfseries}{\thesection}{1em}{}[{\titlerule[0.8pt]}]
  
 \setlist[description]{leftmargin=12.8mm,labelindent=4mm}
 
\begin{document}

\begin{flushleft}
  
    Quin Darcy\par
    Dr. Ebrahimzadeh\par
    MATH 117\par
    11/27/18
  
\end{flushleft}
  
\centerline{\boxed{\text{Homework: 11F'}}}
 
\vspace{4mm}
 
\noindent\textsc{Section: Direct Sums and Diagonalizability}\par
 
\justifying
 
\vspace{1mm}
 
\hline
 
\vspace{4mm}

\noindent\textbf{71.} Let $V$ be a finite dimensional vector space over the field $F$.

\begin{description}
    \item \textbf{ i.} Prove that if $\beta=\{v_1,v_2,\dots,v_n\}$ is a basis for $V$ and that $W=\{w_i\in V\mid i\leq n\}$, then there there exists exactly one linear operator $T$, on $V$, such that $T(v_i)=w_i$, for all $i$.
    \item\textbf{ii.} Prove that if $V=W_1\oplus\cdots\oplus W_k$ and $\{\lambda_i\}_{i=1}^{k}$ are $k$ distinct scalars, then there exists exactly one linear operator $T$, on $V$, such that the distinct eigenvalues of $T$ are $\{\lambda_i\}_{i=1}^{k}$ and $E_{\lambda_i}=W_i$, for all $i$.
\end{description}

\vspace{6mm}

\noindent\textbf{\underline{Solutions}}

\vspace{4mm}

\textbf{(i.) \textit{Proof.}} Assume that $\beta=\{v_1,\dots,v_n\}$ is a basis for $V$, and that \\ $W=\{w_i\in V\mid\exists i(i\in\mathbb{N}\wedge[1\leq i\leq n])\}$,

\vspace{4mm}

\textbf{(ii.) \textit{Proof.}} 

\newpage



\end{document}