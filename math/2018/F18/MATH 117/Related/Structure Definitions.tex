\documentclass[12pt, a4paper]{article}
\usepackage[margin=1in]{geometry}
\usepackage[latin1]{inputenc}
\usepackage{titlesec}
\usepackage{amsmath}
\usepackage{amsthm}
\usepackage{amsfonts}
\usepackage{amssymb}
\usepackage{array}
\usepackage{booktabs}
\usepackage{ragged2e}
\usepackage{enumerate}
\usepackage{enumitem}
\usepackage{cleveref}
\usepackage{slashed}
\usepackage{commath}
\usepackage{lipsum}
\usepackage{colonequals}
\usepackage{addfont}
\addfont{OT1}{rsfs10}{\rsfs}
\renewcommand{\baselinestretch}{1.1}
\usepackage[mathscr]{euscript}
\let\euscr\mathscr \let\mathscr\relax
\usepackage[scr]{rsfso}
\newcommand{\powerset}{\raisebox{.15\baselineskip}{\Large\ensuremath{\wp}}}
\usepackage{longtable}
\usepackage{multirow}
\usepackage{multicol}
\usepackage{calligra}
\usepackage[T1]{fontenc}
\newcounter{proofc}
\renewcommand\theproofc{(\arabic{proofc})}
\DeclareRobustCommand\stepproofc{\refstepcounter{proofc}\theproofc}
\newenvironment{twoproof}{\tabular{@{\stepproofc}c|l}}{\endtabular}
\newcolumntype{C}{>$c<$}
\usepackage{fancyhdr}
\pagestyle{fancy}
\fancyhf{}
\renewcommand{\headrulewidth}{0pt}
\fancyhead[R]{\thepage}
\usepackage{enumitem}
\usepackage{tikz}
\usepackage{commath}
\usepackage{colonequals}
\usepackage{bm}
\usepackage{tikz-cd}
\renewcommand{\baselinestretch}{1.1}
\usepackage[mathscr]{euscript}
\let\euscr\mathscr \let\mathscr\relax
\usepackage[scr]{rsfso}
\usepackage{titlesec}

\newcommand*{\logeq}{\ratio\Leftrightarrow}

\titleformat{\section}
  {\normalfont\Large\bfseries}{\thesection}{1em}{}[{\titlerule[0.8pt]}]
  
  \setlist[description]{leftmargin=10mm,labelindent=10mm}

\begin{document}

\section{Models}

\vspace{4mm}

\noindent\large{\textsc{Rings}}\normalsize

\vspace{4mm}

\noindent A ring is a set $R$ with two binary operations $\oplus\colon R\times R\rightarrow R$ and $\otimes\colon R\times R\rightarrow R$, such that\par

\begin{description}

\item\textbf{R1.} $\forall x\forall y\forall z[x\oplus(y\oplus z)=(x\oplus y)\oplus z]$

\item\textbf{R2.} $\forall x\forall y\forall z[x\otimes(y\otimes z)=(x\otimes y)\otimes z]$

\item\textbf{R3.} $\forall x\forall y[x\oplus y=y\oplus x]$

\item\textbf{R4.} $\exists e_{\oplus}\forall x[x\oplus e_{\oplus}=x]$

\item\textbf{R5.} $\forall x\exists y[x\oplus y=e_{\oplus}]$

\item\textbf{R6.} $\forall x\forall y\forall z[x\otimes(y\oplus z)=x\otimes y\oplus x\otimes z]$

\item\textbf{R7.} $\forall x\forall y\forall z[(x\oplus y)\otimes z=x\otimes z\oplus y\otimes z]$

\item*\textbf{R8.} $\forall x\forall y[x\otimes y=y\otimes x]$ (Commutative Ring)

\item*\textbf{R9.} $\exists e_{\otimes}\forall x[x\otimes e_{\otimes}=e_{\otimes}\otimes x=x]$ (Ring with Unity)

\end{description}

\vspace{4mm}

\noindent\blacksquare\rm\hspace{2mm} DEFINITION\par

\vspace{4mm}

If $R$ is the domain of a ring, $a,b\in R\backslash\{e_{\oplus}\}$, then if $a\otimes b=e_{\oplus}$, then $a$ and $b$ are called \textbf{zero divisors}.\par

\vspace{4mm}

\noindent\blacksquare\rm\hspace{2mm}DEFINITION\par

\vspace{4mm}

An \textbf{integral domain} is a commutative ring with unity that does not have zero divisors.\par

\vspace{4mm}

\noindent\blacksquare\rm\hspace{2mm}DEFINITION\par

\vspace{4mm}

Let $(R,\oplus,\otimes)$ be a ring with unity. If $u\in R$ has the property that there exists $v\in R$ such that $u\otimes v=v\otimes u=e_{\otimes}$, then $u$ is called a \textbf{unit}.

\vspace{4mm}

\noindent\blacksquare\rm\hspace{2mm}DEFINITION\par

\vspace{4mm}

Let $(R,\oplus,\otimes)$ be a ring with unity. $R$ is called a \textbf{division ring} if\par

\begin{description}

\item $\forall x\exists y[x\neq e_{\oplus}\wedge y\neq e_{\oplus}\wedge (x\otimes y=y\otimes x=e_{\otimes})]$.

\end{description}

\noindent Another way to say this is that if every nonzero element in $R$ is a unit, then $R$ is a division ring.

\vspace{4mm}

\noindent\blacksquare\rm\hspace{2mm}DEFINITION\par

\vspace{4mm}

A commutative division ring is called a \textbf{field}. That is, a \textbf{field} is a ring where the multiplication is commutative, the set contains a multiplicative identity, and every element that is not equal to the additive identity has a multiplicative inverse. And yet one more way to state what a field is, is: A field is a set $F$ with at least two elements and two binary operations called \textit{addition} and \textit{multiplication}. $F$ is an abelian group under addition, the set $F$ without the additive inverse is an abelian group under multiplication, and multiplication distributes over addition on both the left and right. 







\end{document}