\documentclass[12pt, a4paper]{article}
\usepackage[margin=1in]{geometry}
\usepackage[latin1]{inputenc}
\usepackage{titlesec}
\usepackage{amsmath}
\usepackage{amsthm}
\usepackage{amsfonts}
\usepackage{amssymb}
\usepackage{array}
\usepackage{booktabs}
\usepackage{ragged2e}
\usepackage{enumerate}
\usepackage{enumitem}
\usepackage{cleveref}
\usepackage{slashed}
\usepackage{commath}
\usepackage{lipsum}
\usepackage{colonequals}
\usepackage{addfont}
\addfont{OT1}{rsfs10}{\rsfs}
\renewcommand{\baselinestretch}{1.1}
\usepackage[mathscr]{euscript}
\let\euscr\mathscr \let\mathscr\relax
\usepackage[scr]{rsfso}
\newcommand{\powerset}{\raisebox{.15\baselineskip}{\Large\ensuremath{\wp}}}
\usepackage{longtable}
\usepackage{multirow}
\usepackage{multicol}
\usepackage{calligra}
\usepackage[T1]{fontenc}
\newcounter{proofc}
\renewcommand\theproofc{(\arabic{proofc})}
\DeclareRobustCommand\stepproofc{\refstepcounter{proofc}\theproofc}
\usepackage{fancyhdr}
\pagestyle{fancy}

\renewcommand{\headrulewidth}{0pt}
\fancyhead[R]{}
\usepackage{enumitem}
\usepackage{tikz}
\usepackage{commath}
\usepackage{colonequals}
\usepackage{bm}
\usepackage{tikz-cd}
\renewcommand{\baselinestretch}{1.1}
\usepackage[mathscr]{euscript}
\let\euscr\mathscr \let\mathscr\relax
\usepackage[scr]{rsfso}
\usepackage{titlesec}
\usepackage{scrextend}
\usepackage{lscape}

\usepackage[english]{babel}
\usepackage{blindtext}



\newcommand*{\logeq}{\ratio\Leftrightarrow}

\titleformat{\section}
  {\normalfont\Large\bfseries}{\thesection}{1em}{}[{\titlerule[0.8pt]}]
  
 \setlist[description]{leftmargin=4mm,labelindent=4mm}
  
 \begin{document}
 
 \section{Vector Space Axioms}
 
 \noindent A \textbf{vector space} $V$ over a field $F$ consists of a set on which two operations, $+\colon V\times V\rightarrow V$ and $\cdot\colon F\times V\rightarrow V$, are defined so that for each pair of elements $x,y$ in $V$ there is a unique element $x+y$ in $V$, and for each $a$ in $F$ and each element $x$ in $V$ there is a unique $ax$ in $V$, such that the following conditions hold.
 
 \vspace{4mm}
 
 \begin{addmargin}[1.5em]{1.5em}
 
    \textbf{(VS 1)} $\forall x\forall y(x\in V\wedge y\in V\rightarrow x+y=y+x)$
    
    \vspace{4mm}
    
    \noindent\textbf{(VS 2)} $\forall x\forall y\forall z(x\in V\wedge y\in V\wedge z\in V\rightarrow (x+y)+z=x+(y+z))$
    
    \vspace{4mm}
    
    \noindent\textbf{(VS 3)} $\exists 0_V(0_V\in V\wedge\forall x[x\in V\rightarrow 0_V+x=x])$
    
    \vspace{4mm}
    
    \noindent\textbf{(VS 4)} $\forall x(x\in V\rightarrow\exists y[y\in V\wedge x+y=0])$
    
    \vspace{4mm}
    
    \noindent\textbf{(VS 5)} $\exists 1(1\in F\wedge\forall x[x\in V\rightarrow 1\cdot x=x])$
    
    \vspace{4mm}
    
    \noindent\textbf{(VS 6)} $\forall a\forall b\forall x(a\in F\wedge b\in F\wedge x\in V\rightarrow (ab)\cdot x=a\cdot(b\cdot x))$
    
    \vspace{4mm}
    
    \noindent\textbf{(VS 7)} $\forall a\forall x\forall y(a\in F\wedge x\in V\wedge y\in V\rightarrow a\cdot(x+y)=a\cdot x+a\cdot y)$
    
    \vspace{4mm}
    
    \noindent\textbf{(VS 8)} $\forall a\forall b\forall x(a\in F\wedge b\in F\wedge x\in V\rightarrow (a+b)\cdot x=a\cdot x+b\cdot x)$
 
 \end{addmargin}
 
 \vspace{4mm}
 
 \noindent The elements $x+y$ and $a\cdot x$ are called the \textbf{sum} of $x$ and $y$ and the \textbf{product} of $a$ and $x$, respectively. The elements of the field $F$ are called \textbf{scalars} and the elements of the vector space $V$ are called \textbf{vectors}.
 
 \newpage
 
 \section{Theorems}
 
 \noindent\blacksquare\textbf{ THEOREM 1.1}\par
 
 \vspace{2mm}
 
 If $x,y$ and $z$ are vectors in a vector space $V$ such that $x+z=y+z$, then $x=y$.
 
 \vspace{4mm}
 
 \noindent\blacktriangle\textbf{ COROLLARY 1.}\par
 
 \vspace{2mm}
 
 The vector $0_V$ is unique.
 
 \vspace{4mm}
 
 \noindent\blacktriangle\textbf{ COROLLARY 2.}\par
 
 \vspace{2mm}
 
 The vector $y$ described in \textbf{(VS 4)} is unique.
 
 \vspace{4mm}
 
 \noindent\blacksquare\textbf{ THEOREM 1.2}\par
 
 \vspace{2mm}
 
 In any vector space $V$, the following statements are true:
 
 \vspace{2mm}
 
 \begin{addmargin}[1.5em]{1.5em}
 
 \par
 
    (a) $0x=0$ for each $x\in V$\par
    \noindent(b) $(-a)x=-(ax)=a(-x)$ for each $a\in F$ and each $x\in V$.\par
    \noindent(c) $a0=0$ for each $a\in F$.
 
 \end{addmargin}
 
 \vspace{4mm}
 
 \noindent\blacksquare\textbf{ THEOREM 1.3}\par
 
 \vspace{2mm}
 
 Let $V$ be a vector space and $W$ a subset of $V$. Then $W$ is a subspace of $V$ if and only\par if the following three conditions hold for the operations defined in $V$.
 
 \vspace{2mm}
 
 \begin{addmargin}[1.5em]{1.5em}
 
    (a) $0\in W$\par
    \noindent(b) $x+y\in W$ whenever $x\in W$ and $y\in W$.\par
    \noindent(c) $cx\in W$ whenever $c\in F$ and $x\in W$.
 
 \end{addmargin}
 
 \vspace{4mm}
 
 \noindent\blacksquare\textbf{ THEOREM 1.4}\par
 
 \vspace{2mm}
 
 Any intersection of subspaces of a vector space $V$ is a subspace of $V$.
 
 \vspace{4mm}
 
 \noindent\blacksquare\textbf{ THEOREM 1.5}\par
 
 \vspace{2mm}
 
 The span of any subset $S$ of a vector space $V$ is a subspace of $V$. Moreover, any\par subspace of $V$ that contains $S$ must also contain the span of $S$.
 
 \vspace{4mm}
 
 \noindent\blacksquare\textbf{ THEOREM 1.6}
 
 \vspace{2mm}
 
 Let $V$ be a vector space and let $S_1\subseteq S_2\subseteq V$. If $S_1$ is linearly dependent, then $S_2$ is\par linearly dependent.
 
 \vspace{4mm}
 
 \noindent\blacktriangle\textbf{ COROLLARY.}
 
 \vspace{2mm}
 
  Let $V$ be a vector space and let $S_1\subseteq S_2\subseteq V$. If $S_2$ is linearly independent, then $S_1$\par is linearly independent.
  
  \newpage
  
  \noindent\blacksquare\textbf{ THEOREM 1.7}
  
  \vspace{2mm}
  
  Let $S$ be a linearly independent subset of a vector space $V$, and let $v$ be a vector in\par $V$ that is not in $S$. Then $S\cup\{v\}$ is linearly dependent if and only if $v\in span(S)$.
  
  \vspace{4mm}
  
  \noindent\blacksquare\textbf{ THEOREM 1.8}
  
  \vspace{2mm}
  
  Let $V$ be a vector space and $\beta=\{u_1,u_2,\dots, u_n\}$ be a subset of $V$. Then $\beta$ is a basis\par for $V$ if and only if each $v\in V$ can be uniquely expressed as a linear combination of\par vectors of $\beta$, that is, can be expressed in the form
  
  \vspace{2mm}
  
  \centerline{$v=a_1u_1+\cdots +a_nu_n$}
  
  \vspace{2mm}
  
  \par for unique scalars $a_1,\dots a_n$.
  
  \vspace{4mm}
  
  \noindent\blacksquare\textbf{ THEOREM 1.9}
  
  \vspace{2mm}
  
  If a vector space $V$ is generated by a finite set $S$, then some subset of $S$ is a basis for\par $V$.
  
  \vspace{4mm}
  
  \noindent\blacksquare\textbf{ THEOREM 1.10}
  
  \vspace{2mm}
  
  Let $V$ be a vector space that is generated by a set $G$ containing exactly $n$ vectors,\par and let $L$ be a linearly independent subset of $V$ containing exactly $m$ vectors. Then\par $m\leq n$ and there exists a subset $H$ of $G$ containing exactly $n-m$ vectors such that\par $L\cup H$ generates $V$.
  
  \vspace{4mm}
  
  \noindent\blacktriangle\textbf{ COROLLARY 1.}
  
  \vspace{2mm}
  
  Let $V$ be a vector space having a finite basis. Then every basis for $V$ contains the\par same number of vectors.
  
  \vspace{4mm}
  
  \noindent\blacktriangle\textbf{ COROLLARY 2.}
  
  \vspace{2mm}
  
  Let $V$ be a vector space with dimension $n$.
  
  \vspace{2mm}
  
  \begin{addmargin}[1.5em]{1.5em}
  
    (a) Any finite generating set for $V$ contains at least $n$ vectors, and a generating\par set for $V$ that contains exactly $n$ vectors is a basis for $V$.\par
    \noindent(b) Any linearly independent subset of $V$ that contains exactly $n$ vectors is a basis\par for $V$.\par
    \noindent(c) Every linearly independent subset of $V$ can be extended to a basis for $V$.
  
  \end{addmargin}
  
  \vspace{4mm}
  
  \noindent\blacksquare\textbf{ THEOREM 1.11}
  
  \vspace{2mm}
  
  Let $W$ be a subspace of a finite-dimensional vector space $V$. Then $W$ is finite-\par dimensional and $dim(W)\leq dim(V)$. Moreover, if $dim(W)=dim(V)$, then $W=V$.
  
  \vspace{4mm}
  
  \noindent\blacktriangle\textbf{ COROLLARY.}
  
  \vspace{2mm}
  
  If $W$ is a subspace of a finite-dimensional vector space $V$, then any basis for $W$ can\par be extended to a basis for $V$.
 
 
 
 
 
 \end{document}