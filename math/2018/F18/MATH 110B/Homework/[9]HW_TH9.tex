\documentclass[12pt, a4paper]{article}
\usepackage[margin=1in]{geometry}
\usepackage[latin1]{inputenc}
\usepackage{titlesec}
\usepackage{amsmath}
\usepackage{amsthm}
\usepackage{amsfonts}
\usepackage{amssymb}
\usepackage{array}
\usepackage{booktabs}
\usepackage{ragged2e}
\usepackage{enumerate}
\usepackage{enumitem}
\usepackage{cleveref}
\usepackage{slashed}
\usepackage{commath}
\usepackage{lipsum}
\usepackage{colonequals}
\usepackage{addfont}
\addfont{OT1}{rsfs10}{\rsfs}
\renewcommand{\baselinestretch}{1.1}
\usepackage[mathscr]{euscript}
\let\euscr\mathscr \let\mathscr\relax
\usepackage[scr]{rsfso}
\newcommand{\powerset}{\raisebox{.15\baselineskip}{\Large\ensuremath{\wp}}}
\usepackage{longtable}
\usepackage{multirow}
\usepackage{multicol}
\usepackage{calligra}
\usepackage[T1]{fontenc}
\newcounter{proofc}
\renewcommand\theproofc{(\arabic{proofc})}
\DeclareRobustCommand\stepproofc{\refstepcounter{proofc}\theproofc}
\usepackage{fancyhdr}
\pagestyle{fancy}

\renewcommand{\headrulewidth}{0pt}
\fancyhead[R]{}
\usepackage{enumitem}
\usepackage{tikz}
\usepackage{commath}
\usepackage{colonequals}
\usepackage{bm}
\usepackage{tikz-cd}
\renewcommand{\baselinestretch}{1.1}
\usepackage[mathscr]{euscript}
\let\euscr\mathscr \let\mathscr\relax
\usepackage[scr]{rsfso}
\usepackage{titlesec}
\usepackage{scrextend}
\usepackage{lscape}
\usepackage{relsize}

\usepackage[english]{babel}
\usepackage{blindtext}
\usepackage{polynom}



\newcommand*{\logeq}{\ratio\Leftrightarrow}

\titleformat{\section}
  {\normalfont\Large\bfseries}{\thesection}{1em}{}[{\titlerule[0.8pt]}]
  
\setlist[description]{leftmargin=4mm,labelindent=4mm}
  
\begin{document}
  
\begin{flushleft}
  
    Quin Darcy\par
    Dr. Shannon\par
    MATH 110B\par
    11/11/18
  
\end{flushleft}
  
\centerline{Homework 9}
 
\vspace{4mm}
 
\noindent\textsc{Section: }\par
 
\justifying
 
\vspace{1mm}
 
\hline
 
\vspace{4mm}

\noindent\textbf{3.} Recall from class (and HW7) $\omega=-\frac{1}{2}+i\frac{\sqrt{3}}{2}$, $\omega^2=-\frac{1}{2}-i\frac{\sqrt{3}}{2}$, and $\omega^3=1$. Find $[\mathbb{Q}(\omega\sqrt[3]{2})\colon\mathbb{Q}]$.

\vspace{4mm}

\noindent\textbf{Solution:} In class we showed that if $c$ is algebraic over $F$, then the monic irreducible polynomial $p(X)\in F[X]$ that has $c$ as a root is called the minimal polynomial of $c$ over $F$. Additionally, it was shown $[F(c)\colon F]=\deg(p(X))$. Thus, we must find the minimal polynomial of $\omega\sqrt[3]{2}$ over $\mathbb{Q}$.\par
We can see that $(\omega\sqrt[3]{2})^3=(1)(2)=2$, and this tells us that our polynomial must be of degree 3. Thus, if we let $p(X)=X^3-2$, then $p(\omega\sqrt[3]{2})=0$. Thus, $p(X)$ is the minimal polynomial of $\omega\sqrt[3]{2}$ over $\mathbb{Q}$ and we then have

\vspace{2mm}

\centerline{$[\mathbb{Q}(\omega\sqrt[3]{2})\colon\mathbb{Q}]=\deg(p(X))=3$.}

\vspace{4mm}

\noindent\textbf{4.} Determine (with explanation) if the following pairs of rings are isomorphic.

\vspace{4mm}\par

\textbf{(a)} $\mathbb{Q}(\pi)$, $\mathbb{Q}(\pi\sqrt{2})$.

\vspace{4mm}\par

\textbf{Solution:} Because both $\pi$ and $\pi\sqrt{2}$ are transcendental over $\mathbb{Q}$, then there are no polynomials in $\mathbb{Q}[X]$ for which $\pi$ or $\pi\sqrt{2}$ is a root. Thus, by the theorem on p.2 of the Extension Fields handout, we have $\mathbb{Q}(\pi)\cong\mathbb{Q}(X)$ and $\mathbb{Q}(\pi\sqrt{2})\cong\mathbb{Q}(X)$. Thus, $\mathbb{Q}(\pi)\cong\mathbb{Q}(\pi\sqrt{2})$.

\vspace{4mm}\par

\textbf{(b)} $\mathbb{Q}[\sqrt{3}]$, $\mathbb{Q}[X]/(g(X))_i$, where $g(X)=X^3-2X^2-3X+6$.

\vspace{4mm}\par

\textbf{Solution:} Although $\sqrt{3}$ is a root of $g(X)$, and so FHT would tell us the two rings are isomorphic; however, $g(X)$ is not irreducible since $g(X)=X^3-2X^2-3X+6=(X^2-3)(X-2)$. The implications of this is that we know from class $\mathbb{Q}[\sqrt{3}]$ is a field. But since $g(X)$ is not irreducible, then $(g(X))_i$ is not maximal, and so $\mathbb{Q}[X]/(g(X))_i$ is not a field. Thus, the two rings cannot be isomorphic.

\vspace{4mm}\par

\textbf{(c)} $\mathbb{Q}(\sqrt{7})$, $\mathbb{Q}(\sqrt{5})$.

\vspace{4mm}\par

\textbf{Solution:} Since $\sqrt{7}$ is a root of $X^2-7$ and $\sqrt{5}$ is a root of $X^2-5$, then by p.2 of the Field Extensions handout, we have $\mathbb{Q}(\sqrt{7})\cong\mathbb{Q}[\sqrt{7}]$, and $\mathbb{Q}(\sqrt{5})\cong\mathbb{Q}[\sqrt{5}]$. Thus, from HW4, we know $\mathbb{Q}[\sqrt{7}]\ncong\mathbb{Q}[\sqrt{5}]$. Thus, $\mathbb{Q}(\sqrt{7})\ncong\mathbb{Q}(\sqrt{5})$.

\vspace{4mm}

\noindent\textbf{5.} Recall $e\in\mathbb{R}$ is the base of the natural logarithm. Give the form of each element of $\mathbb{Q}(e)$, and give (with explanation) a field that is isomorphic to $\mathbb{Q}(e)$.

\vspace{4mm}

\noindent\textbf{Solution:} The form of each element in $\mathbb{Q}(e)$ can be derived by looking at the set itself. We have that 

\vspace{2mm}

\centerline{$ \mathbb{Q}(e)=\{\frac{a_0+a_1e+a_2e^2+\cdots+a_ne^n}{b_0+b_1e^1+b_2e^2+\cdots+b_me^m}\mid a_i,b_j\in\mathbb{Q}\wedge\exists b_k\neq 0\} $.}

\vspace{2mm}

\noindent So for any given $u\in\mathbb{Q}(e)$, we have that 

\vspace{2mm}

\centerline{$ u=(\sum\limits_{i=0}^{n} a_ie^i)(\sum\limits_{j=0}^{m} b_je^j)^{-1} $.}

\vspace{2mm}

\noindent Since $e$ is transcendental over $\mathbb{Q}$, then $\mathbb{Q}(e)\cong\mathbb{Q}(X)$. Thus, in problem 4.(a) we saw that $\mathbb{Q}(\pi)\cong\mathbb{Q}(X)$. Thus, $\mathbb{Q}(e)\cong\mathbb{Q}(\pi)$. 

\vspace{4mm}

\noindent\textbf{6.}

\vspace{4mm}\par

\textbf{(a)} Assume that $E$ is a field, $F$ is a subfield of $E$, $g(X)\in F[X]$ is irreducible over $F[X]$, $c,d\in E$, and $c$ and $d$ are each roots of $g(X)$ over $E[X]$. Explain why $F(c)\cong F(d)$.

\vspace{4mm}\par

\textbf{Solution:} Since $g(X)$ is assumed to be irreducible and it has both $c$ and $d$ as a root, then $F[X]/(g(X))_i\cong F(c)$ and $F[X]/(g(X))_i\cong F(d)$. Thus, $F(c)\cong F(d)$.

\vspace{4mm}\par

\textbf{(b)} Explain why $\mathbb{Q}(\omega\sqrt[3]{2})\cong\mathbb{Q}(\sqrt[3]{2})$.

\vspace{4mm}\par

\textbf{Solution:} Consider the polynomial $X^3-2$. It is an irreducible polynomial in $\mathbb{Q}[X]$ and because $(\omega\sqrt[3]{2})^3=(\sqrt[3]{2})^3=2$, then both $\omega\sqrt[3]{2}$, $\sqrt[3]{2}$ are roots to $X^3-2$. Thus, $\mathbb{Q}[X]/(X^3-2)_i\cong\mathbb{Q}(\omega\sqrt[3]{2})$ and $\mathbb{Q}[X]/(X^3-2)_i\cong\mathbb{Q}(\sqrt[3]{2})$. Therefore, $\mathbb{Q}(\omega\sqrt[3]{2})\cong\mathbb{Q}(\sqrt[3]{2})$.

\vspace{4mm}\par

\textbf{(c)} Determine (with explanation) whether or not $\mathbb{Q}(\omega\sqrt[3]{2})=\mathbb{Q}(\sqrt[3]{2})$.

\vspace{4mm}\par

\textbf{Solution:} These two sets are not equal. This is because the element $\omega\sqrt[3]{2}\in\mathbb{Q}(\omega\sqrt[3]{2})$, thus $(-\frac{\sqrt[3]{2}}{2}+i\frac{\sqrt[3]{2}\sqrt{3}}{2})\in\mathbb{Q}(\omega\sqrt[3]{2})$, which is an element with an imaginary component. Since $\mathbb{Q}(\sqrt[3]{2})$ has no elements with imaginary components, then the two sets cannot be equal.

\vspace{6mm}

\noindent\textbf{7.}

\vspace{4mm}\par

\textbf{(a)} Find $[\mathbb{Q}(\sqrt{2},\sqrt{3})\colon\mathbb{Q}(\sqrt{3})]$, and find $[\mathbb{Q}(\sqrt{2},\sqrt{3})\colon\mathbb{Q}]$ by using the result on the bottom of p.2 of the Vector Spaces handout (and explain your answers).

\vspace{4mm}\par

\textbf{Solution:} To find $[\mathbb{Q}(\sqrt{2},\sqrt{3})\colon\mathbb{Q}(\sqrt{3})]$ we can look at a basis for $\mathbb{Q}(\sqrt{2},\sqrt{3})$ over $\mathbb{Q}(\sqrt{3})$

\vspace{4mm}\par

\textbf{(b)} Find $[\mathbb{Q}(\sqrt[4]{3}, i\sqrt{3}, \sqrt{3})\colon\mathbb{Q}]$, and explain your answer.

\vspace{4mm}\par

\textbf{Solution:}

\vspace{6mm}

\noindent\textbf{8.} Find the minimal polynomial of $4-\sqrt{2}$ over $\mathbb{Q}$ (and show your work).

\vspace{4mm}

\noindent\textbf{Solution:} Let $b=4-\sqrt{2}$. Then $4-b=\sqrt{2}$. Thus, $(4-b)^2=(\sqrt{2})^2$. Thus, $16-8b+b^2=2$. And so $b$ is a root $X^2-8X+14$. We can see that $X^2-8X+14$ is irreducible over $\mathbb{Q}$ since

\begin{equation*}
        \begin{split}
            X &= \frac{8\pm\sqrt{64-56}}{2} \\
            &= \frac{8\pm 2\sqrt{2}}{2} \\
            &= 4\pm\sqrt{2}.
        \end{split}
    \end{equation*}
    
\vspace{4mm}

\noindent Thus, since $X^2-8X+14$ is a monic and irreducible polynomial over $\mathbb{Q}$, whose roots include $4+\sqrt{2}$, then it is the minimal polynomial for $4+\sqrt{2}$. 
 
\end{document}