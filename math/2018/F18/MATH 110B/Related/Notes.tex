\documentclass[12pt, a4paper]{article}
\usepackage[margin=1in]{geometry}
\usepackage[latin1]{inputenc}
\usepackage{titlesec}
\usepackage{amsmath}
\usepackage{amsthm}
\usepackage{amsfonts}
\usepackage{amssymb}
\usepackage{array}
\usepackage{booktabs}
\usepackage{ragged2e}
\usepackage{enumerate}
\usepackage{enumitem}
\usepackage{cleveref}
\usepackage{slashed}
\usepackage{commath}
\usepackage{lipsum}
\usepackage{colonequals}
\usepackage{addfont}
\addfont{OT1}{rsfs10}{\rsfs}
\renewcommand{\baselinestretch}{1.1}
\usepackage[mathscr]{euscript}
\let\euscr\mathscr \let\mathscr\relax
\usepackage[scr]{rsfso}
\newcommand{\powerset}{\raisebox{.15\baselineskip}{\Large\ensuremath{\wp}}}
\usepackage{longtable}
\usepackage{multirow}
\usepackage{multicol}
\usepackage{calligra}
\usepackage[T1]{fontenc}
\newcounter{proofc}
\renewcommand\theproofc{(\arabic{proofc})}
\DeclareRobustCommand\stepproofc{\refstepcounter{proofc}\theproofc}
\usepackage{fancyhdr}
\pagestyle{fancy}

\renewcommand{\headrulewidth}{0pt}
\fancyhead[R]{}
\usepackage{enumitem}
\usepackage{tikz}
\usepackage{commath}
\usepackage{colonequals}
\usepackage{bm}
\usepackage{tikz-cd}
\renewcommand{\baselinestretch}{1.1}
\usepackage[mathscr]{euscript}
\let\euscr\mathscr \let\mathscr\relax
\usepackage[scr]{rsfso}
\usepackage{titlesec}
\usepackage{scrextend}
\usepackage{lscape}
\usepackage{relsize}

\usepackage[english]{babel}
\usepackage{blindtext}
\usepackage{polynom}



\newcommand*{\logeq}{\ratio\Leftrightarrow}

\titleformat{\section}
  {\normalfont\Large\bfseries}{\thesection}{1em}{}[{\titlerule[0.8pt]}]
  
 \setlist[description]{leftmargin=12.8mm,labelindent=4mm}
 
 \begin{document}
 
 \section{Ideals, Homomorphisms, and Quotient Rings}
 
 \noindent\textbf{Definition.} The mapping $\varphi\colon R\rightarrow R'$ of the ring $R$ into the ring $R'$ is a \textit{homomorphism} if
 
 \begin{description}
    \item (a) $\varphi(a+b)=\varphi(a)+\varphi(b)$, and
    \item (b) $\varphi(ab)=\varphi(a)\varphi(b)$ for all $a,b\in R$.
 \end{description}
 
 \vspace{2mm}
 
 \noindent\textbf{Definition.} Let $R$ be a ring. A nonempty subset $I$ or $R$ is called a \textit{ideal} of $R$ if:
 
 \begin{description}
    \item (a) $I$ is an additive subgroup of $R$.
    \item (b) Given $r\in R$, $a\in I$, then $ra\in I$ and $ar\in I$. 
 \end{description}
 
 \vspace{2mm}
 
 \noindent\textbf{Definition.} A proper ideal $M$ of $R$ is a \textit{maximal ideal} if the only ideals of $R$ that contain $M$ are $M$ itself and $R$.
 
 \vspace{4mm}
 
 \noindent\textbf{Theorem 1.} \textit{Let $R$ be a commutative ring with identity, let $M$ be an ideal of $R$, then $R/M$ is a field if and only if $M$ is maximal.}
 
 \section{Polynomial Rings}
 
 
 
 \end{document}