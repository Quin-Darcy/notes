\documentclass[12pt, a4paper]{article}
\usepackage[margin=1in]{geometry}
\usepackage[latin1]{inputenc}
\usepackage{titlesec}
\usepackage{amsmath}
\usepackage{amsthm}
\usepackage{amsfonts}
\usepackage{amssymb}
\usepackage{array}
\usepackage{booktabs}
\usepackage{ragged2e}
\usepackage{enumerate}
\usepackage{enumitem}
\usepackage{commath}
\usepackage{colonequals}
\renewcommand{\baselinestretch}{1.1}
\usepackage[mathscr]{euscript}
\let\euscr\mathscr \let\mathscr\relax
\usepackage[scr]{rsfso}
\usepackage{bm}
\newcommand{\powerset}{\raisebox{.15\baselineskip}{\Large\ensuremath{\wp}}}
\newcolumntype{C}{>$c<$}
\usepackage{fancyhdr}
\pagestyle{fancy}
\fancyhf{}
\renewcommand{\headrulewidth}{0pt}
\fancyhead[R]{\thepage}


\newcommand*{\logeq}{\ratio\Leftrightarrow}

\setlist[description]{leftmargin=60mm,labelindent=60mm}



\begin{document}

\begin{flushleft}
Quin Darcy\linebreak 
MATH 110A\linebreak	
Prof. Schulte\linebreak
4/6/18\linebreak
\end{flushleft}




\section*{\centerline{Quiz 4}}

\justifying

\vspace{4mm}

\begin{flushleft}
\textbf{\textsc{Exercise 7.4.1}}
\end{flushleft}

\vspace{4mm}


Let $\tilde{\gamma}$ be a reparametrization of $\gamma$, so that $\tilde{\gamma}(t)=\gamma(\varphi(t))$ for some smooth function $\varphi$ with $d\varphi/dt\neq 0$ for all vaules of $t$. If $\textbf{v}$ is a tangent vector field along $\gamma$, show that $\tilde{\textbf{v}}(t)=\textbf{v}(\varphi(t))$ is one along $\tilde{\gamma}$. Prove that\par

\vspace{7mm}

\centerline{\Large{$\nabla_{\tilde{\gamma}}\tilde{\textbf{v}}=\frac{d\varphi}{dt}\nabla_{\gamma}\textbf{v}$},}

\vspace{4mm}

\begin{flushleft}
and deduce that \textbf{v} is parallel along $\gamma$ if and only $\tilde{\textbf{v}}$ is parallel along $\tilde{\gamma}$.
\end{flushleft}



\vspace{4mm}

\begin{flushleft}
\textbf{Solution}
\end{flushleft}



\vspace{4mm}


If we assume $\gamma\colon I\subseteq\mathbb{R}\rightarrow S$ and \textbf{v} is a differentiable vector field along $\gamma$, then \textbf{v} is a correspondence that assigns to each $t\in I$, a vector \textbf{v}$(t)\in T_{\gamma(t)}S$. Now suppose that $\gamma(t)=\bm{x}(u(t), v(t))$, then we may write\par

\vspace{4mm}

\centerline{\textbf{v}$(t)=\frac{d\bm{x}}{dt}=a(u(t),v(t))\bm{x}_u+b(u(t),v(t))\bm{x}_v$}
\vspace{2mm}
\hspace{52.5mm}$=a(t)\bm{x}_u+b(t)\bm{x}_v$.\par

\vspace{4mm}

By computing $d$\textbf{v}$/dt$ and dropping the normal component, we find that it is independent of our choice of $\gamma$. Thus, if \textbf{v} is a tangent vector field along $\gamma(t)$ and $\tilde{\gamma}(\varphi(t))$ is a reparametrization of $\gamma(t)$, then $\tilde{\textbf{v}}=d\tilde{\bm{x}}/dt$ is a tangent vector field along $\tilde{\gamma}(t)$.\par

\vspace{4mm}

Finally, since $d\varphi/dt\neq 0$ for all values of $t$, then we have 
\par

\vspace{4mm}

\centerline{$\nabla_{\tilde{\gamma}}\tilde{\textbf{v}}=\dot{\tilde{\textbf{v}}}-(\dot{\tilde{\textbf{v}}}\cdot N)N$}
\vspace{2mm}
\centerline{$=\frac{\partial \textbf{v}}{\partial \varphi}\frac{d\varphi}{dt}-((\frac{\partial \textbf{v}}{\partial \varphi}\frac{d\varphi}{dt})\cdot N)N$.}

\vspace{4mm}

Thus,\par

\vspace{4mm}

\centerline{$\dot{\tilde{\textbf{v}}}=(\dot{\tilde{\textbf{v}}}\cdot N)N\Leftrightarrow \dot{\textbf{v}}=(\dot{\textbf{v}}\cdot N)N$.}




\end{document}