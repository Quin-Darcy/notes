\documentclass[12pt, a4paper]{article}
\usepackage[margin=1in]{geometry}
\usepackage[latin1]{inputenc}
\usepackage{titlesec}
\usepackage{amsmath}
\usepackage{amsthm}
\usepackage{amsfonts}
\usepackage{amssymb}
\usepackage{array}
\usepackage{booktabs}
\usepackage{ragged2e}
\usepackage{enumerate}
\usepackage{mathtools}
\usepackage{verbatim}
\usepackage{enumitem}
\usepackage{tikz}
\usepackage{commath}
\usepackage{colonequals}
\usepackage{bm}
\usepackage{tikz-cd}
\renewcommand{\baselinestretch}{1.1}
\usepackage[mathscr]{euscript}
\let\euscr\mathscr \let\mathscr\relax
\usepackage[scr]{rsfso}
\newcommand{\powerset}{\raisebox{.15\baselineskip}{\Large\ensuremath{\wp}}}
\newcolumntype{C}{>$c<$}
\usepackage{stackengine}



\newcommand*{\logeq}{\ratio\Leftrightarrow}

\setlist[description]{leftmargin=60mm,labelindent=60mm}



\begin{document}


\section*{\centerline{Notes and Definitions}}

\justifying

\section*{\centerline{Tangent Vectors}}

\vspace{6mm}

    We begin by considering \textit{geometric tangent vectors} in $\mathbb{R}^n$, which can be visualized as ``arrows" attached to points. Because the definition of smooth manifolds is built around the idea of identifying which functions are smooth, the property of a geometric tangent vector that is amenable to generalization is its action on smooth functions as a ``directional derivative". The process of taking directional derivatives gives a one-to-one correspondence between geometric tangent vectors and linear maps from $C^{\infty}(\mathbb{R}^n)$ to $\mathbb{R}$ satisfying the product rule. (Such maps are called \textit{derivations}.)\par
    
    Any smooth coordinate chart $(U, \varphi)$ gives a natural isomorphism from the space of tangent vectors to $M$ at $p$ to the space of tangent vectors to $\mathbb{R}^n$ at $\varphi(p)$, which in turn is isomorphic to the space of geometric tangent vectors at $\varphi(p)$. Thus, any smooth any smooth coordinate chart yields a basis for each tangent space.\par
    
    Next we will show how a smooth curve determines a tangent vector at each point, called its \textit{velocity}, which can be regarded as the derivation of $C^{\infty}(M)$ that takes the derivative of each function along the curve.\par
    
\vspace{10mm}


    
    Imagine a manifold in Euclidean space--for example, the unit sphere $\mathbb{S}^{n-1}\subseteq\mathbb{R}^n$. What do we mean by a ``tangent vector" at a point of $\mathbb{S}^{n-1}$? We first must acknowledge the way we think about elements of $\mathbb{R}^n$. On one hand, we usually think of them as \textit{points} in space, whose only property is location, expressed by coordinates $(x^1,\dots, x^n)$. On the other hand, when doing calculus we sometimes think of them instead as \textit{vectors}, which are objects that have magnitude and direction, but whose location is irrelevant. A vector $v=v^ie_i$ (where $e_i$ denotes the $i$th standard basis vector) can be visualized as an arrow with its initial point anywhere in $\mathbb{R}^n$.\par

    What we really have in mind here is a separate copy of $\mathbb{R}^n$ at each point. When we talk about vectors tangent to the sphere at a point $a$, for example, we imagine them as living in a copy of $\mathbb{R}^n$ with its origin translated to $a$.









\end{document}