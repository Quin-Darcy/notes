\documentclass[12pt, a4paper]{article}
\usepackage[margin=1in]{geometry}
\usepackage[latin1]{inputenc}
\usepackage{titlesec}
\usepackage{amsmath}
\usepackage{amsthm}
\usepackage{amsfonts}
\usepackage{amssymb}
\usepackage{array}
\usepackage{booktabs}
\usepackage{ragged2e}
\usepackage{enumerate}
\usepackage{enumitem}
\usepackage{commath}
\usepackage{colonequals}
\renewcommand{\baselinestretch}{1.1}
\usepackage[mathscr]{euscript}
\let\euscr\mathscr \let\mathscr\relax
\usepackage[scr]{rsfso}
\newcommand{\powerset}{\raisebox{.15\baselineskip}{\Large\ensuremath{\wp}}}
\newcolumntype{C}{>$c<$}
\usepackage{fancyhdr}
\pagestyle{fancy}
\fancyhf{}
\renewcommand{\headrulewidth}{0pt}
\fancyhead[R]{\thepage}


\newcommand*{\logeq}{\ratio\Leftrightarrow}

\setlist[description]{leftmargin=60mm,labelindent=60mm}



\begin{document}

\begin{flushleft}
Quin Darcy\linebreak 
MATH 110A\linebreak	
Prof. Krauel\linebreak
4/13/18\linebreak
\end{flushleft}




\section*{\centerline{Homework 11}}

\justifying

\vspace{10mm}

\begin{flushleft}
\textsc{Exercise 6.1}
\end{flushleft}

    List all of the elements in the group $\mathbb{Z}/3\mathbb{Z}\times\mathbb{Z}/4\mathbb{Z}$:\par

\vspace{6mm}
    
    \hspace{28.5mm}$(3\mathbb{Z},4\mathbb{Z})\hspace{12mm}(1+3\mathbb{Z}, 4\mathbb{Z})\hspace{12mm}(2+3\mathbb{Z}, 4\mathbb{Z})$
    
\vspace{2mm}

    \centerline{$(3\mathbb{Z},1+4\mathbb{Z})\hspace{5mm}(1+3\mathbb{Z}, 1+4\mathbb{Z})\hspace{5mm}(2+3\mathbb{Z}, 1+4\mathbb{Z})$} 
    
\vspace{2mm}

    \centerline{$(3\mathbb{Z},2+4\mathbb{Z})\hspace{5mm}(1+3\mathbb{Z}, 2+4\mathbb{Z})\hspace{5mm}(2+3\mathbb{Z}, 2+4\mathbb{Z})$}
    
\vspace{2mm}

    \centerline{$(3\mathbb{Z},3+4\mathbb{Z})\hspace{5mm}(1+3\mathbb{Z}, 3+4\mathbb{Z})\hspace{5mm}(2+3\mathbb{Z}, 3+4\mathbb{Z})$}
    
\vspace{6mm}

\begin{flushleft}
\textsc{Exercise 6.2}
\end{flushleft}

    Find the order of the element $(6+30\mathbb{Z}, 15+45\mathbb{Z}, 4+24\mathbb{Z})$ in the group\par $\mathbb{Z}/30\mathbb{Z}\times\mathbb{Z}/45\mathbb{Z}\times\mathbb{Z}/24\mathbb{Z}$.
    
\vspace{6mm}

    \hspace{21mm}$(6+30\mathbb{Z}, 15+45\mathbb{Z}, 4+24\mathbb{Z})^1=(6+30\mathbb{Z}, 15+45\mathbb{Z}, 4+24\mathbb{Z})$
    
\vspace{2mm}

    \centerline{$(6+30\mathbb{Z}, 15+45\mathbb{Z}, 4+24\mathbb{Z})^3=(18+30\mathbb{Z}, 45\mathbb{Z}, 12+24\mathbb{Z})$}
    
\vspace{2mm}

    \centerline{$(6+30\mathbb{Z}, 15+45\mathbb{Z}, 4+24\mathbb{Z})^5=(30\mathbb{Z}, 30+45\mathbb{Z}, 20+24\mathbb{Z})$}
    
\vspace{2mm}

    \hspace{20mm}
    $(6+30\mathbb{Z}, 15+45\mathbb{Z}, 4+24\mathbb{Z})^6=(6+30\mathbb{Z}, 45\mathbb{Z}, 24\mathbb{Z})$
    
\vspace{2mm}

    \hspace{21mm}$(6+30\mathbb{Z}, 15+45\mathbb{Z}, 4+24\mathbb{Z})^{30}=(30\mathbb{Z}, 45\mathbb{Z}, 24\mathbb{Z})$
    
\vspace{6mm}

    The order of this element was the least common multiple of each of the components orders. Thus, the order of this element is 30.
    

\newpage

\begin{flushleft}
\textsc{exercise 6.3}
\end{flushleft}

    \textsc{Proposition: }\textit{The group $G\times G'$ is abelian if and only if $G$ and $G'$ are abelian.}
    
\vspace{4mm}

    \textsc{proof. }Assume $G\times G'$ is abelian. Then for all $g_1,g_2\in G\times G'$, we have $g_1g_2=g_2g_1$. Considering the type of elements from this group, we may write the following:\par
    
\vspace{6mm}

        \centerline{$\forall (g_1,g'_1),(g_2,g'_2)\in G\times G'\colon$}\par
        
\vspace{2mm}

        \centerline{$(g_1,g'_1)\star(g_2,g'_2)=(g_1\star g_2,g'_1\star g'_2)$}\par
        
\vspace{2mm}

        \centerline{$=(g_2\star g_1,g'_2\star g_2)=(g_2,g'_2)\star(g_1,g'_1)$}\par
        
\vspace{2mm}

        \centerline{$\Leftrightarrow$}\par
        
\vspace{2mm}

        \centerline{$[\forall g_1,g_2\in G\colon(g_1\star g_2=g_2\star g_1)]\wedge[\forall g'_1,g'_2\in G'\colon (g'_1\star g'_2=g'_2\star g'_1)]$}\par
        
\vspace{2mm}

        \centerline{$\Leftrightarrow$}\par
        
\vspace{2mm}

        \centerline{\textit{$G$ and $G'$ are abelian}.}
        
\vspace{6mm}

    We now assume both $G$ and $G'$ are abelian. Under this assumption it follows that\par
    
\vspace{6mm}

        \centerline{$[\forall g_1,g_2\in G\colon(g_1\star g_2=g_2\star g_1)]\wedge[\forall g'_1,g'_2\in G'\colon (g'_1\star g'_2=g'_2\star g'_1)]$}\par
        
\vspace{2mm}

        \centerline{$\Leftrightarrow$}
        
\vspace{2mm}

        \centerline{$(g_1,g'_1)\star(g_2,g'_2)=(g_1\star g_2,g'_1\star g'_2)$}\par
        
\vspace{2mm}

        \centerline{$=(g_2\star g_1,g'_2\star g_2)=(g_2,g'_2)\star(g_1,g'_1)$}\par
        
\vspace{2mm}

        \centerline{$\forall (g_1,g'_1),(g_2,g'_2)\in G\times G'$}\par
        
\vspace{2mm}

        \centerline{$\Leftrightarrow$}
        
\vspace{2mm}

        \centerline{\textit{$G\times G'$ is abelian.}}

\vspace{6mm}

\begin{flushleft}
\textsc{Exercise 6.4}
\end{flushleft}

    \textsc{Proposition: }\textit{If $H$ and $H'$ are subgroups of $G$ and $G'$ respectively, then $H\times H'$ is a subgroup of $G\times G'$}.
    
\vspace{4mm}

    \textsc{Proof. }Assume $H\leqslant G$ and $H'\leqslant G'$. Now consider the following sets:\par
    
\vspace{6mm}

        \centerline{$G\times G'=\{(g,g')\colon (g\in G)\wedge(g'\in G')\}$,}\par
        
\vspace{2mm}

        \centerline{$H\times H'=\{(h,h')\colon (h\in H)\wedge(h'\in H')\}$.}\par
        
\vspace{6mm}

    Let $(a,a'),(b,b')\in H\times H'$. We must show $(a,a')(b,b')^{-1}\in H\times H'$. We first observe that $(b, b')^{-1}\in H\times H'$ since $H$ and $H'$ are both subgroups and so it the following follows from this fact:\par
    
\newpage

        \centerline{$(b\in H\implies b^{-1}\in H)\wedge(b'\in H'\implies(b')^{-1}\in H')$.}
        
\vspace{6mm}

    Thus, by the definition of the set $H\times H'$, we can conclude that the element $(b, b')^{-1}\in H\times H'$. Now consider the following product:\par
    
\vspace{6mm}

        \centerline{$(a,a')(b,b')^{-1}=(a,a')(b^{-1},(b')^{-1})=(ab^{-1}, a'(b')^{-1})$.}
        
\vspace{2mm}

    Thus,
    
\vspace{2mm}

        \centerline{$[(ab^{-1}\in H)\wedge(a'(b')^{-1}\in H')]\Leftrightarrow (a,a')(b,b')^{-1}\in H\times H'$.}\par
        
\vspace{6mm}

    Therefore, $H\times H'\leqslant G\times G'$.

    


    



\end{document}}