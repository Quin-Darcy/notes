\documentclass[12pt, a4paper]{article}
\usepackage[margin=1in]{geometry}
\usepackage[latin1]{inputenc}
\usepackage{titlesec}
\usepackage{amsmath}
\usepackage{amsthm}
\usepackage{amsfonts}
\usepackage{amssymb}
\usepackage{array}
\usepackage{booktabs}
\usepackage{ragged2e}
\usepackage{enumerate}
\usepackage{enumitem}
\usepackage{commath}
\usepackage{colonequals}
\renewcommand{\baselinestretch}{1.1}
\usepackage[mathscr]{euscript}
\let\euscr\mathscr \let\mathscr\relax
\usepackage[scr]{rsfso}
\newcommand{\powerset}{\raisebox{.15\baselineskip}{\Large\ensuremath{\wp}}}
\newcolumntype{C}{>$c<$}
\usepackage{fancyhdr}
\pagestyle{fancy}
\fancyhf{}
\renewcommand{\headrulewidth}{0pt}
\fancyhead[R]{\thepage}


\newcommand*{\logeq}{\ratio\Leftrightarrow}

\setlist[description]{leftmargin=8mm,labelindent=10mm}


\begin{document}

\begin{flushleft}
Quin Darcy\linebreak 
MATH 110A\linebreak	
Prof. Krauel\linebreak
4/29/18\linebreak
\end{flushleft}




\section*{\centerline{Homework 12}}

\justifying

\vspace{10mm}

\begin{flushleft}
\textsc{Exercise 7.1}
\end{flushleft}

    \textsc{Proposition: }Suppose $G$ is a group such that $\abs{G}=p$, where $p$ is prime. Then $G\cong \mathbb{Z}/p\mathbb{Z}$.

\vspace{4mm}

    \textsc{Proof. }In order for the set $G$ to be isomorphic to $\mathbb{Z}/p\mathbb{Z}$, we must show that there exists a bijective homomorphism between the two groups. This means that we need to show that there exists some map, $\phi\subseteq (G\times\mathbb{Z}/p\mathbb{Z})$, or $\phi\colon G\rightarrow\mathbb{Z}/p\mathbb{Z}$ such that the following conditions are satisfied:\par
    
\vspace{4mm}

\begin{description}

    \item \textbf{Well Defined }\par$\forall(g,k+p\mathbb{Z}),(g',k'+p\mathbb{Z})\in(G\times\mathbb{Z}/p\mathbb{Z})\colon[(g=g')\Rightarrow (k+p\mathbb{Z}=k'+p\mathbb{Z})]$
    
    \item \textbf{Injective }\par$\forall g_1,g_2\in G\colon[(\phi(g_1)=\phi(g_2))\Rightarrow(g_1=g_2)]$
    
    \item \textbf{Surjective }\par$\forall (k+p\mathbb{Z})\in\mathbb{Z}/p\mathbb{Z}\colon\exists g\in G\colon[\phi(g)=k+p\mathbb{Z}]$
    
    \item \textbf{Homomorphic }\par$\forall g_1,g_2\in G\colon[\phi(g_1+g_2)=\phi(g_1)+\phi(g_2)]$

\end{description}

\vspace{4mm}

    To test each of these conditions, we first need a definition for our map. Consider any arbitrary index for the elements in $G$ such that $G=\{g_0, \cdots g_{p-1}\}$ and then define the map in the following way:\par
    
\vspace{4mm}

        \centerline{$\phi(g_i)=i+p\mathbb{Z}$.}
        
\vspace{4mm}

\newpage

    Let us now see if this definition is well defined. Consider any two elements $g_i,g_j\in G$ and assume $g_i=g_j$. We need that this assumption implies that the images of these elements under $\phi$ are equal. Thus, \par
    
\vspace{4mm}

\hspace{10mm} $g_i=g_j$

\vspace{2mm}

\hspace{18mm} $\Leftrightarrow [(\phi(g_i)=i+p\mathbb{Z})\wedge(\phi(g_j)=j+p\mathbb{Z})]$

\vspace{2mm}

\hspace{18mm} $\Leftrightarrow\exists m\in\mathbb{Z}\colon[\phi(g_i)=(j+pm)+p\mathbb{Z}=\phi(g_{j+pm})]$

\vspace{2mm}

\hspace{18mm} $\Leftrightarrow (j+pm)+p\mathbb{Z}=j+(pm+p\mathbb{Z})=j+p\mathbb{Z}$

\vspace{2mm}

\hspace{18mm} $\Leftrightarrow \phi(g_i)=j+p\mathbb{Z}=\phi(g_j)$

\vspace{2mm}

\hspace{18mm} $\therefore$\hspace{3mm}$(g_i=g_j)\Rightarrow\phi(g_i)=\phi(g_j)$.

\vspace{4mm}

We have shown that $\phi$ is well defined. Now we move on to see if it is injective. Consider any arbitrary $g_i,g_j\in G$ and assume $\phi(g_i)=\phi(g_j)$. Then\par

\vspace{4mm}

\hspace{10mm} $\phi(g_i)=\phi(g_j)$

\vspace{2mm}

\hspace{18mm} $\Leftrightarrow\phi(g_i)=i+p\mathbb{Z}=j+\mathbb{Z}=\phi(g_j)$

\vspace{2mm}

\hspace{18mm} $\Leftrightarrow\phi(g_i)-\phi(g_j)=(i-j)+p\mathbb{Z}=p\mathbb{Z}$

\vspace{2mm}

\hspace{18mm} $\Leftrightarrow(i-j)+p\mathbb{Z}=p\mathbb{Z}$

\vspace{2mm}

\hspace{18mm} $\Leftrightarrow\exists m\in\mathbb{Z}\colon[i-j=pm]$

\vspace{2mm}

\hspace{18mm} $\Leftrightarrow i=pm+j$

\vspace{2mm}

\hspace{18mm} $\Leftrightarrow g_i=g_{j+pm}=g_j$

\vspace{2mm}

\hspace{18mm} $\therefore$\hspace{3mm}$(\phi(g_i)=\phi(g_j))\Rightarrow (g_i=g_j)$

\vspace{4mm}

    From the above proof we can conclude that $\phi$ is injective. To show that $\phi$ is surjective we will choose any arbitrary element $(k+p\mathbb{Z})\in\mathbb{Z}/p\mathbb{Z}$, and show that there exists some element in $G$ such that its image under the map is equal to the arbitrarily chosen element. Thus, if our chosen element is $k+p\mathbb{Z}$, then we observe that for $g_k\in G$, we have $\phi(g_k)=k+p\mathbb{Z}$. Therefore, $\phi$ is surjective. Finally, we must show that $\phi$ is homomorphic.\par
    
    Consider any two arbitrary elements $g_i,g_j\in G$. Now consider the image of their sum.\par
    
\vspace{4mm}

        \centerline{$\phi(g_i+g_j)=(i+j)+p\mathbb{Z}=\phi(g_i)+\phi(g_j)$.}
        
\vspace{4mm}

    We see that the definition of $\phi$ results in the indices being added. The consequence of this definition is that $\phi$ is in fact homomorhpic. Having satisfied all the conditions, we may conclude that for some group $G$, such that $\abs{G}=p$, we have $G\cong\mathbb{Z}/p\mathbb{Z}$.
    
\newpage

\begin{flushleft}
    \textsc{Exercise 7.2}
\end{flushleft}

    \textsc{Proposition: }Consider any two groups $G$ and $G'$. It follows that for any two such groups, we have $G\times G'\cong G'\times G$.
    
\vspace{4mm}

    \textsc{Proof. }Assume $\abs{G}=m$ and $\abs{G'}=n$. By the properties of Direct Products, we have $\abs{G\times G'}=mn=\abs{G'\times G}$. Now consider a map $\phi\colon G\times G'\rightarrow G'\times G$, such that $\phi(g,g')=(g',g)$. Assume for some $(g_1,g'_1),(g_2,g'_2)\in G\times G'$, we have $(g_1,g'_1)=(g_2,g'_2)$. By the definition of the cartesian product, this implies that $g_1=g_2$ and $g'_1=g'_2$. Now consider the following\par
    
\vspace{4mm}

        \centerline{$\phi((g_1,g'_1))=(g'_1,g_1)$\hspace{5mm}and\hspace{5mm}$\phi((g_2,g'_2))=(g'_2,g_2)$.}
        
\vspace{4mm}

    In order for $\phi$ to be well defined we need it to be the case  that $(g'_1,g_1)=(g'_2,g_2)$. This is true if and only if $g'_1=g'_2$ and $g_1=g_2$. However, this was an immediate consequence of our initial assumption. Thus, $(g'_1,g_1)=(g'_2,g_2)$ and our function is well defined.\par
    
    We need to show that $\phi$ is injective. Consider the same elements from before and assume $\phi((g_1,g'_1))=\phi((g_2,g'_2))$. It follows that $(g'_1,g_1)=(g'_2,g_2)$. Thus, $g'_1=g'_2$ and $g_1=g_2$. Thus, $(g_1,g'_1)=(g_2,g'_2)$. $\phi$ is therefore injective. Now consider some element $(g',g)\in G'\times G$. We choose an element $(g,g')\in G\times G'$ and observe $\phi(g,g')=(g',g)$. Thus, $\phi$ is surjective.\par
    
    Finally, consider two arbitrary elements $(g_1,g'_1),(g_2,g'_2)\in G\times G'$. Now consider their `product', and the image of their product:\par
    
\vspace{4mm}

        \centerline{$(g_1,g'_1)(g_2,g'_2)=(g_1g_2,g'_1g'_2)$,}
        
\vspace{2mm}

    and
    
\vspace{2mm}

        \centerline{$\phi((g_1g_2,g'_1g'_2))=(g'_1g'_2,g_1g_2)=(g'_1,g_1)(g'_2,g_2)=\phi((g_1,g'_1))\phi((g_2,g'_2))$.}
        
\vspace{4mm}

    Thus, $\phi$ is homomorphic. Therefore, $G\times G'\cong G'\times G$.
    
\vspace{4mm}

\begin{flushleft}
    \textsc{Exercise 7.3}
\end{flushleft}

    \textsc{Proposition: }Let $G$ be any group, $g$ a fixed element in $G$, and defines $\phi\colon G\rightarrow G$ by $\phi(x)=gxg^{-1}$. Then $\phi$ is an isomorphism of $G$ onto $G$.
    
\vspace{4mm}

    \textsc{Proof. }Consider any two arbitrary elements $x,y\in G$ and assume $x=y$. It follows that $\phi(x)=gxg^{-1}$ and $\phi(y)=gyg^{-1}$. In order for $\phi$ to be well defined we need it to be the case that $gxg^{-1}=gyg^{-1}$. Since $x=y$, by assumption, then it follows that $gxg^{-1}=gyg^{-1}$. Thus, $\phi$ is well defined. Now assume that $gxg^{-1}=gyg^{-1}$ for some $x,y\in G$. Now observe the following\par
    
\vspace{4mm}

\hspace{10mm} $gxg^{-1}=gyg^{-1}$

\vspace{2mm}

\hspace{18mm} $\Leftrightarrow gxg^{-1}(g)=gyg^{-1}(g)$

\newpage

\hspace{18mm} $\Leftrightarrow gx=gy$

\vspace{2mm}

\hspace{18mm} $\Leftrightarrow (g^{-1})gx=(g^{-1})gy$

\vspace{2mm}

\hspace{18mm} $\Leftrightarrow x=y$

\vspace{4mm}

    Thus, if $\phi(x)=\phi(y)$, then $x=y$ and $\phi$ is therefore injective. Now let $z\in G$, such that $z=gxg^{-1}$ for some $x\in G$. We must show that there exists an element in $G$ such that its image under $\phi$ is equal to $z$. Let $x=g^{-1}zg$, then we have\par
    
\vspace{4mm}

        \centerline{$\phi(x)=g(g^{-1}zg)g^{-1}=z$.}
        
\vspace{4mm}

    Thus, $\phi$ is surjective. Now let $x,y\in G$ be any arbitrary elements and consider the following\par
    
\vspace{4mm}

        \centerline{$\phi(x)=gxg^{-1}$\hspace{5mm}and\hspace{5mm}$\phi(y)=gyg^{-1}$.}
        
\vspace{4mm}

    Thus,
    
\vspace{4mm}

                \centerline{$\phi(x)\phi(y)=(gxg^{-1})(gyg^{-1})=(gx)(g^{-1}g)(yg^{-1})=gxyg^{-1}=\phi(x)\phi(y)$.}
                
\vspace{4mm}

    Thus, $\phi$ is a homomorphism. Therefore, $\phi$ is an isomorphism from $G$ onto $G$.
    
\vspace{4mm}

\begin{flushleft}
    \textsc{Exercise 7.4}
\end{flushleft}

    \textsc{Proposition: }Let $G$ be a group and let Aut($G$) be the set of all automorphisms on $G$. Then Aut($G$) is a group with respect to composition of functions.
    
\vspace{4mm}

    \textsc{Proof. }\par
    
\vspace{4mm}

\begin{description}
    
    \item \textbf{Closure }Let $\phi,\psi\in$Aut($G$). We recall that $\phi\colon G\rightarrow G$ and $\psi\colon G\rightarrow G$, such that both maps are isomorphisms. It is clear that $\phi\circ\psi\colon G\rightarrow G$. However, we need to show that $\phi\circ\psi$ is a isomorphism. Let $g,g'\in G$ and assume $g=g'$. Now consider the following\par
    
\vspace{4mm}

        \centerline{$(\phi\circ\psi)(g)=\phi(\psi(g))$\hspace{5mm}and\hspace{5mm}$(\phi\circ\psi)(g')=\phi(\psi(g'))$.}
        
\vspace{4mm}

    Since $\psi(g)$ and $\psi(g')$ are both elements of $G$, let $\psi(g)=a$ and $\psi(g')=b$. Our assumption is that $g=g'$, and since $\psi$ is an isomorphism, then $\psi(g)=a=b=\psi(g')$. Since $a=b$ and $\phi$ is an isomorphism, then this implies that $\phi(a)=\phi(b)$. Thus, $\phi\circ\psi$ is well defined. Let $g,g'\in G$ and assume\par
    
\vspace{4mm}

        \centerline{$(\phi\circ\psi)(g)=(\phi\circ\psi)(g')$.}
        
\vspace{4mm}

    Now let $\psi(g)=a$ and $\psi(g')=b$. Then we have $\phi(a)=\phi(b)$. Thus, $a=b$. Thus, $\psi(g)=\psi(g')$. Therefore, $g=g'$ and $(\phi\circ\psi)$ is injective. Now consider some $g'\in G$. We must show that there exists $g\in G$, such that $(\phi\circ\psi)(g)=g'$.
    
\newpage

    Let $\phi(g)=g'$ and let $\psi(g)=id(g)$, then $\psi(\phi(g))=\phi(g)=g'$. Thus, $(\phi\circ\psi)$ is surjective. Finally, consider two elements $g_1,g_2\in G$. We then observe\par
    
\vspace{4mm}

\hspace{10mm} $(\phi\circ\psi)(g_1g_2)$

\vspace{2mm}

\hspace{18mm} $=\phi(\psi(g_1g_2))$

\vspace{2mm}

\hspace{18mm} $=\phi(\psi(g_1)\psi(g_2))$

\vspace{2mm}

\hspace{18mm} $=\phi(\psi(g_1))\phi(\psi(g_2))$

\vspace{4mm}

    Thus, $(\phi\circ\psi)$ is a bijective homomorphism. Therefore, Aut($G$) is closed with respect to composition of functions.
    
\vspace{4mm}
    
    \item \textbf{Associativity }Let $\phi,\psi,\rho\in$Aut($G$) and let $g\in G$. Then consider the following\par

\vspace{4mm}

        \centerline{$(\phi\circ\psi)\circ\rho(g)$.}
        
\vspace{4mm}

    By definition of composition, we have\par
    
\vspace{4mm}

        \centerline{$(\phi\circ\psi)\circ\rho(g)=\phi(\psi(\rho(g)))=\phi\circ(\psi\circ\rho)(g)$.}
        
\vspace{4mm}

    Thus, composition of functions is associative in Aut($G$).
    
\vspace{4mm}
    
    \item \textbf{Identity }Consider the map $id_G\colon G\rightarrow G$ defined by $id_G(x)=x$. Let $g_1,g_2\in G$ and assume $g_1=g_2$. Then $id_G(g_1)=g_1=g_2=id_G(g_2)$. Thus, this map is well defined. Now consider the same elements as before and assume $id_G(g_1)=id_G(g_2)$. Then $g_1=g_2$ by definition. Thus, this map is injective. Next, consider some element $g\in G$. It is easily seen that since $id_G(g)=g$, the map is surjective. Finally, consider $id_G(g_1g_2)=g_1g_2=id_G(g_1)id_G(g_2)$. Thus, the identity map is an isomoprphism and $id_G\in$Aut($G$). We also note that for all $\phi\in$Aut($G$), we have $(\phi\circ id_G)(g)=\phi(g)=(id_G\circ\phi)(g)$. Thus, it is the identity element.
    
\vspace{4mm}

    \item \textbf{Inverse }We must show that for all $\phi\in$Aut($G$), there exists $\phi^{-1}\in$Aut($G$), such that $(\phi^{-1}\circ\phi)(g)=g$. We can show this by showing $\phi^{-1}$ is an isomorphism. Let $\phi\in$Aut($G$) and let $g_1,g_2\in G$. Since $\phi$ is an isomorphism, we know that $\phi^{-1}$ is a function and thus well defined. Now assume $\phi^{-1}(g_1)=\phi^{-1}(g_2)$. Let $\phi^{-1}(g_1)=a$ and $\phi^{-1}(g_2)=b$, where $a,b\in G$, by definition. Then\par
    
\vspace{4mm}

\hspace{10mm} $\phi^{-1}(g_1)=\phi^{-1}(g_2)$

\vspace{2mm}

\hspace{18mm} $\Leftrightarrow\phi^{-1}(g_1)=a\Rightarrow g_1=\phi(a)$

\vspace{2mm}

\hspace{18mm} $\Leftrightarrow\phi^{-1}(g_2)=b\Rightarrow g_2=\phi(b)$

\vspace{2mm} 

\hspace{18mm} $\Leftrightarrow a=b$

\vspace{2mm}

\hspace{18mm} $\therefore g_1=g_2$

\vspace{4mm}

    Thus, $\phi^{-1}$ is injective. Now consider some $g'\in G$. We know that there exists $g\in G$ such that $\phi(g)=g'$. Thus, $\phi^{-1}(g')=g$. Thus, for all $g\in G$, there exists $g'\in G$ such that $\phi^{-1}(g')=g$. Therefore, $\phi^{-1}$ is surjective, Now consider some $g_1,g_2\in G$, and let $\phi(g_1)=a$ and $\phi(g_2)=b$. We have\par
    
\vspace{4mm}

        \centerline{$\phi(g_1g_2)=\phi(g_1)\phi(g_2)=ab$,}
        
\vspace{4mm}

    and
    
\vspace{4mm}

        \centerline{$\phi^{-1}(\phi(g_1g_2))=\phi^{-1}(\phi(g_1)\phi(g_2))=\phi^{-1}(ab)$,}
        
\vspace{4mm}

    thus
    
\vspace{4mm}

        \centerline{$\phi^{-1}(ab)=g_1g_2=\phi^{-1}(a)\phi^{-1}(b)$.}
        
\vspace{4mm}

    Thus, $\phi^{-1}$ is a bijective homomorphism. Therefore, for all $\phi\in$Aut($G$), there exists $\phi^{-1}\in$Aut($G$). The set of all automorphisms on $G$ is a group with respect to composition of functions.
    
\end{description}

\end{document}