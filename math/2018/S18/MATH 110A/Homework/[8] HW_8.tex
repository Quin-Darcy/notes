\documentclass[12pt, a4paper]{article}
\usepackage[margin=1in]{geometry}
\usepackage[latin1]{inputenc}
\usepackage{titlesec}
\usepackage{amsmath}
\usepackage{amsfonts}
\usepackage{amssymb}
\usepackage{array}
\usepackage{booktabs}
\usepackage{ragged2e}


\begin{document}

\begin{flushleft}
Quin Darcy\linebreak 
MATH 110A\linebreak	
Prof. Krauel\linebreak
3/18/18\linebreak
\hfill


\section*{\centerline{Homework 8}}
\hfil

\justifying
\hspace{2mm}\textbf{Exercise 4.13.} \it{The structure} $(\mathbb{Z}/ n\mathbb{Z}, +)$ \it{is a cyclic group for any} $n\in \mathbb{N}$.\linebreak
\vspace{6mm}

\hspace{2mm}\it{\textbf{Proof.}} \rm To show that $(\mathbb{Z}/ n\mathbb{Z}, +)$ is a cyclic group, we must show that there exists some element $p\in \mathbb{Z}/ n\mathbb{Z}$, such that\hfill

\vspace{6mm}
\centerline{$\langle p \rangle=\hspace{0.5mm} \mathbb{Z}/ n\mathbb{Z}$.}

\vspace{6mm}
\begin{flushleft}
Suppose we let $p=1+n\mathbb{Z}$, and let $A = \{0, \ldots , (n-1)\}$. Then the set generated by $1+n\mathbb{Z}$ is the following:
\end{flushleft}

\vspace{6mm}
\centerline{$\langle(1+n\mathbb{Z})\rangle=\{(1+n\mathbb{Z}), (1+n\mathbb{Z})+(1+n\mathbb{Z}), \ldots , (1+n\mathbb{Z})+\ldots+(1+n\mathbb{Z})\}$}\hfil

\vspace{3mm}
\centerline{$=\{n\mathbb{Z}, 1+n\mathbb{Z}, \ldots , (n-1)+n\mathbb{Z}\}$}\hfil

\vspace{3mm}
\centerline{$=\{k+n\mathbb{Z}\mid k\in A\}$}\hfil

\vspace{3mm}
\centerline{$=\mathbb{Z}/n\mathbb{Z}$.}\hfil

\vspace{6mm}
\begin{flushleft}
Thus, we have shown that there exists an element in $\mathbb{Z}/n\mathbb{Z}$ such that it generates $\mathbb{Z}/n\mathbb{Z}$. Therefore, the structure $(\mathbb{Z}/n\mathbb{Z}, +)$ is a cyclic group for any $n\in \mathbb{N}$.
\end{flushleft}

\vspace{6mm}
\hspace{2mm}\textbf{Exercise 4.14.} \it{The group} $((\mathbb{Z}/8\mathbb{Z})^{*}, \star)$ \it{is not cyclic.}\hfil
\vspace{10mm}

\hspace{2mm}\it{\textbf{Proof.}} \rm In order for a group to be cyclic, there must exist an element within the group such that it generates the whole group. The claim is that this group is not cyclic and so we must show that such an element does not exist.\hfil
\vspace{16mm}

\begin{flushleft}
The definition of the set $(\mathbb{Z}/8\mathbb{Z})^{*}$ is as follows:\hfil
\end{flushleft}

\vspace{6mm}
\centerline{$(\mathbb{Z}/8\mathbb{Z})^{*}=\{k+8\mathbb{Z}\in\mathbb{Z}/8\mathbb{Z}\mid gcd(k, n)=1\}$.}\hfil

\vspace{2mm}
\begin{flushleft}
The values of $k$ for which this set is defined is the subset of $\{0, \ldots, 7\}$, such that for each $k\in\{0, \ldots, 7\}$, there exists $x\in\{0, \ldots, 7\}$ in which $n\mid(kx-1)$. This can be stated equivalently as requiring the existence of some integer $l$, such that $nl=kx-1$. It is easily checked with a multiplication table that for all evaluations of the expression $kx-1$, with $k$ and $x$ ranging over $\{0, \ldots, 7\}$, there are only four multiples of $8$. Those are the following:
\end{flushleft}

\vspace{6mm}

\begin{center}
\begin{enumerate}
\centering
\item \hspace{1.5mm}$1\star1-1=0=8\star0\Longrightarrow l=0$
\item \hspace{1.5mm}$3\star3-1=8=8\star1\Longrightarrow l=1$
\item $5\star5-1=24=8\star3\Longrightarrow l=3$
\item $7\star7-1=48=8\star6\Longrightarrow l=6$
\end{enumerate}
\end{center}

\vspace{6mm}
\begin{flushleft}
The following list tells us that the values of $x\in\{0, \ldots, 7\}$ for which an inverse exists are $\{1, 3, 5, 7\}$. Thus,
\end{flushleft}

\vspace{6mm}
\centerline{$(\mathbb{Z}/8\mathbb{Z})^{*}=\{(1+8\mathbb{Z}), (3+8\mathbb{Z}), (5+8\mathbb{Z}), (7+8\mathbb{Z})\}$}
\vspace{6mm}
 
\begin{flushleft}
Now that we have all of the elements to our set, we can directly evaluate the sets that each of them generate respectively. 
\end{flushleft}

\vspace{6mm}
\centerline{$\langle(1+8\mathbb{Z})\rangle=\{(1+8\mathbb{Z})\}\neq(\mathbb{Z}/8\mathbb{Z})^{*}$}\hfil
\vspace{2mm}
\centerline{$\langle(3+8\mathbb{Z})\rangle=\{(1+8\mathbb{Z}), (3+8\mathbb{Z})\}\neq(\mathbb{Z}/8\mathbb{Z})^{*}$}\hfil
\vspace{7mm}
\centerline{$\langle(5+8\mathbb{Z})\rangle=\{(1+8\mathbb{Z}), (5+8\mathbb{Z})\}\neq(\mathbb{Z}/8\mathbb{Z})^{*}$}\hfil
\vspace{7mm}
\centerline{$\langle(7+8\mathbb{Z})\rangle=\{(1+8\mathbb{Z}), (7+8\mathbb{Z})\}\neq(\mathbb{Z}/8\mathbb{Z})^{*}$}\hfil
\vspace{6mm}

\begin{flushleft}
We have now shown that there does not exist an element in $(\mathbb{Z}/8\mathbb{Z})^{*}$ such that the set it generates is equal to $(\mathbb{Z}/8\mathbb{Z})^{*}$, and the set is therefore not cyclic.
\end{flushleft}

\vspace{26mm}
\hspace{2mm}\textbf{Exercise 4.15.} \it The group $(\mathbb{Z}/9\mathbb{Z})^{*}$ is cyclic.
\vspace{10mm}

\hspace{2mm}\textbf{Proof.} \rm To show that $(\mathbb{Z}/9\mathbb{Z})^{*}$ is a cyclic group, we must find the element(s) within it that generate the whole set. To do this, we will note, as we did in the previous exercise, the elements in the set that have an inverse also in the set. Those elements are the subset of $\{0, \ldots, 8\}$ that consist of the numbers $\{1, 2, 4, 5, 7, 8\}$. Thus, we have the following equality:\hfil

\vspace{6mm}
\centerline{$(\mathbb{Z}/9\mathbb{Z})^{*}=\{(1+9\mathbb{Z}), (2+9\mathbb{Z}), (4+9\mathbb{Z}), (5+9\mathbb{Z}), (7+9\mathbb{Z}), (8+9\mathbb{Z})\}$}\hfil
\vspace{2mm}

\begin{flushleft}
By checking the set that each element generates, it can be confirmed that the element $(2+9\mathbb{Z})$ in fact generates all of $(\mathbb{Z}/9\mathbb{Z})^{*}$. To see why this is, it should be observed that by repeatedly taking powers of $2$ and 'modding' the result with $9$, the subset listed earlier is generated and repeats itself every $6th$ power of $2$. Therefore, with the existence of this element we can conclude that the group $(\mathbb{Z}/9\mathbb{Z})^{*}$ is cyclic.
\end{flushleft}
\vspace{6mm}

\hspace{2mm}\textbf{Exercise 4.17} Consider the function\hfil
\vspace{12mm}

\centerline{$\phi\colon\mathbb{Z}\rightarrow\mathbb{Z}/n\mathbb{Z}$,}\hfil

\vspace{0.2mm}
\hspace{66mm}$k\mapsto\ k+n\mathbb{Z}$.

\vspace{8mm}
\hspace{2mm}\textbf{(a)} In order to prove that $\phi$ is a group homomorphism, we must show that for the groups $(\mathbb{Z}, +)$, $(\mathbb{Z}/n\mathbb{Z}, +)$ and elements $g_1,g_2\in \mathbb{Z}$, the following is true:\hfil
\vspace{10mm}

\centerline{$\phi(g_1+ g_2)=\phi(g_1)+\phi(g_2)$.}

\vspace{6mm}
\begin{flushleft}
Using the definition provided, we see that each integer is mapped to a set. Consider the integers $g_1$ and $g_2$ and the image of their sum:\hfil
\end{flushleft}
\vspace{10mm}

\centerline{$\phi(g_1+g_2)=(g_1+g_2)+n\mathbb{Z}=\{m\in\mathbb{Z}\mid \exists k\in\mathbb{Z}\colon m=(g_1+g_2)+nk\}$.}

\vspace{10mm}
\begin{flushleft}
We can use the definition of addition of sets to split the last equality into the addition of two sets and verify that $\phi$ is a group homomorphism.
\end{flushleft}
\vspace{25mm}

\centerline{$(g_1+g_2)+n\mathbb{Z}=(g_1+n\mathbb{Z})+(g_2+n\mathbb{Z})$}\hfil
\vspace{2mm}
\centerline{$=\{m\in\mathbb{Z}\mid\exists k\in\mathbb{Z}\colon m=g_1+nk\}+\{m\in\mathbb{Z}\mid\exists k\in\mathbb{Z}\colon m=g_2+nk\}$}\hfil
\vspace{8mm}
\centerline{$=\phi(g_1)+\phi(g_2)$.}\hfil
\vspace{8mm}

\begin{flushleft}
Therefore, the function $\phi$ is a group homomorphism.
\end{flushleft}

\vspace{6mm}
\hspace{2mm}\textbf{(b)} The kernel of the function $\phi$ is defined to be the set $\phi^{-1}(\{e\})$, where $e$ is the identity element of $\mathbb{Z}/n\mathbb{Z}$. Under addition, this identity element is the set $n\mathbb{Z}$. Thus, $ker(\phi)=\phi^{-1}(n\mathbb{Z})$. To determine the elements in this set we can first look at the elements of $n\mathbb{Z}$.
\vspace{6mm}

\centerline{$n\mathbb{Z}=\{m\in\mathbb{Z}\mid\exists k\in\mathbb{Z}\colon m=nk\}$.}
\vspace{6mm}

\begin{flushleft}
So this is the set of all integer multiples of $n$. Thus, with the definition of our function, we can see that the elements that get mapped to this set are those that are congruent to $0$ modulo $n$, or rather integer multiples of $n$. Thus, $ker(\phi)=kn, \forall k\in\mathbb{Z}$.
\end{flushleft}




\end{flushleft}
\end{document}