\documentclass[12pt, a4paper]{article}
\usepackage[margin=1in]{geometry}
\usepackage[latin1]{inputenc}
\usepackage{titlesec}
\usepackage{amsmath}
\usepackage{amsfonts}
\usepackage{amssymb}
\usepackage{array}
\usepackage{booktabs}


\begin{document}

\begin{flushleft}
Quin Darcy\linebreak
MATH 110A\linebreak
Prof. Krauel\linebreak
1/27/18\linebreak
\hfill
\end{flushleft}

\section*{\hfil\hfil Homework 2\hfil}
\hfil


\begin{flushleft}
\hfil\textbf{Exercise 1.3} Define the function $\phi$: $\mathbb{R}\rightarrow\mathbb{R}$ by $\phi(k)=e^k$. Show that $\phi$ is a\linebreak morphism from the magma $(\mathbb{R}, +)$ to the magma $(\mathbb{R}, \cdot)$.



\end{flushleft}

\begin{flushleft}
\textbf{Solution:}  Let $a,b\in\mathbb{R}$ be any two real numbers. Now consider the image of the sum of $a$ and $b$ under the mapping $\phi$:\linebreak
\hfil
\hfil
\end{flushleft}

\begin{flushright}\hfil\hfil\hfil$\phi(a+b)=e^{a+b}$\hfil\hfil\hfil\hfil(1)\linebreak\end{flushright}

\begin{flushleft}If we now compare (1) to the product of $\phi(a)$ and $\phi(b$), while making use of the properties of exponents, we see the product, (2),\linebreak
\hfil
\hfil
\hfil
\end{flushleft}

\begin{flushright}\hfil\hfil$\phi(a)\cdot\phi(b)=e^a\cdot e^b=e^{a+b}$,\hfil\hfil\hfil(2)\linebreak\end{flushright}

\begin{flushleft}
is equal to (1). Thus, $\phi(a+b)=\phi(a)\cdot \phi(b)$. We can see then that the mapping $\phi$ preserves the operation between the two magmas $(\mathbb{R}, +)$ and $(\mathbb{R}, \cdot)$. Therefore, $\phi$ is a morphism.\linebreak
\hfil
\hfil
\hfil 
\hfil
\end{flushleft}

\begin{flushleft}
\hfil\textbf{Exercise 1.4.} Suppose $(S, \star)$ is a magma and $(T, \circ)$ is a semigroup. Additionally, suppose $f$: $S\rightarrow T$ is a morphism of magmas. Show that if $f$ is injective, the $(S, \star)$ is also a semigroup. 
\end{flushleft}

\begin{flushleft}
\textbf{Solution} Assume $f$ is injective. Then for any $x_1,x_2\in S$, if $f(x_1)=f(x_2)$, then $x_1=x_2$. Let $a,b,c\in S$. Now let $\rho=(a\star b)\star c$ and $\sigma=a\star(b\star c)$. Consider the image of $\rho$ under $f$:\linebreak
\hfil
\hfil
\hfil
\hfil
\end{flushleft}

\begin{flushright}
\hfil$f(\rho)=f((a\star b)\star c)=f(a\star b)\circ f(c)=(f(a)\circ f(b))\circ f(c)$\hfil\hfil(1)\linebreak
\end{flushright}

\begin{flushleft}
Now let us look at the image of $\sigma$ under $f$:\linebreak
\hfil
\hfil
\hfil
\hfil
\end{flushleft}

\begin{flushright}
\hfil$f(\sigma)=f(a\star (b\star c))=f(a)\circ f(b\star c)=f(a)\circ (f(b)\circ f(c))$\hfil\hfil(2)\linebreak
\end{flushright}

\begin{flushleft}
Since $(T, \circ)$ is a semigroup, then the operator $\circ$ is associative. We can then rewrite (1) as:\linebreak
\hfil
\hfil
\hfil
\hfil
\end{flushleft}

\begin{flushright}
\hfil$f(\rho)=(f(a)\circ f(b))\circ f(c)=f(a)\circ (f(b)\circ f(c))$\hfil\hfil(3)\linebreak
\end{flushright}

\begin{flushleft}
We can see (2) is equal to (3), thus $f(\rho)=f(\sigma)$. Then by assumption, $\rho=\sigma$. Hence,\linebreak
\hfil
\hfil
\hfil
\hfil
\end{flushleft}

\begin{flushright}
$(a\star b)\star c=a\star (b\star c)$.\hfil\hfil
\linebreak
\end{flushright}

\begin{flushleft}
Thus, if $f$ is injective then $\star$ is associative. $(S, \star)$ is then a semigroup, by definition.\linebreak
\hfil
\hfil
\hfil
\linebreak
\end{flushleft}

\begin{flushleft}
\hfil\textbf{Exercise 1.5.} Find an example of the following structures. In each case, show\linebreak
that the structure satisfies the axioms of that structure, and also provide the axiom\linebreak (or axioms) that fail for the more complex structure.\vspace*{3mm}\linebreak
\hspace*{10mm}(a) A set $S$ and operation $\star$ that is not a magma.\linebreak
\hspace*{10mm}(b) A magma that is not a semigroup.\linebreak
\hspace*{10mm}(c) A semigroup that is not a monoid.\linebreak
\hspace*{10mm}(d) A monoid that is not a group.\linebreak
\end{flushleft}

\begin{flushleft}
\textbf{Solution}\vspace*{3mm}\linebreak
\hspace*{10mm}(a) Since a magma is a set equipped with a binary operation in which it is\linebreak
\hspace*{16.7mm}closed under, then an example of something that fails the closure axiom\linebreak
\hspace*{16.7mm}would be the integers with division, or $(\mathbb{Z}, \div)$. Consider the two integers\linebreak
\hspace*{16.7mm}$2$ and $3$. It is clear that $\div(2, 3)\notin\mathbb{Z}$. Thus, this set is not closed under\linebreak
\hspace*{16.7mm}this operation, as demonstrated by counterexample.\vspace*{3mm}\linebreak

\hspace*{10mm}(b) A semigroup requires its binary operation be associative. We can choose\linebreak
\hspace*{16.7mm}our set to be the integers and our operator to be substraction. Consider, for\linebreak
\hspace*{16.7mm}example, the equation $(1-1)-1=1-(1-1)$. Simplifying yields $-1=1$. 
\hspace*{16.7mm}Since this is not true, then we can conclude associativity does not hold on\linebreak
 \hspace*{16.7mm}this magma. Thus, $(\mathbb{Z}, -)$ is a magma that is not a semigroup.\vspace*{3mm}\linebreak

\hspace*{10mm}(c) A monoid is a semigroup with an identity element. A trivial, but correct\linebreak
\hspace*{16.7mm}example would be the set $\mathbb{N}$ and the operator $+$. This is an example\linebreak
\hspace*{16.7mm}of a semigroup that is not a monoid because $0$ is the additive identity on\linebreak
\hspace*{16.7mm}the positive integers. Restricting our set to $\mathbb{N}$ removes $0$, and this is what    \hspace*{16.7mm}makes this set fail the identity axiom needed to be a monoid. Thus, $(\mathbb{N}, +)$ is \hspace*{16.7mm}an example of a semigroup that is not a monoid.\vspace*{3mm}\linebreak

\hspace*{10mm}(d) A monoid is a set equipped with an associative operation in which it is\linebreak \hspace*{16.7mm}closed under, and it also contains an identity element with respect to the\linebreak
\hspace*{16.7mm}operation. A group is a monoid that has an inverse for every element in\linebreak
\hspace*{16.7mm}the set. Consider the set of all 2x2 square matrices $\mathbb{M}_2(\mathbb{R})$, with our binary\linebreak
\hspace*{16.7mm}associative operation being multiplication. To prove that this is not a group\linebreak
\hspace*{16.7mm}we need only find an element in our set with no inverse. \vspace*{3mm}\linebreak

\begin{itemize}

\item Let $A=\left[ \begin{array}{cc} a & b \\ c & d \end{array} \right]$ be an element of $\mathbb{M}_2(\mathbb{R})$, such that $ad-bc=0$. Then it is clear\vspace*{2mm} that $A$ does not have an inverse. Thus, $\mathbb{M}_2(\mathbb{R})$ is not a group.\linebreak \par

\end{itemize}

\end{flushleft}


\begin{flushleft}
\hspace*{8mm}\textbf{Exercise 1.13} Is the function $\phi$ given in Exercise 1.3 a group homomorphism? How about the same function from $(\mathbb{R}, +)$ to $(\mathbb{R}\backslash \{ 0\}, \cdot)$?\linebreak

\end{flushleft}

\begin{flushleft}
\textbf{Solution} To answer these two questions we must first determine if $(\mathbb{R}, +)$, $(\mathbb{R}, \cdot)$, and $(\mathbb{R}\backslash \{ 0\}, \cdot)$ are groups. Let us first look at $(\mathbb{R}, +)$:\linebreak

\begin{itemize}
\item \textbf{Closure} Let $a$,$b\in \mathbb{R}$ be any two real numbers. Since $a+b\in \mathbb{R}$ is also a real number, we can conclude the closure axiom is satisfied.
\item \textbf{Associativity} Let $a$,$b$,$c\in \mathbb{R}$ be any three real numbers. Since $(a+b)+c=a+(b+c)$, we can conclude that this axiom is satisfied.
\item \textbf{Identity} Let $a,e\in \mathbb{R}$ such that $e=0$, then it is clear that $a+e=a+0=a=0+a=e+a$. Thus, this axiom is satisfied.
\item \textbf{Inverse} Let $a,\sigma \in \mathbb{R}$ such that $\sigma=-a$. Then $a+\sigma=a+(-a)=0$. Thus, every element has an inverse. 
\end{itemize}
\vspace*{5mm}

With all four axioms being satisfied, this shows that $(\mathbb{R}, +)$ is a group. Now let us look at $(\mathbb{R}, \cdot)$:\linebreak

\begin{itemize}
\item \textbf{Closure} Let $a,b\in \mathbb{R}$ be any two real numbers. Since the product of reals is itself a real, then $a\cdot b\in \mathbb{R}$. Thus, this set is closed with respect to multiplication.
\item \textbf{Associativity} Let $a,b,c\in \mathbb{R}$ be any three real numbers. Since it is true that $(a\cdot b)\cdot c=a\cdot (b\cdot c)$, then the operator is associative.
\item \textbf{Identity} Let $a,e\in \mathbb{R}$ be two real numbers such that $e=1$. Then $a\cdot e=a\cdot 1=a=1\cdot a=e\cdot a$. Thus, this set has an identity with $e=1$. 
\item \textbf{Inverse} This set contains an inverse for every element except for $0$. This axiom is thereby unsatisfied with respect to multiplication.
\end {itemize}
\vspace*{5mm}

From the failure of the axiom regarding the existence of inverses, we can conclude $(\mathbb{R}, \cdot)$ is not a group. Thus, the function $\phi$ is not a group homomorphism.\linebreak
\par
We now focus our attention on whether or not $\phi$ constitutes a group homomorphism if we remove $0$ from the codomain and have $\phi$ defined instead as $\phi: \mathbb{R} \rightarrow \mathbb{R} \backslash \{ 0\}$. However, based on the fact that the inverse axiom was the only axiom that failed, and its failure was because the element $0$ did not have an inverse, then the removal of $0$ will then satisfy the inverse axiom and $\phi$ becomes a group homomorphism as a result.\linebreak
\par

\begin{flushleft}
\hspace*{8mm} \textbf{Exercise 1.15} Recall Examples 1.6, 1.10, 1.13, 1.17, and 1.20.  (Note: You\linebreak may use the result of Exercise 1.14 whenever you'd like.)\linebreak
\end{flushleft}

\begin{flushleft}

\hspace*{10mm}(a) Using cycle notation, what are all of the distinct elements of $S_3$?\linebreak
\hspace*{16.7mm}(Hint: $|S_3|=6$ and we have already listed three. Note: You only need to\linebreak
\hspace*{16.7mm}give one expression for an element, as there are often more than one way\linebreak
\hspace*{16.7mm}to write such a permutation (for example, $(12)=(21)$.)\linebreak

\hspace*{10mm}(b) Argue that $S_3$ is closed under the operation of composition of permuta-\linebreak
\hspace*{16.7mm}tions.\linebreak

\hspace*{10mm}(c) Show that the operation of composition of permutations is associative on\linebreak
\hspace*{16.7mm}$S_3$.\linebreak

\hspace*{10mm}(d) Show that $(1)$ is an identity element for $S_3$.\linebreak

\hspace*{10mm}(e) For each element in $S_3$, calculate its inverse element.\linebreak

\hspace*{10mm}(f) Use the previous part to conclude that $S_3$ under the operation of compo-\linebreak
\hspace*{16.7mm}sition is a group.\linebreak

\hspace*{10mm}(g) Is $S_3$ an abelian group? Justify your answer.\linebreak

\end{flushleft}


\begin{flushleft}

\textbf{Solution}\vspace*{3mm}\linebreak
\hspace*{10mm}(a)\vspace*{8mm}\linebreak
\hspace*{28mm}$\epsilon=\left (\begin{array}{ccc} 1 & 2 & 3 \\ 1 & 2 & 3 \end{array} \right)$, 
$\alpha=\left (\begin{array}{ccc} 1 & 2 & 3 \\ 2 & 3 & 1 \end{array} \right)$, 
$\beta=\left (\begin{array}{ccc} 1 & 2 & 3 \\ 3 & 1 & 2 \end{array} \right)$,\vspace*{10mm}\linebreak
\hspace*{27.5mm}$\gamma=\left (\begin{array}{ccc} 1 & 2 & 3 \\ 1 & 3 & 2 \end{array} \right)$, 
\hspace*{1mm}$\delta=\left (\begin{array}{ccc} 1 & 2 & 3 \\ 3 & 2 & 1 \end{array} \right)$, 
\hspace*{1mm}$\epsilon=\left (\begin{array}{ccc} 1 & 2 & 3 \\ 2 & 1 & 3 \end{array} \right)$.\vspace*{14mm}\linebreak


\end{flushleft}

\begin{flushleft}

\hspace*{10mm}(b) To argue that $S_3$ is closed under composition of permutations we first\linebreak
\hspace*{16.7mm}note two things: the set is finite, and the permutations are bijective maps. \linebreak
\hspace*{16.7mm}This means that the maps are injective, and the entries in the second row \linebreak
\hspace*{16.7mm}are thus all distinct. There are $27$ possible maps from the set $\{1,2,3\}$ to\linebreak
\hspace*{16.7mm}itself, but only $6$ of them are bijections (e.g. permutations), so regardless\linebreak
\hspace*{16.7mm}of the composition, the result can only be another permutation, otherwise the\linebreak
\hspace*{16.7mm}mappings would not be bijections.\vspace*{3mm}\linebreak

\hspace*{16.7mm}The following table will aid in answering the remaining parts of this problem.\vspace*{5mm}
\linebreak

\end{flushleft}



\begin{table}[htb]
\centering
\begin{tabular}{c|cccccc}
\large$\circ$ & \large$\epsilon$ & \large$\alpha$ & \large$\beta$ & \large$\gamma$ & \large$\delta$ & \large$\phi$ \\
\hline
\large$\epsilon$ & \large$\epsilon$ & \large$\alpha$ & \large$\beta$ & \large$\gamma$ & \large$\delta$ & \large$\phi$ \\
$\alpha$ & $\alpha$ & $\beta$ & $\epsilon$ & $\phi$ & $\gamma$ & $\delta$ \\
\large$\beta$ & \large$\beta$ & \large$\epsilon$ & \large$\alpha$ & \large$\delta$ & \large$\phi$ & \large$\gamma$ \\
\large$\gamma$ & \large$\gamma$ & \large$\delta$ & \large$\phi$ & \large$\epsilon$ & \large$\alpha$ & \large$\beta$ \\
\large$\delta$ & \large$\delta$ & \large$\phi$ & \large$\gamma$ & \large$\beta$ & \large$\epsilon$ & \large$\alpha$ \\
\large$\phi$ & \large$\phi$ & \large$\gamma$ & \large$\delta$ & \large$\alpha$ & \large$\beta$ & \large$\epsilon$ \\
\end{tabular}
\end{table}



\begin{flushleft}

\vspace*{5mm}\hspace*{10mm}(c) Let $f,g,h\in S_3$ be any three permutations. Then by definition, for all 
\hspace*{16.7mm}$x\in \{1,2,3\}$,\vspace*{3mm}\linebreak


\hspace*{44.5mm}$((h\circ g)\circ f)(x)$\hspace*{3mm} $=$ $\hspace*{3mm} (h\circ g)(f(x))$\vspace*{5mm}\linebreak
\hspace*{76mm}$=$ \hspace*{3mm}$h(g(f(x)))$\vspace*{5mm}\linebreak
\hspace*{76mm}$=$ \hspace*{3mm}$h(g\circ f(x))$\vspace*{5mm}\linebreak
\hspace*{76mm}$=$ \hspace*{3mm}$(h\circ (g\circ f))(x).$\vspace*{3mm}\linebreak
 

\end{flushleft}

\begin{flushleft}

\hspace*{16.7mm}It follows that the maps $(h\circ g)\circ f$ and $h\circ (g\circ f)$ are equal. Therefore,\linebreak
\hspace*{16.7mm}the operation of composition of permutation is associative on $S_3$.\vspace*{3mm}\linebreak

\end{flushleft}

\begin{flushleft}

\hspace*{10mm}(d) The element $(1)$ is the mapping equivalent to $\epsilon$ listed above. Let $f \in S_3$,\linebreak
\hspace*{16.7mm}be any permutation. Then by the definition of $(1)$ (e.g. $\epsilon$), which tells us
\hspace*{16.7mm}that this element takes each number in the set $\{1,2,3\}$ and maps it to itself.
\hspace*{16.7mm}We now have the following: $f\circ \epsilon= f = \epsilon\circ f$.\vspace*{25mm}\linebreak

\hspace*{10mm}(e) With the aid of the previous Cayley table, we can see the inverses are those
\hspace*{16.7mm}maps which when applied, result in $\epsilon$. The list of inverses is as follows:\vspace*{4mm}\linebreak

\end{flushleft}

\centering
$\large\epsilon^{-1}=\large\epsilon$, \hspace*{3mm} $\large\alpha^{-1}=\large\beta$, \hspace*{3mm} 
$\large\beta^{-1}=\large\alpha$\vspace*{3mm}\linebreak
$\large\gamma^{-1}=\large\gamma$, \hspace*{3mm} $\large\delta^{-1}=\large\delta$, \hspace*{3mm}
$\large\phi^{-1}=\large\phi$.\vspace{4mm}\linebreak


\begin{flushleft}

\hspace*{10mm}(f) As we can see from parts (b), (c), (d), and (e), the set $S_3$ with the operation of\linebreak
\hspace*{16.7mm}composition of permutations is closed, associative, contains an identity, and 
\hspace*{16.7mm}each element has an inverse. $(S_3, \circ)$ is therefore a group.


\end{flushleft}

\begin{flushleft}

\hspace*{10mm}(g) $(S_3, \circ)$ is not an abelian group and this will be demonstrated by counter 
\hspace*{16.7mm}example: Consider elements $\beta, \gamma \in S_3$. We can see from the table that the\linebreak
\hspace*{16.7mm}composition $\beta \circ \gamma = \delta$, but the composition $\gamma \circ \beta = \phi$. Thus, 
$\beta \circ \gamma \neq \gamma \circ \beta$.\linebreak
\hspace*{16.7mm}Therefore, $(S_3, \circ)$ is not abelian.


\end{flushleft}



\end{flushleft}










\end{document}
