\documentclass[12pt, a4paper]{article}
\usepackage[margin=1in]{geometry}
\usepackage[latin1]{inputenc}
\usepackage{titlesec}
\usepackage{amsmath}
\usepackage{amsthm}
\usepackage{amsfonts}
\usepackage{amssymb}
\usepackage{array}
\usepackage{booktabs}
\usepackage{ragged2e}
\usepackage{enumerate}
\usepackage{enumitem}
\usepackage{commath}
\usepackage{colonequals}
\renewcommand{\baselinestretch}{1.1}
\usepackage[mathscr]{euscript}
\let\euscr\mathscr \let\mathscr\relax
\usepackage[scr]{rsfso}
\newcommand{\powerset}{\raisebox{.15\baselineskip}{\Large\ensuremath{\wp}}}
\usepackage{longtable}
\usepackage{multirow}
\newcolumntype{C}{>$c<$}

\pagestyle{fancy}
\renewcommand{\headrulewidth}{0pt}

\usepackage{enumitem}
\usepackage{tikz}
\usepackage{commath}
\usepackage{colonequals}
\usepackage{bm}
\usepackage{tikz-cd}
\renewcommand{\baselinestretch}{1.1}
\usepackage[mathscr]{euscript}
\let\euscr\mathscr \let\mathscr\relax
\usepackage[scr]{rsfso}



\newcommand*{\logeq}{\ratio\Leftrightarrow}

\setlist[description]{leftmargin=10mm,labelindent=10mm}





\begin{document}


\section*{Discussion}


The first group isomorphism theorem states that if $\phi\colon G\rightarrow F$ is a group homomorphism from groups $G$ to $F$, then $G/$ker$(\phi)\cong$ im$(\phi)$. The previous consequence leads us to see that, by definition:\par

\vspace{4mm}

Let $\phi\colon G\rightarrow F$ is a group homomorphism, then in order for $G/$ker$(\phi)\cong$ im$(\phi)$ to be true, there must exist a map, call it $\eta$, from $G/$ker$(\phi)$ to im$(\phi)$ such that this map is a bijective homomorphism. This means that $\par

\vspace{4mm}

1.\hspace{2mm} $G/$ker$(\phi)\cong$ im$(\phi)$\par
\vspace{2mm}
2.\hspace{3mm}$\logeq\exists\eta\in(G/$ker$(\phi)\times$ im$(\phi))\colon$\par
\vspace{2mm}
3.\hspace{10mm}$\forall g$ker$(\phi)\in G/$ker$(\phi)\colon$\par
\vspace{2mm}
4...............................$(g_1$ker$(\phi)\in G/$ker$(\phi))\wedge(g_2$ker$(\phi)\in G/$ker$(\phi))\colon$\par
\vspace{2mm}
5....................................... $[g_1$ker$(\phi)=g_2$ker$(\phi)]\Rightarrow[\eta(g_1$ker$(\phi))=\eta(g_2$ker$(\phi))]$\par 
\vspace{2mm}
6.........................................$[\eta(g_1$\ker$(\phi))=\eta(g_2$ker$(\phi)]\Rightarrow[g_1$ker$(\phi)=g_2$ker$(\phi)]$\par
\vspace{2mm}
7.........................................$[g$ker$(\phi)\in G/$ker$(\phi)]\Rightarrow[\exists f\in$ im$(\phi)\colon\phi(g$ker$(\phi)=f]$\par
\vspace{2mm}
8.........................................$[\eta(g_1$ker$(\phi)g_2$ker$(\phi))=\eta(g_1$ker$(\phi))\eta(g_1$ker$(\phi))]$

\begin{flushleft}

What the above tells us is that the two groups are isomorphic $G/$ker$(\phi)\cong$ im$(\phi)$ if and only if there exists a map $\eta\colon G/$ker$(\phi)\rightarrow$ im$(\phi)$ such that $\eta$ is a bijective homomorphism. Looking at the domain of of $\eta$, we notice that im$(\phi)\subseteq F$. Since we are given a map $\phi\colon G\rightarrow F$, that maps us to the domain of $\eta$, then we will need for there to be a map $\psi\colon G/$ker$(\phi)\rightarrow G$, such that $\phi\circ\psi\colon G/$ker$(\phi)\rightarrow F$ is a map equivalent to $\eta$.

\end{flushleft}

Let us consider some relation $\psi\in(G/$ker$(\phi)\times G)$. Suppose we choose to define this relation as follows:\par

\vspace{4mm}

1.\hspace{2mm}$(g$ker$(\phi),g)\in\psi\logeq$\par
\vspace{2mm}
2...............$\forall g$ker$(\phi),g'$ker$(\phi)\in G/$ker$(\phi)\colon$\par
\vspace{2mm}
3...............................$[(g$ker$(\phi)=g'$ker$(\phi))\Rightarrow(g=g')]$\par



\vspace{4mm}

Let us now try to prove line 3.\par

\vspace{4mm}

\hspace{2mm} $g$ker$(\phi)=g'$ker$(\phi)$\par
\vspace{2mm}
\hspace{10mm} $\Rightarrow (gg'^{-1})$ker$(\phi)=(e_G)$ker$(\phi)$\par
\vspace{2mm}
\hspace{10mm} $\Rightarrow\exists\hat{g}\in$ker$(\phi)\colon[\hat{g}(gg'^{-1})\hat{g}^{-1}=e_G]$\par
\vspace{2mm}
\hspace{10mm} $\Rightarrow \hat{g}(gg'^{-1})\hat{g}^{-1}=e_G$\par
\vspace{2mm}
\hspace{10mm} $\Rightarrow \hat{g}(gg'^{-1})=\hat{g}$\par
\vspace{2mm}
\hspace{10mm} $\Rightarrow gg'^{-1}=e_G$\par
\vspace{2mm}
\hspace{2mm} $\therefore g=g'$\par

\vspace{4mm}

\textbf{THE ABOVE PROOF IS INCORRECT}






\end{document}