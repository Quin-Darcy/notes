\documentclass[12pt, a4paper]{article}
\usepackage[margin=1in]{geometry}
\usepackage[latin1]{inputenc}
\usepackage{titlesec}
\usepackage{amsmath}
\usepackage{amsthm}
\usepackage{amsfonts}
\usepackage{amssymb}
\usepackage{array}
\usepackage{booktabs}
\usepackage{ragged2e}
\usepackage{enumerate}
\usepackage{enumitem}
\usepackage{commath}
\usepackage{colonequals}
\renewcommand{\baselinestretch}{1.1}
\usepackage[mathscr]{euscript}
\let\euscr\mathscr \let\mathscr\relax
\usepackage[scr]{rsfso}
\newcommand{\powerset}{\raisebox{.15\baselineskip}{\Large\ensuremath{\wp}}}
\usepackage{longtable}
\usepackage{multirow}
\newcolumntype{C}{>$c<$}

\pagestyle{fancy}
\renewcommand{\headrulewidth}{0pt}

\usepackage{enumitem}
\usepackage{tikz}
\usepackage{commath}
\usepackage{colonequals}
\usepackage{bm}
\usepackage{tikz-cd}
\renewcommand{\baselinestretch}{1.1}
\usepackage[mathscr]{euscript}
\let\euscr\mathscr \let\mathscr\relax
\usepackage[scr]{rsfso}



\newcommand*{\logeq}{\ratio\Leftrightarrow}

\setlist[description]{leftmargin=10mm,labelindent=10mm}


%\section*{\centerline{\underline{First Isomorphism Theorem}}}


\begin{document}

\vspace{10mm}

\begin{flushleft}

\textbf{\large{\underline{RELATIONS}}}

\vspace{8mm}

\blacksquare \textbf{ DEFINITION}

\end{flushleft}

A set $R$ is an $(\mathbf{n}$\textbf{-ary}$)$ \textbf{relation} if there exists $A_0,A_1,\dots,A_{n-1}$ such that\par

\vspace{4mm}

\centerline{$R\subseteq A_0\times A_1\times\cdots A_{n-1}$.}

\vspace{4mm}

In particular, $R$ is a \textbf{unary} relation if $n=1$ and a \textbf{binary} relation if $n=2$. If\par$R\subseteq A\times A$ for some set $A$, then $R$ is a \textbf{relation on} $A$ and we write $(A,R)$.

\begin{flushleft}

\blacksquare \textbf{ EXAMPLE}

\end{flushleft}

For any set $A$, define\par

\vspace{4mm}

\centerline{$I_A=\{(a,a)\colon a\in A\}$.}

\vspace{4mm}

Call this set the \textbf{identity} on $A$. 

\begin{flushleft}

\blacksquare \textbf{ DEFINITION}

\end{flushleft}

Let $R\subseteq A\times B$. The \textbf{domain} of $R$ is the set\par

\vspace{4mm}

\centerline{dom$(R)=\{x\in A\colon\exists y(y\in B\wedge(x,y)\in R)\}$,}

\vspace{4mm}

and the \textbf{range} of $R$ is the set\par

\vspace{4mm}

\centerline{ran$(R)=\{y\in B\colon\exists x(x\in A\wedge(x,y)\in R)\}$.}

\begin{flushleft}

\textbf{COMPOSITION}

\vspace{4mm}

Given the relations $R$ and $S$, let us define a new relation. Suppose $(a,b)\in S$ and $(b,c)\in R$. Therefore, $a$ is related to $c$ through $c$. The new relation will contain the ordered pair $(a,c)$ to represent this relationship.

\vspace{4mm}

\blacksquare \textbf{ DEFINITION}

\end{flushleft}

Let $R\subseteq A\times B$ and $S\subseteq B\times C$. The \textbf{composition} of $S$ and $R$ is the subset of $A\times C$ defined as\par

\vspace{4mm}

\centerline{$S\circ R=\{(x,z)\colon\exists y(y\in B\wedge(x,y)\in R\wedge(y,z)\in S\}$.}

\newpage

\begin{flushleft}

\blacksquare \textbf{ EXAMPLE}

\end{flushleft}

Define\par

\vspace{4mm}

\centerline{$R=\{(x,y)\in\mathbb{R}^2\colon x^2+y^2=1\}$}

\vspace{4mm}

and\par

\vspace{4mm}

\centerline{$S=\{(x,y)\in\mathbb{R}^2\colon y=x+1\}$.}

\vspace{4mm}

Notice that $R$ is the unit circle and $S$ is the line with slope of 1 and $y$-intercept of\par$(0,1)$. Let us find $R\circ S$.\par

\vspace{4mm}

\centerline{$R\circ S=\{(x,z)\in\mathbb{R}^2\colon\exists y(y\in\mathbb{R}\wedge(x,y)\in S\wedge(y,z)\in\mathbb{R}\}$}

\vspace{2mm}

\hspace{31.5mm} $=\{(x,z)\in\mathbb{R}^2\colon\exists y(y\in\mathbb{R}\wedge y=x+1\wedge y^2+z^2=1)\}$

\vspace{2mm}

\hspace{31.5mm} $=\{(x,z)\in\mathbb{R}^2\colon(x+1)^2+z^2=1\}$.

\begin{flushleft}

\blacksquare \textbf{ EXAMPLE}

\end{flushleft}

\textbf{\textsc{PROPOSITION }}\par 
\hspace{5mm}If $R\subseteq A\times B$, then $R\circ I_A=R$ and $I_B\circ R=R$.

\vspace{4mm}

\textbf{\textsc{PROOF}}\par

\hspace{5mm}Assume $R\subseteq A\times B$. Then,\par

\vspace{4mm}

\hspace{10mm}$(x,z)\in R\circ I_A$\par
\vspace{2mm}
\hspace{20mm}$\Leftrightarrow\exists y(y\in A\wedge(x,y)\in I_A\wedge(y,z)\in R)$\par
\vspace{2mm}
\hspace{20mm}$\Leftrightarrow\exists y(y\in A\wedge y=x\wedge(y,z)\in R)$\par
\vspace{2mm}
\hspace{20mm}$\Leftrightarrow\exists x(x\in A\wedge(x,z)\in R)$\par
\vspace{2mm}
\hspace{20mm}$\Leftrightarrow(x,z)\in R$. \blacksquare

\vspace{4mm}

\hspace{10mm}$(x,z)\in I_B\circ R$\par
\vspace{2mm}
\hspace{20mm}$\Leftrightarrow\exists y(y\in B\wedge(x,y)\in R\wedge(y,z)\in I_B)$\par
\vspace{2mm}
\hspace{20mm}$\Leftrightarrow\exists y(y\in B\wedge(x,y)\in R\wedge y=z)$\par
\vspace{2mm}
\hspace{20mm}$\Leftrightarrow\exists z(z\in B\wedge(x,z)\in R)$\par
\vspace{2mm}
\hspace{20mm}$\Leftrightarrow (x,z)\in R$. \blacksquare


\begin{flushleft}

\textbf{INVERSES}

\vspace{4mm}

Let $R\subseteq A\times B$. We know that $R\circ I_A=R$ and $I_B\circ R=R$. If we want $I_A=I_B$, we need $A=B$ so that $R$ is a relation on $A$. Then,\par

\vspace{4mm}

\centerline{$R\circ I_A=I_A\circ R=R$.}

\end{flushleft}

\newpage

\begin{flushleft}

For example, if we define $R=\{(2,4),(1,3),(2,5)\}$ and view it as a relation on $\mathbb{Z}$, then $R$ composed on either side by $I_{\mathbb{Z}}$ yields $R$. To illustrate, consider the ordered pair $(1,3)$. It is an element of $R\circ I_{\mathbb{Z}}$ because\par

\vspace{4mm}

\centerline{$(1,1)\in I_{\mathbb{Z}}\wedge(1,3)\in R$,}

\vspace{4mm}

and it is also an element of $I_{\mathbb{Z}}\circ R$ because\par

\vspace{4mm}

\centerline{$(1,3)\in R\wedge(3,3)\in I_{\mathbb{Z}}$.}

\vspace{4mm}

Notice that not every identity relation will have this property. Using the same definition of $R$ as above,\par

\vspace{4mm}

\centerline{$R\circ I_{\{1,2\}}=R$,}

\vspace{4mm}

but\par

\vspace{4mm}

\centerline{$I_{\{1,2\}}\circ R=\varnothing$.}

\end{flushleft}

\begin{flushleft}

\blacksquare \textbf{ DEFINITION}

\end{flushleft}

Let $R$ be a binary relation. The \textbf{inverse} of $R$ is the set\par

\vspace{4mm}

\centerline{$R^{-1}=\{(y,x)\colon(x,y)\in R\}$.}

\begin{flushleft}

\blacksquare \textbf{ EXAMPLE}

\end{flushleft}

We now check whether $R\circ R^{-1}=R^{-1}\circ R=I_A$ for any relation on $A$. Consider $R=\{(2,1),(4,3)\}$, which is a relation on $\{1,2,3,4\}$. Then,\par

\begin{flushleft}

\centerline{$R^{-1}=\{(1,2),(3,4)\}$,}

\vspace{4mm}

and we see that composing does not yield the identity on $\{1,2,3,4\}$ because\par

\vspace{4mm}

\centerline{$R\circ R^{-1}=\{(1,1),(3,3)\}=I_{\{1,3\}}$}

\vspace{4mm}

and\par

\vspace{4mm}

\centerline{$R^{-1}\circ R=\{(2,2),(4,4)\}=I_{\{2,4\}}$.}

\vspace{4mm}

The situation is worse when we define $S=\{(2,1),(2,3),(4,3)\}$. In this case, we have that\par

\vspace{4mm}

\centerline{$S^{-1}=\{(1,2),(3,2),(3,4)\}$}

\vspace{4mm}

but\par

\vspace{4mm}

\centerline{$S\circ S^{-1}=\{(1,1),(1,3),(3,1),(3,3)\}$}

\end{flushleft}

\newpage

\begin{flushleft}

and\par

\vspace{4mm}

\centerline{$S^{-1}\circ S=\{(2,2),(2,4),(4,4)\}$.}

\vspace{4mm}

Neither of these compositions leads to an identity, but at least we have that\par

\vspace{4mm}

\centerline{$\{(1,1),(3,3)\}\subseteq S\circ S^{-1}$}

\vspace{4mm}

and\par

\vspace{4mm}

\centerline{$\{(2,2),(4,4)\}\subseteq S^{-1}\circ S$.}

\vspace{4mm}

This can be generalized.


\end{flushleft}

\begin{flushleft}

\blacksquare \textbf{ THEOREM}

\end{flushleft}

$I_{ran(R)}\subseteq R\circ R^{-1}$ and $I_{dom(R)}\subseteq R^{-1}\circ R$ for any binary relation.










\newpage

$\phi\circ\psi\colon G/$ker$(\phi)\rightarrow$ im$(\phi)$, where $\psi\colon G/$ker$(\phi)\rightarrow G$ defined by $g$ker$(\phi)\mapsto g$. \par

\vspace{4mm}

Let $(g_1$ker$(\phi)),(g_2$ker$(\phi))\in G/$ker$(\phi)$.

\vspace{6mm}

\begin{center}
\begin{tabular}{ |p{3mm}||p{26mm}||p{8.5cm}| }
 \hline
 \hspace{1mm} &\textbf{Connectives} & \hspace{30mm}\textbf{Predicates}\\
 \hline
 
1. & \hspace{12mm}A & \hspace{3mm}$\forall h\in$ker$(\phi)(h=e_G)$\\
\hline
 
2. & \hspace{12mm}A & \hspace{3mm}$g_1$ker$(\phi)=g_2$ker$(\phi)$\\
\hline
 
3. & \hspace{11.5mm}$\Leftrightarrow$ & \hspace{3mm}$g_1$ker$(\phi)\subseteq g_2$ker$(\phi)\wedge g_2$ker$(\phi)\subseteq g_1$ker$(\phi)$\\
 \hline
 
4. & \hspace{11.5mm}$\Leftrightarrow$ & \hspace{3mm}$g_1$ker$(\phi)\subseteq g_2$ker$(\phi)$\\
 \hline
 
5. & \hspace{11.5mm}$\Leftrightarrow$ & \hspace{3mm}$g_1\in g_1$ker$(\phi)$\\
 \hline
 
6. & \hspace{11.5mm}$\Leftrightarrow$ & \hspace{3mm}$\exists h(h\in$ker$(\phi)\wedge(g_1=g_2h))$\\
 \hline
 
7. & \hspace{11.5mm}$\Leftrightarrow$ & \hspace{3mm}$\exists h(h\in$ker$(\phi)\wedge(g_1=g_2e_G))$\\
 \hline
 
8. & \hspace{12mm}$\therefore$ & \hspace{3mm}$g_1=g_2$ \\
\hline
 
 \hline
\end{tabular}
\end{center}






\end{document}