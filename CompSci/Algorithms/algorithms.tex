\documentclass[12pt]{article}
\usepackage[margin=1in]{geometry}
\usepackage{graphicx}
\usepackage{amsmath}
\usepackage{amsthm}
\usepackage{amsfonts}
\usepackage{amssymb}
\usepackage{array}
\usepackage{enumerate}
\usepackage{slashed}
\usepackage{colonequals}
\usepackage{fancyhdr}
\usepackage{import}
\usepackage{xifthen}
\usepackage{pdfpages}
\usepackage{transparent}
\usepackage{enumitem}

\newcommand{\incfig}[1]{%
    \def\svgwidth{\columnwidth}
    \import{/home/arbegla/figures/}{#1.pdf_tex}
}

\pagestyle{fancy}
\fancyhf{}
\rhead{}
\lhead{}
\rfoot{\thepage}
\setlength{\headheight}{10pt}

\newtheorem{theorem}{Theorem}[section]
\newtheorem{corollary}{Corollary}[theorem]
\newtheorem{prop}{Proposition}[section]
\newtheorem{lemma}[theorem]{Lemma}
\theoremstyle{definition}
\newtheorem{definition}{Definition}[section]
\theoremstyle{definition}
\newtheorem{exmp}{Example}[section]

\newcommand{\abs}[1]{\lvert #1 \rvert}
\newcommand{\bigabs}[1]{\Bigl \lvert #1 \Bigr \rvert}
\newcommand{\bigbracket}[1]{\Bigl [ #1 \Bigr ]}
\newcommand{\bigparen}[1]{\Bigl ( #1 \Bigr )}
\newcommand{\ceil}[1]{\lceil #1 \rceil}
\newcommand{\bigceil}[1]{\Bigl \lceil #1 \Bigr \rceil}
\newcommand{\floor}[1]{\lfloor #1 \rfloor}
\newcommand{\bigfloor}[1]{\Bigl \lfloor #1 \Bigr \rfloor}
\newcommand{\norm}[1]{\| #1 \|}
\newcommand{\bignorm}[1]{\Bigl \| #1 \Bigr \| #1}
\newcommand{\inner}[1]{\langle #1 \rangle}
\newcommand{\set}[1]{{ #1 }}

    %\begin{figure}[htp!]
    %    \centering
    %      \incfig{S5}
    %      \caption{Group Action}
    %      \label{fig:action}
    %  \end{figure}\

\begin{document}
\title{Time Complexity and Algorithms}
\author{Quin Darcy}
\date{Jan, 26 2021}
\maketitle
    \section{Linked Lists}
    Non-empty \textit{lists} can be represented by \textit{two-cells}, in each
    of which the first cell contains a pointer to a list element and the second
    cell contains a \textit{pointer} to either the empty list or another
    two-cell.For instance, the list $[3, 1, 2, 4, 5]$.\par\hspace{4mm} More abstractly, a list can be constructed by two
    constructors:
        \begin{enumerate}
            \item \texttt{EmptyList}, which gives you the empty list, and 
            \item \texttt{MakeList(element, list)}, which puts an element at the
                top of an existing list.
        \end{enumerate}
    Using this, our last example list can be constructed as 
        \begin{equation*}
            \text{\texttt{MakeList(3, MakeList(1, MakeList(4, MakeList(2,
            MakeList(5, EmptyList)))))}}.
        \end{equation*}
    In order to access the elements of a linked list, it is important to note
    that they are constructed from the first elememt and the rest of the list. 
    \
\end{document}
