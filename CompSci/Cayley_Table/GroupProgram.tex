\documentclass[12pt]{article}
\usepackage[margin=1in]{geometry}
\usepackage{graphicx}
\usepackage{amsmath}
\usepackage{amsthm}
\usepackage{amsfonts}
\usepackage{amssymb}
\usepackage{array}
\usepackage{enumerate}
\usepackage{slashed}
\usepackage{colonequals}
\usepackage{fancyhdr}
\usepackage{import}
\usepackage{xifthen}
\usepackage{pdfpages}
\usepackage{transparent}

\newcommand{\incfig}[1]{%
    \def\svgwidth{\columnwidth}
    \import{/home/arbegla/figures/}{#1.pdf_tex}
}

\pagestyle{fancy}
\fancyhf{}
\rhead{}
\lhead{}
\rfoot{\thepage}
\setlength{\headheight}{10pt}

\newtheorem{theorem}{Theorem}[section]
\newtheorem{corollary}{Corollary}[theorem]
\newtheorem{prop}{Proposition}[section]
\newtheorem{lemma}[theorem]{Lemma}
\theoremstyle{definition}
\newtheorem{definition}{Definition}[section]
\theoremstyle{definition}
\newtheorem{exmp}{Example}[section]

\newcommand{\abs}[1]{\lvert #1 \rvert}
\newcommand{\bigabs}[1]{\Bigl \lvert #1 \Bigr \rvert}
\newcommand{\bigbracket}[1]{\Bigl [ #1 \Bigr ]}
\newcommand{\bigparen}[1]{\Bigl ( #1 \Bigr )}
\newcommand{\ceil}[1]{\lceil #1 \rceil}
\newcommand{\bigceil}[1]{\Bigl \lceil #1 \Bigr \rceil}
\newcommand{\floor}[1]{\lfloor #1 \rfloor}
\newcommand{\bigfloor}[1]{\Bigl \lfloor #1 \Bigr \rfloor}
\newcommand{\norm}[1]{\| #1 \|}
\newcommand{\bignorm}[1]{\Bigl \| #1 \Bigr \| #1}
\newcommand{\inner}[1]{\langle #1 \rangle}
\newcommand{\set}[1]{{ #1 }}

\begin{document}
\title{Visualizing $S_n$}
\author{Quin Darcy}
\date{23, Jan 2021}
\maketitle
    Below is a demonstration of the various images produced by the program
    \texttt{group.py}. This program starts prompting the user to enter a value
    \texttt{n}. From here the program recursively generates the set of all
    permutations of the numbers $1,\dots, n$. To do this \textit{Heap's
    Algorithm} is used. The number of permutations generated is $n!$. At this
    point, each permutation is assigned a unique color in the form $(R, G, B)$,
    where $0\leq R\leq 255$, $0\leq G\leq 255$, $0\leq B\leq 255$. This
    assignment is accomplished by  dividing $2\pi/n!$, and to
    the $i$th permutation $p_i$, we assign $i(2\pi/n!)$ for each $i$. Thus
    there is a one-to-one correspondance between the $n!$ permutations and $n!$
    angles $0\leq\theta_i<2\pi$. From here the set of angles is sent to
    a function which maps the given angle $\theta_i$ to a point $(R_i, G_i,
    B_i)$.\par\hspace{4mm} Equipped with a set of $n!$ $(R, G, B)$ tuples, we
    then compute the Cayley table for $S_n$. After this, the program then
    computes the set of all cyclic subgroups, it computes the alternating group
    $A_n$, and promts the user to choose an element from the group and computes
    the orbit and stabilizer of this element. After all this sets are computed,
    an image is constructed, pixel by pixel, referencing the set of $(R, G, B)$
    tuples. \par\hspace{4mm} Before showing the images, it is worth noting how
    the permutations are ordered. If $n=3$, we should expect the top row of the
    Cayley table to ordered as
      \begin{equation*}
        [1, 2, 3], [1, 3, 2], [2, 1, 3], [2, 3, 1], [3, 1, 2], [3, 2, 1].
      \end{equation*}
    One thing to notice is that the program keeps the permuations as is instead
    of converting them to cycle notation.\par\hspace{4mm} Below are the Cayley
    tables for $n=3,4,5, 6, 7$, and various other of the above mentioned
    sets. $n = 7$ sadly the limit for my current
    CPU.
      \begin{figure}[htp!]
        \centering
          \incfig{S3}
          \caption{Cayley Table for $S_3$}
          \label{fig:orb}
      \end{figure}
      \begin{figure}[htp!]
        \centering
          \incfig{S4}
          \caption{Cayley Table for $S_4$}
          \label{fig:orb}
      \end{figure}
      \begin{figure}[htp!]
        \centering
          \incfig{S5}
          \caption{Cayley Table for $S_5$}
          \label{fig:orb}
      \end{figure}
      \begin{figure}[htp!]
        \centering
          \incfig{S6}
          \caption{Cayley Table for $S_6$}
          \label{fig:orb}
      \end{figure}
      \begin{figure}[htp!]
        \centering
          \incfig{S7}
          \caption{Cayley Table for $S_7$}
          \label{fig:orb}
      \end{figure}
      \begin{figure}[htp!]
        \centering
          \incfig{S5O6}
          \caption{Cyclic Subgroup of $(2453)$ in $S_5$}
          \label{fig:orb}
      \end{figure}
      \begin{figure}[htp!]
        \centering
          \incfig{S4stab}
          \caption{Stabilizer of $(12)(34)$ in $S_4$}
          \label{fig:orb}
      \end{figure}
      \begin{figure}[htp!]
        \centering
          \incfig{S4orb}
          \caption{Orbit of $(12)(34)$ in $S_4$}
          \label{fig:orb}
      \end{figure}
      \begin{figure}[htp!]
        \centering
          \incfig{A3}
          \caption{Alternating group of $S_3$}
          \label{fig:orb}
      \end{figure}
      \begin{figure}[htp!]
        \centering
          \incfig{A4}
          \caption{Alternating group of $S_4$}
          \label{fig:orb}
      \end{figure}
      \begin{figure}[htp!]
        \centering
          \incfig{A5}
          \caption{Alternating group od $S_5$}
          \label{fig:orb}
      \end{figure}
\end{document}
