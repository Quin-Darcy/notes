\documentclass[12pt]{article}
\usepackage[margin=1in]{geometry}
\usepackage{graphicx}
\usepackage{amsmath}
\usepackage{amsthm}
\usepackage{amsfonts}
\usepackage{amssymb}
\usepackage{array}
\usepackage{enumerate}
\usepackage{fancyhdr}
\pagestyle{fancy}
\fancyhf{}
\rhead{Darcy}
\lhead{MATH 230B}
\rfoot{\thepage}
\setlength{\headheight}{10pt}

\newenvironment{solution}
{\renewcommand\qedsymbol{$\blacksquare$}\begin{proof}[Solution]}
{\end{proof}}
\newenvironment{psmall}{\left(\begin{smallmatrix}}{\end{smallmatrix}\right)}

\begin{document}
    \thispagestyle{empty}\hrule

    \begin{center}
        \vspace{.4cm} { \large MATH 230B}
    \end{center}
    {Name:\ Quin Darcy \hspace{\fill} Due Date: 02/18/2022   \\
    { Instructor:}\ Dr. Ricciotti \hspace{\fill} Assignment:
    Homework 02 \\ \hrule}

    \begin{enumerate}
        \item Let $f:[0, 1]\to\mathbb{R}$ be defined as 
            \begin{equation*}
                f(x)=\begin{cases}\frac{1}{n}&\text{if $x=\frac{1}{n}$ where
                    $n\in\mathbb{N}$} \\
                0 &\text{otherwise}\end{cases}.
            \end{equation*}
            Determine (with proof) whether $f\in\mathcal{R}[a, b]$ if so,
            compute $\int_{0}^{1}f$.
            \begin{solution}
                Let $\varepsilon>0$ and choose a partition $P=\{x_0, \dots,
                x_n\}$ such that the mesh $||P||<\varepsilon/n$. Since the
                irriationals are dense in $\mathbb{R}$, then for any $\Delta
                x_i$, there exists an irrational number $r$ such that
                $x_{i-1}<r<x_i$ and $f(r)=0$. This implies that $m_i=0$ for all
                $i\in\{1, \dots, n\}$ and so $L(f, P)=0$. Hence
                \begin{align}
                    U(f, P)-L(f, P) &= U(f, P) \\
                    &= \sum_{i=1}^{n}M_i\Delta x_i \\
                    &\leq \sum_{i=1}^{n}\Delta x_i \\
                    &< \sum_{i=1}^{n}\frac{\varepsilon}{n} \\
                    &= \varepsilon.
                \end{align}
                Note that the inequality from (2) to (3) holds since (3)
                represents the case where $M_i=1$ for all $i$, which in general
                is not true (consider $x_{i-1}=3/4$ and $x_i=5/6$, there is no
                $n\in\mathbb{N}$ such that $3/4\leq 1/n\leq 5/6$ and thus
                $M_i=0$). Therefore by Theorem 6.6, $f\in\mathcal{R}[0,
                1]$.\par\hspace{4mm} To compute $\int_{0}^{1}f$, we first note
                that the integrability of $f$ implies that
                \begin{equation*}
                    \sup\{L(f, P)|\;P\in\mathcal{P}([0,
                        1])=\int_{0}^{1}f=\inf\{U(f, P)|\;P\in\mathcal{P}([0,
                        1])\}.
                \end{equation*} 
                So computing $\sup\{L(f, P)|\; P\in\mathcal{P}([0, 1])\}$ is
                sufficient. As was noted earlier, for any partition $P$, $L(f,
                P)=0$ since the irrationals are dense in $\mathbb{R}$. This
                means the set has 0 as its only element. Thus the set is closed
                and bounded and so it attains its $\sup$ which is 0. Therefore
                \begin{equation*}
                    \int_{0}^{1}f = 0.
                \end{equation*}
            \end{solution}
        \item Let $f$ and $g$ be boundeds function on $[a, b]$. Assume that
            $f\in\mathcal{R}[a, b]$ and $f(x)=g(x)$ for all $x\in [a,
            b]\backslash F$, where $F$ is a finite subset of $[a, b]$. Prove
            that
            \begin{equation*}
                g\in\mathcal{R}[a,
                b]\;\;\;\text{and}\;\;\;\int_{a}^{b}f=\int_{a}^{b}g.
            \end{equation*}
            Does the same conclusion hold if $F$ is not finite? Give a proof or
            counterexample. 
            \begin{proof}
                Assume that $F=\{s\}$ contains only one element. Let
                $\varepsilon>0$, then since $f\in\mathcal{R}[a, b]$, there
                exists a partition $P=\{x_0, \dots, x_n\}$ such that 
                \begin{equation*}
                    U(f,P)-L(f, P)<\frac{\varepsilon}{2}. 
                \end{equation*}
                Moreover, because $F\subset[a, b]$,
                then for some $i\in\{1, \dots, n\}$, we have that $x_{i-1}\leq
                s\leq x_i$. Now select $p, r\in[a, b]$ such that $x_{i-1}\leq
                p\leq s\leq r\leq x_i$ and $K|r-q|<\varepsilon/2$, where
                $K=\max\{|\inf f(x)-g(s)|, |\sup f(x)-g(s)|\}$ for $p\leq x\leq
                r$. Now
                define $P^*=P\cup\{p, r\}$. Then $P^*$ is a refinement of $P$
                and thus 
                \begin{equation*}
                    U(f, P^*)-L(f, P^*)<\frac{\varepsilon}{2}.
                \end{equation*}
                Now we consider the following 3 cases:
                \begin{enumerate}[(i)]
                    \item $\inf f(x)\leq g(s)\leq \sup f(x)$ for $p\leq x\leq
                        r$. This implies that $\inf f(x)=\inf g(x)$ and $\sup
                        f(x)=\sup g(x)$ for $p\leq x\leq r$ and thus $L(f,
                        P^*)=L(g, P^*)$ and $U(f, P^*)=U(g, P^*)$. Thus $U(g,
                        P^*)-L(g, P^*)<\varepsilon/2<\varepsilon$.
                    \item $g(s)<\inf f(x)$ for $p\leq x\leq r$. This implies
                        that $\inf g(x)<\inf f(x)$ for $p\leq x\leq r$. Thus
                        $L(g, P^*)<L(f, P^*)$ and $U(g, P^*)=U(f, P^*)$. Note
                        that 
                        \begin{align*}
                            L(f, P^*)-L(g, P^*) &= (\inf f(x)-g(s))(r-q) \\
                            &\leq K(r-p) \\
                            &<\varepsilon/2 \\
                            &\Rightarrow L(g, P^*)< L(f, P^*)-\varepsilon/2.
                        \end{align*}
                        With this we have 
                        \begin{align*}
                            U(g, P^*)-L(g, P^*) &= U(f, P^*)-L(g, P^*) \\
                            &< U(f, P^*)-L(f, P^*)+\frac{\varepsilon}{2} \\
                            &<\frac{\varepsilon}{2}+\frac{\varepsilon}{2}=\varepsilon.
                        \end{align*}
                    \item $\sup f(x)<g(s)$ for $p\leq x\leq r$. This case is
                        shown with a similar argument to part (ii), the only
                        difference is that we use the fact that $U(g, P^*)-U(f,
                        P^*)<\varepsilon/2$. 
                \end{enumerate}
                This shows that for all $\varepsilon>0$, there exists
                a partition $P$ such that $U(g, P)-L(g, P)<\varepsilon$. Thus
                $g\in\mathcal{R}[a, b]$. Since $g$ is integrable on $[a, b]$,
                then 
                \begin{align*}
                    \int_{a}^{b}g &=\int_{a}^{s}g+\int_{s}^{b}g \\
                    &=\int_{a}^{s}f+\int_{s}^{b}f \\
                    &=\int_{a}^{b}f.
                \end{align*}
                The above argument can then be repeated to show
                that it holds for $|F|=n$ for all
                $n\in\mathbb{N}$.\par\hspace{4mm} Finally, let
                \begin{equation*}
                    f(x)=1\;\;\;\text{and}\;\;\;g(x)=\begin{cases}
                        1&\text{if $x\in[0, 1]\cap\mathbb{Q}$} \\
                        0&\text{if $x\in[0, 1]\cap\mathbb{I}$}.
                    \end{cases}
                \end{equation*}
                Clearly $f$ is integrable and both $f$ and $g$ are bounded.
                Also $f(x)=g(x)$ for all $x\in[0, 1]\backslash([0,
                1]\cap\mathbb{I})$. Since $g$ is not integrable, then the
                result does not hold. 
            \end{proof}
        \item Let $f:[a, b]\to\mathbb{R}$ be an increasing function. Prove that
            $f\in\mathcal{R}[a, b]$.
            \begin{proof}
                Let $\varepsilon>0$. Since $f$ is increasing then either 
                $f(a)=f(b)$, in which case $f$ is constant and 
                therefore integrable, or $f(a)<f(b)$.
                Supposing the latter case, let $P=\{x_0, \dots, x_n\}$ such
                that for some $\delta<\varepsilon/(f(b)-f(a))$, 
                $\Delta x_i=\delta$
                all $i\in\{1, \dots, n\}$. Then because $f$ is increasing, it
                follows that $m_i=\inf f(x)=f(x_{i-1})$ and $M_i=\sup
                f(x)=f(x_i)$ for $x_{i-1}\leq x\leq x_i$. Hence,
                \begin{align*}
                    U(f, P)-L(f, P) &= \sum_{i=1}^{n}M_i\Delta
                    x_i-\sum_{i=1}^{n}m_i\Delta x_i \\
                    &=\sum_{i=1}^{n}(M_i-m_i)\Delta x_i \\
                    &=\sum_{i=1}^{n}(f(x_i)-f(x_{i-1}))\delta \\
                    &=\delta(f(b)-f(a)) \\
                    &<\frac{\varepsilon}{f(b)-f(a)}(f(b)-f(a))=\varepsilon.
                \end{align*}
                Therefore $f\in\mathcal{R}[a, b]$. 
            \end{proof}
        \item Let $f\in C[a, b]$ satisfy
            \begin{equation*}
                \int_{a}^{x}f=\int_{x}^bf\;\;\;\text{for all $x\in[a, b]$.}
            \end{equation*}
            Prove that $f(x)=0$ for all $x\in[a, b]$. 
            \begin{proof}
                Since $f$ is continuous on $[a, b]$. Then by part 1 of
                the Fundamental Theorem of Calucus, $I_f'(x)=f(x)$ for all
                $x\in[a, b]$. It then follows from part 2 of FHT that $I_f(x)$
                is an antiderivative of $f$ over $[a, b]$ and thus over $[a,
                y]$ and $[y, b]$ for any $y\in[a, b]$. Moreover, 
                \begin{equation*}
                    \int_{a}^y
                    f=I_f(y)-I_f(a)\;\;\text{and}\;\;\int_{y}^b=I_f(b)-I_f(y).
                \end{equation*}
                Combining this with our assumption of equality, we get that 
                \begin{equation*}
                    I_f(y)-I_f(a)=I_f(b)-I_f(y)\Rightarrow
                    I_f(y)=\frac{I_f(b)+I_f(a)}{2}.
                \end{equation*}
                The right-hand side being a constant gives 
                \begin{equation*}
                    I_f'(y)=0=f(y)
                \end{equation*}
                for all $y\in [a, b]$. Therefore $f(x)=0$ for all $x\in[a, b]$.
            \end{proof}
        \item Let $f, g\in C[a, b]$ with $g(x)\geq 0$ for all $x\in[a, b]$.
            Prove that there exists $c\in[a, b]$ such that 
            \begin{equation*}
                \int_{a}^{b}fg=f(c)\int_{a}^{b}g.
            \end{equation*}
            Does the same conclusion hold if $g$ is not assumed to be
            nonnegative? Give a proof or a counterexample.
            \begin{proof}
                Let $m=\inf f(x)$ and $M=\sup f(x)$ for $x\in[a, b]$. Then
                since $f$ is continuous and $[a, b]$ is closed, then $m=\min
                f(x)=f(a^*)$ and $M=\max f(x)=f(b^*)$ for $x\in[a, b]$ and
                $a^*, b^*\in[a, b]$. Thus $m\leq f(x)\leq
                M$ for all $x\in[a, b]$. Moreover, since $g(x)\geq 0$ for all
                $x\in[a, b]$, then 
                \begin{equation*}
                    mg(x)\leq f(x)g(x)\leq Mg(x)\;\;\text{for all $x\in[a, b]$}
                \end{equation*}
                and by monotonicity we get that
                \begin{equation}
                    m\int_{a}^bg\leq\int_{a}^bfg\leq M\int_{a}^bg.
                \end{equation}
                Assuming $\int_{a}^b g>0$, then from (6) it follows that 
                \begin{equation}
                    f(a^*)=m\leq\frac{\int_{a}^bfg}{\int_{a}^bg}\leq M=f(b^*).
                \end{equation}
                Now consider the integral function $I_f(x)=\int_{a}^x f$. Since
                $f$ is continuous on $[a, b]$, then by part 1 of FHT,
                $I_f'(x)=f(x)$ for all $x\in [a, b]$. From 7 and Theorem 5.12
                (Rudin) it follows that there exists some $c\in(a^*, b^*)$ such
                that 
                \begin{equation*}
                     I_f'(c)=f(c)=\frac{\int_{a}^bfg}{\int_a^b g}\Rightarrow
                    \int_{a}^bfg=f(c)\int_{a}^b g.
                \end{equation*}
                If $\int_a^b g=0$, then since $g(x)\geq 0$ and since $g$ is
                continuous on $[a, b]$, then $g(x)=0$ for all $[a, b]$. This
                implies that $f(x)g(x)=0$ for all $x\in[a, b]$. Hence
                \begin{equation*}
                    \int_{a}^bfg = \int_a^b 0=0\Rightarrow
                    \int_a^bfg=f(c)\int_a^b g=f(c)\int_a^b0=0.
                \end{equation*}
                Which holds for any $c\in[a, b]$.
                For a counter example let $a=0.5$, $b=1.5$, $f(x)=x$, and
                $g(x)=\log x$. Then we have that
                \begin{equation*}
                    \int_{a}^bfg=\int_{0.5}^{1.5}x\log x\approx 0.0427916
                \end{equation*}
                and 
                \begin{equation*}
                    \int_{a}^b g=\int_{0.5}^{1.5}\log x\approx -0.0452287.
                \end{equation*}
                Seeing as the range of $f$ is $[0.5, 0.5]$ which is all
                positive values, then it is clear
                that there does not exist any $c\in[0.5, 1.5]$ such that 
                \begin{equation*}
                    \int_{a}^bfg=f(c)\int_{a}^b g.
                \end{equation*}
            \end{proof}
    \end{enumerate}
\end{document}
