\documentclass[12pt]{article}
\usepackage[margin=1in]{geometry}
\usepackage{graphicx}
\usepackage{amsmath}
\usepackage{amsthm}
\usepackage{amsfonts}
\usepackage{amssymb}
\usepackage{array}
\usepackage{enumerate}
\usepackage{slashed}
\usepackage{colonequals}
\usepackage{fancyhdr}

\pagestyle{fancy}
\fancyhf{}
\rhead{Darcy}
\lhead{MATH 230B}
\rfoot{\thepage}
\setlength{\headheight}{10pt}

\newtheorem{theorem}{Theorem}[section]
\newtheorem{corollary}{Corollary}[theorem]
\newtheorem{prop}{Proposition}[section]
\newtheorem{lemma}[theorem]{Lemma}
\theoremstyle{definition}
\newtheorem{definition}{Definition}[section]
%\theoremstyle{remark}
%\newtheorem*{remark}{Remark}

\newcommand{\abs}[1]{\lvert #1 \rvert}
\newcommand{\bigabs}[1]{\Bigl \lvert #1 \Bigr \rvert}
\newcommand{\bigbracket}[1]{\Bigl [ #1 \Bigr ]}
\newcommand{\bigparen}[1]{\Bigl ( #1 \Bigr )}
\newcommand{\ceil}[1]{\lceil #1 \rceil}
\newcommand{\bigceil}[1]{\Bigl \lceil #1 \Bigr \rceil}
\newcommand{\floor}[1]{\lfloor #1 \rfloor}
\newcommand{\bigfloor}[1]{\Bigl \lfloor #1 \Bigr \rfloor}
\newcommand{\norm}[1]{\| #1 \|}
\newcommand{\bignorm}[1]{\Bigl \| #1 \Bigr \| #1}
\newcommand{\inner}[1]{\langle #1 \rangle}
\newcommand{\set}[1]{{ #1 }}

\begin{document}
    \thispagestyle{empty}\hrule

    \begin{center}
        \vspace{.4cm} { \large MATH 230B}
    \end{center}
    {Name:\ Quin Darcy \hspace{\fill} Due Date: 04/29/2022 \\
    { Instructor:}\ Dr.Ricciotti \hspace{\fill} Assignment:
    Writeup  \\ \hrule}\


    \section*{Homework 1 Summary}

    The problems in HW1 probed at certain properties of differentiation and
    continuity. There was one result that was used in almost all of the
    exercises in this homework, the Mean Value Theorem. This theorem states
    that a  function $f$ which is
    continuous over an interval $[a, b]$ and differentiable over $(a, b)$ is
    said to have a derivative at a particular point $c\in(a, b)$ such that
    $f'(c)=(f(b)-f(a))/(b-a)$. The power of this theorem is that it presents an
    explicit relationship between the function evaluated at the endpoints and
    the derivative of the function. In the case where you know more about the
    function than you do about its derivative, then this theorem allows you to
    take advantage of that asymmetry, and vice versa.\par\hfill\par Several
    exercises included the condition of a function with a bounded derivative (see
    problems 2, A1, A3). What this condition gives you is Lipschitz continuity
    of the function. This type of continuity allows you to directly relate the
    distance between the images of two points in the domain and the distance
    between the two points themselves. For example, in problem 2 the functions
    derivative was strictly bounded 1. This means that for any $x, y$ in our domain,
    $|f(x)-f(y)|<|x-y|$. How this was used in this exercise was two take $x,
    y$ to be two terms in a Cauchy sequence whose distance was less than some
    $\varepsilon$, which in turn allowed us to show that the distance between
    the two image of the two
    sequence terms was less than $\varepsilon$, hence showing that the sequence
    of image terms was Cauchy and therefore convergent.\par\hfill\par Something
    worth noting is that problem A1 is a generalization of problem 2. Both
    required that the derivative be strictly bounded by 1 which gives rise to
    a contraction mapping. The same
    proof structure was used but instead of referring to a specific Cauchy
    sequence, we used an arbitrary one. The one in problem 2 converged to the left
    end point (to 0), whereas in problem A1, we worked with a general Cauchy
    sequence which converged to the right endpoint of the domain. 
    \par\hfill\par Another
    concept present in this homework was that of using the derivative of the
    function to indicate whether a function is increasing or decreasing (see
    problems 4 and 5). In problem 4, we were able to show that the given
    function was increasing by showing that its derivative was greater than
    0 over its domain. Specifically, the function was a quotient function and
    we needed only to show that the numerator in the resultant derivative was
    greater than or equal to 0.\par\hfill\par Two other problems that I found
    noteworthy were problems 3 and A3. Both of them had to do with showing the
    existence of unique fixed points, but the methods used to do it were very
    different for each. In problem 4, the approach was to use contradiction and
    assume there was no fixed points. From there one exhausts all the
    implications of the other premises to show that each one leads to
    a contradiction. In problem A3, we had another bounded derivative and
    a recursive function definition. In this case, the derivative was bounded
    by one and with the recursive definition, we were dealing with
    a contraction mapping. The approach to this problem began by showing that
    the sequence defined by the difference between two consecutive terms in the
    sequence went to 0. From there one shows the general case that the
    difference between arbitrary terms go to zero which is to show that the
    sequence is Cauchy. To show uniqueness of the fixed point used the Mean
    Value Theorem to produce a contradiction when assuming the existence of two
    distinct fixed points.\par\hfill\par Lastly, in problem A4 the approach was
    to start with any nonzero $x$ and show that any derivative of the function
    can be re-expressed as a rational polynomial multiplied by the original
    function. This then gives rise to the
    infinite differentiability that we were set out to prove. 

\end{document}
