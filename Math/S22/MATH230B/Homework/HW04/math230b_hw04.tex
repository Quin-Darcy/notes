\documentclass[12pt]{article}
\usepackage[margin=1in]{geometry}
\usepackage{graphicx}
\usepackage{amsmath}
\usepackage{amsthm}
\usepackage{amsfonts}
\usepackage{amssymb}
\usepackage{array}
\usepackage{enumerate}
\usepackage{fancyhdr}
\pagestyle{fancy}
\fancyhf{}
\rhead{Darcy}
\lhead{MATH 230B}
\rfoot{\thepage}
\setlength{\headheight}{10pt}

\newenvironment{solution}
{\renewcommand\qedsymbol{$\blacksquare$}\begin{proof}[Solution]}
{\end{proof}}
\newenvironment{psmall}{\left(\begin{smallmatrix}}{\end{smallmatrix}\right)}

\begin{document}
    \thispagestyle{empty}\hrule

    \begin{center}
        \vspace{.4cm} { \large MATH 230B}
    \end{center}
    {Name:\ Quin Darcy \hspace{\fill} Due Date: 04/01/2022   \\
    { Instructor:}\ Dr. Ricciotti \hspace{\fill} Assignment:
    Homework 04 \\ \hrule}

    \begin{enumerate}
        \item Let $a_k\geq 0$, $b_k>0$ for all $k\in\mathbb{N}$. Assume that
            $\lim_{k\to\infty}\frac{a_k}{b_k}=\lambda$, where
            $0<\lambda<\infty$. Prove that $\sum_{k=1}^{\infty}a_k$ is
            convergent if and only if $\sum_{k=1}^{\infty}b_k$ is convergent. 
            \begin{proof}
                Assume that $\sum_{k=1}^{\infty}b_k$ is convergent. Let
                $\varepsilon=1$. Then since $\lim_{n\to\infty}a_n/b_n=\lambda$,
                then there exists $N\in\mathbb{N}$ such that 
                \begin{align*}
                    \bigg|\frac{a_n}{b_n}-\lambda\bigg|<1 
                    \Leftrightarrow \bigg|\frac{a_n}{b_n}\bigg|-|\lambda|<1 
                    \Leftrightarrow \frac{a_n}{b_n}<1+\lambda 
                    \Leftrightarrow a_n<(1+\lambda)b_n
                \end{align*}
                for all $n\geq N$. Hence for any $M\geq N$
                \begin{equation*}
                    \sum_{k=N}^{M}a_k<\sum_{k=N}^{M}(1+\lambda)b_k
                    \leq(1+\lambda)\sum_{k=1}^{\infty}b_k<\infty.
                \end{equation*}
                The last inquality holds by assumption. Noting that
                $\sum_{k=1}^{N-1}a_k$ is finite, then the above inequality
                implies that the sequence of partial sums
                $S_n=\sum_{k=1}^{n}a_k$ is bounded from above. Moreover, this
                sequence is monotonically increasing as all of its terms are
                positive. Therefore the sequence of parthial sums is convergent
                and so 
                \begin{equation*}
                    \sum_{k=1}^{\infty}a_k<\infty. 
                \end{equation*}
                Now assume that $\sum_{k=1}^{\infty}a_k<\infty$. Let
                $\varepsilon>0$ such that $\varepsilon<\lambda$. Then there
                exists $N\in\mathbb{N}$ such that 
                \begin{equation*}
                    \bigg|\frac{a_n}{b_n}-\lambda\bigg|<\varepsilon
                    \Leftrightarrow(\lambda-\varepsilon)b_n<a_n<(\lambda+\varepsilon)b_n 
                \end{equation*}
                for all $n\geq N$. This implies that for any $M\geq N$
                \begin{equation*}
                    \sum_{k=N}^{M}b_k<(\lambda-\varepsilon)\sum_{k=N}^{M}a_k
                    \leq(\lambda-\varepsilon)\sum_{k=1}^{\infty}a_k<\infty.
                \end{equation*}
                Since $\sum_{k=1}^{N-1}b_k<\infty$, then the sequence of
                partial sums $S_n=\sum_{k=1}^n b_k$ is bounded above. Since the
                sequence is monotincally increasing, then it converges.
                Therefore
                \begin{equation*}
                    \sum_{k=1}^{\infty}b_k<\infty.
                \end{equation*}
            \end{proof}
        \item Let $a_k\geq 0$ for all $k\in\mathbb{N}$. Prove that if
            $\sum_{k=1}^{\infty}a_k$ is convergent, then
            $\sum_{k=1}^{\infty}a^2_k$ is convergent. 
            \begin{proof}
                Since $\sum_{k=1}^{\infty}a_k$ converges, then from Theorem
                3.23 (Rudin), it follows that $\lim_{k\to\infty}a_k=0$. Letting
                $\varepsilon=1$, there exists $N\in\mathbb{N}$ such that
                $|a_k|<1$ for all $k\geq N$. Since $a_k\geq 0$ for all
                $k\in\mathbb{N}$, then $a_k<1$ for all $k\geq N$. Hence,
                $a_k^{2}\leq a_k<1$ for all $k\geq N$. This implies that for
                all $M\geq N$ 
                \begin{equation*}
                    \sum_{k=N}^{M}a^2_k\leq\sum_{k=N}^{M}a_k\leq\sum_{k=1}^{\infty}a_k<\infty.
                \end{equation*}
                By Theorem 3.25 (Rudin), 
                \begin{equation*}
                    \sum_{k=1}^{\infty}a^2_k<\infty.
                \end{equation*}
            \end{proof}
        \item Let $a_k\geq0$ for all $k\in\mathbb{N}$. Prove that if
            $\sum_{k=1}^{\infty}a_k$ is convergent, then
            $\sum_{k=1}^{\infty}\sqrt{a_ka_{k+1}}$ is convergent.
            \begin{proof}
                Before proceeding, we note that if $a, b\geq 0$, then
                $(\sqrt{a}-\sqrt{b})^2\geq 0$, which implies that 
                \begin{equation}
                    a+b-2\sqrt{ab}\geq 0\Leftrightarrow
                    \sqrt{ab}\leq\frac{a+b}{2}.  
                \end{equation}
                From (1) it follows that for all $k\in\mathbb{N}$
                \begin{equation*}
                    \sqrt{a_ka_{k+1}}\leq\frac{a_k+a_{k+1}}{2}. 
                \end{equation*}
                Thus for any $n\in\mathbb{N}$
                \begin{equation*}
                    \sum_{k=1}^{n}\sqrt{a_ka_{k+1}}
                    \leq\frac{1}{2}\sum_{k=1}^{n}a_k+a_{k+1}
                    =\frac{a_1}{2}+\sum_{k=1}^{n-1}a_{k+1}.
                \end{equation*}
                This implies that the sequence of partial sums 
                $S_n=\sum_{k=1}^{n}\sqrt{a_ka_{k+1}}$ is bounded above and
                therefore converges. 
            \end{proof}
        \item Let $a_k\geq 0$ for all $k\in\mathbb{N}$. Prove that
            $\sum_{k=1}^{\infty}a_k$ is convergent if and only if
            $\sum_{k=1}^{\infty}\frac{a_k}{1+a_k}$ is convergent. 
            \begin{proof}
                Assume that $\sum_{k=1}^{\infty}a_k$ is convergent. Then
                since $a_k\geq 0$ for all $k\in\mathbb{N}$, we have that
                \begin{equation*}
                    \frac{a_k}{1+a_k}\leq a_k
                \end{equation*}
                for all $k\in\mathbb{N}$. By Theorem 3.25 (Rudin) it follows
                that 
                \begin{equation*}
                    \sum_{k=1}^{\infty}\frac{a_k}{1+a_k}<\infty.
                \end{equation*}
                Now assume that $\sum_{k=1}^{\infty}\frac{a_k}{1+a_k}$ is
                convergent. Then $\lim_{k\to\infty}\frac{a_k}{1+a_k}=0$.
                Letting $0<\varepsilon<1$, then there exists $N\in\mathbb{N}$ such
                that for all $k\geq N$
                \begin{align*}
                    \frac{a_k}{1+a_k}<\varepsilon
                    &\Leftrightarrow a_k<\varepsilon(1+a_k) \\
                    &\Leftrightarrow a_k(1-\varepsilon)<\varepsilon \\
                    &\Leftrightarrow a_k<\frac{\varepsilon}{1-\varepsilon}.
                \end{align*}
                As $\varepsilon\to 0$, then
                $\frac{\varepsilon}{1-\varepsilon}\to0$ and therefore $a_k\to
                0$ as $k\to\infty$. Since $\lim_{k\to\infty}a_k=0$, then for
                any $1<B$, there exists $N$ such that for all $k\geq N$,
                we have that $1+a_k<B$. Then $\frac{B}{1+a_k}>1$ for all $k\geq
                N$. Thus for all $k\geq N$
                \begin{equation*}
                    a_k\leq\frac{Ba_k}{1+a_k}.
                \end{equation*}
                By Theorem 3.23 (Rudin), 
                \begin{equation*}
                    \sum_{k=1}^{\infty}a_k<\infty.
                \end{equation*}
            \end{proof}
        \item Prove that if $\sum_{k=1}^{\infty}a_k$ is conditionally
            convergent, then $\sum_{k=1}^{\infty}k^2a_k$ is not convergent. 
            \begin{proof}
                If $\sum_{k=1}^{\infty}a_k$ is conditionally convergent, then
                $\sum_{k=1}^{\infty}|a_k|$ is divergent. For contradiction,
                assume that $\sum_{k=1}^{\infty}k^2a_k$ converges. Then
                $\lim_{k\to\infty}k^2a_k=0$. If $\varepsilon=1$, there exists
                $N\in\mathbb{N}$ such that for all $k\geq N$,
                \begin{equation*}
                    |k^2a_k|=k^2|a_k|<\varepsilon\Rightarrow
                    |a_k|<\frac{1}{k^2}.
                \end{equation*}
                Given that by Theorem 3.28 (Rudin), $\sum_{k=1}^{\infty}\frac{1}{k^2}$ 
                converges, then by Theorem 3.23 (Rudin),
                $\sum_{k=1}^{\infty}|a_k|$ converges. This contradicts our
                assumption that $\sum_{k=1}^{\infty}a_k$ converges
                conditionally. Therefore $\sum_{k=1}^{\infty}k^2a_k$ is
                divergent. 
            \end{proof}
        \item Let $a_n\geq 0$ for all $k\in\mathbb{N}$. Prove that if
            $\sum_{k=1}^{\infty}a_k$ is convergent, then $\lim_{k\to\infty}\inf
            ka_k=0$. Is it true that $\lim_{k\to 0}ka_k=0$?
            \begin{proof}
                For contradiction, assume that $\lim_{n\to\infty}\inf ka_k>0$.
                Then no subsequences of $\{ka_k\}$ converge to 0. Hence, there
                exists $r>0$ and some $N\in\mathbb{N}$ such that for all $n\geq
                N$ we have that $na_n>r$ which implies that $a_n>r/n$. Thus 
                \begin{equation*}
                    \sum_{k=1}^{\infty}a_k=\sum_{k=1}^{N-1}a_k+\sum_{k=N}^{\infty}a_k
                    \geq \sum_{k=1}^{N-1}a_k+\sum_{k=N}^{\infty}\frac{r}{k}.
                \end{equation*}
                Since $\sum_{k=N}^{\infty}r/k=\infty$, then this contradicts
                the assumption that $\sum_{k=1}^{\infty}a_k$ is convergent. 
            \end{proof}
    \end{enumerate}
\end{document}
