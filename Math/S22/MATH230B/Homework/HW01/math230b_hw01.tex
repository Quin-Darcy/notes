\documentclass[12pt]{article}
\usepackage[margin=1in]{geometry}
\usepackage{graphicx}
\usepackage{amsmath}
\usepackage{amsthm}
\usepackage{amsfonts}
\usepackage{amssymb}
\usepackage{array}
\usepackage{enumerate}
\usepackage{fancyhdr}
\usepackage{enumitem}
\pagestyle{fancy}
\fancyhf{}
\rhead{Darcy}
\lhead{MATH 230B}
\rfoot{\thepage}
\setlength{\headheight}{10pt}

\newenvironment{solution}
{\renewcommand\qedsymbol{$\blacksquare$}\begin{proof}[Solution]}
{\end{proof}}
\newenvironment{psmall}{\left(\begin{smallmatrix}}{\end{smallmatrix}\right)}

\begin{document}
    \thispagestyle{empty}\hrule

    \begin{center}
        \vspace{.4cm} { \large MATH 230B}
    \end{center}
    {Name:\ Quin Darcy \hspace{\fill} Due Date: 02/04/2022   \\
    { Instructor:}\ Dr. Ricciotti \hspace{\fill} Assignment:
    Homework 01 \\ \hrule}

    \begin{enumerate}
        \item Let
            \begin{equation*}
                f(x)=\begin{cases}
                    x^5\sin(\frac{1}{x^3}), & \text{if $x\neq0$} \\
                    0, & \text{if $x=0$}
                \end{cases}
            \end{equation*}
            How many times is $f$ differentiable on $\mathbb{R}$?\\
            For which $n\in\mathbb{N}$ do we have 
            $f\in C^n(\mathbb{R})\backslash C^{n+1}(\mathbb{R})$?
            \begin{solution}
                For $x\in\mathbb{R}-\{0\}$, we have
                $f(x)=x^5\sin(\frac{1}{x^3})$. Seeing as $f$ is the product,
                composition, and quotient of differentiable functions, then by
                Theorem 5.3 and Theorem 5.7, we can conclude that $f$ is
                differentiable for all $x\in\mathbb{R}-\{0\}$. Using the
                Product and Chain Rule, we obtain
                \begin{equation*}
                    f'(x)=5x^4\sin\big(\frac{1}{x^3}\big)-3x\cos\big(\frac{1}{x^3}\big)
                \end{equation*}
                for all $x\in\mathbb{R}-\{0\}$. To obtain $f'(0)$, if it
                exists, we can use the definition of derivative and compute
                \begin{equation*}
                    \lim_{h\to
                    0}\frac{f(0+h)-f(0)}{h}=\lim_{h\to
                    0}\frac{h^5\sin\big(\frac{1}{h^3}\big)}{h}
                    =\lim_{h\to 0}h^4\sin\big(\frac{1}{h^3}\big)=0.
                \end{equation*}
                The last equality holding since
                $0\leq|h^4\sin\big(\frac{1}{h^3}\big)|\leq h^4$ and since $h^4\to 0$
                as $h\to 0$. Thus
                \begin{equation*}
                    f'(x)=\begin{cases}
                        5x^4\sin\big(\frac{1}{x^3}\big)-3x\cos\big(\frac{1}{x^3}\big),
                        & \text{if $x\neq 0$} \\
                        0, & \text{if $x=0$}.
                    \end{cases}
                \end{equation*}
                Seeing as $f'(x)$ is the product, composition, and quotient of
                continuous functions, provided that $x\neq 0$, then by Theorem
                4.7 and Theorem 4.9, $f'(x)$ is continuous for $x\neq 0$. For
                $x=0$, we note that since both the trigonometric terms are
                bounded as $x\to 0$ and that $5x^4, 3x\to 0$ as $x\to 0$, then
                $\lim_{x\to 0}f'(x)=f'(0)=0$, meaning $f'$ is continuous at
                $x=0$. Thus $f\in C^1(\mathbb{R})$.\par\hspace{4mm} We now
                compute $f''$. By the same reasoning as above, for $x\neq 0$,
                we have that $f'(x)$ is differentiable and the Product and
                Chain Rules gives us
                \begin{equation*}
                    f''(x)=\bigg(20x^3-\frac{9}{x^3}\bigg)
                    \sin\big(\frac{1}{x^3}\big)-(15x+3)\cos\big(\frac{1}{x^3}\big).
                \end{equation*}
                Using the same method as before, we let $x=0$ and get 
                \begin{align*}
                    \lim_{h\to 0}\frac{f'(0+h)-f(0)}{h}
                    &=\lim_{h\to
                    0}\frac{5h^4\sin\big(\frac{1}{h^3}\big)-3h\cos\big(\frac{1}{h^3}\big)}{h}
                      \\
                    &=\lim_{h\to
                      0}\bigg(5h^3\sin\big(\frac{1}{h^3}\big)-3\cos\big(\frac{1}{h^3}\big)\bigg)
                        \\
                    &=\lim_{h\to 0}5h^3\sin\big(\frac{1}{h^3}\big)-\lim_{h\to
                    0}3\cos\big(\frac{1}{h^3}\big) \\
                    &=\lim_{h\to 0}3\cos\big(\frac{1}{h^3}\big).
                \end{align*}
                Seeing as the limit in the last equality does not exist, then
                we can conclude that $f'$ is not differentiable at $x=0$.
                Therefore $f$ is differentiable once on $\mathbb{R}$ and for
                $m=1$ is it true that $f\in C^m(\mathbb{R})\backslash
                C^{m+1}(\mathbb{R})$. 
            \end{solution}
        \item Let $f:(0, 1]\to\mathbb{R}$ be differentiable with $0<f'(x)<1$
            for all $x\in(0,1]$. Prove that the sequence $\{f(1/n)\}_n$ has
            a limit. 
            \begin{proof}
                Let $x, y\in(0, 1]$. Then $f$ is continuous on $[x, y]$ and
                differentiable on $(x, y)$. By the Mean Value Theorem, there
                exists $z\in(x, y)$ such that 
                \begin{equation*}
                    f(y)-f(x)=f'(z)(x-y)<(1)(x-y)<|x-y|,
                \end{equation*}
                and since this is true for all $x, y\in(0, 1]$, then we can conclude
                that $f$ is Lipschitz on this interval and thus uniformly
                continuous on $(0,1]$. With this in mind, we note that $\{1/n\}_n$ 
                is a Cauchy sequence. Now letting $\varepsilon>0$, there exists
                $N\in\mathbb{N}$ such that for all $m, n\geq N$ we get
                \begin{equation*}
                    \bigg|\frac{1}{n}-\frac{1}{m}\bigg|<\varepsilon.
                \end{equation*}
                Combining this with the fact that $f$ is uniformly continuous
                on $(0, 1]$, we have 
                \begin{equation*}
                    f(1/n)-f(1/m)<\bigg|\frac{1}{n}-\frac{1}{m}\bigg|<\varepsilon.
                \end{equation*}
                Hence, $\{f(1/n)\}_n$ is Cauchy and therefore convergent. 
            \end{proof}
        \item Let $f:[0, 1]\to[0,1]$ be continuous on $[0,1]$ and
            differentiable on $(0,1)$, with $f'(x)\neq 1$ for all $x\in(0, 1)$.
            Prove that there exists a unique fixed point for $f$ in $[0, 1]$.
            \begin{proof}
                For contradiction, assume that $f(x)\neq x$ for all $x\in[0,
                1]$. Then define the set
                \begin{equation*}
                    S=\{|f(x)-x|:x\in[0, 1]\}.
                \end{equation*}
                Seeing as $f(x)$, $x$, and $|x|$ are all continuous on $[0,
                1]$, then $g(x)=|f(x)-x|$ is continuous on $[0, 1]$ as a composition
                and difference of continuous functions. Thus the image of $[0,
                1]$ under $g$, $S$,  is closed and bounded.  
                \par\hspace{4mm} By the greatest lower bound property,
                $\alpha=\inf(S)$ exists and $\alpha\in S$ since $S$ is closed.
                Hence there exists $x_0\in[0, 1]$ such that
                $\alpha=|f(x_0)-x_0|$. If $\alpha=0$ and $f(x)>x$, then
                $|f(x)-x|=f(x)-x=0$ and so $f(x)=x$, which is a contradiction.
                Similarly, if $x>f(x)$, then $|f(x)-x|=x-f(x)=0$ and so
                $x=f(x)$, another contradiction. Thus
                $\alpha>0$.\par\hspace{4mm}Define $g(x)=f(x)-x$. Then $g$ is
                continuous on $[0, 1]$, differentiable on $(0, 1)$, and
                $\alpha$ is a local minimum at some $x_0$. If $x_0\in(0, 1)$,
                then by Theorem 5.8 $g'(x_0)=0$, which implies $f'(x_0)=1$
                which is not possible by assumption. Thus either $g(0)=\alpha$
                or $g(1)=\alpha$. Coupled with the fact that $g(x)\neq 0$ for
                all $x\in (0, 1)$ implies that $g$ is strictly monotone. 
                \par\hspace{4mm} If $g(0)=\alpha$, then $g$ must be strictly
                monotone increasing. Thus $g(1)>\alpha$ which implies
                $f(1)>1+\alpha\notin[0, 1]$. This is not possible by definition
                of $f$.\par\hspace{4mm} If $g(1)=\alpha$, then
                $f(1)=1+\alpha\notin[0, 1]$, which is not possible by the same
                reasoning above. Therefore there must exist some $x_0\in[0, 1]$
                such that $f(x_0)=x_0$.\par\hspace{4mm} To show uniqueness, let
                $x_1, x_2\in [0, 1]$ such that $f(x_1)=x_1$ and $f(x_2)=x_2$.
                Then $g(x_1)=0=g(x_2)$. By Rolle's Theorem, there exists
                $x_0\in(x_1, x_2)$ such that $g'(x_0)=0$ which implies
                $f'(x_0)=1$, a contradiction unless $x_1=x_2$. [I'm sorry this
                was so long! I got carried away and had too much fun with it.]  
            \end{proof}
        \item Let $f:[0, \infty)\to\mathbb{R}$ be continuous on $[0, \infty)$
            with $f(0)=0$ and differentiable on $(0, \infty)$ with $f'$
            increasing on $(0, \infty)$. Prove that the function
            $\frac{f(x)}{x}$ is increasing on $(0, \infty)$.
            \begin{proof}
                Let $g(x)=\frac{f(x)}{x}$ and select an arbitrary $x_0\in (0,
                \infty)$. Since $f$ and $x$ are continuous on $[0, \infty)$,
                then $g$ is continuous on this interval. Similarly, since $f$
                and $x$ are differentiable on $(0, \infty)$, then $g$ is
                differentiable on this interval.\par\hspace{4mm}
                By assumption, $f$ is continuous on $[0,
                x_0]\subset[0, \infty)$ and differentiable on $(0, x_0)$. Thus
                by Theorem 5.10, there exists $a\in(0, x_0)$ such that 
                \begin{equation}
                    f'(a)=\frac{f(x_0)-f(0)}{x_0-0}=\frac{f(x_0)}{x_0}\Leftrightarrow
                    x_0f'(a)=f(x_0).
                \end{equation}
                We established that $g$ is differentiable on $(0,
                \infty)\supset(0, x_0)$. Thus for each $x\in (0, x_0)$, it
                follows from Theorem 5.3(c)
                \begin{equation}
                    g'(x)=\frac{xf'(x)-(1)f(x)}{x^2}.
                \end{equation}
                If we want to prove that $g$ is increasing on $(0, \infty)$,
                then we can use Theorem 5.11(a) to show that $g'(x)\geq 0$ for
                all $x\in(0, \infty)$. Using (2), we then need to show
                \begin{equation*}
                    \frac{xf'(x)-f(x)}{x}\geq 0
                \end{equation*}
                for all $x\in(0, \infty)$. By assumption, $f'(a)\leq f'(x_0)$ and
                thus $x_0f'(a)\leq x_0f'(x_0)$ since $a<x_0$. However, by (1) we have 
                $x_0f'(a)=f(x_0)$. Hence $f(x_0)\leq x_0f'(x_0)$ which implies
                that $x_0f'(x_0)-f(x_0)\geq 0$. Since $x_0>0$, then 
                \begin{equation*}
                    g'(x_0)=\frac{x_0f'(x_0)-f(x_0)}{x}\geq 0.
                \end{equation*}
                Finally, since $x_0$ was arbitrary, then $g'(x)\geq 0$ for all
                $x\in(0, \infty)$ as desired. 
            \end{proof}
        \item Let $a<b$, $x_0\in(a, b)$ and $f\in C^{2n}((a, b))$ for some
            $n\in\mathbb{N}$. Suppose that $f^{(k)}(x_0)=0$ for all $k\in\{1,
            2, \dots, 2n-1\}$ and $f^{(2n)}(x_0)>0$. Prove that $f$ has a local
            minimum at $x_0$. 
            \begin{proof}
                Since $f^{(2n)}$ is continuous and $f^{(2n)}(x_0)>0$, then
                there exists $\delta>0$ such that for all $x\in
                N_{\delta}(x_0)=\{x\in\mathbb{R}\mid |x-x_0|<\delta\}$, we have
                $f^{(2n)}(x)\geq 0$. Let $x\in N_{\delta}(x_0)$. Then by
                Taylor's theorem, there exists a $c$ in between $x_0$ and $x$
                such that 
                \begin{align}
                    f(x)&=\sum_{k=0}^{2n-1}\frac{f^{(k)}(x_0)}{k!}
                    (x-x_0)^k+\frac{f^{(2n)}(c)}{(2n)!}(x-x_0)^{2n} \\
                    &=f(x_0)+\frac{f^{(2n)}(c)}{(2n)!}(x-x_0)^{2n},
                \end{align}
                where (4) was a result of one of our
                assumptions. Since $c$ is in between $x_0$ and $x$, then $c\in
                N_{\delta}(x_0)$ and thus $f^{(2n)}(c)\geq 0$. Moreover, since
                $2n$ is even, then $(x-x_0)^{2n}\geq 0$ for all $x$. Hence
                \begin{equation}
                    \frac{f^{(2n)}(c)}{(2n)!}(x-x_0)^{2n}\geq 0.
                \end{equation}
                Combining (4) and (5), the implication is
                \begin{equation}
                    f(x)-f(x_0)\geq 0
                \end{equation}
                for all $x\in N_{\delta}(x_0)$. Thus $f(x_0)\leq f(x)$ for all
                $x\in N_{\delta}(x_0)$. Therefore $f(x_0)$ is a local minimum. 
            \end{proof}
        \item Let $a<b$ be real numbers and $f:(a, b)\to\mathbb{R}$ be
            differentiable with $|f'(x)|\leq M$ for all $x\in(a, b)$, where
            $M>0$. Prove that $\lim_{x\to b^{-}}f(x)$ exists. 
            \begin{proof}
                To prove this claim, we will appeal to the sequential
                definition of limit and show that for any sequence $\{x_n\}$ in
                $(a, b)$ that converges to $b$, we have that
                $\{f(x_n)\}$ is a convergent sequence.\par\hspace{4mm} We begin
                by showing that $f$ is uniformly continuous on $(a, b)$. Let
                $x, y\in(a, b)$. Then $f$ is differentiable on $(x, y)$ and
                continuous on $[x, y]$. Thus, by the MVT, there exists some
                $z\in(x, y)$ such that 
                \begin{equation*}
                    f'(z)=\frac{f(y)-f(x)}{y-x}\Leftrightarrow |f(y)-f(x)|\leq
                    M|y-x|,
                \end{equation*}
                which shows that $f$ is Lipschitz continuous. Simply selecting
                $x, y\in(a, b)$ such that $|y-x|<\varepsilon/<M$, for any
                $\varepsilon$, then shows that $f$ is uniformly continuous on
                $(x, y)$ and as $x$ and $y$ were arbitrary, then $f$ is
                uniformly continuous on $(a, b)$. \par\hspace{4mm} Let
                $\varepsilon>0$, then since $f$ is uniformly continuous, there
                exists $\delta>0$ such that for $x, y\in(a, b)$,
                $|x-y|<\delta$, then $|f(y)-f(x)|<\varepsilon$. Letting $c_n\to
                b$ be a sequence in $(a, b)$, then by its convergence it
                follows that it is a Cauchy sequence. Hence, there exists
                $N\in\mathbb{N}$, such that for any $m>n\geq N$, we have that 
                \begin{equation*}
                    |c_m-c_n|<\delta\rightarrow|f(c_m)-f(c_n)|<\varepsilon. 
                \end{equation*}
                Therefore $\{f(c_n)\}$ is Cauchy and therefore convergent. 
            \end{proof}
        \item Let $f:(0, \infty)\to\mathbb{R}$ be differentiable. Prove that if
            $\lim_{x\to\infty}f(x)=M\in\mathbb{R}$, then there exists
            a sequence $\{x_n\}_n$ in $(0, \infty)$ such that $f'(x_n)$
            converges to 0. 
            \begin{proof}
                Let $\varepsilon>0$. Since $\lim_{x\to\infty}f(x)=M$, then
                there exists $\delta>0$ such that for any $x>\delta$, we have
                that $|f(x)-M|<\varepsilon$. Moreover, by the Archemedian
                principle, there exists an integer $N>\delta$.
                Thus for any integer $n\geq N$, we have
                $|f(n)-M|<\varepsilon$. Therefore $\{f(n)\}_n$ is a convergent
                sequence and hence Cauchy. Taking the same $\varepsilon$ as
                before, there exists $N_{\varepsilon}$ such that for any $m,
                n\geq N_{\varepsilon}$, we have $|f(n)-f(m)|<\varepsilon$.
                Hence, we will let $n=N_{\varepsilon}$ and $m=n+1$. Then 
                \begin{equation}
                    |f(m)-f(n)|<\varepsilon\Rightarrow
                    \frac{f(m)-f(n)}{m-n}\leq \frac{|f(m)-f(n)|}{|m-n|}<\varepsilon. 
                \end{equation}
                Since $f$ is differentiable on $[n, m]\subset(0, \infty)$,
                then there exists $z\in(n, m)$ such that 
                \begin{equation*}
                    f'(z)=\frac{f(m)-f(n)}{m-n}<\varepsilon.
                \end{equation*}
                We now define the following sequence $\{z_k\}_k$. For each
                $k\in\mathbb{N}$, let $\varepsilon=1/k$, then we obtain
                $N_{\varepsilon}$ and let $n=N_{\varepsilon}$, $m=n+1$. And
                finally by the MVT, there exists $z_k\in(n, m)$ such that
                $f'(z_k)<\varepsilon=1/k$. Since $1/k\to 0$ as $k\to\infty$,
                then $f'(z_k)\to 0$ as $k\to\infty$.  
            \end{proof}
        \item Let $f:\mathbb{R}\to\mathbb{R}$ be differentiable and assume that
            $|f'(x)|\leq M$ for all $x\in\mathbb{R}$, where $0\leq M<1$. Let
            $x_1\in\mathbb{R}$ and consider the recursion
            \begin{equation*}
                x_{n+1}=f(x_n)\text{ for all $n\in\mathbb{N}$}.
            \end{equation*}
            Prove that $\{x_n\}_n$ converges to the unique fixed point of $f$. 
            \begin{proof}
                Let $a, b\in\mathbb{R}$. Then $f$ is differentiable on $(a, b)$
                and continuous on $[a, b]$. By the MVT, there exists $z\in(a,
                b)$ such that 
                \begin{equation*}
                    f'(z)=\frac{f(b)-f(a)}{b-a}\Leftrightarrow |f(b)-f(a)\leq
                    M|b-a|. 
                \end{equation*}
                Thus, $f$ is Lipschitz continuous on $\mathbb{R}$. It follows
                that 
                \begin{equation*}
                    |f(x_{n+1})-f(x_n)|=|x_{n+2}-x_{n+1}|\leq
                    M|x_{n+1}-x_n|\leq M^{n-1}|x_{2}-x_1|.
                \end{equation*}
                Since $0\leq M<1$, then $(x_{n+1}-x_n)\to 0$ as $n\to\infty$.
                Letting $m, n\in\mathbb{N}$ with $m<n$, then call $k=n-m$. We
                have that 
                \begin{align*}
                    |x_n-x_m|&=|x_{m+k}-x_m| \\
                    &\leq |x_{m+k}-x_{m+k-1}|+\cdots+|x_{m+1}-x_m| \\
                    &< (M^{k-1}+\cdots+M+1)|x_{m+1}-x_m| \\
                    &<\sum_{i=0}^{\infty}M^i|x_{m+1}-x_m| \\
                    &=\frac{|x_{m+1}-x_m|}{1-M}.
                \end{align*}
                Finally, since the last equality converges to 0 as
                $m\to\infty$, then $|x_n-x_m|\to 0$ and is therefore Cauchy.
                Thus $\{x_n\}$ is convergent. \par\hspace{4mm} Let 
                $x=\lim_{n\to\infty}x_n$. Then
                \begin{equation*}
                    f(x)=f(\lim_{n\to\infty}x_n)=\lim_{n\to\infty}f(x_n)
                    =\lim_{n\to\infty}x_{n+1}=x.
                \end{equation*}
                Thus $x$ is a fixed point of $f$. Finally, assume that $y$ is
                another fixed point of $f$ and take $x<y$. Then $f$ is
                differentiable on $(x, y)$ and continuous on $[x, y]$. By the
                MVT, there exists $z\in(x, y)$ such that 
                \begin{equation*}
                    f'(z)=\frac{f(y)-f(x)}{y-x}=\frac{y-x}{y-x}=1.
                \end{equation*}
                This contradicts that $|f'(x)|<1$ for all $x\in\mathbb{R}$.
                Therefore $x$ must be unique. 
            \end{proof}
        \item Let 
            \begin{equation*}
                f(x)=\begin{cases}
                    e^{-\frac{1}{x^2}} &\text{if $x\neq 0$} \\
                    0 &\text{if $x=0$}.
                \end{cases}
            \end{equation*}
            Prove that $f\in C^{\infty}(\mathbb{R})$ and $f^{(n)}(0)=0$ for all
            $n\in\mathbb{N}$. 
            \begin{proof}
                Letting $x\in\mathbb{R}-\{0\}$, then $f(x)=e^{-1/x^2}$ and 
                \begin{equation*}
                    f'(x)=2\frac{e^{-\frac{1}{x^2}}}{x^3}.
                \end{equation*}
                To obtain $f'(0)$, if it exists, we consider
                \begin{equation*}
                    \lim_{h\to 0}\frac{f(h+0)-f(0)}{h}=\lim_{h\to
                    0}\frac{f(h)}{h}.
                \end{equation*}
                Letting $g(h) = h$, then we can write the previous statement
                as
                \begin{equation*}
                    \lim_{h\to 0}\frac{f(h)}{g(h)}
                \end{equation*}
                and since both $f$ and $g$ are differentiable on $(0, \infty)$
                and $g'(h)=1\neq 0$. 
            \end{proof}
    \end{enumerate}
\end{document}
