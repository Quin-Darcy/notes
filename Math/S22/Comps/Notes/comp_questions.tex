\documentclass[12pt]{article}
\usepackage[margin=1in]{geometry}
\usepackage{graphicx}
\usepackage{amsmath}
\usepackage{amsthm}
\usepackage{amsfonts}
\usepackage{amssymb}
\usepackage{array}
\usepackage{enumerate}
\usepackage{fancyhdr}
\pagestyle{fancy}
\fancyhf{}
\rhead{Darcy}
\lhead{COMPS}
\rfoot{\thepage}
\setlength{\headheight}{10pt}

\newenvironment{solution}
{\renewcommand\qedsymbol{$\blacksquare$}\begin{proof}[Solution]}
{\end{proof}}
\newenvironment{psmall}{\left(\begin{smallmatrix}}{\end{smallmatrix}\right)}

\begin{document}
    \thispagestyle{empty}\hrule

    \begin{center}
        \vspace{.4cm} { \large COMPS}
    \end{center}
    {Name:\ Quin Darcy \hspace{\fill} Due Date: NONE   \\
    { Instructor:} NONE \hspace{\fill} Assignment:
    PRACTICE \\ \hrule}

    \begin{enumerate}
        \item[AL.1.2] Let $G$ be a finite group. 
            \begin{enumerate}[(a)]
                \item If $[G:Z(G)]=n$ where $n$ is positive, show that every
                    conjugacy class has at most $n$ elements. 
                    \begin{proof}
                        Our premise implies that 
                        \begin{equation*}
                            G/Z(G)=\{Z(G), g_1Z(G), \dots, g_{n-1}Z(G)\},
                        \end{equation*}
                        where $g_i\notin Z(G)$ for all $1\leq i\leq n-1$. Now
                        let $x\in G$. Then if $x\in Z(G)$, it follows that $x$
                        commutes with every $g\in G$. Thus 
                        \begin{equation*}
                            c(x)=\{gxg^{-1}:g\in G\}=\{gg^{-1}x:g\in G\}=\{x\}.
                        \end{equation*}
                        Assume $x\notin Z(G)$. Let $y\in c(x)$ such that $y\neq
                        x$ and $y\in Z(G)$. Then since $y\in c(x)$, there is
                        some $g\in G$ such that $y=gxg^{-1}$. Then
                        $g^{-1}yg=g^{-1}gy=y=x$, a contradiction. Thus, if
                        $x\notin Z(G)$, then for all $y\in c(x)$, $y\notin
                        Z(G)$. Hence $y=g_i$ for some $1\leq i\leq n-1$.
                        Therefore $|c(x)|\leq n$.   
                    \end{proof}
                \item Suppose the size of each conjugacy class in $G$ is at
                    most $2$. Show that for all $g\in G$, the centralizer
                    $C_G(g)$ is a normal subgroup of $G$.
                    \begin{proof}
                        Let $g\in G$. Then by assumption $c(g)=\{g\}$ or
                        $c(g)=\{g, x\}$, where $x\neq g$. If the only conjugate
                        of $g$ is itself, then $C_G(g)=G$ and $G\unlhd G$
                        trivially. Assume that $c(g)=\{g, x\}$. Then for any
                        $h\in G$, $hgh^{-1}=g$ or $hgh^{-1}=x$. This implies
                        that 
                        \begin{equation*}
                            G/C_G(g)=\{C_G(g), G\backslash C_G(g)\}.
                        \end{equation*}
                        Hence $[G:C_G(g)]=2$ and therefore $C_G(g)\unlhd G$. 
                    \end{proof}
            \end{enumerate}
        \item[AN.1.1]\hfill\par
            \begin{enumerate}[(a)]
                \item Prove that the set of isolated points of a subset
                    $S\subseteq\mathbb{R}^k$ is countable. 
                    \begin{proof}
                        Let $A\subseteq S$ be the set of all isolated points of
                        $S$. For any $x\in A$, there exists some
                        $r_x\in\mathbb{R}^{+}$ such that $N_{r_x}(x)\cap
                        S=\{x\}$. With this define the following set
                        \begin{equation*}
                            R=\{r_x:x\in A\}.
                        \end{equation*}
                        This set is bounded below by 0 and thus its infimum
                        exists. Let $r=\inf R/2$. The set $T=\{N_r(x):x\in A\}$
                        is a set of disjoint open sets and is therefore
                        countable (since from each open set, we can select
                        a unique rational $k$-tuple and from this create an
                        injective map into $\mathbb{Q}^k$ which is countable.)
                        Finally, if $f:A\to T$ is a map defined by $x\mapsto
                        N_r(x)$, then since each neighborhood is dijoint, it
                        follows that $f$ is injective. Moreover, for each
                        neighborhood there is an isolated point which maps to
                        it. The map $f$ is therefore a bijection. Thus $A\sim
                        T$ and so $A$ is countable. 
                    \end{proof}
                \item Prove that the set of all binary sequences is
                    uncountable. 
                    \begin{proof}
                        Let $S$ be the set of all binary sequences and let
                        $A\subseteq S$ be any countable subset of $S$. Suppose
                        the elements of $A$ are the sequences $s_1, s_2,
                        \dots$. Define a new sequence $s$ in the following way:
                        If the first digit of $s_1$ is a 1, then let the first
                        digit of $s$ be 0, whereas if the first digit is 0,
                        then let the first digit of $s$ be 1. Generally, if 
                        the $n$th digit of $s_n$ is 1, the let the $n$th digit
                        of $s$ be 0 and if the $n$th digit is $0$, let the
                        $n$th digit of $s$ be 1. In this way, $s$ differs from
                        each sequence in $A$ by at least one digit and so
                        $s\notin A$. However, as $s$ is still a binary
                        sequence, then $s\in S$. This means that $A$ is
                        a proper subset of $S$. Hence, every countable subset
                        of $S$ is a proper subset of $S$. Therefore $S$ is
                        uncountable. 
                    \end{proof}
            \end{enumerate}
        \item[AL.1.4]\hfill\par
            \begin{enumerate}[(a)]
                \item If $G$ is a cyclic group, prove that every subgroup of
                    $G$ is also cyclic. 
                    \begin{proof}
                        If $G$ is cyclic, then there is some $g\in G$ such that
                        $G=\langle g\rangle$. Let $H\leq G$ be a subgroup of
                        $G$. Let $k$ be the smallest nonzero positive
                        integer such that $g^k\in H$. 
                        We claim that $H=\langle g^k\rangle$. Let $h\in H$.
                        Then for some $m\in\mathbb{N}$, $h=g^m$. If $k\mid m$,
                        then $m=tk$ for some $t\in\mathbb{N}$ and hence 
                        \begin{equation*}
                            h=g^m=g^{tk}=(g^k)^t\in\langle g^k\rangle. 
                        \end{equation*}
                        If $k\nmid m$, then by the QR theorem, there exists
                        $q, r\in\mathbb{Z}$ such that $m=qk+r$ and $0\leq r<k$.
                        Hence 
                        \begin{equation*}
                            h=g^m=g^{qk+r}=g^{qk}g^r\Rightarrow g^{-qk}h=g^r.
                        \end{equation*}
                        We have that $g^{-qk}, h\in H$ but $g^r$ cannot be in
                        $H$ as that would contradict that $k$ is the smallest
                        nonzero integer such that $g^k\in H$. Thus $k\mid m$
                        and $H\subseteq\langle g^k\rangle$.\par\hspace{4mm} Let
                        $(g^k)^t\in\langle g^k\rangle$. Then since $g^k\in H$,
                        $(g^k)^t=(g^k)\cdots(g^k)\in H$ by closure. Hence
                        $H=\langle g^k\rangle$ and so $H$ is cyclic. 
                    \end{proof}
                \item Suppose $a\in G$ and ord$(a)=n$. Given a positive integer
                    $m$, if $d=\gcd(n, m)$, prove that 
                    \begin{equation*}
                        \text{ord}(a^m)=\text{ord}(a^d). 
                    \end{equation*}
                    \begin{proof}
                        Let $x=\text{ord}(a^m)$ and $y=\text{ord}(a^d)$. Since
                        $d\mid m$, then $m=kd$ for some $k\in\mathbb{Z}$. Thus 
                        \begin{equation*}
                            (a^m)^y=(a^{kd})^y=((a^d)^y)^k=e^k=e.
                        \end{equation*}
                        Thus $x\mid y$. Since $d=\gcd(n, m)$, then by the
                        Euclidien algorithm, there exists $t_1,
                        t_2\in\mathbb{Z}$ such that $d=nt_1+mt_2$. Thus
                        \begin{equation*}
                            (a^d)^x=(a^{nt_1+mt_2})^x=(a^n)^{xt_1}((a^m)^x)^{t_2}=e^{xt_1}e^x=e.
                        \end{equation*}
                        Thus $y\mid x$. Therefore $x=y$. 
                    \end{proof}
            \end{enumerate}
        \item[AN.4.9] Let $f:\mathbb{R}\to\mathbb{R}$ be defined as follows
            \begin{equation*}
                f(x)=\begin{cases}
                    x^5\sin\big(\frac{1}{x^3}\big)&\text{if $x\neq 0$} \\
                    0&\text{if $x=0$}. 
                \end{cases}
            \end{equation*}
            Study the continuity and differentiability of $f$ on $\mathbb{R}$.
            How many times is $f$ differentiable?
            \begin{proof}
                Let $x\in\mathbb{R}-\{0\}$. Then $f(x)=x^5\sin(1/x^3)$. Since
                $x^5$, $\sin(x)$, and $1/x^3$ are all differentiable for $x\neq 0$,
                then $f$ is differentiable as the product and composition of
                differentiable functions. This further implies that $f$ is
                continuous. Using the product and chain rule we obtain
                \begin{equation*}
                    f'(x)=5x^4\sin(\frac{1}{x^3})-3x\cos(\frac{1}{x^3})
                \end{equation*}
                for all $x\in\mathbb{R}-\{0\}$. To see if $f'(0)$ is defined we
                check 
                \begin{equation*}
                    \lim_{x\to 0}\frac{f(x+0)-f(0)}{x}=\lim_{x\to
                    0}\frac{x^5\sin(\frac{1}{x^3})}{x}=\lim_{x\to
                    0}x^4\sin(\frac{1}{x^3}).
                \end{equation*}
                Since $|\sin(1/x^3)|\leq 1$ for all $x$, then it follows that
                $0\leq|x^4\sin(1/x^3)|\leq x^4$. Thus by the Squeeze theorem,
                $f'(0)=0$. We can now write 
                \begin{equation*}
                    f'(x)=\begin{cases}
                        5x^4\sin\big(\frac{1}{x^3}\big)-3x\cos\big(\frac{1}{x^3}\big)&\text{if
                        $x\neq 0$} \\
                        0&\text{if $x=0$}.
                    \end{cases}
                \end{equation*}
                By the same reasoning above, for all $x\neq 0$, $f$ is
                differentaible as the product, difference, and composition of
                differentiable functions. Using the product and chain rules we
                obtain 
                \begin{equation*}
                    f''(x)=
                    \big(20x^3-\frac{9}{x^3}\big)\sin\big(\frac{1}{x^3}\big)
                    -(15x+3)\cos\big(\frac{1}{x^3}\big)
                \end{equation*}
                for all $x\neq 0$. Using the same method before, we use the
                limit definition of the derivative to see if $f'(x)$ is
                differentiable at $x=0$. 
                \begin{align*}
                    f''(0)&=\lim_{x\to 0}\frac{f'(x+0)-f'(0)}{x} \\
                    &=\lim_{x\to
                    0}\frac{5x^4\sin\big(\frac{1}{x^3}\big)-3x\cos\big(\frac{1}{x^3}\big)}{x}
                      \\
                    &=\lim_{x\to0}5x^3\sin\big(\frac{1}{x^3}\big)-3\cos\big(\frac{1}{x^3}\big)
                    \\
                    &=\lim_{x\to 0}-3\cos\big(\frac{1}{x^3}\big).
                \end{align*}
                Seeing as the limit of $\cos(1/x^3)$ as $x\to 0$ does not
                exist, then we can conclude that $f''(x)$ is not differentiable
                at $x=0$. Hence, $f$ is twice differentiable. 
            \end{proof}
        \item[AL.2.10] Let $\omega=-\frac{1}{2}+\frac{\sqrt{3}}{2}i$, so
            $\omega$ is a primitive cube root of unity. 
            \begin{enumerate}[(a)]
                \item Let 
                    \begin{align*}
                        \alpha_1 &= \omega^{1/3}+\omega^{-1/3} \\
                        \alpha_2 &= \omega^{2/3}+\omega^{-2/3} \\
                        \alpha_3 &= \omega^{4/3}+\omega^{-4/3}
                    \end{align*}
                    Show that $\alpha_1$ is a root of $f(x)=x^3-3x+1$. (After this 
                    you may assume that $\alpha_2$ and $\alpha_3$ are each roots of $f(x)$.)
                    \begin{proof}
                        To show that $\alpha_1$ is a root of $f(x)$, we must
                        show that $f(\alpha_1)=0$. With this we have
                        \begin{align*}
                            f(\alpha_1)&=\alpha_1^3-3\alpha_1+1 \\
                            &=(\omega^{1/3}+\omega^{-1/3})^3-3(\omega^{1/3}+\omega^{-1/3})+1
                            \\
                            &=\omega+3(\omega^{1/3}+\omega^{-1/3})
                            -3(\omega^{1/3}+\omega^{-1/3})+\omega^{-1}+1 \\
                            &=\omega+\omega^{-1}+1 \\
                            &=\frac{\omega^2+\omega+1}{\omega} \\
                            &=\frac{0}{\omega}=0.
                        \end{align*}
                    \end{proof}
                \item Show that the splitting field for $f(x)$ over
                    $\mathbb{Q}$ is $E=\mathbb{Q}(\alpha_1)$. (Hint consider
                    $(\alpha_1)^2)$. 
                    \begin{proof}
                        Part (a) showed us that $(x-\alpha_1)$ is a factor of
                        $f(x)$. To show that $E=\mathbb{Q}(\alpha_1)$ is the
                        splitting field for $f(x)$ over $\mathbb{Q}$, we must
                        show that $E$ contains all the roots of $f(x)$ and that
                        any proper subfield of $E$ does not contain all the
                        roots of $f(x)$.\par\hspace{4mm} 
                        Clearly $\alpha_1=0+1\cdot\alpha_1\in\mathbb{Q}(\alpha_1)$. The question 
                        now is if we can obtain $\alpha_2$ and $\alpha_3$ from the 
                        elements in $\mathbb{Q}(\alpha_1)$. Notice that $\alpha_1^2=\alpha_2+2$ 
                        and so $\alpha_2\in\mathbb{Q}(\alpha_1)$. Similarly, 
                        $\alpha_2^2=\alpha_3+2$ and so $\alpha_3\in\mathbb{Q}(\alpha_1)$. This 
                        shows that all of the roots of $f$ are in $E$. We now need to show 
                        that it is the smallest field which contains these roots. 
                        The smallest field which contains $\alpha_1, \alpha_2, \alpha_3$ 
                        and $\mathbb{Q}$ is $\mathbb{Q}(\alpha_1, \alpha_2, \alpha_3)$ and so 
                        we must show that $E=\mathbb{Q}(\alpha_1, \alpha_2, \alpha_3)$. Since 
                        the left-to-right inclusion is clear, we will prove the inclusion 
                        $\supseteq$ holds.\par\hspace{4mm} Let 
                        $x\in\mathbb{Q}(\alpha_1, \alpha_2, \alpha_3)$. Then 
                        \begin{equation*}
                            x=a+b\alpha_1+c\alpha_2+d\alpha_3+e\alpha_1\alpha_2
                            +f\alpha_1\alpha_3+g\alpha_2\alpha_3
                            +h\alpha_1\alpha_2\alpha_3
                        \end{equation*}
                        for some $a,b,c,d,e,f,g,h\in\mathbb{Q}$. Based on the equalities 
                        we used before, we have that 
                        \begin{align*}
                            \alpha_2&=\alpha_1^2-2 & \alpha_3&=\alpha_2^2-2 \\
                            & & &=(\alpha_1^2-2)^2-2 \\
                            & & &=\alpha_1^4-4\alpha_1^2-2.
                        \end{align*}
                        Thus $x$ can be written in terms of only  the element $\alpha_1$ using 
                        multiplication, addition, and subtraction. Therefore 
                        $x\in\mathbb{Q}(\alpha_1)$. Hence $E$ is the splitting field 
                        for $f(x)$. 
                    \end{proof}
                \item Determine, with proof, the degree of
                    $\mathbb{Q}(\omega^{1/3})$ over $\mathbb{Q}(\omega)$. 
                    \begin{proof}
                        We first wish to compute $[\mathbb{Q}(\omega^{1/3}):\mathbb{Q}]$. Letting 
                        $x-\omega^{1/3}=0$, then manipulating until all radicals are 
                        removed, we get that $x^6+x^3+1=0$. Seeing that this polynomial 
                        is in $\mathbb{Z}[x]$, its degree is greater than 1, then we note 
                        that the polynomial is irreducible over $\mathbb{Z}_2[x]$ and that 
                        its degree over that field is the same as the degree of the 
                        original polynomial. It then follows from the $\mathbb{Z}_p$ test, 
                        that this polynomial is irreducible over $\mathbb{Q}$ is is therefore 
                        the minimal polynomial of $\omega^{1/3}$ over $\mathbb{Q}$. Therefore 
                        $[\mathbb{Q}(\omega^{1/3}):\mathbb{Q}]=6$. Next, we consider 
                        $[\mathbb{Q}(\omega):\mathbb{Q}]$.\par\hspace{4mm} Doing the same as we 
                        did above, we let $x-\omega=0$ and manipulate until we remove the 
                        radicals to obtain $x^2+x+1=0$. This polynomial is irreducible over 
                        $\mathbb{Z}_2$ and is therefore irreducible over $\mathbb{Q}$. Hence, 
                        $[\mathbb{Q}(\omega):\mathbb{Q}]=2$.\par\hspace{4mm} Finally, we note 
                        that $\omega\in\mathbb{Q}(\omega^{1/3})$ and so 
                        $\mathbb{Q}(\omega)\subseteq\mathbb{Q}(\omega^{1/3})$. 
                        Thus we have that 
                        \begin{align*}
                            &[\mathbb{Q}(\omega^{1/3}):\mathbb{Q}]
                            =[\mathbb{Q}(\omega^{1/3}):\mathbb{Q}(\omega)]
                            [\mathbb{Q}(\omega):\mathbb{Q}] \\
                            &\Rightarrow [\mathbb{Q}(\omega^{1/3}):\mathbb{Q}(\omega)]
                            =\frac{[\mathbb{Q}(\omega^{1/3}):\mathbb{Q}]}
                            {[\mathbb{Q}(\omega):\mathbb{Q}]}=\frac{6}{2}=3
                        \end{align*}
                    \end{proof}
            \end{enumerate}
        \item[AN.1.14]\hfill\par
            \begin{enumerate}[(a)]
                \item Suppose $\{f_n\}$ is a sequence of continuous real-valued
                    functions on a metric space $X$ and $f_n\to f$ uniformly.
                    Let $\{x_n\}$ be a sequence  of points in $X$ converging to
                    $x\in X$. Prove that $\lim_{n\to\infty}f_n(x_n)=f(x)$. 
                    \begin{proof}
                        Let $\varepsilon>0$. Since $\{f_n\}$ converges
                        uniformly on $X$, then there exists $N_1$ such that for
                        any $n>N_1$ we have 
                        \begin{equation*}
                            |f_n(y)-f(y)|<\frac{\varepsilon}{2}
                        \end{equation*}
                        for all $y\in X$. It also follows from the uniform
                        convergence (and that each $f_n$ is continuous) that
                        $f$ is continuous on $X$. Specifically, $f$ is
                        continuous at $x$. This means there is some $\delta$
                        such that for any $y\in X$ such that $|x-y|<\delta$,
                        then $|f(y)-f(x)|<\varepsilon/2$. Moreover, since
                        $x_n\to x$, then there exists some $N_2$ such that for
                        all $n>N_2$, $|x_n-x|<\delta$. Letting $N>\max\{N_1,
                        N_2\}$, then for any $n>N$, we have 
                        \begin{equation*}
                            |x_n-x|<\delta\Rightarrow|f(x_n)-f(x)|<\frac{\varepsilon}{2}
                        \end{equation*}
                        and since $n>N>N_1$, then 
                        \begin{equation*}
                            |f_n(x_n)-f(x_n)|<\frac{\varepsilon}{2}.
                        \end{equation*}
                        Finally, from the triangle inequality it follows that 
                        \begin{equation*}
                            |f_n(x_n)-f(x)|\leq|f_n(x_n)-f(x_n)|+|f(x_n)+f(x)|<\varepsilon. 
                        \end{equation*}
                        Therefore $\lim_{n\to\infty}f_n(x_n)=f(x)$.
                    \end{proof}
                \item Suppose $\{f_n\}$ is a uniformly bounded sequence of
                    Riemann integrable functions on $[a, b]$. Let $F_n:[a,
                    b]\to\mathbb{R}$ be defined by
                    $F_n(x)=\int_{a}^{x}f_n(t)dt$. Prove that $\{F_n\}$
                    contains a uniformly convergent subsequence. 
                    \begin{proof}
                        Since $\{f_n\}$ is uniformly bounded, then there exists
                        some $M\in\mathbb{R}$ such that $|f_n(x)|<M$ for all
                        $n$ and for all $x\in[a, b]$. For each $n$, $f_n$ is
                        Riemann integrable and so for any $x\in[a, b]$, since
                        $\sup_{[a, b]}f_n<M$, it follows that 
                        \begin{equation*}
                            |F_n(x)|=\int_{a}^{x}f_n(t)dt< M(b-a).
                        \end{equation*}
                        Thus $\{F_n\}$ is uniformly bounded on $[a, b]$.
                        Letting $\varepsilon>0$, then we want to show there
                        exists some $\delta$ such that for any $x, y\in[a, b]$,
                        that if $|y-x|<\delta$, then
                        $|F_n(y)-F_n(x)|<\varepsilon$. And this is to apply for
                        all $n$. Note that if $x<y$, then 
                        \begin{equation}
                            |F_n(y)-F_n(x)|=\int_{a}^yf_n(t)dt-\int_{a}^xf_n(t)dt
                            =\int_{x}^yf_n(t)dt<M|y-x|.
                        \end{equation}
                        Hence, if we choose $\delta=\varepsilon/M$, then for
                        any $x<y$ such that $|y-x|<\delta$, it follows from (1)
                        that 
                        \begin{equation*}
                            |F_n(y)-F_n(x)|<\varepsilon.
                        \end{equation*}
                        Since this is for every $n$, then $\{F_n\}$ is
                        equicontinuous. Thus we have that $[a, b]$ is compact,
                        $\{F_n\}$ is uniformly continuous (and thus pointwise
                        continous), and that $\{F_n\}$ is equicontiuous. By
                        Theorem 7.25, $\{F_n\}$ contains a convergent
                        subsequence. 
                    \end{proof}
            \end{enumerate}
        \item[AL.3.2]\hfill\par
            \begin{enumerate}[(a)]
                \item Let $G$ be a group with $|G|=520=2^3\cdot5\cdot13$. Prove
                    that $G$ is not simple. 
                    \begin{proof}
                        By the third Sylow Theorem, we have that the number of
                        Sylow 5-subgroups, $n_5$, and the number of Sylow
                        13-subgroups, $n_{13}$, must satisfy
                        \begin{align*}
                            &n_5\mid2^3\cdot13\;\;\land\;\;n_5\equiv1(\text{mod
                            }5), \\
                            &n_{13}\mid 2^3\cdot 5\;\;\land\;\;
                            n_{13}\equiv1(\text{mod }13).
                        \end{align*}
                        This implies that $n_5\in\{1, 26\}$ and $n_{13}\in\{1,
                        40\}$. Assuming that neither $n_5$ and $n_{13}$ are
                        equal to 1, then it follows that there are
                        $26(5-1)=104$ elements in $G$ of order 5, and
                        $40(13-1)=480$ elements in $G$ of order 13. A total of
                        584 distict elements. As this is a contradiction, it
                        follows that either $n_5=1$ or $n_{13}=1$ is whichever
                        it is presents a nontrivial normal subgroup. Therefore
                        $G$ is not simple. 
                    \end{proof}
                \item Let $G$ be a group with $|G|=36=2^2\cdot3^2$. Prove that
                    $G$ is not simple. 
                    \begin{proof}
                        By the third Sylow Theorem, the number of Sylow
                        3-subgroups, $n_3$, must satisfy
                        \begin{equation*}
                            n_3\mid 2^2\;\;\land\;\; n_3\equiv 1(\text{mod }3).
                        \end{equation*}
                        This implies that $n_3=1$ $n_3=4$. If $n_3=1$, then
                        there is only one such subgroup and by Corollary 20
                        (Dummit \& Foote), it is normal. If $n_3=4$, then let
                        $H$ and $K$ be two distinct subgroups of order $9$. By
                        Proposition 13 (Dummit \& Foote), 
                        \begin{equation*}
                            |HK|=\frac{|H| |K|}{|H\cap K|}=\frac{81}{|H\cap
                            K|}.
                        \end{equation*}
                        Since $HK\subseteq G$, then $|HK|\leq 36$ which implies
                        that $|H\cap K|\geq 3$. Also note that the normalizer
                        $N(H\cap K)$ includes $H$ and $K$ and since these are
                        both distinct groups of order 9, then $N(H\cap K)$ must
                        have at least 18 elements. Moreover, since $N(H\cap
                        K)\leq G$, then its order must divide 36. Thus
                        $|N(H\cap K)|=18$ or $|N(H\cap K)|=36$. If it is 18,
                        then it has index 2 in $G$ any is thus a normal
                        subgroup. If it has order $36$, then $N(H\cap K)=G$
                        which implies that $H\cap K$ is normal in $G$.
                        Therefore $G$ is not simple. 
                    \end{proof}
            \end{enumerate}
        \item[AN.1.6]\hfill\par
            \begin{enumerate}[(a)]
                \item Suppose $a_n\geq 0$ for all $n\in\mathbb{N}$. Let
                    $s_k=\sum_{n=1}^{k}a_n$. Prove $\sum_{n=1}^{\infty}a_n$
                    converges if and only if its sequence of partial sums
                    $\{s_k\}$ is bounded. 
                    \begin{proof}
                        Assume that $\sum a_k$ converges. Then by definition
                        \begin{equation*}
                            \lim_{n\to\infty}s_n=s\in\mathbb{R}.
                        \end{equation*}
                        For contradiction assume that $\{s_k\}$ is not bounded.
                        Then for each $M\in\mathbb{R}$ there exists
                        $k\in\mathbb{N}$ such that $s_k>M$. Thus letting $M=s$,
                        then for all $n\geq k$
                        \begin{equation*}
                            s_n=\sum_{i=1}^n a_i=\sum_{i=1}^k a_i+\sum_{i=k+1}^
                            n a_k\geq s.
                        \end{equation*}
                        This contradicts that
                        $\lim_{n\to\infty}s_n=s$.\par\hspace{4mm} Now assume
                        that $\{s_k\}$ is bounded. Then because the terms of
                        $\{a_k\}$ are all nonnegative, then $\{s_k\}$ is
                        a monotone increasing sequence which is bounded above. Thus
                        $\{s_k\}$ is convergent and so $\sum a_k$ converges. 
                    \end{proof}
                \item Let $\alpha=\lim\sup_{n\to\infty}\sqrt[n]{|a_n|}$. Prove
                    that if $\alpha>1$ then $\sum_{n=1}^{\infty}a_n$ diverges. 
                    \begin{proof}
                        We assume that $\alpha<\infty$. If $\alpha>1$, then
                        since $\alpha$ is the limit of some subsequence, 
                        \begin{equation*}
                            \lim_{k\to\infty}\sqrt[n_k]{|a_{n_k}|}=\alpha>1.  
                        \end{equation*}
                        This implies that there are infinitely many terms
                        $\sqrt[n_k]{|a_{n_k}|}>1$ which implies that for
                        infinitely terms $|a_{n_k}|>1$. Thus $\lim_{n\to\infty} a_n>0$
                        which, by contrapositive, implies that $\sum a_k$ does
                        not converge. 
                    \end{proof}
            \end{enumerate}
        \item[AL.4.3] Let $G$ be a group of order $539=7^2\cdot 11$. 
            \begin{enumerate}[(a)]
                \item Prove that $G$ is Abelian. 
                    \begin{proof}
                        By the third Sylow theorem, the number of Sylow
                        7-subgroups, $n_7$, and the number of Sylow
                        11-subgroups, $n_{11}$ must satisfy
                        \begin{align*}
                            &n_7\mid 11\;\land\;n_7\equiv1(\text{ mod  }11). \\
                            &n_{11}\mid 7\;\land\;n_{11}\equiv1(\text{ mod
                            }11).
                        \end{align*}
                        With these conditions, it follows that $n_7=n_{11}=1$.
                        Thus there is one unique Sylow 7-subgroup, call it $S$,
                        and one unique Sylow 11-subgroup, call it $E$. Since
                        $S$ and $E$ are unique, then $S\unlhd G$ and $E\unlhd G$. 
                        Moreover, since $E$ is of prime order, then $E$ is
                        cyclic and thus abelian.\par\hspace{4mm} Now let $g_1,
                        \dots, g_r$ be representatives of the distinct
                        non-central conjugacy classes. Then by the class
                        equation
                        \begin{equation*}
                            |S|=|Z(S)|+\sum_{i=1}^{r}[S:C_S(g_i)].
                        \end{equation*}
                        Since each conjugacy class is noncentral, then
                        $C_S(g_i)\neq S$ for all $i=1, \dots, r$. We have that
                        $7$ divides the left hand side and $7$ divides
                        $[S:C_S(g_i)]$, then $7$ divides $Z(S)$. Thus $Z(S)$ is
                        nontrivial. Since $S$ is the only Sylow 7-subgroup,
                        then either $|Z(S)|=7$ or $|Z(S)|=49$. If the latter is
                        true, then $S$ is abelian and so since $E\unlhd G$,
                        then $ES\leq G$ and since both groups are normal and
                        there intersection is the identity, then $ES\cong
                        E\times S$ which is abelian. If $|Z(S)|=7$, then
                        $|S/Z(S)|=7$ and thus is cyclic. Thus $S/Z(S)=\langle
                        xZ(S)\rangle$ for some $x\in S$. Letting $g, h\in S$
                        then for some $m, n\in\mathbb{Z}$ we have that
                        $gZ(S)=x^mZ(S)$ and $hZ(S)=x^nZ(S)$. This implies that
                        there exists $z_1, z_2\in Z(S)$ such that $g=x^mz_1$
                        and $h=x^nz_2$. Thus 
                        \begin{align*}
                            gh &=x^mz_1x^nz_2 \\
                            &=x^{m+n}z_1z_2 \\
                            &=x^{m+n}z_2z_1 \\
                            &=x^nz_2x^mz_1 \\
                            &=hg.
                        \end{align*}
                        Thus $S$ is abelian. Since $S\cap E=\{e\}$ then $SE=G$
                        and $G$ is therefore abelian as the product of abelian
                        groups. 
                    \end{proof}
                \item Give an example from each isomorphism class of groups of
                    order $539$. 
                    \begin{solution}
                        Since $G$ was shown to be abelian in part (a), and
                        since it is finite, then by the Fundamental Theorem of
                        Finite Abelian Groups we have that 
                        \begin{equation*}
                            G\cong\mathbb{Z}_{49}\times\mathbb{Z}_{11},
                            \;\;\;\;
                            G\cong\mathbb{Z}_7\times\mathbb{Z}_7\times\mathbb{Z}_{11}.
                        \end{equation*}
                    \end{solution}
                \item For each isomorphism class of groups of order 539,
                    determine (with explanation) the number of elements of
                    order 7. 
                    \begin{solution}
                        For $\mathbb{Z}_{49}\times\mathbb{Z}_{11}$, we note
                        that the order of an element $(a,
                        b)\in\mathbb{Z}_{49}\times\mathbb{Z}_{11}$ is $o(a,
                        b)=lcm(o(a), o(b))$. Thus if 
                        \begin{equation*}
                            o(a, b)=7
                        \end{equation*}
                        and since $o(a)=1, 7, 49$ and $o(b)=1, 11$ then we are
                        looking for $(a, b)$ such that $o(a)=7$ and $o(b)=1$.
                        The only element in $\mathbb{Z}_{11}$ with order 1 is
                        $1$. Thus we are looking for elements of the form $(a,
                        1)$. The only element in $\mathbb{Z}_{49}$ with order
                        1 is $1$. Every other element in $\mathbb{Z}_{49}$
                        either has order 7 or order 49. The elements with order
                        49 are those elements which are relatively prime to 49.
                        The elements which are not relatively prime to 49 are
                        $7, 14, 21, 28, 35, 42$. Thus there are 6 elements of
                        order 7 in
                        $\mathbb{Z}_{49}\times\mathbb{Z}_{11}$.\par\hspace{4mm}
                        For
                        $\mathbb{Z}_7\times\mathbb{Z}_7\times\mathbb{Z}_{11}$,
                        we are looking for elements $(a, b, c)$ such that 
                        \begin{equation*}
                            lcm(o(a), o(b), o(c))=7.
                        \end{equation*}
                        Since $\mathbb{Z}_7\times\mathbb{Z}_7$ is a group of
                        order 49 and no element can have order 49, then there
                        are $49-1=48$ many elements order 7. 
                    \end{solution}
            \end{enumerate}
        \item[AN.1.13] Suppose that $\{f_n\}$ is a sequence of real-valued
            functions on a compact metric space $K$. Suppose $\{f_n\}$ is
            equicontinuous and pointwise convergent. Prove that $\{f_n\}$ is
            uniformly convergent. 
            \begin{proof}
                Let $\varepsilon>0$, then since $\{f_n\}$ is equicontinuous,
                there exists $\delta>0$ such that for all $x, y\in K$ with
                $|x-y|<\delta$, 
                \begin{equation*}
                    |f_n(x)-f_n(y)|<\frac{\varepsilon}{4}
                \end{equation*}
                for all $n\in\mathbb{N}$. Define 
                \begin{equation*}
                    K\subseteq S=\bigcup_{x\in K}N_{\delta}(x).
                \end{equation*}
                Then this forms an open cover of $K$ and since $K$ is compact,
                there exists a finite subcover. That is, there exists  $x_1,
                \dots, x_n\in K$ such that 
                \begin{equation*}
                    K\subseteq\bigcup_{i=1}^{n}N_{\delta}(x_i).
                \end{equation*}
                For each $x_i\in K$, by the assumed pointwise convergence,
                there exists $N_i\in\mathbb{N}$ such that for any $n\geq N_i$,
                we have 
                \begin{equation*}
                    |f_n(x_i)-f(x_i)|<\frac{\varepsilon}{4}. 
                \end{equation*}
                Let $N=\max\{N_1, \dots, N_n\}$ and take $n, m\geq N$. For each
                $x\in K$, there exists $x_i\in K$ such that $x\in
                N_{\delta}(x_i)$ and 
                \begin{align*}
                    |f_n(x)-f_m(x)|&=|f_n(x)-f_n(x_i)+f_n(x_i)-f(x_i)
                    +f(x_i)-f_m(x_i)+f_m(x_i)-f_m(x)| \\
                    &\leq|f_n(x)-f_n(x_i)|+|f_n(x_i)-f(x_i)|+|f(x_i)-f_m(x_i)|
                    +|f_m(x_i)-f_m(x)| \\
                    &<\frac{\varepsilon}{4}+\frac{\varepsilon}{4}+
                    \frac{\varepsilon}{4}+\frac{\varepsilon}{4}=\varepsilon. 
                \end{align*}
                Therefore $\{f_n\}$ is uniformly convergent over $K$. 
            \end{proof}
        \item[AL.3.3] Note: $675=3^3\cdot 5^2$.
            \begin{enumerate}[(a)]
                \item Up to isomorphism, describe all the Abelian groups of
                    order 675.
                    \begin{proof}
                        By the Fundamental Theorem of Abelian groups, every
                        finite abelian group is isomorphic to the direct
                        product of cyclic groups of prime power order. Thus we
                        have the following groups 
                        \begin{align*}
                            \mathbb{Z}_{27}\times\mathbb{Z}_{25}
                            & &\mathbb{Z}_{27}\times\mathbb{Z}_{5}\times\mathbb{Z}_5
                            \\
                            \mathbb{Z}_{3}\times\mathbb{Z}_9\times\mathbb{Z}_{25}
                            & &\mathbb{Z}_{3}\times\mathbb{Z}_{9}\times\mathbb{Z}_{5}
                            \times\mathbb{Z}_{5} \\
                            \mathbb{Z}_{3}\times\mathbb{Z}_{3}\times\mathbb{Z}_{3}
                            \times\mathbb{Z}_{25}
                            & &\mathbb{Z}_{3}\times\mathbb{Z}_{3}\times\mathbb{Z}_{3}
                            \times\mathbb{Z}_{5}\times\mathbb{Z}_{5}. 
                        \end{align*}
                    \end{proof}
                \item Consider
                    $G=\mathbb{Z}_9\times\mathbb{Z}_3\times\mathbb{Z}_{25}$.
                    \begin{enumerate}[i.]
                        \item Determine, with explanation, the number of
                            elements of order 15 in $G$.
                            \begin{proof}
                                Note that $15=3\cdot 5$. Since for any $(a, b,
                                c)\in\mathbb{Z}_{9}\times\mathbb{Z}_{3}\times\mathbb{Z}_{25}$,
                                we have that 
                                \begin{equation*}
                                    |(a, b, c)|=lcm(|a|, |b|, |c|)
                                \end{equation*}
                                then we are looking for $(a, b, c)$ such that
                                $|(a, b, c)|=15$. This gives us the following
                                options
                                \begin{align*}
                                    1\cdot3\cdot 5\to&(0, 1, 5),(0,
                                    1, 10), (0, 1, 15), (0, 1, 20), \\
                                    &(0, 2, 5), (0, 2, 10), (0, 2, 15), (0, 2,
                                    20) \\
                                    3\cdot 1\cdot 5\to&(3, 0, 5), (3, 0, 10),
                                    (3, 0, 15), (3, 0, 20), \\
                                    &(6, 0, 5), (6, 0, 10), (6, 0, 15), (6, 0,
                                    20).
                                \end{align*}
                                There are 16 elements of order 15. 
                            \end{proof}
                        \item Determine, with explanation, the number of
                            elements of order 45 in $G$. 
                            \begin{proof}
                                Note that $45=9\cdot 5$. So we are in
                                need of elements with orders: $9, 1, 5$; 
                                This gives the following:
                                \begin{align*}
                                    9\cdot1\cdot5\to&(1, 0, 5), (1, 0, 10), (1,
                                    0, 15), (1, 0, 20), \\
                                    &(2, 0, 5), (2, 0, 10), (2, 0, 15), (2, 0,
                                    20), \\
                                    &(4, 0, 5), (4, 0, 10), (4, 0, 15), (4, 0,
                                    20), \\
                                    &(5, 0, 5), (5, 0, 10), (5, 0, 15), (5, 0,
                                    20), \\
                                    &(7, 0, 5), (7, 0, 10), (7, 0, 15), (7, 0,
                                    20), \\
                                    &(8, 0, 5), (8, 0, 10), (8, 0, 15), (8, 0,
                                    20).\\
                                \end{align*}
                                There are 24 elements of order 45. 
                            \end{proof}
                    \end{enumerate}
            \end{enumerate}
        \item[AL.6.6] Assume that $R$ is a commutative ring with identity, and
            that $M$ is an ideal of $R$.
            \begin{enumerate}[(a)]
                \item Prove that $M$ is a maximal ideal of $R$ iff $R/M$ is
                    a field. 
                    \begin{proof}
                        ($\Rightarrow$) Assume that $M$ is a maximal ideal of
                        $R$. Let $a\in R$ and define 
                        \begin{equation*}
                            S=\{ar+m:r\in R, m\in M\}.
                        \end{equation*}
                        We claim that $S$ is an ideal of $R$ containing $M$.
                        \begin{equation*}
                            \text{[Supply proof if there is time]}
                        \end{equation*}
                        Since $M$ is maximal, then $S=R$ and thus $1\in S$.
                        Hence, for some $r\in r$ and $m\in M$ we have that
                        $1=ar+m$ which implies that $(a+M)(r+M)=1+M$. Since
                        $r\in R$ was arbirary, then every element of $R/M$ is
                        a unit and therefore $R/M$ is a field.\par\hfill
                        ($\Leftarrow$) Assume that $R/M$ is a field and that
                        $B$ is an ideal of $R$ properly containing $M$. Let
                        $b\in B-M$. Then $b+M$ is a nonzero element of $R/M$
                        and is therefore a unit. Then for some $c\in R$ we have
                        that $(b+M)(c+M)=bc+M=1+M$. This implies that $1-bc\in
                        M\subset B$. And since $b\in B$ and $c\in R$, then
                        $bc\in B$. Thus $(1-bc)+bc=1\in B$. Hence $B=R$.
                        Therefore $M$ is a maximal ideal. 
                    \end{proof}
                \item Assume that $R$ is a PID. Prove that $r\in R$ is prime
                    iff $R/(r)_i$ is a field. 
                    \begin{proof}
                        ($\Rightarrow$) Assume that $r\in R$ is prime. Then
                        $(r)_i$ is a prime ideal. We want to show that $(r)_i$
                        is maximal. Let $I$ be an ideal of $R$ that properly
                        contains $(r)_i$. Since $R$ is a PID, then for some
                        $a\in R$, $I=(a)_i$. By the assumed inclusion, $r\in
                        (a)_i$ and so for some $t\in R$ we have that $r=ta$.
                        Thus $ta\in(r)_i$, which is a prime ideal, and so
                        either $t\in(r)_i$ or $a\in(r)_i$. If $t\in(r)_i$, then
                        $t=rs$ for some $s\in R$ and so $r=ta=rsa$ thus 
                        \begin{equation*}
                            r(1-sa)=0\rightarrow 1=sa.
                        \end{equation*}
                        Thus $1\in (a)_i$ and so $(a)_i=R$. Otherwise, if
                        $a\in(r)_i$, then $(r)_i=(a)_i$ which contradicts the
                        proper inclusion we assumed. Therefore $(r)_i$ is
                        maximal and $R/(r)_i$ is a field.\hfill\par
                        ($\Leftarrow$) Assume that $R/(r)_i$ is a field. Let
                        $a, b\in R$ be nonzero such that $ab\in (r)_i$. Then
                        $(ab+(r)_i)=(r)_i$. This implies that
                        $(a+(r)_i)(b+(r)_i)=(r)_i$. However, since $R/(r)_i$ is
                        a field, then there are no zero divsors and thus either
                        $(a+(r)_i)=(r)_i$ or $(b+(r)_i)=(r)_i$. That is, either
                        $a\in (r)_i$ or $b\in (r)_i$. Therefore $(r)_i$ is
                        a prime ideal and so $r\in R$ is prime. 
                    \end{proof}
            \end{enumerate}
        \item[AL.6.7] Let $I=(x^4+7x^2)_i$ in $\mathbb{Q}[x]$.
            \begin{enumerate}[(a)]
                \item Assume that $J$ is an ideal of $\mathbb{Q}[x]$. Prove
                    that $I\subseteq J$ iff $J=(f(x))_i$ for some monic
                    $f(x)\in\mathbb{Q}[x]$ such that $f(x)\mid x^4+7x^2$. 
                    \begin{proof}
                        ($\Rightarrow$) Assume that $I\subseteq J$. We first
                        note that since $\mathbb{Q}$ is a field, then
                        $\mathbb{Q}$ is a PID. Thus for some
                        $f(x)\in\mathbb{Q}[x]$, we have that $J=(f(x))_i$.
                        Since $I\subseteq J$, then $x^4+7x^2=h(x)f(x)$ for some
                        $h(x)\in\mathbb{Q}[x]$. Hence $f(x)\mid
                        x^4+7x^2$.\hfill\par ($\Leftarrow$) Assume that
                        $J=(f(x))_i$ for some monic $f(x)\in\mathbb{Q}[x]$ such
                        that $f(x)\mid x^4+7x^2$. This implies that
                        $x^4+7x^2=h(x)f(x)$ for some $h(x)\in\mathbb{Q}[x]$.
                        Hence $x^4+7x^2\in(f(x))_i$. Thus $I\subseteq J$.  
                    \end{proof}
                \item Determine, with explanation, all ideals $J$ in
                    $\mathbb{Q}[x]$ such that $I\subseteq J$. 
                    \begin{proof}
                        By part (a), if $I\subseteq J$, then $J$ must be
                        generated by a monic factor of $x^4+7x^2$. Given that
                        $x^4+7x^2=x^2(x^2+7)$, then the only possiblities for
                        $J$ are
                        \begin{align*}
                            J=(x)_i & &J=(x^2)_i \\
                            J=(x^2+7)_i & &J=(x^3+7x)_i \\
                            J=(x^4+7x)_i.
                        \end{align*}
                    \end{proof}
                \item Determine, with explanation, all ideals $J$ from (b) such
                    that $\mathbb{Q}[x]/J$ is a field, and express each of
                    these fields in the form $\mathbb{Q}(d)$ for some
                    $d\in\mathbb{C}$. 
                    \begin{proof}
                        In order for $\mathbb{Q}[x]/J$ to be a field, we need
                        for $J$ to be generated by an irreducible polynomial
                        over $\mathbb{Q}$. Of the options from (b), we have
                        $J=(x)_i$ and $J=(x^2+7)_i$. This gives the following
                        fields: $\mathbb{Q}(0)=\mathbb{Q}$ and
                        $\mathbb{Q}(\sqrt{7} )$. 
                    \end{proof}
            \end{enumerate}
        \item[AL.6.8] Assume that $F$ is finite field. 
            \begin{enumerate}[(a)]
                \item Indicate why $\text{char}(F)$ is a prime number, $p$.
                    \begin{proof}
                        For contradiction, assume that $\text{char}(F)=mn$ for
                        $m, n\in\mathbb{N}$ such that $m, n>1$. Then for any
                        $\alpha\in F$ we have that 
                        \begin{equation*}
                            (mn)\alpha=m(n\alpha)=m(n\cdot 1\cdot\alpha)=0.
                        \end{equation*}
                        Since $n\alpha=\beta\in F$, then $m\beta=0$. Since
                        $m<n$, then this contradicts that $mn$ is the smallest
                        integer such that $mn\cdot 1=0$. Therefore the
                        characteristic must be a prime $p$.  
                    \end{proof}
                \item Indicate why $F$ is a vector space over $\mathbb{Z}_p$. 
                    \begin{proof}
                        Let $m, n\in\mathbb{Z}_p$ and $\alpha, \beta\in F$.
                        Then we have that
                        \begin{align*}
                            m\alpha\in F \\(m+n)\alpha = m\alpha+n\alpha \\
                            (\alpha+\beta)m=m\alpha+m\beta.
                        \end{align*}
                        This coupled with the facts that $F$ is an abelian
                        additive group shows that $F$ is a vector space over
                        $\mathbb{Z}_p$. 
                    \end{proof}
                \item Assume that $[F:\mathbb{Z}_p]=n$. Determine (with proof)
                    $|F|$. 
                    \begin{proof}
                        By part (b), $F$ is a vector space over $\mathbb{Z}_p$.
                        Let $\{1, \alpha_1, \dots, \alpha_n\}$ be a basis for
                        this vector space. Then $\text{span}\{1, \alpha_1,
                        \dots\alpha_n\}=F$. Thus every element of $F$ can be
                        expressed as a linear combination of the $n$ many basis
                        elements. Each basis element can take on any one of $p$
                        many coefficients from $\mathbb{Z}_p$. Thus, there are
                        $p^n$ many possible linear combinations and so
                        $|F|=p^n$. 
                    \end{proof}
                \item Explain why $F$ is the splitting field over
                    $\mathbb{Z}_p$ of a seperable polynomial. 
                    \begin{proof}
                        Let $\alpha\in F$, then if $\alpha=0$, we have that
                        $\alpha^{p^n}=0=\alpha$. Note that the nonzero elements
                        of $F$ form a group under multiplication of order
                        $p^n-1$ and from this
                        it follows that for any nonzero $\alpha\in F$, we have
                        that $\alpha^{p^n-1}=1$ and so $\alpha^{p^n}-\alpha=0$.
                        Now consier the polynomial $x^{p^n}-x$ in
                        $\mathbb{Z}_p$. This polynomial has at most $p^n$ roots
                        and each element of $F$ is a root, of which there are
                        exactly $p^n$ many of them. Therefore $F$ is the
                        splitting field of $x^{p^n}-x$ and since all the roots
                        are distinct, then the polynomial is seperable. 
                    \end{proof}
            \end{enumerate}
        \item[AL.6.9]\hfill\par
            \begin{enumerate}[(a)]
                \item Find the splitting field of $x^3-2$ over $\mathbb{Q}$.
                    Also, if we denote the splitting field by $L$, then find
                    $[L:\mathbb{Q}]$. 
                    \begin{proof}
                        Using DeMoivre's theorem, we can write
                        $x=r(\cos\theta+i\sin\theta)$ and so we have that
                        $r^3(\cos3\theta+i\sin3\theta)=2$. This implies that
                        $r=\sqrt[3]{2}$ and $\theta=0, 2\pi/3, 4\pi/3$. Thus,
                        the roots of $x^3-2$ are 
                        \begin{equation*}
                            c_1=\sqrt[3]{2}, \;\;c_2=\omega\sqrt[3]{2},
                            \;\;c_2=\omega^2\sqrt[3]{2},    
                        \end{equation*}
                        where $\omega=-\frac{1}{2}+i\frac{\sqrt{3}}{2}$. The
                        splitting field must therefore contain $\sqrt[3]{2}$.
                        Note that $1, \sqrt[3]{2}, \sqrt[3]{4}$ is a basis of
                        $\mathbb{Q}(\sqrt[3]{2})$ over $\mathbb{Q}$. Also note
                        that if $i\sqrt{3}\in\mathbb{Q}$, then
                        $-\frac{1}{2}+i\frac{\sqrt{3}}{2}=\omega\in\mathbb{Q}$.
                        Since $1, i\sqrt{3}$ is a basis for
                        $\mathbb{Q}(\sqrt[3]{2}, i\sqrt{3})$ over
                        $\mathbb{Q}(\sqrt[3]{2})$, then it follows that the
                        product of these two bases give us a basis: $1,
                        \sqrt[3]{2}, \sqrt[3]{4}, i\sqrt{3}, \sqrt[3]{2}
                        i\sqrt{3}, \sqrt[3]{4} i\sqrt{3}$ over $\mathbb{Q}$.
                        Letting $L=\mathbb{Q}(\sqrt[3]{2}, i\sqrt{3})$, then
                        clearly $x^3-2$ splits over $L$. Given the number of
                        basis elements, it follows that $[L:\mathbb{Q}]=6$.           
                    \end{proof}
                \item If we denote the roots of $x^3-2$ by $c_1, c_2, c_3$,
                    then express each element of $G(L/\mathbb{Q})$ as
                    a permutation of the subscripts of $c_1, c_2, c_3$. 
                    \begin{proof}
                        Given any $\sigma\in\text{Aut}(L/\mathbb{Q})$, it is
                        completely determined by where it sends the basis
                        elements. The pair $\{\sqrt[3]{2}, \omega\}$ generates
                        the basis mentioned above and so we
                        will then consider what each map does to $\sqrt[3]{2}$
                        and $\omega$. We have 6 possibilities
                        \begin{align*}
                            &\sigma_1:=\begin{cases}
                                \sqrt[3]{2}&\mapsto\sqrt[3]{2} \\
                                \omega&\mapsto \omega
                            \end{cases} & &
                            &\sigma_2:=\begin{cases}
                                \sqrt[3]{2}&\mapsto\omega\sqrt[3]{2} \\
                                \omega&\mapsto\omega  
                            \end{cases} & &
                            &\sigma_3:=\begin{cases}
                                \sqrt[3]{2}&\mapsto\omega^2\sqrt[3]{2} \\
                                \omega&\mapsto\omega
                            \end{cases} \\
                            &\sigma_4:=\begin{cases}
                                \sqrt[3]{2}&\mapsto\sqrt[3]{2} \\
                                \omega&\mapsto\omega^2
                            \end{cases} & & 
                            &\sigma_5:=\begin{cases}
                                \sqrt[3]{2}&\mapsto\omega\sqrt[3]{2} \\
                                \omega&\mapsto\omega^2
                            \end{cases} & &
                            &\sigma_6:=\begin{cases}
                                \sqrt[3]{2}&\mapsto\sqrt[3]{2} \\
                                \omega&\mapsto\omega^2.
                            \end{cases}
                        \end{align*}
                        With these in hand, we see the following
                        correspondances with the index permutations:
                        \begin{align*}
                            \sigma_1:(1) & &\sigma_2:(123) & &\sigma_3:(132) \\
                            \sigma_4:(23) & &\sigma_5:(12) & &\sigma_6:(13).
                        \end{align*}
                    \end{proof}
                \item For $H=\langle(123)\rangle$, find the subfield of $L$
                    that corresponds to $H$ as given by the Fundamental Theorem
                    of Galois Theory. 
                    \begin{proof}
                        By part (b), we see that $H=\langle(123)\rangle$
                        corresponds to the following automorphisms $\{\sigma_1,
                        \sigma_2, \sigma_3\}$. By the definitions of each of
                        these mappings, we see that each one of them fix
                        $\omega$ and only fix $\omega$. Therefore the
                        corresponding fixed field is $\mathbb{Q}(\omega)$.   
                    \end{proof}
            \end{enumerate}
        \item[AL.6.10] Assume that $F$ is a subfield of $\mathbb{C}$, $n$ is
            a positive integer, and that $\xi$ is a primitive $n$th root of
            unity. Assume that $\xi\in F$. 
            \begin{enumerate}[(a)]
                \item Let $c\in F$, let $E$ be the splitting field of $x^n-c$
                    over $F$, and let $\delta$ be any root of $x^n-c$. Prove
                    that $E=F(\delta)$. 
                    \begin{proof}
                        Using DeMoive's Theorem, we can deduce that the roots
                        of $x^n-c$ are of the form $\xi^k\sqrt[n]{c}$ for
                        $0\leq k\leq n-1$. Seeing as $\xi\in F$, then adjoining
                        $\delta=\xi^m\sqrt[n]{c}$ to $F$, we obtain all the
                        terms needed to generate all $n$ roots of the
                        polynomial. And so $x^n-c$ splits in $F(\delta)$.  
                    \end{proof}
                \item Prove that $G(E/F)$ is Abelian. 
                    \begin{proof}
                        Because $\xi\in F$, then any $\sigma\in\text{Aut}(E/F)$
                        will fix $\xi$. Thus the only basis elements to be
                        permuted are those generated by $\sqrt[n]{c}$. This
                        implies that $\sigma_i(\sqrt[n]{c})=\xi^i\sqrt[n]{c}$
                        for all $0\leq i\leq n-1$. This induces a cyclic group
                        of automorphisms and therefore $G(E/F)$ is Abelian. 
                    \end{proof}
            \end{enumerate}
    \end{enumerate}
\end{document}
