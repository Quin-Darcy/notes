\documentclass[12pt]{article}
\usepackage[margin=1in]{geometry}
\usepackage{graphicx}
\usepackage{amsmath}
\usepackage{amsthm}
\usepackage{amsfonts}
\usepackage{amssymb}
\usepackage{array}
\usepackage{enumerate}
\usepackage{fancyhdr}
\pagestyle{fancy}
\fancyhf{}
\rhead{Darcy}
\lhead{STAT 215A}
\rfoot{\thepage}
\setlength{\headheight}{10pt}

\newenvironment{solution}
{\renewcommand\qedsymbol{$\blacksquare$}\begin{proof}[Solution]}
{\end{proof}}
\newenvironment{psmall}{\left(\begin{smallmatrix}}{\end{smallmatrix}\right)}

\begin{document}
    \thispagestyle{empty}\hrule

    \begin{center}
        \vspace{.4cm} { \large STAT 215A}
    \end{center}
    {Name:\ Quin Darcy \hspace{\fill} Due Date: NONE   \\
    { Instructor:}\ Dr. Cetin \hspace{\fill} Assignment:
    PS03 \\ \hrule}

    \begin{enumerate}
        \item Out of 12 balls in an urn, 5 of them are green. Consider a sample
            size of $n=4$ without replacement. Let $A_k$ be the event that
            a sample of size 4 contains exactly $k$ green balls for $0\leq
            k\leq 4$. Moreover, let $B_j$ be the event that the ball selected
            at the $j$th step is green for $1\leq j\leq 4$.
            \begin{enumerate}[(a)]
                \item Determine $P(B_j)$ for each $1\leq j\leq 4$. 
                    \begin{solution}
                        Using Lemma 4 from section 1.4, we get the following
                        probabilities 
                        \begin{align*}
                            P(B_1) &= \frac{5}{12} \approx 0.42\\
                            P(B_2) &= P(B_2\mid B_1)P(B_1)+P(B_2\mid
                            B_1^c)P(B_1^c) \\
                            &=
                            \frac{4}{11}\frac{5}{12}
                            +\frac{5}{11}\frac{7}{12}
                            \approx 0.42 \\
                            P(B_3)&=P(B_3\mid B_2\cap B_1)P(B_2\cap
                            B_1)+P(B_3\mid B_2^c\cap B_1)P(B_2^c\cap
                            B_1)\\
                            &\hspace{4mm}+P(B_3\mid B_2\cap B_1^c)P(B_2\cap
                            B_1^c)+P(B_3\mid B_2^c\cap B_1^c)P(B_2^c\cap B_1^c) \\
                            &=\frac{3}{10}\frac{4}{11}\frac{5}{12}
                            +\frac{4}{10}\frac{7}{11}\frac{5}{12}
                            +\frac{4}{10}\frac{5}{11}\frac{7}{12}
                            +\frac{5}{10}\frac{6}{11}\frac{7}{12} 
                            \approx 0.42
                        \end{align*}
                    \end{solution}
                \item Compute $P(A_3\mid B_2)$.
                    \begin{solution}
                        We want to know the probability of our sample size
                        containing 3 green balls given that the second ball
                        selected was green. This can happen in the following
                        ways: balls 1, 2, and 3 are green; balls 1, 2, and
                        4 are green; balls 2, 3, and 4 are green. The
                        probabilities associated with these three events are
                        \begin{equation*}
                            \frac{5}{12}\frac{4}{11}\frac{3}{10}\frac{7}{9}
                            +\frac{5}{12}\frac{4}{11}\frac{7}{10}\frac{3}{9}
                            +\frac{7}{12}\frac{5}{11}\frac{4}{10}\frac{3}{9}
                            \approx 0.12
                        \end{equation*}
                    \end{solution}
                \item Compute $P(B_j\mid A_3)$ for each $1\leq j\leq 4$. 
                    \begin{solution}
                        
                    \end{solution}
                \item Would you answers to any of the parts (a)-(c) above
                    change if the sampling is done with replacemet?
                    \begin{solution}
                        
                    \end{solution}
            \end{enumerate} 
    \end{enumerate}
\end{document}
