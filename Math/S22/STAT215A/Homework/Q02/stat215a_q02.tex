\documentclass[12pt]{article}
\usepackage[margin=1in]{geometry}
\usepackage{graphicx}
\usepackage{amsmath}
\usepackage{amsthm}
\usepackage{amsfonts}
\usepackage{amssymb}
\usepackage{array}
\usepackage{enumerate}
\usepackage{fancyhdr}
\pagestyle{fancy}
\fancyhf{}
\rhead{Darcy}
\lhead{STAT 215A}
\rfoot{\thepage}
\setlength{\headheight}{10pt}

\newenvironment{solution}
{\renewcommand\qedsymbol{$\blacksquare$}\begin{proof}[Solution]}
{\end{proof}}
\newenvironment{psmall}{\left(\begin{smallmatrix}}{\end{smallmatrix}\right)}

\begin{document}
    \thispagestyle{empty}\hrule

    \begin{center}
        \vspace{.4cm} { \large STAT 215A}
    \end{center}
    {Name:\ Quin Darcy \hspace{\fill} Due Date: 04/07/2022   \\
    { Instructor:}\ Dr. Cetin \hspace{\fill} Assignment:
    Quiz 02 \\ \hrule}

    \begin{enumerate}
        \item Let $X_1, X_2, \dots, X_{10}$ be 10 independent random variables
            on a probability space $(\Omega, \mathcal{F}, P)$, each with
            a continuous unifrom distribution on the interval $(0, a)$ where
            $a>0$. 
            \begin{enumerate}[(a)]
                \item Let $Y=\max\{X_i:i=1, 2, \dots, 10\}$. More explicitly,
                    $Y(\omega)=\max\{X_i(\omega):i=1, 2, \dots, 10\}$ for
                    $\omega\in\Omega$. Show that $Y$ is a random variable on
                    $(\Omega, \mathcal{F}, P)$. In other words, prove that
                    $\{\omega\in\Omega:Y(\omega)\leq y\}\in\mathcal{F}$ for all
                    $y\in\mathbb{R}$. 
                    \begin{proof}
                        We begin by noting that if $x\in\mathbb{R}$, then
                        $\{\omega\in\Omega:X(\omega)\leq x\}=X^{-1}((-\infty,
                        x])\in\mathcal{F}$ since $X$ is a random variable and
                        thus a measurable function. Taking that same
                        $x\in\mathbb{R}$, we have that 
                        \begin{align*}
                            \{\omega\in\Omega:Y(\omega)\leq x\}
                            &=\{\omega\in\Omega:\max\{X_1(\omega), \dots,
                            X_{10}(\omega)\} \leq x\} \\ 
                            &=\{\omega\in\Omega:(X_1(\omega)\leq
                            x)\land\cdots\land(X_{10}(\omega)\leq x)\} \\
                            &=\{\omega\in\Omega:X_1(\omega)\leq
                            x\}\cap\cdots\cap\{\omega\in\Omega:X_{10}(\omega)\leq
                            x\} \\
                            &\in\mathcal{F},
                        \end{align*}
                        as the intersection of a sequence of elements of
                        $\mathcal{F}$, and this follows from the properties of
                        $\sigma$-algebras in which $\mathcal{F}$ is one on
                        $\Omega$. Therefore, for all $x\in\mathbb{R}$,
                        $Y^{-1}((-\infty, x])=\{\omega\in\Omega:Y(\omega)\leq
                        x\}\in\mathcal{F}$ and so $Y$ is an
                        $\mathcal{F}$-measurable function and thus $Y$ is a random
                        variable on $(\Omega, \mathcal{F})$. 
                    \end{proof}
                \item In fact, $Y$ is a continuous random variable. Justify
                    this by writing the cdf of $Y$, $F_Y(y)$, as an integral of
                    an explicit nonnegative function $f_Y(\cdot)$ such that
                    $F_Y(y)=\int_{-\infty}^{y}f_Y(t)dt$, for all
                    $y\in\mathbb{R}$. Recall that $f_Y(\cdot)$ is a pdf for
                    $Y$.
                    \begin{solution}
                        Since for each $i\in\{1, \dots, 10\}$,
                        $X_i\sim\textit{Unif }(0,
                        a)$, then the pdf of $X_i$ is 
                        \begin{equation*}
                            f_{X_i}(x)=\begin{cases}
                                \frac{1}{a} &\text{if $0<x<a$} \\
                                0 &\text{otherwise}.
                            \end{cases}
                        \end{equation*}
                        for all $x\in\mathbb{R}$ and for all $i\in\{1, \dots,
                        10\}$. And the cdf of $X_i$ is
                        \begin{equation*}
                            F_{X_i}(x)=\int_{-\infty}^{x}f_{X_i}(t)dt=\begin{cases}
                                0 &\text{if $x\leq 0$} \\
                                \frac{x}{a} &\text{if $0<x<a$} \\
                                1 &\text{if $a\leq x$}
                            \end{cases}.
                        \end{equation*}
                        With this, we have that 
                        \begin{align}
                            F_Y(x) &= P(\{\omega\in\Omega:Y(\omega)\leq x\}) \\
                            &=P(\{\omega\in\Omega:\max\{X_i(\omega), \dots,
                            X_{10}(\omega)\}\leq x\}) \\
                            &=P(\{\omega\in\Omega:(X_1(\omega)\leq
                            x)\land\cdots\land(X_{10}(\omega)\leq x)\}) \\
                            &=P(\{\omega\in\Omega:X_1(\omega)\leq
                                x\}\cap\cdots\cap\{\omega\in\Omega:X_{10}(\omega)\leq
                            x\}) \\
                            &=P(\{\omega\in\Omega:X_1(\omega)\leq x\})\cdots
                            P(\{\omega\in\Omega:X_{10}(\omega)\leq x\}) \\
                            &=\prod_{i=1}^{10}F_{X_i}(x) \\
                            &=(F_{X_1}(x))^{10}
                        \end{align}
                        Note that (5) holds since the $X_i$'s are
                        independent. Additionally, the last equality holds
                        since the cdf's of each $X_i$ are all equal so we
                        arbitrarily selected $F_{X_1}(x)$. Finally, we can
                        obtain $f_Y(x)$ by differentiating and getting 
                        \begin{equation*}
                            f_Y(x)=\frac{d}{dx}F_Y(x)
                            =\frac{d}{dx}(F_{X_1}(x))^{10}=10(F_{X_1}(x))^9F'_{X_1}(x)
                            =10(F_{X_1}(x))^9f_{X_1}(x).
                        \end{equation*}
                        Therefore
                        \begin{equation*}
                            F_Y(x)=10\int_{-\infty}^{x}(F_{X_1}(t))^9f_{X_1}(t)dt.
                        \end{equation*}
                    \end{solution}
                \item Compute $P(Y>2a/3)$. 
                    \begin{solution}
                        We begin by noting 
                        \begin{equation*}
                            P(Y>2a/3)=1-P(Y\leq 2a/3)=F_Y(2a/3).
                        \end{equation*}
                        Then using (7) from above we have that 
                        \begin{equation*}
                            F_Y(2a/3)=(F_X(2a/3))^{10}.
                        \end{equation*}
                        Since $0<\frac{2a}{3}<a$, then $F_X(2a/3)=2/3$. Thus 
                        \begin{equation*}
                            P(Y>2a/3)=1-\bigg(\frac{2}{3}\bigg)^{10}\approx
                            0.983.
                        \end{equation*}
                    \end{solution}
            \end{enumerate}
        \item An urn contains 5 balls numbered 1, 2, 3, 4, and 5. We randomly
            select two ballss, one after another and without replacement, and
            record the number on each ball. Let $X_1$ denote the number on the
            first ball selected, and $X_2$ denote the number on the second ball
            selected. So, clearly $X_1$ and $X_2$ are two discrete random
            variables on $(\Omega, \mathcal{F})$ where $\Omega=\{1, 2, 3, 4,
            5\}$ and $\mathcal{F}$ is the power set of $\Omega$. Moreover,
            define $Z=\min\{X_1, X_2\}$. One can check that $Z$ is also
            a discrete random variable on $(\Omega, \mathcal{F})$. 
            \begin{enumerate}[(a)]
                \item Determine the pmf of $Z$. 
                    \begin{solution}
                        To begin, we first note that there are $5$ ways to
                        select the first ball, and 4 ways to select the second
                        ball. This gives us a sample space $|\Omega|=20$. For
                        any $x\in\{1, 2, 3, 4, 5\}$, there are two ways in
                        which the $Z=x$.
                        However, if $x=5$, then there is no outcome in which
                        $Z=x$ and so $p_Z(Z\geq 5)=0$. Thus for $x\in\{1, 2, 3,
                        4\}$, then either $X_1=x<X_2$ or $X_2=x<X_1$. With the
                        two options mentioned, there remains $5-x$ many numbers
                        that are bigger than $x$ in the set $\{1, 2, 3, 4,
                        5\}$. Thus 
                        \begin{equation*}
                            p_Z(x)=\begin{cases}
                                \frac{2(5-x)}{20} &\text{if $x\in\{1, 2, 3, 4\}$} \\
                                0 &\text{otherwise}.
                            \end{cases}
                        \end{equation*}
                    \end{solution}
                \item Compute $E[Z]$ and $E[Z^2]$.
                    \begin{solution}
                        To compute the expexted value of $Z$, we have 
                        \begin{equation*}
                            E[Z]=\sum_{x=1}^{5}xp_Z(x)
                            =\frac{8}{20}+\frac{12}{20}+\frac{12}{20}+\frac{8}{20}=
                            2.
                        \end{equation*}
                        Letting $g(x)=x^2$, then using LOTUS, we can compute
                        \begin{equation*}
                            E[Z^2]=\sum_{x=1}^{5}g(x)p_Z(x)=\frac{100}{20}=50.
                        \end{equation*}
                        To be honest, this last answer does not seem correct. 
                    \end{solution}
            \end{enumerate}
    \end{enumerate}
\end{document}
