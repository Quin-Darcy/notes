\documentclass[12pt]{article}
\usepackage[margin=1in]{geometry}
\usepackage{graphicx}
\usepackage{amsmath}
\usepackage{amsthm}
\usepackage{amsfonts}
\usepackage{amssymb}
\usepackage{array}
\usepackage{enumerate}
\usepackage{fancyhdr}
\usepackage{enumitem}
\pagestyle{fancy}
\fancyhf{}
\rhead{Darcy}
\lhead{STAT 215A}
\rfoot{\thepage}
\setlength{\headheight}{10pt}

\newenvironment{solution}
{\renewcommand\qedsymbol{$\blacksquare$}\begin{proof}[Solution]}
{\end{proof}}
\newenvironment{psmall}{\left(\begin{smallmatrix}}{\end{smallmatrix}\right)}

\begin{document}
    \thispagestyle{empty}\hrule

    \begin{center}
        \vspace{.4cm} { \large STAT 215A}
    \end{center}
    {Name:\ Quin Darcy \hspace{\fill} Due Date: 02/22/2022   \\
    { Instructor:}\ Dr. Cetin \hspace{\fill} Assignment:
    Homework 02 \\ \hrule}

    \begin{enumerate}
        \item Depending on the risk level of home insurance contacts/policies,
            an insurance company divided a city into three regions: high risk,
            medium risk, and low risk.
            \begin{itemize}
                \item About 10\% of its customers live in the high risk region
                    where about one fifth of the customers filed a claim last
                    year.
                \item About 40\% of its customers live in the medium risk
                    region where about 6\% of the customers filed a claim last
                    year. 
                \item Only about 2\% of the customers in the low risk region
                    filed a claim last year. 
            \end{itemize}
            One of the customers is selected at random. Determine the
            probability that 
            \begin{enumerate}[(a)]
                \item the customer filed a claim last year
                    \begin{solution}
                        We begin by labelling some of the events with the first
                        being $C$ and it will denote the event of filing
                        a claim. The other labels
                        \begin{align*}
                            &A_1:\text{Lives in high risk region} \\
                            &A_2:\text{Lives in medium risk region} \\
                            &A_3:\text{Lives in low risk region}
                        \end{align*}
                        Then based on the given information we have that 
                        \begin{align*}
                            &\mathbb{P}(A_1)=0.1 & &\mathbb{P}(C\mid A_1)=0.2 \\
                            &\mathbb{P}(A_2)=0.4 & &\mathbb{P}(C\mid A_2)=0.06 \\
                            &\mathbb{P}(A_3)=0.5 & &\mathbb{P}(C\mid A_3)=0.02.
                        \end{align*}
                        With this we can calculate $\mathbb{P}(C)$ by using the
                        total probability formula
                        \begin{equation*}
                            \mathbb{P}(C)=\sum_{i}^3\mathbb{P}(A_i)\mathbb{P}(C\mid
                            A_i)=0.02+0.024+0.01=0.054.
                        \end{equation*}
                    \end{solution}
                \item the customer lived in the high risk region, given that
                    the customer filed a claim last year. 
                    \begin{solution}
                        We want to calculate the probability of $A_1$ given
                        $C$. Using the information above we get
                        \begin{equation*}
                            \mathbb{P}(A_1\mid C)=\frac{\mathbb{P}(C\mid
                            A_1)\mathbb{P}(A_1)}{\sum_{i=1}^3\mathbb{P}(C\mid
                            A_i)\mathbb{P}(A_i)}=\frac{\mathbb{P}(C\mid
                            A_1)\mathbb{P}(A_1)}{\mathbb{P}(C)}=\frac{0.02}{0.054}=0.37.
                        \end{equation*}
                    \end{solution}
            \end{enumerate} 
        \item Assume that three events $A, B$, and $C$ are pairwise disjoint,
            each with positive probability. Prove or disprove each of the
            following statements: 
            \begin{enumerate}[(a)]
                \item The events $A\cup B$ and $C$ are independent.
                    \begin{solution}
                        Consider an experiment where a coin is tossed twice.
                        The possible outcomes of this experiment are
                        $\Omega=\{HH, HT, TH, TT\}$. Let $A$ denote the event
                        that a heads occurred on the first flip. Then $A=\{HH,
                        HT\}$. Let $B$ denote the event that a heads occurred
                        on the last flip, then $B=\{HH, TH\}$. Finally, let
                        $C$ denote the event that both flips were the same,
                        then $C=\{HH, TT\}$. From this it follows that 
                        \begin{equation*}
                            (A\cup B)\cap C = {HH}.
                        \end{equation*}
                        Thus 
                        \begin{equation*}
                            \mathbb{P}((A\cup B)\cap C)=1/4
                        \end{equation*}
                        whereas
                        \begin{equation*}
                            \mathbb{P}(A\cup B)\mathbb{P}(C)=(3/4)(1/4)=3/16.
                        \end{equation*}
                        Therefore $\mathbb{P}((A\cup B)\cap
                        C)\neq\mathbb{P}(A\cup B)\mathbb{P}(C)$ and so $A\cup
                        B$ and $C$ are not
                        independent. 
                    \end{solution}
                \item The events $A\cap B$ and $C\cap B$ are independent. 
                    \begin{solution}
                        Using the same example as above, we have that $A\cap
                        B=\{HH\}$ and $B\cap C=\{HH\}$ and so $(A\cap
                        B)\cap(B\cap C)=\{HH\}$. Thus
                        \begin{equation*}
                            \mathbb{P}((A\cap B)\cap (B\cap C))=1/4,
                        \end{equation*}
                        whereas
                        \begin{equation*}
                            \mathbb{P}(A\cap B)\mathbb{P}(B\cap
                            C)=(1/4)(1/4)=1/16.
                        \end{equation*}
                        Thus 
                        \begin{equation*}
                            \mathbb{P}((A\cap B)\cap(B\cap
                            C))\neq\mathbb{P}(A\cap B)\mathbb{P}(B\cap C).
                        \end{equation*}
                        Therefore $A\cap B$ and $B\cap C$ are not independent. 
                    \end{solution}
            \end{enumerate}
    \end{enumerate}
\end{document}
