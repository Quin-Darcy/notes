\documentclass[12pt]{article}
\usepackage[margin=1in]{geometry}
\usepackage{graphicx}
\usepackage{amsmath}
\usepackage{amsthm}
\usepackage{amsfonts}
\usepackage{amssymb}
\usepackage{array}
\usepackage{enumerate}
\usepackage{fancyhdr}
\usepackage{enumitem}
\pagestyle{fancy}
\fancyhf{}
\rhead{Darcy}
\lhead{STAT 215A}
\rfoot{\thepage}
\setlength{\headheight}{10pt}

\newenvironment{solution}
{\renewcommand\qedsymbol{$\blacksquare$}\begin{proof}[Solution]}
{\end{proof}}
\newenvironment{psmall}{\left(\begin{smallmatrix}}{\end{smallmatrix}\right)}

\begin{document}
    \thispagestyle{empty}\hrule

    \begin{center}
        \vspace{.4cm} { \large STAT 215A}
    \end{center}
    {Name:\ Quin Darcy \hspace{\fill} Due Date: 02/08/2022   \\
    { Instructor:}\ Dr. Cetin \hspace{\fill} Assignment:
    Homework 01 \\ \hrule}

    \begin{enumerate}
        \item Let $F_1$ and $F_2$ be two sigma-algebras on a sample space
            $\Omega$. We know that $F_1\cap F_2$ is a sigma-algebra on $\Omega$ 
            Show that $F_1\cup F_2$ is not necessarily
            a sigma-algebra on $\Omega$. 
            \begin{solution}
                Consider an experiment where two coins are flipped at the same
                time. If a coin lands on heads, we will denote that with
                a 1 and tails with a 0. Then the sample space is defined as 
                \begin{equation*}
                    \Omega=\{(0, 0), (0, 1), (1, 0), (1, 1)\}.
                \end{equation*}
                Let $A=\{(0, 0)\}$ and $B=\{(1, 0)\}$ be two subsets of
                $\Omega$. Then consider the two following $\sigma$-algebras
                \begin{align*}
                    &\mathcal{F}_1=\{\varnothing, A, A^c, \Omega\} \\
                    &\mathcal{F}_2=\{\varnothing, B, B^c, \Omega\}
                \end{align*}
                which happen to be the smallest $\sigma$-algebras containing
                $A$ and $B$, respectively. The union of these two
                $\sigma$-algebras is 
                \begin{align*}
                    \mathcal{F}_1\cup\mathcal{F}_2&=\{\varnothing, A, B, A^c,
                    B^c, \Omega\} \\
                    &=\bigg\{\varnothing, \;\{(0, 0)\}, \;\{(1, 0)\}, \;\{(0, 1), (1, 0),
                    (1, 1)\}, \;\{(0, 0), (0, 1), (1, 1)\}, \;\Omega\bigg\}.
                \end{align*}
                With this we can see that $A\cup B=\{(0, 0), (1, 0)\}$ is not
                an element of $\mathcal{F}_1\cup\mathcal{F}_2$. Thus it is not
                true that $\mathcal{F}_1\cup\mathcal{F}_2$ is closed under
                unions. Therefore, it is not a $\sigma$-algebra. 
            \end{solution}
        \item Let $\{A_k:k=1, 2, \dots, n\}$ be finite collection of events in
            a probability space $(\Omega, \mathcal{F}, \mathbb{P})$. Prove the
            following inequality for any such finite collection of $n$ events:
            \begin{equation*}
                \mathbb{P}\big(\bigcup_{k=1}^nA_k\big)\geq\sum_{k=1}^n\mathbb{P}(A_k)-\sum_{1\leq
                j<k\leq n}\mathbb{P}(A_j\cap A_k).
            \end{equation*}
            \begin{proof}
                We proceed by induction on $n$. For our base case we let $n=2$.
                Then 
                \begin{align*}
                    \mathbb{P}(A_1\cup A_2) &= \mathbb{P}(A_1\cup(A_2\;\backslash
                    \;A_1)) \\
                    &=\mathbb{P}(A_1)+\mathbb{P}(A_2\;\backslash\; A_1) \\
                    &=\mathbb{P}(A_1)+\mathbb{P}(A_2\;\backslash\;(A_1\cap
                    A_2)) \\
                    &=\mathbb{P}(A_1)+\mathbb{P}(A_2)-\mathbb{P}(A_1\cap A_2).
                \end{align*}
                With equality we satisfy the claim for $n=2$.\par\hspace{4mm}
                Now assume the claim is true for some $n\geq 2$. Let
                $A_{n+1}\in\mathcal{F}$ and let $A=\bigcup_{k=1}^{n} A_k$. Then 
                \begin{align*}
                    \mathbb{P}\big(\bigcup_{k=1}^{n+1}A_k\big)&=
                    \mathbb{P}\big(\bigcup_{k=1}^{n}A_k\cup A_{n+1}\big) \\
                    &=\mathbb{P}(A\cup A_{n+1}) \\
                    &= \mathbb{P}(A)+\mathbb{P}(A_{n+1})-\mathbb{P}(A\cap
                    A_{n+1}) \\
                    &= \mathbb{P}\big(\bigcup_{k=1}^n
                    A_k\big)+\mathbb{P}(A_{n+1})-\mathbb{P}\big(\bigcup_{k=1}^{n}A_k\cap
                    A_{n+1}\big) \\
                    &\geq
                    \sum_{k=1}^{n}\mathbb{P}(A_k)+\mathbb{P}(A_{n+1})-\sum_{1\leq
                    j<k\leq n}\mathbb{P}(A_j\cap A_k)-\mathbb{P}(A_{n+1}) \\
                    &\geq\sum_{k=1}^{n+1}\mathbb{P}(A_k)-\sum_{1\leq j<k\leq
                    n+1}\mathbb{P}(A_j\cap A_k).
                \end{align*}
                Therefore the claim is true for $n+1$ and so it is true for all
                $n$. 
            \end{proof}
        \item Let $A$, $B$, and $C$ be three events in a probaility space
            $(\Omega, \mathcal{F}, \mathbb{P})$. 
            \begin{enumerate}
                \item Let $D$ be the event that exactly one event in the set
                    ${A, B}$ occurs. Show that
                    $\mathbb{P}(D)=\mathbb{P}(A)+\mathbb{P}(B)-2\mathbb{P}(A\cap
                    B)$.
                    \begin{proof}
                        We want to determine the probability of either $A$ or
                        $B$ happening but not both. Thus we are looking for
                        $\mathbb{P}(A\vartriangle B)$. Since $A\vartriangle
                        B=(A\;\backslash\;B)\cup(B\;\backslash\;A)$, then we have
                        \begin{align*}
                            \mathbb{P}(A\vartriangle
                            B)&=\mathbb{P}((A\;\backslash\;B)\cup(B\;\backslash\;A))
                            \\
                            &=\mathbb{P}(A\;\backslash\;B)+\mathbb{P}(B\;\backslash\;A)\\
                            &=\mathbb{P}(A\;\backslash\;(A\cap
                            B))+\mathbb{P}(B\;\backslash\;(A\cap B)) \\
                            &=\big(\mathbb{P}(A)-\mathbb{P}(A\cap
                            B)\big)+\big(\mathbb{P}(B)-\mathbb{P}(A\cap B)\big)
                            \\
                            &=\mathbb{P}(A)+\mathbb{P}(B)-2\mathbb{P}(A\cap B).
                        \end{align*}
                    \end{proof}
                \item Let $E$ be the event that at least one event in $\{A, B,
                    C\}$ occurs. Describe $\mathbb{P}(E)$ in terms of the
                    probabilities of the events in $\{A, B, C\}$ and/or their
                    intersection probabilities.
                    \begin{solution}
                       To approach this we can ask in what ways can at least
                       one of these events happen. At least one of these events
                       can happen in the following ways
                       \begin{equation*}
                           \{A, A\cap B, A\cap C, B, B\cap C, C\}.
                       \end{equation*}
                       Thus 
                       \begin{equation*}
                           E=A\cup(A\cap B)\cup(A\cap C)\cup B\cup(B\cap C)\cup
                           C=A\cup B\cup C.
                       \end{equation*}
                       Then by the Inclusion-Exclusion Principle, we have
                       \begin{align*}
                           \mathbb{P}(A\cup B\cup C) =
                           \mathbb{P}(A)&+\mathbb{P}(B)+\mathbb{P}(C) \\
                           &-\mathbb{P}(A\cap
                           B)-\mathbb{P}(A\cap C)-\mathbb{P}(B\cap
                           C)+\mathbb{P}(A\cap B\cap C).
                       \end{align*}
                    \end{solution}
            \end{enumerate}
    \end{enumerate}
\end{document}
