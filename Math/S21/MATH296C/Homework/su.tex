\documentclass[12pt]{article}
\usepackage[margin=1in]{geometry}
\usepackage{graphicx}
\usepackage{amsmath}
\usepackage{amsthm}
\usepackage{amsfonts}
\usepackage{amssymb}
\usepackage{array}
\usepackage{enumerate}
\usepackage{slashed}
\usepackage{colonequals}
\usepackage{fancyhdr}
\usepackage{import}
\usepackage{xifthen}
\usepackage{pdfpages}
\usepackage{transparent}
\usepackage{enumitem}

\newcommand{\incfig}[1]{%
    \def\svgwidth{\columnwidth}
    \import{/home/arbegla/figures/}{#1.pdf_tex}
}

\pagestyle{fancy}
\fancyhf{}
\rhead{}
\lhead{}
\rfoot{\thepage}
\setlength{\headheight}{10pt}

\newtheorem{theorem}{Theorem}[section]
\newtheorem{corollary}{Corollary}[theorem]
\newtheorem{prop}{Proposition}[section]
\newtheorem{lemma}[theorem]{Lemma}
\theoremstyle{definition}
\newtheorem{definition}{Definition}[section]
\theoremstyle{definition}
\newtheorem{exmp}{Example}[section]

\newcommand{\abs}[1]{\lvert #1 \rvert}
\newcommand{\bigabs}[1]{\Bigl \lvert #1 \Bigr \rvert}
\newcommand{\bigbracket}[1]{\Bigl [ #1 \Bigr ]}
\newcommand{\bigparen}[1]{\Bigl ( #1 \Bigr )}
\newcommand{\ceil}[1]{\lceil #1 \rceil}
\newcommand{\bigceil}[1]{\Bigl \lceil #1 \Bigr \rceil}
\newcommand{\floor}[1]{\lfloor #1 \rfloor}
\newcommand{\bigfloor}[1]{\Bigl \lfloor #1 \Bigr \rfloor}
\newcommand{\norm}[1]{\| #1 \|}
\newcommand{\bignorm}[1]{\Bigl \| #1 \Bigr \| #1}
\newcommand{\inner}[1]{\langle #1 \rangle}
\newcommand{\set}[1]{{ #1 }}


\begin{document}
\title{The Lie algebra $\mathfrak{sl}_n(\mathbb{C})$ and its eigenvalues}
\author{Quin Darcy}
\date{March, 5 2021}
\maketitle
    \section{Definition and remarks}
    The Lie algebra $\mathfrak{sl}_n(\mathbb{C})$ is defined to be the set of all $n\times
    n$ matrices with trace 0 and whose entries are complex numbers. As the
    title of this paper suggests, this set is a Lie algebra. Recall the
    following definition:
    \begin{definition}
        A Lie algebra is a vector space $L$ over some field $F$ equipped with
        a binary operation $[\cdot, \cdot]:L\times L\to L$ such that for all
        $x, y, z\in L$, the
        following conditions hold:
        \begin{enumerate}[label=(\roman*)]
            \item $[\cdot, \cdot]$ is a bilinear map.
            \item $[x, x] = 0$
            \item $[x, [y, z]]+[y, [z, x]]+[z, [x, y]] = 0$.
            \end{enumerate}
    \end{definition}
    \noindent With the above definition in hand, if we define $[X, Y]
    = XY-YX$ for all $X, Y\in\mathfrak{sl}_n(\mathbb{C})$, then this will
    satisfy the above three conditions.
\end{document}
