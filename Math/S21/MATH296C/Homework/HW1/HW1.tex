\documentclass[12pt]{article}
\usepackage[margin=1in]{geometry}
\usepackage{graphicx}
\usepackage{amsmath}
\usepackage{amsthm}
\usepackage{amsfonts}
\usepackage{amssymb}
\usepackage{array}
\usepackage{enumerate}
\usepackage{slashed}
\usepackage{colonequals}
\usepackage{fancyhdr}
\usepackage{enumitem}
\pagestyle{fancy}
\fancyhf{}
\rhead{Darcy}
\lhead{MATH 296C}
\rfoot{\thepage}
\setlength{\headheight}{10pt}

\newtheorem{theorem}{Theorem}[section]
\newtheorem{corollary}{Corollary}[theorem]
\newtheorem{prop}{Proposition}[section]
\newtheorem{lemma}[theorem]{Lemma}
\theoremstyle{definition}
\newtheorem{definition}{Definition}[section]
%\theoremstyle{remark}
%\newtheorem*{remark}{Remark}
\newenvironment{solution}
{\renewcommand\qedsymbol{$\blacksquare$}\begin{proof}[Solution]}
{\end{proof}}
\newenvironment{psmall}{\left(\begin{smallmatrix}}{\end{smallmatrix}\right)}

\newcommand{\R}{\mathbb{R}}
\newcommand{\C}{\mathbb{C}}
\newcommand{\slin}[2]{\mathfrak{sl}_{#1}(#2)}
\newcommand{\mat}[2]{\text{Mat}_{#1}(#2)}
\newcommand{\End}[1]{\text{End}(#1)}
\newcommand{\dm}[1]{\text{dim}(#1)}
\newcommand{\dmm}[2]{\text{dim}_{#1}(#2)}
\newcommand{\abs}[1]{\lvert #1 \rvert}
\newcommand{\bigabs}[1]{\Bigl \lvert #1 \Bigr \rvert}
\newcommand{\bigbracket}[1]{\Bigl [ #1 \Bigr ]}
\newcommand{\bigparen}[1]{\Bigl ( #1 \Bigr )}
\newcommand{\ceil}[1]{\lceil #1 \rceil}
\newcommand{\bigceil}[1]{\Bigl \lceil #1 \Bigr \rceil}
\newcommand{\floor}[1]{\lfloor #1 \rfloor}
\newcommand{\bigfloor}[1]{\Bigl \lfloor #1 \Bigr \rfloor}
\newcommand{\norm}[1]{\| #1 \|}
\newcommand{\bignorm}[1]{\Bigl \| #1 \Bigr \| #1}
\newcommand{\inner}[1]{\langle #1 \rangle}
\newcommand{\set}[1]{{ #1 }}

\begin{document}
    \thispagestyle{empty}\hrule

    \begin{center}
        \vspace{.4cm} { \large MATH 296C}
    \end{center}
    {Name:\ Quin Darcy \hspace{\fill} Due Date: \#/\#/\#   \\
    { Instructor:}\ Dr. Krauel \hspace{\fill} Assignment:
    Homework 1 \\ \hrule}

    \begin{enumerate}
        \item[1.1] Find the matrix $A$ that represents $f$ in example 1.26.
            \begin{solution}
                We can solve for $A$ as follows:
                \begin{equation*}
                    \begin{pmatrix} a & b \\
                    c & d\end{pmatrix}\begin{pmatrix}1 \\
                    0\end{pmatrix} = \begin{pmatrix} a \\
                    c\end{pmatrix}=\begin{pmatrix}2 \\
                    1\end{pmatrix}
                \end{equation*}
                Thus $a=2$ and $c=1$. Which gives 
                \begin{equation*}
                    \begin{pmatrix} 2 & b \\
                    1 & d\end{pmatrix}\begin{pmatrix} 1 \\
                    1\end{pmatrix}=\begin{pmatrix} 2+b \\
                    1+d\end{pmatrix}=\begin{pmatrix}1 \\
                    2\end{pmatrix}
                \end{equation*}
                the last system yields $b=-1$ and $d=1$. Hence
                \begin{equation*}
                    A=\begin{pmatrix}2 & -1\\
                    1 & 1\end{pmatrix}.
                \end{equation*}
            \end{solution}
        \item[1.5] Let
            $\mathfrak{sl}_2(\R):=\{\begin{psmall}a&b\\c&d\end{psmall}\mid
            a+d=0, a, b, c, d\in\R\}$. That is, $\mathfrak{sl}_2(\R)$ consists
            of $2\times 2$ trace 0 matrices with entries in $\R$.
            \begin{enumerate}[label=(\alph*)]
                \item Show that $\mathfrak{sl}_2(\R)$ is a Lie algebra with
                    bracket $[A, B]=AB-BA$.
                    \begin{proof}
                        As a subset of $\mathfrak{gl}_{2}(\R)$, which
                        is a Lie algebra, we can prove the claim by
                        showing that $\mathfrak{sl}_{2}(\mathbb{R})$
                        is a Lie subalgebra of $\mathfrak{gl}_{2}(\R)$.
                        Recall that a Lie subalgebra is a subspace
                        closed under the bracket of the outer space. So
                        do prove our claim,
                        consider any $c\in\mathbb{R}$ and any $u,
                        v\in\mathfrak{sl}_{2}(\mathbb{R})$. It follows that
                        $\text{tr}(u+v)=\text{tr}(u)+\text{tr}(v)=0+0=0$
                        and so $u+v\in\mathfrak{sl}_{2}(\mathbb{R})$.
                        Similarly, we have that
                        $\text{tr}(cu)=c\text{tr}(u)=c0=0$. Thus,
                        $cu\in\mathfrak{sl}_{2}(\mathbb{R})$ and it is
                        therefore a subspace.\par\hspace{4mm} Taking
                        the same $u,
                        v\in\mathfrak{sl}_{2}(\mathbb{R})$, and letting 
                        \begin{equation*}
                            u=\begin{pmatrix} a&b\\c&d
                            \end{pmatrix}\quad\text{and}\quad
                            v=\begin{pmatrix}e&f\\g&h
                            \end{pmatrix},  
                        \end{equation*}
                        then 
                        \begin{equation*}
                            \begin{split}
                                [u, v]&=\begin{pmatrix}a&b\\c&d
                                \end{pmatrix}\begin{pmatrix}e&f\\g&h\end{pmatrix}-\begin{pmatrix}e&f\\g&h
                                \end{pmatrix}\begin{pmatrix}a&b\\c&d\end{pmatrix}\\
                                &=\begin{pmatrix}ae+bg&af+bh\\ce+dg&cf+dh\end{pmatrix}-
                                \begin{pmatrix}ea+fc&eb+fd\\ga+hc&gb+hd\end{pmatrix} \\
                                &=\begin{pmatrix}bg-fc&af+bh-eb-fd\\ce+dg-ga-he&cf-gb\end{pmatrix}. 
                            \end{split}
                        \end{equation*}
                        By the commutativity of the reals, it follows
                        that $\text{tr}([u, v])=bg-fc+cf-gb=0$. Hence,
                        $[u, v]\in\mathfrak{sl}_{2}(\mathbb{R})$.
                    \end{proof}
                \item Find a basis for $\mathfrak{sl}_{2}(\mathbb{R})$. 
                    \begin{solution}
                        Considering that the definition of this set requires
                        that the entries be elements of $\mathbb{R}$ and that
                        the trace must be zero, then we find that we have two
                        ``free" entries in the upper right and lower left, and
                        that we have only one free entry along the diagonal
                        since the other must be its additive inverse. Hence,
                        a basis for this Lie algebra is 
                        \begin{align*}
                            \begin{pmatrix}0&1\\0&0 \end{pmatrix},&
                            &\begin{pmatrix}0&0\\1&0\end{pmatrix},
                            & &\begin{pmatrix}1&0\\0&-1\end{pmatrix}.
                        \end{align*}
                    \end{solution}
                \item Find
                    $\text{dim}_{\mathbb{R}}(\mathfrak{sl}_{2}(\mathbb{R}))$.
                    \begin{solution}
                        Considering the basis provided in part (b), we can
                        conclude that this Lie algebra has dimension 3.
                    \end{solution}
            \end{enumerate}
        \item[1.7] Prove that every Lie algebra $(L,[\cdot, \cdot])$ over
            a field $K$ is a $K$-algebra if one sets $x\cdot y=[x, y]$.
            \begin{proof}
                We first recall the definition of a $K$-algebra and hence need
                to show that $L$ is a vector space over $K$ such that $L$ is
                a ring with under addition and $[\cdot, \cdot]$, however,
                associativity is not required for the second operation.
                Additionally, we need to show that for any $k, l\in K$, and any
                $x, y\in L$, we have $[kx, ly]=(kl)[x, y]$.\par\hspace{4mm} By
                the definition of a Lie algebra, $L$ is a vector space over
                some field $K$. As such, it is an abelian group with respect to
                addition and a magma with respect to $[\cdot, \cdot]$. By the
                biliearity of $[\cdot, \cdot]$, it follows that $[kx, ly]=k[x,
                ly]=l[kx, y ]=(kl)[x, y]$. Thus, $(L, [\cdot, \cdot])$ is
                a $K$-algebra.
            \end{proof}
        \item[2.9] Prove $\mathfrak{sl}_{2}(K)$ is simple if and only if the
            characteristic of $K$ is not 2.
            \begin{proof}
                Assume that the characteristic of $K$ is 2 and consider the
                following set
                \begin{equation*}
                    S = \text{span}\{\begin{pmatrix} 1&0\\0&-1 \end{pmatrix}
                    \}.
                \end{equation*}
                $S$ is a subspace of $\mathfrak{sl}_{2}(K)$ by the definition
                of span. Moreover, if we notice that 
                \begin{equation*}
                    \forall u=\begin{pmatrix}a&b\\c&-a
                    \end{pmatrix}\in\mathfrak{sl}_{2}(K)\quad
                    \text{and}\quad\forall v=\begin{pmatrix}k&0\\0&-k\end{pmatrix}\in
                    S, 
                \end{equation*}
                we get that 
                \begin{equation*}
                    \begin{split}
                        [u, v] &= \begin{pmatrix} 0&-2bk\\2ck&0\end{pmatrix} \\
                        &=\begin{pmatrix}0&0\\0&0 \end{pmatrix} \in S.
                    \end{split}
                \end{equation*}
                Seeing as we just found a one-dimensional (non-trivial) ideal
                of $\mathfrak{sl}_{2}(K)$, we can conclude that
                $\mathfrak{sl}_{2}(K)$ is not simple when the characteristic of
                $K$ is two.\par\hspace{4mm} Now assume that
                $\mathfrak{sl}_{2}(K)$ is simple and consider the following basis:
                \begin{align*}
                    x=\begin{pmatrix}0&1\\0&0\end{pmatrix},& &y=\begin{pmatrix}
                    0&0\\1&0\end{pmatrix},& &h=\begin{pmatrix}1&0\\0&-1\end{pmatrix}.
                \end{align*}
                We have the following commutation relations: 
                \begin{align*}
                    [x, y]=h,& &[x, h]=-2x,& &[y, h]=2y.
                \end{align*}
                Now suppose that $I$ is a nonempty ideal of $\mathfrak{sl}_2(K)$. 
                Then since the latter was assumed to be simple, it follows 
                that $I=\mathfrak{sl}_2(K)$. However, if $K$ had characteristic 2, 
                then this would imply that $[x, h]=0$ and $[y, h]=0$, rendering 
                $I$ one-dimensional which is a contradiction. Hence, the 
                characteristic is not two.
            \end{proof}
        \item[2.10] Prove that $\mathfrak{sl}_n(\mathbb{C})$ is an ideal of $\mathfrak{gl}_n(\mathbb{C})$.
            \begin{proof}
                To show that $\mathfrak{sl}_n(\mathbb{C})$ is an ideal, 
                we must show that it is a subspace such that $[x, y]\in\mathfrak{sl}_n(\mathbb{C})$ 
                for all $x\in\mathfrak{gl}_n(\mathbb{C})$ and all 
                $y\in\mathfrak{sl}_n(\mathbb{C})$. That it is a subspace is 
                immediate since it is a subset and it is a vector space under 
                the same operations. Now let $x\in\mathfrak{gl}_n(\mathbb{C})$ 
                and $y\in\mathfrak{sl}_n(\mathbb{C})$. We want to show that 
                $[x, y]\in\mathfrak{sl}_n(\mathbb{C})$. Thus we need that 
                $\text{tr}([x, y])=0$. By the fact that $\text{tr}(AB)=\text{tr}(BA)$ 
                for all $A, B\in\text{Mat}_{n\times n}(\mathbb{C})$, then we 
                have that $\text{tr}([x, y])=\text{tr}(xy)-\text{tr}(yx)=0$, as desired. 
            \end{proof}
        \item[2.12] Find the structure constants for
            $\mathfrak{sl}_{2}(\mathbb{C})$.
            \begin{solution}
                By the above commutation relations, the structure constants are
                as follows: Letting $a_{xy}^k$ denote the $k$th coefficient on
                the linear combination representing $[x, y]$, then we have
                \begin{align*}
                    &a_{xy}^0=0;& &a_{xy}^1=0;& &a_{xy}^2=1; \\
                    &a_{xh}^0=-2;& &a_{xh}^1=0;& &a_{xh}^2=0; \\
                    &a_{yh}^0=0;& &a_{yh}^1=2;& &a_{yh}^2=0.
                \end{align*}
            \end{solution}
        \item[5.1] Recall $\mathcal{V}_2$ of Example 4.6.
            \begin{enumerate}
                \item Find all weights of $h$ for $\mathcal{V}_2$.
                    \begin{solution}
                        We first recall that  
                        \begin{equation*}
                            h.(aX^2+bXY+cY^2):=(X\frac{\partial}{\partial
                            X}-Y\frac{\partial}{\partial
                        Y})(aX^2+bXY+cY^2)=2aX^2-2cY^2.
                        \end{equation*}
                        Moreover, using $\{X^2, XY, Y^2\}$ as a basis for
                        $\mathcal{V}_2$, then we can apply $h$ to each of the
                        basis elements to obtain:
                            \begin{align*}
                                &h.X^2 = (X\frac{\partial}{\partial
                                X}-Y\frac{\partial}{\partial Y})X^2 =2X^2& \\
                                &h.XY = (X\frac{\partial}{\partial
                                X}-Y\frac{\partial}{\partial Y})XY =XY-XY=0& \\
                                &h.Y^2 = (X\frac{\partial}{\partial
                                X}-Y\frac{\partial}{\partial Y})Y^2=-2Y^2.
                            \end{align*}
                        This suffices to show that the only weights of $h$ for
                        $\mathcal{V}_2$ are 2, 0, and -2.
                    \end{solution}
                \item Describe (find a basis) all weight spaces. 
                    \begin{solution}
                        By the above calculation, we can see that $w_0=X^2$ is the
                        highest weight vector with weight $2$. By Lemma 5.6, it
                        follows that 
                        \begin{equation*}
                            \begin{split}
                                &w1 = \frac{1}{1!}y^1.w_0=Y\frac{\partial}{\partial
                                X}(X^2)=2XY \\
                                &w2
                                = \frac{1}{2!}y^2.w_0=\frac{1}{2}(Y^2\frac{\partial^2}{\partial
                                X^2})(X^2)=Y^2.
                            \end{split}
                        \end{equation*}
                        Thus $w_0=X^2, w_1=2XY, w_2=Y^2$ gives us a basis for
                        $V(2)$. Moreover, by Theorem 5.8, $V(2)\cong\mathcal{V}_2$.
                        Thus by part (a), we have that $V_{-2}=\text{span}(w_2)$,
                        $V_0=\text{span}(w_1)$, and $V_2=\text{span}(w_0)$.
                    \end{solution}
                \item Express $\mathcal{V}_2$ as a direct sum of its weight
                    spaces as in Theorem 5.1.
                    \begin{solution}
                        By parts (a) and (b), we found that the weight spaces
                        of $h$ are $V_{-2}$, $V_0$, and $V_2$ and as such it
                        follows that 
                        \begin{equation*}
                            \mathcal{V}_2=\bigoplus_{i=0}^2V_{-2+2i}=\text{span}(X^2)\oplus\text{span}(2XY)\oplus\text{span}(Y^2).
                        \end{equation*}
                    \end{solution}
            \end{enumerate}
        \item[5.2] Define a vector space $\mathcal{V}_3$ similar to example
            4.6.
            \begin{enumerate}
                \item Prove that $\mathcal{V}_3$ is an
                    $\mathfrak{sl}_2$-module.
                    \begin{solution}
                        Letting $\mathcal{V}_3=\{aX^3+bX^2Y+cXY^2+dY^3\mid a, b,
                        c, d\in\mathbb{C}\}$, then for the same reason that
                        $\mathcal{V}_2$ is a $\mathbb{C}$-linear space, we have
                        that $\mathcal{V}_3$ is a $\mathbb{C}$-linear space. We
                        will also re-use the same following relations. For all
                        $v\in\mathcal{V}_3$, and for $x, y, h$ as basis elements of
                        $\mathfrak{sl}_2$, then 
                        \begin{align*}
                            &x.v = (X\frac{\partial}{\partial Y})v,&
                            &y.v = (Y\frac{\partial}{\partial X})v,&
                            &h.v = (X\frac{\partial}{\partial X}
                            - Y\frac{\partial}{\partial Y})v.
                        \end{align*}
                        With this we will let $\alpha_1x+\beta_1y+\gamma_1h,
                        \alpha_2x+\beta_2y+\gamma_2h\in\mathfrak{sl}_2$ and we let
                        $u=a_1X^3+b_1X^2Y+c_1XY^2+d_1Y^3$ and
                        $v=a_2X^3+b_2X^2Y+c_2XY^2+d_2Y^3$ be two arbitrary elements
                        of $\mathcal{V}_3$. Then
                        \begin{equation*}
                            \begin{split}
                                ((\alpha_1x+\beta_1y+\gamma_1h)+(\alpha_2x+\beta_2y+\gamma_2h)).u
                                &=((\alpha_1+\alpha_2)x+(\beta_1+\beta_2)y+(\gamma_1+\gamma_2)h).u
                                \\ 
                                &=(\alpha x+\beta y+\gamma h).u \\
                                &=(\alpha X\frac{\partial}{\partial
                                Y}+\beta Y\frac{\partial}{\partial X}+\gamma(X\frac{\partial}{\partial X}
                                -Y\frac{\partial}{\partial Y}))u \\
                                &=\alpha (X\frac{\partial}{\partial
                                Y})u+\beta (Y\frac{\partial}{\partial
                                X})u+\gamma(X\frac{\partial}{\partial
                                X}-Y\frac{\partial}{\partial Y})u \\
                                &=\alpha(x.u)+\beta(y.u)+\gamma(h.u) \\
                                &=(\alpha_1+\alpha_2)(x.u)+(\beta_1+\beta_2)(y.u)+(\gamma_1+\gamma_2)(h.u)
                                \\
                                &= \alpha_1(x.u)+\beta_1(y.u)+\gamma_1(h.u) \\
                                &\quad\quad
                                +\alpha_2(x.u)+\beta_2(y.u)+\gamma_2(h.u),
                            \end{split}
                        \end{equation*}
                        which proves M1. REMINDER: Finish showing M2 and M3.
                    \end{solution}
                \item Find all the weights of $h$ for $\mathcal{V}_3$.
                    \begin{solution}
                        Letting $\{X^3, X^2Y, XY^2, Y^3\}$ be a basis for
                        $\mathcal{V}_2$, then we can find the weights of $h$ by
                        applying $h$ to each basis element. Doing this we get
                        \begin{equation*}
                            \begin{split}
                                &h.X^3=(X\frac{\partial}{\partial
                                X}-Y\frac{\partial}{\partial Y})X^3=3X^3 \\
                                &h.X^2Y=(X\frac{\partial}{\partial
                                X}-Y\frac{\partial}{\partial Y})X^2Y=X^2Y \\
                                &h.XY^2=(X\frac{\partial}{\partial
                                X}-Y\frac{\partial}{\partial Y})XY^2=-XY^2 \\
                                &h.Y^3=(X\frac{\partial}{\partial
                                X}-Y\frac{\partial}{\partial Y})Y^3=-3Y^3.
                            \end{split}
                        \end{equation*}
                        Hence, the weights of $h$ are $-3, -1, 1, 3$.
                    \end{solution}
                \item Describe all weight spaces.
                    \begin{solution}
                        By part (b), we see that the highest weight is $3$ with
                        weight vector $w_0=X^3$. With this we can use Lemma 5.6 to
                        calculate the remaining weight spaces.
                        \begin{equation*}
                            \begin{split}
                                &w_1 =\frac{1}{1}y.w_0=Y\frac{\partial}{\partial
                                X}(X^3)=3X^2Y \\ 
                                &w_2=\frac{1}{2}y^2.w_0=\frac{1}{2}(Y^2\frac{\partial^2}{\partial
                                X^2})X^3=3XY^2 \\
                                &w_3=\frac{1}{6}(Y^3\frac{\partial^3}{\partial
                                X^3})X^3=Y^3.
                            \end{split}
                        \end{equation*}
                        Thus, we have that the weight spaces of $\mathcal{V}_3$
                        are $V_{-3}=\text{span}(w_3)$, $V_{-1}=\text{span}(w_2)$,
                        $V_1=\text{span}(w_1)$, and $V_3=\text{span}(w_0)$.
                    \end{solution}
                \item Express $\mathcal{V}_3$ as a direct sum of its weight
                    spaces as in Theorem 5.1.
                    \begin{solution}
                        With $-3, -1, 1, 3$ being distinct eigenvalues for $h$,
                        then by Theorm 5.1 it follows that 
                        \begin{equation*}
                            \mathcal{V}_3\cong\bigoplus_{i=0}^{3}V_{-3+2i}
                        \end{equation*}
                    \end{solution}
                \item Check that the results of Corollary 5.12 hold for
                    $\mathcal{V}_3$.
                    \begin{solution}
                        By part (d), we can see that (a) of Corollary 5.12
                        holds. Part (b) shows that (b) of Corollary 5.12 holds.
                        As the dimension of each of the weight spaces is one,
                        this shows that (c) of the Corollary holds. 
                    \end{solution}
            \end{enumerate}
    \end{enumerate}
\end{document}
