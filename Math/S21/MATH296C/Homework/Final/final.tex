\documentclass[12pt]{article}
\usepackage[margin=1in]{geometry}
\usepackage{graphicx}
\usepackage{amsmath}
\usepackage{amsthm}
\usepackage{amsfonts}
\usepackage{amssymb}
\usepackage{array}
\usepackage{enumerate}
\usepackage{slashed}
\usepackage{colonequals}
\usepackage{fancyhdr}
\usepackage{enumitem}
\usepackage{import}
\usepackage{xifthen}
\usepackage{pdfpages}
\usepackage{transparent}

\pagestyle{fancy}
\fancyhf{}
\rhead{Darcy}
\lhead{MATH 296C}
\rfoot{\thepage}
\setlength{\headheight}{10pt}

\newtheorem{theorem}{Theorem}[section]
\newtheorem{corollary}{Corollary}[theorem]
\newtheorem{prop}{Proposition}[section]
\newtheorem{lemma}[theorem]{Lemma}
\theoremstyle{definition}
\newtheorem{definition}{Definition}[section]
%\theoremstyle{remark}
%\newtheorem*{remark}{Remark}

\newcommand{\incfig}[1]{%
    \def\svgwidth{\columnwidth}
    \import{/home/arbegla/figures/}{#1.pdf_tex}
}

\newenvironment{solution}
  {\renewcommand\qedsymbol{$\blacksquare$}\begin{proof}[Solution]}
  {\end{proof}}

\newcommand{\abs}[1]{\lvert #1 \rvert}
\newcommand{\bigabs}[1]{\Bigl \lvert #1 \Bigr \rvert}
\newcommand{\bigbracket}[1]{\Bigl [ #1 \Bigr ]}
\newcommand{\bigparen}[1]{\Bigl ( #1 \Bigr )}
\newcommand{\ceil}[1]{\lceil #1 \rceil}
\newcommand{\bigceil}[1]{\Bigl \lceil #1 \Bigr \rceil}
\newcommand{\floor}[1]{\lfloor #1 \rfloor}
\newcommand{\bigfloor}[1]{\Bigl \lfloor #1 \Bigr \rfloor}
\newcommand{\norm}[1]{\| #1 \|}
\newcommand{\bignorm}[1]{\Bigl \| #1 \Bigr \| #1}
\newcommand{\inner}[1]{\langle #1 \rangle}
\newcommand{\set}[1]{{ #1 }}

\begin{document}
    \thispagestyle{empty}\hrule

    \begin{center}
        \vspace{.4cm} { \large MATH 296C}
    \end{center}
    {Name:\ Quin Darcy \hspace{\fill} Due Date: 5/17/21 \\
    { Instructor:}\ Dr. Krauel \hspace{\fill} Assignment:
    Final Exam \\  \hrule}

    \begin{enumerate}
        \item Explain your work for everything done.
            \begin{enumerate}[label=\textbf{\alph*}.]
                \item Do \textbf{one} of the following:
                    \begin{enumerate}[label=(\Roman*)]
                        \item Construct a rank 2 root system associated with
                            the angle $\theta=3\pi/4$. (Include a diagram of
                            the roots, along with their labels.)
                            \begin{solution}
                                Taking $\alpha, \beta\in\Phi$ such that
                                $\beta\neq\pm\alpha$ and the angle between
                                these roots is $3\pi/4$, then as this is an
                                obtuse angle, we get that
                                $\alpha+\beta\in\Phi$. Additionally, we get
                                that $-\alpha, -\beta, -\alpha-\beta\in\Phi$.
                                Next we want to calculate the angle between
                                $\alpha$ and $\alpha+\beta$. Thus, we want to
                                calculate the $\theta$ which satisfies
                                \begin{equation*}
                                    \langle\alpha+\beta,
                                    \alpha\rangle\langle\alpha,
                                    \alpha+\beta\rangle=4\cos^2\theta.
                                \end{equation*}
                                By Table 1 from the notes, we have that
                                $\langle\alpha, \beta\rangle = -1$ and
                                $\langle\beta, \alpha\rangle=-2$. Using this, we find that 
                                \begin{equation*}
                                    \begin{split}
                                        \langle\alpha+\beta,
                                        \alpha\rangle&=\frac{2(\alpha+\beta,
                                        \alpha)}{(\alpha, \alpha)} \\
                                        &=\frac{2(\alpha, \alpha)+2(\beta,
                                        \alpha)}{(\alpha, \alpha)} \\
                                        &=2+\langle\beta, \alpha\rangle \\
                                        &=2+(-2) \\
                                        &=0.
                                    \end{split}
                                \end{equation*}
                                Hence, $\cos^2\theta=0$. This implies that
                                $\theta=\pi/2$. Next, we consider the
                                reflection of $-\beta$ over $\alpha$:
                                \begin{equation*}
                                    \begin{split}
                                        \sigma_{\alpha}(-\beta)&=-\beta-\frac{2(-\beta,
                                        \alpha)}{(\alpha, \alpha)}\alpha \\
                                        &=-\beta+\frac{2(\beta,
                                        \alpha)}{(\alpha, \alpha)}\alpha \\
                                        &=-\beta+\langle\beta,
                                        \alpha\rangle\alpha \\
                                        &=-\beta-2\alpha.
                                    \end{split}
                                \end{equation*}
                                From this we also get that
                                $\pm(2\alpha+\beta)\in\Phi$. Using the same
                                technique from above, we can determing the
                                angle between $\alpha$ and $2\alpha+\beta$. We
                                get that $\theta=\pi/4$. Thus our root system
                                is $\Phi=\{\pm\alpha, \pm\beta,
                                \pm(\alpha+\beta), \pm(2\alpha+\beta)\}$. This
                                root system is isomorphic to $B_2$.
                            \end{solution}
                    \end{enumerate}
                \item Find the Cartan matrix for your choice of (a).
                    \begin{solution}
                        Seeing that $\Delta=\{\alpha, \beta\}$ is a base for
                        our choice in (a). This is true since $\alpha$ and
                        $\beta$ generated the root system in (a). With this we
                        calculate the Cartan matrix as 
                        \begin{equation*}
                            C=\begin{bmatrix}\langle\alpha,
                            \alpha\rangle&\langle\alpha,
                        \beta\rangle\\ \langle\beta,
                    \alpha\rangle&\langle\beta, \beta\rangle
                \end{bmatrix}=\begin{bmatrix}2&-1\\-2&2 \end{bmatrix}  
                        \end{equation*}
                    \end{solution}
            \end{enumerate}
            \begin{enumerate}[label=(\Alph*)]
                \item Performing the same procedure as above we get that with
                    the two roots $\alpha, \beta\in\Phi$, it follows that
                    $-\alpha, -\beta\in\Phi$. And since $\theta=5\pi/6$ is
                    obtuse, then $\pm(\alpha+\beta)$. Via reflections across
                    the roots in $\Phi$, we also obtain $\pm(\alpha+2\beta)$,
                    $\pm(2\alpha+\beta)$, $\pm(3\alpha+2\beta)$, and
                    $\pm(3\alpha+\beta)$. This gives us a root system
                    isomorphic to $G_2$.
                \item In part I, we found that $\Phi$ had 8 roots, 4 of them
                    positive. So the Weyl group would be the set of all
                    compositions of the elements 
                    $\{\sigma_{\alpha},\sigma_{\beta},\sigma_{\alpha+\beta},\sigma_{2\alpha_\beta}\}$,
                    since $\sigma_{\alpha}(\beta)=\sigma_{-\alpha}(\beta)$ (at
                    least I think so?).
                \item If $\langle\alpha, \beta\rangle\langle\beta,
                    \alpha\rangle=2$, then lets assume $\langle\alpha,
                    \beta\rangle=1$. This forces $\langle\beta,
                    \alpha\rangle=2$. Hence, $2=4\cos^2\theta$ which implies
                    $\cos\theta=\pm\sqrt{2}/2$. Note that we have $(\alpha, \beta)>0$, and thus $\theta$ is
                    acute. Hence, we get that $\cos\theta=\sqrt{2}/2$ and thus $\theta=\pi/4$. Moreover, we find
                    that $(\beta, \beta)/(\alpha, \alpha)=2$.\par\hspace{4mm}
                    If instead, $\langle\beta, \alpha\rangle=-2$, everythig
                    above is the same except we have that $(\alpha, \beta)<0$
                    and so the angle is obtuse. Thus, $\cos\theta=-\sqrt{2}/2$,
                    and hence $\theta=3\pi/4$.
            \end{enumerate}
        \item Do the following.
            \begin{enumerate}[label=(\Alph*)]
                \item State the roots of $G_2$.
            \begin{solution}
                As vectors in a 2-dimensional subspace of a 3-dimensional
                space, the roots are
                \begin{align*}
                    &(1, -1, 0)& &(-1, 1, 0)& &(2, -1, -1)& &(-2, 1, 1)& \\
                    &(1, 0, -1)& &(-1, 0, 1)& &(1, -2, 1)& &(-1, 2, -1)& \\
                    &(0, 1, -1)& &(0, -1, 1)& &(1, 1, -2)& &(-1, -1, 2)
                \end{align*}
            \end{solution}
        \item Provide the Cartan matrix of $G_2$.
            \begin{solution}
                The Cartan matrix of $G_2$ is 
                \begin{equation*}
                    C=\begin{bmatrix} 2&-3\\-1&2 \end{bmatrix} 
                \end{equation*}
            \end{solution}
        \item Explain how one can obtain the Cartan matrix from a base of
            roots.
            \begin{solution}
                Given a base $\Delta=\{\alpha_1, \dots, \alpha_n\}$ of a root
                system $E$, the Cartan matrix can be found by calculating
                $\langle\alpha_i, \alpha_j\rangle$ for all $1\leq i, j\leq n$
                and it is the result of each of these calculations which which
                will serve as the entries in the matrix. For example, the entry
                at the $i$-th row and $j$-th column would be $\langle\alpha_i,
                \alpha_j\rangle$.
            \end{solution}
        \item Provide the Dynkin diagram for $G_2$.
            \begin{solution}\hfill\par
                \begin{figure}[htp!]
                    \incfig{G2}
               \end{figure}\hfill\par
               Sorry for how crazy big this turned out. I'm a lot better at
               LaTeX/InkScape than this, I swear!
            \end{solution}
        \item Explain how one can obtain the Cartan matrix from the Dynkin
            diagram. In the explanation, actually construct the Cartan matrix. 
            \begin{solution}
                Given a Dynkin diagram, we first count the number of vertices.
                This will tell us how many simple roots are in our base,
                $\Delta$. Let's
                say there are $n$ many. Between any two vertices, we count the
                number of lines (edges) connecting them. This number is
                $n_{\alpha_i, \alpha_j}$ for $\alpha_i, \alpha_j\in\Delta$.
                Knowing this number gives us $\langle\alpha_i,
                \alpha_j\rangle\langle\alpha_j, \alpha_i\rangle$. This number
                is always going to be an element of $\{0, 1, 2, 3\}$. Since
                $\langle\alpha_i, \alpha_j\rangle\in\mathbb{Z}$, for all $i,
                j$, then this leaves a finite number of possibilities for what
                $\langle\alpha_i, \alpha_j\rangle$, and $\langle\alpha_j,
                \alpha_i\rangle$ can be. Each possibility for one decides the
                value for the other. Supposing $\langle\alpha_i,
                \alpha_j\rangle=a$ and $\langle\alpha_j, \alpha_i\rangle=b$,
                where $ab=n_{\alpha_i, \alpha_j}$, we have now just calculated
                two entries of the Cartan matrix. Namely, the $ij$-th entry and
                the $ji$-th entry. Continuing in this way, looking at every
                consecutive pair of vertices, this will yield all entries in
                the matrix. 
            \end{solution}
        \item Explain how one can obtain the Dynkin diagram from the Cartan
            matrix. In the explanation, actually construct the Dynkin diagram.
            \begin{solution}
                Given the Cartan matrix, $C$, we can begin our Dynkin diagram
                by drawing as many vertices as there are rows or columns in the
                matrix. At this point, we can recall that the $ij$-th entry in
                the matrix is $\langle\alpha_i, \alpha_j\rangle$, and so we can
                associate each consecutive pair of vertices with a particular
                entry in the matrix. The entry will tell us the number of lines
                or edges which connect the consecutive pair of points, via the
                product of the entry with its transpose, i.e.,
                $\langle\alpha_i, \alpha_j\rangle\langle\alpha_j,
                \alpha_i\rangle$.
                Additionally, we stipulate that if this product is greater than
                1, then we draw an arrow pointing towards the vertex whose
                associated root has a greater magnitude betweeen the pair. 
            \end{solution}
            \end{enumerate} 
        \item In your own words, write a 1 page overview of what we covered on
            Lie algebras. Attempt to weave the themes together.
            \begin{solution}
                To start, I must say that this is one of those classes that
                I would happily take again. It is a subject one joyously swan
                dives into if given the time. In this class we began by simply
                giving a definition. The definition was of a Lie algebra, and
                it is essentially (details omitted)
                a vector space equipped with a strange operator. But this
                humble beginning resulted in significant
                implications.\par\hspace{4mm} We were exposed to a few examples
                of these Lie algebras, but we set our sights on one in
                particular as it proved a very demonstrative choice for many of
                the concepts we looked at. The Lie algebra we focused on was
                $\mathfrak{sl}_2(\mathbb{C})$. We quickly moved to studying how
                this Lie algebra ``acted" on other spaces. It was through this
                lense that we found an entry point to the classification of
                certain types of Lie algebras.\par\hspace{4mm} In my naive
                understanding, with an operator called the adjoint operator,
                this allowed us to place the elements into the setting which
                was where we studied the techniques of decomposing a space.
                This operator allowed us to see how the Lie algebra acted on
                itself and we were able to decompose the Lie algebra into
                simpler structures. Moreover, we looked at particular elements of the Lie
                algebra which we called semisimple. And these elements it was
                these elements which specifically provided us with the
                constituents of the decomposition. \par\hspace{4mm} The
                semisimple elements, had adjoint representations which were
                diagonalizable and yielded eigenvalues, which then in turn,
                yielded eigenspaces, whose direct sum was the Lie
                algebra.\par\hspace{4mm} In taking a closer look at the set of
                the adjoint representation of the
                semisimple (and abelian) elements of the Lie algebra, we found
                that these elements had common eigenvectors. For such
                a common eigenvector, we considered the eigenvalues associated
                to it with respect to the particular semisimple element for
                which it was an eigenvector. Said less horribly, being
                a simulateous eigenvector for all of these semisimple elements
                means that for each semisimple element, the eigenvector has
                a different eigenvalue. The set of these eigenvalues is what we
                called a root. We also considered the root to be a linear
                functional whose inputs were the semisimple elements and whose
                outputs were the eigenvalue for that given
                input.\par\hspace{4mm} Okay, we have this set of roots, big
                deal! But wait, we also looked at something called the Killing
                form which is a symmetric bilinear nondegenerate map, you know,
                one of those things $\dots$ This Killing form allowed us to
                define an inner product on the space spanned by our roots which
                is quite significant since this gives rise to a kind of
                geometry in the space! Namely, we were able to discuss lengths
                and angles, like one does on Thanksgiving at grandma's
                house.\par\hspace{4mm} So along with this geometry, we also
                looked at ways to reflect the roots across a hyperplane. This
                one of the tools needed in recovering all of the roots of a Lie
                algebra. In addition to the reflection, there were many small
                results about the inner product and the properties of the
                geometry which all together suggested that at the level of root
                spaces, there are only finitely many of them. And if any
                arbitrary one of these special types of Lie algebras
                corresponds to one of these root systems, then this implies
                that a classifiaction of the root systems might be able to pull
                back into a classification of the Lie algebras themselves. This
                turns out to be true!\par\hspace{4mm} Better yet, the root
                spaces have a few different, very compact, forms. We looked at
                root diagrams, which emphasized the geometry of the root space
                as the diagrams are like a crystallographic arrangement of
                arrows. Then there are the Cartan matrices whose entries give
                you all the information you need about the inner product
                between any two roots. Finally, there are the Dynkin diagrams,
                which, like the Cartan matrix, tells you about the inner
                product of any two roots, but it also conveys information
                regarding the lengths of the roots which is something seen in
                the root diagram.\par\hspace{4mm} All in all, and despite this
                (very) inaccurate summaries, this was one of, if not, the most
                interesting class I have ever had the pleasure of taking.
                Thank you very much for teaching it, Dr. Krauel. You did an
                incredible job!
            \end{solution}
    \end{enumerate}
\end{document}
