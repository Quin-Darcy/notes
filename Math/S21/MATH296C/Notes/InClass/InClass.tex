\documentclass[12pt]{article}
\usepackage[margin=1in]{geometry}
\usepackage{graphicx}
\usepackage{amsmath}
\usepackage{amsthm}
\usepackage{amsfonts}
\usepackage{amssymb}
\usepackage{array}
\usepackage{enumerate}
\usepackage{slashed}
\usepackage{colonequals}
\usepackage{fancyhdr}
\usepackage{import}
\usepackage{xifthen}
\usepackage{pdfpages}
\usepackage{transparent}
\usepackage{enumitem}

\newcommand{\incfig}[1]{%
    \def\svgwidth{\columnwidth}
    \import{/home/arbegla/figures/}{#1.pdf_tex}
}
\newenvironment{psmall}{\left(\begin{smallmatrix}}{\end{smallmatrix}\right)}

\pagestyle{fancy}
\fancyhf{}
\rhead{}
\lhead{}
\rfoot{\thepage}
\setlength{\headheight}{10pt}

\newtheorem{theorem}{Theorem}[section]
\newtheorem{corollary}{Corollary}[theorem]
\newtheorem{prop}{Proposition}[section]
\newtheorem{lemma}[theorem]{Lemma}
\theoremstyle{definition}
\newtheorem{definition}{Definition}[section]
\theoremstyle{definition}
\newtheorem{exmp}{Example}[section]
\newtheorem*{remark}{Remark}

\newcommand{\abs}[1]{\lvert #1 \rvert}
\newcommand{\bigabs}[1]{\Bigl \lvert #1 \Bigr \rvert}
\newcommand{\bigbracket}[1]{\Bigl [ #1 \Bigr ]}
\newcommand{\bigparen}[1]{\Bigl ( #1 \Bigr )}
\newcommand{\ceil}[1]{\lceil #1 \rceil}
\newcommand{\bigceil}[1]{\Bigl \lceil #1 \Bigr \rceil}
\newcommand{\floor}[1]{\lfloor #1 \rfloor}
\newcommand{\bigfloor}[1]{\Bigl \lfloor #1 \Bigr \rfloor}
\newcommand{\norm}[1]{\| #1 \|}
\newcommand{\bignorm}[1]{\Bigl \| #1 \Bigr \| #1}
\newcommand{\inner}[1]{\langle #1 \rangle}
\newcommand{\set}[1]{{ #1 }}
\newcommand{\ad}[1]{\text{ad}_{#1}}
\newcommand{\R}{\mathbb{R}}
\newcommand{\C}{\mathbb{C}}
\newcommand{\mat}[2]{ \text{Mat} _{#1}(#2) }
\newcommand{\slin}[2]{\mathfrak{sl}_{#1}(#2)}
\newcommand{\End}[1]{ \text{End} (#1) }
\newcommand{\dm}[1]{ dim (#1) }
\newcommand{\dmm}[2]{ dim _{#1}(#2) }
    %\begin{figure}[htp!]
    %    \centering
    %      \incfig{S5}
    %      \caption{Group Action}
    %      \label{fig:action}
    %  \end{figure}\

\begin{document}
\title{Lie Theory Notes}
\author{Quin Darcy}
\maketitle
    \section{02/03/21}
        Recall: An ideal $I$ iof $I$ of $L$ is a subspace such that 
            \begin{equation*}
                [x, y]\in I, \quad\forall x\in I, \forall y\in L.
            \end{equation*}
        \begin{definition}
            A \textbf{simple} Lie algebra $L$ is one with no non-trivial ideal
            and $L$ is not abelian. 
        \end{definition}
        \begin{itemize}
            \item \textbf{Q}: What does it mean for $L$ to be abelian?
            \item \textbf{A}: We mean $[x, y]=0$ for any $x, y\in L$.\par
        \end{itemize}
        \noindent Given a vector space, you can always attach an abelian Lie algebra
        structure. Suppose $\dm{L}=1$. Then any basis for $L$ has one
        basis element.
        \begin{itemize}
            \item \textbf{Q}: What are the different types of $[\cdot,\cdot]$ can we define?
            \item \textbf{A}: For all $a, b\in L$, $[a,
                b]=(ab)[e_1, e_1]=0$. So for any 1-dimensional Lie algebra, is has to
                be abelian. Thus, up to isomorphism, there is only one Lie algebra on
                dimension 1 and it is abelian. 
        \end{itemize}
        \begin{lemma}
            Suppose $I, J$ are ideals of $L$. Then 
                \begin{equation*}
                    I+J=\{x+y\mid x\in I, y\in J\}
                \end{equation*}
            and 
                \begin{equation}
                    [I, J]=\{[x, y]\mid x\in I, y\in J\}
                \end{equation}
                are ideals. ((1) is the set of all linear combinations of
                elements, or the span/set generated by, the elements in $[I,
            J]$).
        \end{lemma}\newpage
        \begin{exmp}
            $[L, L]$ is an ideal. Note if $[L, L]=\{0\}$, then $L$ is abelian.
            Suppose $\dm(L)=3$. Then is could be (with $L':=[L, L]$) that 
                \begin{itemize}
                    \item$\dm{L'}=3\rightarrow[L, L]=L$
                    \item $\dm{L'}=2$
                    \item$\dm{L'}=1$
                    \item$\dm{L'}=0\rightarrow$ $L$ is abelian.
                \end{itemize}
        \end{exmp}
        \begin{definition}
            The set 
                \begin{equation*}
                    Z(L):=\{x\in L\mid[x, y]=0, \forall y\in L\}
                \end{equation*}
            is called the \textbf{center} of $L$.
        \end{definition}
        \begin{lemma}
            $Z(L)$ is an ideal of $L$.
        \end{lemma}
        \begin{itemize}
            \item\textbf{Q}: Is $Z(L)$ abelian? i.e., what is $[Z(L),
                Z(L)]=Z(L)'=?$
            \item\textbf{A}: Yes. Note the differences between $L'$ and $Z(L)$. So
                the $L'$ is the non-abelian piece and the center is the abelian
                piece. 
        \end{itemize}
        We want to start classifying simple Lie algebras. Later we'll be
        interested in cases where $L'=L$ since if it doesn't satisfy this it
        will not be simple. 
        \subsection{Some chit-chat}
        For any $x\in L$, with $\text{dim}(L)=n$, we can define a map $\ad{x}:L\to L$ where
        $y\mapsto\ad{x}(y):=[x, y]$ and this is a homomorphism, i.e.,
        $\ad{x}\in\text{gl}(L)=\text{gl}_{n}(\mathbb{C})$. Next level: there
        exists a map 
            \begin{equation}
                \ad{}:L\to\text{gl};\quad x\mapsto\ad{x}
            \end{equation}
        and $\ad{}$ is a Lie algebra homomorphism. Is $\ad{}$ 1-1? In other
        words, does $\ker(\ad{})=\{0\}$? In other words, 
            \begin{equation*}
                \ker(\ad{})=\{x\in L\mid \ad{x}=0\}=\{\ad{x}\mid\forall y\in
                L\}=\{[x, y]=0\mid \forall y\in L\}=Z(L).
            \end{equation*}
    \section{02/08/21}
        \subsection{Structure Constants}
            \begin{remark}
                $\slin{2}{\C}=\{A\in\mat{2}{\C}\mid\text{tr}(A)=0\}$, $[A,
                B]=0$. 3 dimensional Lie algebra with basis
                $\mathcal{B}=\{h=\begin{psmall}1&0\\0&-1\end{psmall},
                    x=\begin{psmall}0&1\\0&0\end{psmall},
                y=\begin{psmall}0&0\\1&0\end{psmall}$\}\par\hspace{4mm} Suppose we
                have 
                    \begin{equation*}
                        [x, y]=h, \quad[h, x0]=2x, \quad[h, y]=-2y.
                    \end{equation*}
            \end{remark}
            For a fixed $x\in L$, define $\ad{x}:L\to L$, by
            $y\mapsto\ad{x}{y}:=[x, y]$. This is a linear map, i.e.,
            $\ad{x}\in\End{L}\cong\mat{n}{\C}\cong\mathfrak{gl}(L)\cong\mathfrak{gl}_{n}(\C)$.
            \subsection{Representations}
                \begin{definition}
                    A Lie algebra homomorphism
                    $\varphi:L\to\text{gl}(V)\cong\text{gl}_n(\C)$ for some
                    vector space $V$ and $\text{dim}(V)=n$ is called
                    \textbf{representation} of $L$. 
                \end{definition}
                \begin{definition}
                    We call a vector space $V$ an $L$-module if there is a map
                        \begin{equation*}
                            \begin{split}
                                &L\times V\to V \\
                                &(x, v)\mapsto x.v
                            \end{split}
                        \end{equation*}
                    such that $\forall x, y\in L$, $u, v\in V$, and
                    $\alpha, \beta\in F$ where
                        \begin{enumerate}
                            \item $(\alpha x+\beta y).u=\alpha(x.u)+\beta(y.u)$
                            \item $x.(\alpha u+\beta v)=\alpha(x.u)+\beta(x.v)$
                            \item [x, y].u=x.(y.u)-y.(x.u)
                        \end{enumerate}
                \end{definition}
    \section{2021-02-10}
        Recall: $\varphi:L\to\mathfrak{gl}_{n}(V)$ if $\varphi$ is a Lie
        homomorphism. Often refter ro $V$ as the representation.
        \par\hspace{4mm} A vector space $V$ is an $L$-module if $L\times V\to
        V$ and $(x, v)\mapsto x.v$. This action is biliniear.
        \begin{exmp}
            Let $\varphi=\text{ad}$, $V=L$. We've seen
            $\text{ad}:L\to\mathfrak{gl}_{n}(L)$ is a representation and
            $x\mapsto\text{ad}(x):=\text{ad}_x$.
        \end{exmp}
\end{document}
