\documentclass[12pt]{article}
\usepackage[margin=1in]{geometry}
\usepackage{graphicx}
\usepackage{amsmath}
\usepackage{amsthm}
\usepackage{amsfonts}
\usepackage{amssymb}
\usepackage{array}
\usepackage{enumerate}
\usepackage{slashed}
\usepackage{colonequals}
\usepackage{fancyhdr}
\usepackage{import}
\usepackage{xifthen}
\usepackage{pdfpages}
\usepackage{transparent}
\usepackage{enumitem}

\newcommand{\incfig}[1]{%
    \def\svgwidth{\columnwidth}
    \import{/home/arbegla/figures/}{#1.pdf_tex}
}

\pagestyle{fancy}
\fancyhf{}
\rhead{}
\lhead{}
\rfoot{\thepage}
\setlength{\headheight}{10pt}

\newtheorem{theorem}{Theorem}[section]
\newtheorem{corollary}{Corollary}[theorem]
\newtheorem{prop}{Proposition}[section]
\newtheorem{lemma}[theorem]{Lemma}
\theoremstyle{definition}
\newtheorem{definition}{Definition}[section]
\theoremstyle{definition}
\newtheorem{exmp}{Example}[section]

\newcommand{\abs}[1]{\lvert #1 \rvert}
\newcommand{\bigabs}[1]{\Bigl \lvert #1 \Bigr \rvert}
\newcommand{\bigbracket}[1]{\Bigl [ #1 \Bigr ]}
\newcommand{\bigparen}[1]{\Bigl ( #1 \Bigr )}
\newcommand{\ceil}[1]{\lceil #1 \rceil}
\newcommand{\bigceil}[1]{\Bigl \lceil #1 \Bigr \rceil}
\newcommand{\floor}[1]{\lfloor #1 \rfloor}
\newcommand{\bigfloor}[1]{\Bigl \lfloor #1 \Bigr \rfloor}
\newcommand{\norm}[1]{\| #1 \|}
\newcommand{\bignorm}[1]{\Bigl \| #1 \Bigr \| #1}
\newcommand{\inner}[1]{\langle #1 \rangle}
\newcommand{\set}[1]{{ #1 }}

    %\begin{figure}[htp!]
    %    \centering
    %      \incfig{S5}
    %      \caption{Group Action}
    %      \label{fig:action}
    %  \end{figure}\

\begin{document}
\title{Lie Theory Notes}
\author{Quin Darcy}
\date{Jan, 26 2021}
\maketitle
    \section{Preliminary Definitions and Examples}
        \begin{definition}
            A \textbf{Lie Algebra} is a finite-dimensional vector space
            $\mathfrak{g}$, over
            a real or complex base field $K$,
            together with a map $[\cdot, \cdot]$ from
            $\mathfrak{g}\times\mathfrak{g}$ into $\mathfrak{g}$ with the
            following properties:
                \begin{enumerate}
                    \item $[\cdot, \cdot]$ is \textbf{bilinear}. For all $u, v, w\in\mathfrak{g}$ and for all
                        $\lambda\in K$ we have
                        \begin{enumerate}[label=(\roman*)]
                            \item $[\lambda u, v]=[u, \lambda
                                v]=\lambda([u, v])$.
                            \item $[u+w, v]=[u, v]+[w, v]$
                            \item $[u, w+v]=[u, w]+[u, v]$
                        \end{enumerate}
                    \item $[\cdot, \cdot]$ is \textbf{skew symmetric}. For all
                        $u, v\in\mathfrak{g}$ we have $[u, v]=-[v, u]$.
                    \item The \textbf{Jacobi identity} holds: For all
                        $u, v, w\in\mathfrak{g}$ 
                        \begin{equation*}
                            [u, [v, w]]+[v, [w, u]]+[w, [u, v]]=0.
                        \end{equation*}
                \end{enumerate}
            Two elements $u, v$ of a Lie Algebra are said to commute provided
            $[u, v]=0$. Moreover, a Lie Algebra is said to be commutative if
            $[u, v]=0$ for all $u, v\in\mathfrak{g}$.
        \end{definition}
        \begin{exmp}
            Let $\mathfrak{g}=\mathbb{R}^3$ and define $[\cdot,
            \cdot]:\mathbb{R}^3\times\mathbb{R}^3\to\mathbb{R}^3$ by
                \begin{equation*}
                    (\mathbf{u},
                    \mathbf{v})\mapsto\norm{\mathbf{u}}\norm{\mathbf{v}}\sin\theta\cdot\mathbf{n},
                \end{equation*}
            where $\theta$ is the angle between $\mathbf{u}$ and $\mathbf{v}$
            and $\mathbf{n}$ is the unit vector perpendicular to the plane
            containing $\mathbf{u}$ and $\mathbf{v}$. Since for any plane,
            there are two vectors perpendicular to it, the ``right-hand rule"
            is used to make the selection. 
        \end{exmp}
        \begin{definition}
            Given a Lie Algebra $\mathfrak{g}$ and a subspace, $\mathfrak{h}$,
            of $\mathfrak{g}$, then $\mathfrak{h}$ is called
            a \textbf{subalgebra} of $\mathfrak{g}$ if for all $h_1,
            h_2\in\mathfrak{h}$, we have that $[h_1, h_2]\in\mathfrak{h}$.
        \end{definition}
        \begin{definition}
            Given a Lie Algebra $\mathfrak{g}$ and a subalgebra, $\mathfrak{h}$
            of $\mathfrak{g}$, we say that $\mathfrak{h}$ is an \textbf{ideal}
            of $\mathfrak{g}$ if for all $u\in\mathfrak{g}$ and for all
            $h\in\mathfrak{h}$, we have $[u, h]\in\mathfrak{h}$.
        \end{definition}
        \begin{definition}
            The \textbf{center} of a Lie Algebra is the set consisting of all
            the elements $u\in \mathfrak{g}$ such that for all $v\in
            \mathfrak{g}$, $[u, v]=0$.
        \end{definition}
    \section{Simple, Solvable, and Nilpotent Lie Algebras}
        Now that we know what a Lie Algebra is, we will explore some different
        types of Lie Algebras.
        \begin{definition}
            Given a Lie Algebra $\mathfrak{g}$, if the only ideals of
            $\mathfrak{g}$ are trivial (meaning $\{0\}$ and $\mathfrak{g}$
            itself), then $\mathfrak{g}$ is called \textbf{irreducible}.
        \end{definition}
        \begin{definition}
            If $\mathfrak{g}$ is an irreducible Lie Algebra such that the
            dimension of $\mathfrak{g}$ is greater than or equal to 2, then
            $\mathfrak{g}$  is called \textbf{simple}.
        \end{definition}
        \begin{prop}
            If $\mathfrak{g}$ is a one-dimensional Lie Algebra, then
            $\mathfrak{g}$ is commutative.
        \end{prop}
            \begin{proof}
                Let $\{e_1\}$ be a basis for $\mathfrak{g}$ as a vector space
                over $K$,  and let
                $u, v\in\mathfrak{g}$. Then for some $a, b\in K$ it follows
                that $u=ae_1$ and $v=be_1$. Thus by the bilinearity of the
                bracket operator we have 
                    \begin{equation*}
                        [u, v] = [ae_1, be_2]=(ab)[e_1, e_1]=(ab)0=0.
                    \end{equation*}
                Hence, $\mathfrack{g}$ is commutative.
            \end{proof}
\end{document}
