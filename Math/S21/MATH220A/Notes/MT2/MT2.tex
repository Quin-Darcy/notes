
\documentclass[12pt]{article}
\usepackage[margin=1in]{geometry}
\usepackage{graphicx}
\usepackage{amsmath}
\usepackage{amsthm}
\usepackage{amsfonts}
\usepackage{amssymb}
\usepackage{array}
\usepackage{enumerate}
\usepackage{slashed}
\usepackage{colonequals}
\usepackage{fancyhdr}
\usepackage{enumitem}

\pagestyle{fancy}
\fancyhf{}
\rhead{Darcy}
\lhead{MATH 220A}
\rfoot{\thepage}
\setlength{\headheight}{10pt}

\newtheorem{theorem}{Theorem}[section]
\newtheorem{corollary}{Corollary}[theorem]
\newtheorem{prop}{Proposition}[section]
\newtheorem{lemma}[theorem]{Lemma}
\theoremstyle{definition}
\newtheorem{definition}{Definition}[section]
%\theoremstyle{remark}
%\newtheorem*{remark}{Remark}

\newcommand{\abs}[1]{\lvert #1 \rvert}
\newcommand{\bigabs}[1]{\Bigl \lvert #1 \Bigr \rvert}
\newcommand{\bigbracket}[1]{\Bigl [ #1 \Bigr ]}
\newcommand{\bigparen}[1]{\Bigl ( #1 \Bigr )}
\newcommand{\ceil}[1]{\lceil #1 \rceil}
\newcommand{\bigceil}[1]{\Bigl \lceil #1 \Bigr \rceil}
\newcommand{\floor}[1]{\lfloor #1 \rfloor}
\newcommand{\bigfloor}[1]{\Bigl \lfloor #1 \Bigr \rfloor}
\newcommand{\norm}[1]{\| #1 \|}
\newcommand{\bignorm}[1]{\Bigl \| #1 \Bigr \| #1}
\newcommand{\inner}[1]{\langle #1 \rangle}
\newcommand{\set}[1]{{ #1 }}

\begin{document} \thispagestyle{empty}\hrule

    \begin{center}
        \vspace{.4cm} { \large MATH 220A}
    \end{center}
    {Name:\ Quin Darcy \hspace{\fill} Due Date: NONE \\
    { Instructor:}\ Dr. Martins \hspace{\fill} Assignment:
    Midterm 2 Notes \\ \hrule}
    \begin{enumerate}
            \item Let $A, B$ and $A_{\alpha}$ denote subsets of a topological space $X$. Prove the following:
            \begin{enumerate}
            \item If $A\subset B$, then $\overline{A}\subset \overline{B}$.
                \begin{proof}
                    We have that $A\subset B$, and that $B\subset \overline{B}$. And so 
                    $A\subset\overline{B}$. This means that $\overline{B}$ is a closed set 
                    containing $A$. Since $\overline{A}$ is defined to be the intersection 
                    of all closed sets containing $A$, then it follows that $\overline{B}$ 
                    is in this intersection. Thus $\overline{A}\subset\overline{B}$.
                \end{proof}
            \item $\overline{A\cup B}=\overline{A}\cup\overline{B}$.
                \begin{proof}
                    Since $A\subset A\cup B$, then by (a), $\overline{A}\subset\overline{A\cup B}$. 
                    Similarly, with $B\subset A\cup B$, it follows that 
                    $\overline{B}\subset\overline{A\cup B}$. Therefore, 
                    $\overline{A}\cup\overline{B}\subset\overline{A\cup B}$. Now with 
                    $A\subset \overline{A}$, comes $A\subset\overline{A}\cup\overline{B}$. 
                    And with $B\subset\overline{B}$, comes $B\subset\overline{A}\cup\overline{B}$. 
                    Thus $A\cup B\subset\overline{A}\cup\overline{B}$. Hence, by (a), 
                    $\overline{A\cup B}\subset\overline{A}\cup\overline{B}$, since 
                    $\overline{A}\cup\overline{B}$ is a closed subset as a finite 
                    union of closed sets.
                \end{proof}
            \item $\bigcup_{\alpha\in J}\overline{A}_{\alpha}\subset\overline{\bigcup_{\alpha\in J}A_{\alpha}}$.
                \begin{proof}
                    Let $\alpha_0\in J$ be any fixed element. Then 
                    $A_{\alpha_0}\subset\bigcup_{\alpha\in J}A_{\alpha}$. Hence, 
                    by (a), $\overline{A}_{\alpha_0}\subset\overline{\bigcup_{\alpha\in J}A_{\alpha}}$. 
                    Since this choice of $\alpha_0$ was arbitrary, then the 
                    containment hold for all such $\alpha\in J$. Hence, 
                    $\bigcup_{\alpha\in J}\overline{A}_{\alpha}\subset
                    \overline{\bigcup_{\alpha\in J}A_{\alpha\in J}}$.
                \end{proof}
        \end{enumerate}
        \item Let $A$ and  $B$ be two subsets of a set $X$. Then $A\cap B\neq\varnothing$
        if and only if $(A\times B)\cap \Delta\neq \varnothing$, where
        $\Delta=\{(x, x)\in X\times X\mid x\in X\}$.
        \begin{proof}
            Suppose that $A\cap B\neq\varnothing$, then there is some $x\in
            X$ such that $x\in A\cap B$. With $x\in X$, we get that $(x,
            x)\in \Delta$ and also that $(x, x)\in A\times B$. Hence, $(x,
            x)\in (A\times B)\cap\Delta$ and thus $(A\times
            B)\cap\Delta\neq\varnothing$.\par\hspace{4mm} Conversely,
            assume that $(A\times B)\cap\Delta\neq\varnothing$. Then there
            exists some $(x, y)\in (A\times B)\cap\Delta$. Since $(x,
            y)\in\Delta$, then we may write $x=y$ and so we have that $(x,
            x)\in A\times B$. Hence, $x\in A$ and $x\in B$. Therefore,
            $A\cap B\neq\varnothing$.
        \end{proof}
        \item Show that $X$ is Hausdorff if and only if the diagonal
            \begin{equation*}
                \Delta=\{(x, x)\in X\times X\mid x\in X\}
            \end{equation*}
            is closed in $X\times X$.
            \begin{proof}
                Assume that $X$ is a Hausdorff topological space. We will show
                that $\Delta$ is closed in $X\times X$ by showing that its
                complement, $X-\Delta$, is open in $X\times X$.\par\hspace{4mm}
                Let $(x, y)\in X-\Delta$. Then $(x, y)\notin \Delta$ which
                implies that $x\neq y$. Since $X$ is Hausdorff, then we may
                find open neighborhoods, $U\subset X$, of $x$, and $V\subset
                X$, of V, such that $U\cap V=\varnothing$. Hence, $U\times V$
                is an open neighborhood of $(x, y)$ in $X\times X$.  Further,
                we recall that $U\cap V=\varnothing$, which by 2., we get that
                $(U\times V)\cap\Delta=\varnothing$. This implies that $U\times
                V\subset(X-\Delta)$. Hence, $X-\Delta$ is open.\par\hspace{4mm}
                Suppose now that $\Delta$ is closed in $X\times X$. Let $x$ and
                $y$ be any two distict points in $X$ and consider the pair $(x,
                y)\in X-\Delta$. Since $X-\Delta$ is open, then there is
                a neighborhood $U\times V$ of $(x, y)$ contained in $X-\Delta$,
                where $U$ is a neighborhood of $x$ and $V$ is a neighborhood of
                $y$. With $U\times V$ being a neighborhood in $X-\Delta$, this
                implies that $(U\times V)\cap\Delta=\varnothing$. Hence, by 2.,
                it follows that $U\cap V=\varnothing$. Therfore, $X$ is
                Hausdorff.
            \end{proof}
        \item If $A, U\subset X$, we define the boundary of $A$ by:
            \begin{equation*}
                \partial A=\overline{A}\cap\overline{X-A}.
            \end{equation*}
            \begin{enumerate}
                \item Show that $\mathring{A}$ and $\partial A$ are disjoint,
                    and that $\overline{A}=\mathring{A}\cup\partial A$.
                    \begin{proof}
                        To show that $\mathring{A}$ and 
                    \end{proof}
            \end{enumerate}
    \end{enumerate}
\end{document}
