\documentclass[12pt]{article}
\usepackage[margin=1in]{geometry}
\usepackage{graphicx}
\usepackage{amsmath}
\usepackage{amsthm}
\usepackage{amsfonts}
\usepackage{amssymb}
\usepackage{array}
\usepackage{enumerate}
\usepackage{slashed}
\usepackage{colonequals}
\usepackage{fancyhdr}
\usepackage{enumitem}

\pagestyle{fancy}
\fancyhf{}
\rhead{Darcy}
\lhead{MATH 220A}
\rfoot{\thepage}
\setlength{\headheight}{10pt}

\newtheorem{theorem}{Theorem}[section]
\newtheorem{corollary}{Corollary}[theorem]
\newtheorem{prop}{Proposition}[section]
\newtheorem{lemma}[theorem]{Lemma}
\theoremstyle{definition}
\newtheorem{definition}{Definition}[section]
%\theoremstyle{remark}
%\newtheorem*{remark}{Remark}

\newcommand{\abs}[1]{\lvert #1 \rvert}
\newcommand{\bigabs}[1]{\Bigl \lvert #1 \Bigr \rvert}
\newcommand{\bigbracket}[1]{\Bigl [ #1 \Bigr ]}
\newcommand{\bigparen}[1]{\Bigl ( #1 \Bigr )}
\newcommand{\ceil}[1]{\lceil #1 \rceil}
\newcommand{\bigceil}[1]{\Bigl \lceil #1 \Bigr \rceil}
\newcommand{\floor}[1]{\lfloor #1 \rfloor}
\newcommand{\bigfloor}[1]{\Bigl \lfloor #1 \Bigr \rfloor}
\newcommand{\norm}[1]{\| #1 \|}
\newcommand{\bignorm}[1]{\Bigl \| #1 \Bigr \| #1}
\newcommand{\inner}[1]{\langle #1 \rangle}
\newcommand{\set}[1]{{ #1 }}

\begin{document}
    \thispagestyle{empty}\hrule

    \begin{center}
        \vspace{.4cm} { \large MATH 220A}
    \end{center}
    {Name:\ Quin Darcy \hspace{\fill} Due Date: 05/02/21 \\
    { Instructor:}\ Dr. Martins \hspace{\fill} Assignment:
    Homework 5 \\ \hrule}

    \begin{enumerate}
        \item[23.2.] Let $\{A_n\}$ be a sequence of connected subspaces of $X$,
            such that $A_n\cap A_{n+1}\neq \varnothing$ for all $n$. Show that
            $\bigcup A_n$ is connected.
            \begin{proof}
                Assume for contradiction, that $\bigcup A_n$ is not connected.
                Then there exists nonempty open subsets $U, V\subset\bigcup
                A_n$ such that $U\cap V=\varnothing$ and $U\cup V=\bigcup A_n$.
                By Lemma 23.2, for each $n$, $A_n$ is entirely contained in
                either $U$ or $V$. Moreover, 
                as $\bigcup A_n$ is a sequence of subspaces, it follows that
                for some $m$, $A_m\subset U$, but $A_{m+1}\not\subset U$, and
                instead, $A_{m+1}\in V$. However, since $A_m\cap
                A_{m+1}\neq\varnothing$, then $A_m\cap A_{m+1}\subset U\cap
                V\neq\varnothing$, which is a contradiction.
            \end{proof}
        \item[23.6.] Let $A\subset X$. Show that if $C$ is a connected subspace
            of $X$ that intersects both $A$ and $X-A$, then $C$ intersected
            $\text{Bd}A$.
            \begin{proof}
                (Two contradiction proofs in a row, don't hate me!) Assume that
                $C$ does not intersect the boundary of $A$. Then since
                $\overline{A}=\text{Int}A\cup\text{Bd}A$, and $C$ intersects
                $A$, it follows that $C\cap\text{Int}A\neq\varnothing$. By
                similar reasoning, we have that
                $C\cap\text{Int}(X-A)\neq\varnothing$. Since
                $\text{Int}A\cap\text{Int}(X-A)=\varnothing$ and
                $\text{Int}A\cup\text{Int}(X-A)=C$, then the two previous sets
                are a seperation of $C$ and therefore $C$ is not connected.
            \end{proof}
        \item[24.1]\hfill\par
            \begin{enumerate}
                \item Show that no two of the spaces $(0, 1)$, $(0, 1]$, $[0,
                    1]$ are homeomorphic.
                    \begin{proof}
                        By Corollary 24.2, each of the above intervals are
                        connected in $\mathbb{R}$. However, note that if
                        $a\in(0, 1)$ and we let $A=(0, 1)/\{a\}$, then $A$ is
                        not connected since we can let $U=(0, a)$ and $V=(a,
                        1)$ and this would constitute a serperation. On the
                        other hand, if we let $B=(0, 1]/\{1\}=(0, 1)$, then $B$
                        is connected. Further note that if $(0, 1)$ was
                        homeomorphic to $(0, 1]$, then $A$ would be
                        homeomorphic to $B$, however this is not the case as
                        $A$ is not connected but $B$ is. Therefore, $(0, 1)$ is
                        not homeomorphic to $(0, 1]$. By the same reasoning, we
                        can show that $(0, 1)$ is not homeomorphic to $[0, 1)$.
                        Now note that if $a\in [0, 1]$ such that $0<a<1$, then
                        $[0, 1]/\{a\}$ is not connected since we can let $U=[0,
                        a)$ and $V=(a, 1]$ and this forms a seperation on
                        $[0,1]/\{a\}$. However, by removing either $0$ or $1$
                        from $[0, 1]$ we preserve its connectedness. Though
                        doing so results in either $(0, 1]$ or $[0, 1)$, both
                        of which we have shown not to be homeomorphic to (0,
                        1). 
                    \end{proof}
                \item Show $\mathbb{R}^n$ and $\mathbb{R}$ are not homeomorphic
                    if $n>1$.
                    \begin{proof}
                        By Example 4 in section 24, we have that the punctured
                        euclidean space $\mathbb{R}^n-\{0\}$ is path connected
                        and therefore connected, for $n>1$. However,
                        $\mathbb{R}-\{0\}$ is not connected since $U=(-\infty,
                        0)$ and $V=(0, \infty)$ forms a seperation on the set.
                        Therefore, for $n>1$, $\mathbb{R}$ and $\mathbb{R}^n$
                        are not homeomorphic.
                    \end{proof}
            \end{enumerate}
        \item[24.3] Let $f:X\to X$ be continuous. Show that if $X=[0, 1]$, there is
            a point $x$ such that $f(x)=x$. What happens if $X$ equals $[0, 1)$
            or $(0, 1)$.
            \begin{proof}
                Consider the funtion $h(x)=f(x)-id(x)$, where $id:X\to X$ is
                the identity map on $X$. Since both $f$ and $id$ are continuous
                on $X$, then $h$ is continuous on $X$. Note that if either
                $f(0)=0$ or $f(1)=1$, then this proves the claim. On the other
                hand, if $f(0)\neq 0$ and $f(1)\neq 1$, then considering the
                range of $f$, this would imply that $f(0)>0$ and $f(1)<1$. Now
                given that $h$ is continuous on $X$, $X$ is connected, and $X$ is an ordered set in the
                order topology, then we may apply the Intermediate Value
                Theorem. Namely, we have that $h(1)<0<h(0)$ and thus by IVP,
                there exists some $x\in X$ such that $h(x)=0$. This implies
                that $f(x)-id(x)=f(x)-x=0$. Hence, $f(x)=x$.\par\hspace{4mm} If
                $X=[0, 1)$ or $X=(0, 1)$, then we want to show that there does not exist
                a fixed point. To do this, we need a counterexample consisting
                of a function continuous on $X$, but such that there is no
                $x\in X$ for which $f(x)=x$. Letting $f(x)=(x+1)/2$, we see
                that $f$ is continuous as a scalar multiple of a continuous
                function. Now if we assume that $f(x)=x$, for some $x\in X$,
                then we have that $(x+1)/2=x$, which implies that $x=1$.
                However, this cannot be as $1\not\in X$ for $X=[0, 1)$, nor for
                $X=(0, 1)$.
            \end{proof}
        \item[24.10] Show that if $U$ is an open connected subspace of
            $\mathbb{R}^2$, then $U$ is path connected.
            \begin{proof}
                Let $x_0\in U$, and define $P$ as the set of all points, $x$, for
                which there exists a path connecting $x_0$ to $x$. Then if
                $x\in P$, then $x\in U$ and since $U$ is open, there exists
                a neighborhood $V$ of $x$ contained in $U$. Specifically, we
                may assume that $V\cap U=V$. Now we cite Example 3 in Section
                24, which states that the unit ball $B^n$ is path connected in
                $\mathbb{R}^n$. Thus, open balls in $\mathbb{R}^2$ are path
                connected. With $V$ being such an open ball around $x$, we
                observe that if $\gamma$ is a path connecting $x_0$ and $x$,
                then for any other $x'\in V$, we can construct a path $\gamma'$
                from $x$ to $x'$. Thus taking $\lambda=\gamma'\circ\gamma$, we
                have a path from $x_0$ to $x'$. This implies that for all
                $x'\in V$, there exists a path connecting it to $x_0$. Hence,
                $V\subset P$. Therefore, $P$ is open.\par\hspace{4mm} To show
                that $P$ is also closed, we want to show that $U-P$ is open.
                Letting $x\in U-P$, then as $U$ is open, there exists an open
                ball $B$ such that $x\in B$ and $B\subset U$. Now assume that
                $x'\in B\cap P$. Then $x'\in P$ which implies that there exists
                a path from $x_0$ to $x'$. But we also have that $x\in B$ and
                so there is a path from $x'$ to $x$. As before we can then
                construct a path from $x_0$ to $x$. This implies that $x\in P$,
                which is a contradiction. Therefore, $B\cap P=\varnothing$ and
                so $U-P$ is open. Hence, $P$ is closed. We have shown that for
                every such $P\subset U$, $P$ is open and closed which implies
                that $P$ is connected.
            \end{proof}
    \end{enumerate}
\end{document}
