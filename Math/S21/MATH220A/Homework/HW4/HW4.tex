\documentclass[12pt]{article}
\usepackage[margin=1in]{geometry}
\usepackage{graphicx}
\usepackage{amsmath}
\usepackage{amsthm}
\usepackage{amsfonts}
\usepackage{amssymb}
\usepackage{array}
\usepackage{enumerate}
\usepackage{slashed}
\usepackage{colonequals}
\usepackage{fancyhdr}
\usepackage{enumitem}
\usepackage{import}
\usepackage{xifthen}
\usepackage{pdfpages}
\usepackage{transparent}

\pagestyle{fancy}
\fancyhf{}
\rhead{Darcy}
\lhead{MATH 220A}
\rfoot{\thepage}
\setlength{\headheight}{10pt}

\newcommand{\incfig}[1]{%
    \def\svgwidth{\columnwidth}
    \import{/home/arbegla/figures/}{#1.pdf_tex}
}

\newtheorem{theorem}{Theorem}[section]
\newtheorem{corollary}{Corollary}[theorem]
\newtheorem{prop}{Proposition}[section]
\newtheorem{lemma}[theorem]{Lemma}
\theoremstyle{definition}
\newtheorem{definition}{Definition}[section]
%\theoremstyle{remark}
%\newtheorem*{remark}{Remark}

\newcommand{\abs}[1]{\lvert #1 \rvert}
\newcommand{\bigabs}[1]{\Bigl \lvert #1 \Bigr \rvert}
\newcommand{\bigbracket}[1]{\Bigl [ #1 \Bigr ]}
\newcommand{\bigparen}[1]{\Bigl ( #1 \Bigr )}
\newcommand{\ceil}[1]{\lceil #1 \rceil}
\newcommand{\bigceil}[1]{\Bigl \lceil #1 \Bigr \rceil}
\newcommand{\floor}[1]{\lfloor #1 \rfloor}
\newcommand{\bigfloor}[1]{\Bigl \lfloor #1 \Bigr \rfloor}
\newcommand{\norm}[1]{\| #1 \|}
\newcommand{\bignorm}[1]{\Bigl \| #1 \Bigr \| #1}
\newcommand{\inner}[1]{\langle #1 \rangle}
\newcommand{\set}[1]{{ #1 }}

\begin{document} \thispagestyle{empty}\hrule

    \begin{center}
        \vspace{.4cm} { \large MATH 220A}
    \end{center}
    {Name:\ Quin Darcy \hspace{\fill} Due Date: 4/4/21 \\
    { Instructor:}\ Dr. Martins \hspace{\fill} Assignment:
    Homework 4 \\ \hrule}

    \begin{enumerate}
        \item[19.1] Prove that for functions $f:\mathbb{R}\to\mathbb{R}$, the $\varepsilon-\delta$ definition 
        of continuity implies the open set definition.
            \begin{proof}
                Assume that $f:\mathbb{R}\to\mathbb{R}$ is continuous and let $V\subset\mathbb{R}$ 
                be any open set in the range of $f$. Then we want to show that $f^{-1}(V)$ is open.
                To do this we let $a\in f^{-1}(V)$ and show that $a$ is contained in a 
                neighborhood which is contained in $f^{-1}(V)$. Since $f$ is continuous, then 
                for any $\varepsilon>0$, there exists $\delta>0$ such that 
                $f(x)\in(f(a)-\varepsilon,f(a)+\varepsilon)$ for all $x\in\mathbb{R}$ such that 
                $\abs{x-a}<\delta$. Hence, $a\in(a-\delta, a+\delta)\subset f^{-1}(V)$. Therefore 
                every point of $f^{-1}(V)$ is an interior point.
            \end{proof}
        \item[19.5.] Show that the subspace $(a,b)$ of $\mathbb{R}$ is homeomorphic with $(0,1)$ and 
        the subspace $[a,b]$ is homeomorphic with $[0,1]$.
            \begin{proof}
                We begin by defining the following function $f:(a,b)\to(0,1)$:
                \begin{equation*}
                    f(x)=\frac{x-a}{b-a}.
                \end{equation*}
                We want to show that $f$ is a bijection, $f$ is continuous, and
                $f^{-1}$ is continuous. We will first let $x_1, x_2\in(a, b)$
                and assume that $f(x_1)=f(x_2)$. Then this implies that 
                \begin{equation*}
                    \frac{x_1-a}{b-a}=\frac{x_2-a}{b-a}\Rightarrow
                    x_1-a=x_2-a\Rightarrow x_1=x_2.
                \end{equation*}
                Hence, $f$ is injective. Next, if $y\in (0, 1)$, then choosing
                $x\in(a, b)$ such that $x=y(b-a)+a$, then we get that $f(x)=y$.
                Hence, $f$ is surjective.\par\hspace{4mm} Having shown that $f$
                is a bjection, we now want to show that $f$ and $f^{-1}$ are
                both continuous over $(a, b)$ and $(0, 1)$, respectively. Let
                $c\in (a, b)$ and let $\varepsilon>0$. Then for
                $\delta=\varepsilon\abs{b-a}$, it follows that for all $x\in(a, b)$
                \begin{equation*}
                    \begin{split}
                        \abs{x-c}<\delta&\Rightarrow
                        \abs{x-c}<\varepsilon\abs{b-a} \\
                        &\Rightarrow \frac{\abs{x-c}}{\abs{b-a}}<\varepsilon \\
                        &\Rightarrow\bigabs{\frac{x-c}{b-a}}<\varepsilon \\
                        &\Rightarrow\bigabs{\frac{x-a}{b-a}-\frac{c-a}{b-a}}<\varepsilon
                        \\
                        &\Rightarrow\abs{f(x)-f(c)}<\varepsilon.
                    \end{split}
                \end{equation*}
                For all $c\in (a, b)$. This shows that $f$ is continuous. Now we need to show
                that $f^{-1}(y)=y(b-a)+a$ is continuous at every $x\in (0,1)$. Letting
                $\varepsilon>0$ and selecting the same
                $\delta=\varepsilon/\abs{b-a}$ as before we find that for all
                $y\in(0, 1)$ 
                \begin{equation*}
                    \begin{split}
                        \abs{y-c} &<\delta \\
                        &\Rightarrow \abs{y-c} < \frac{\varepsilon}{\abs{b-a}}
                        \\
                        &\Rightarrow \abs{y-c}\abs{b-a} <\varepsilon \\
                        &\Rightarrow \abs{(y-c)(b-a)} <\varepsilon \\
                        &\Rightarrow \abs{y(b-a)-c(b-a)} <\varepsilon \\
                        &\Rightarrow \abs{y(b-a)+a-c(b-a)-a}< \varepsilon \\
                        &\Rightarrow \abs{f^{-1}(y)-f^{-1}(c)}<\varepsilon.
                    \end{split}
                \end{equation*}
                Therefore $(a, b)$ is homeomorphic to $(0, 1)$. Lastly, we note
                that both $f$ and $f^{-1}$ are defined on the end points of
                their respective intervals and that $f(a)=0$, $f(b)=1$,
                $f^{-1}(0)=a$, and $f^{-1}(1)=b$. Hence, the given function,
                $f$, also shows that $[a, b]$ and $[0, 1]$ are homeomorphic.
            \end{proof}
        \item[19.13.] Let $A\subset X$; let $f:A\to Y$ be continuous; let $Y$ be
            Hausdorff. Show that if $f$ may be extended to a continuous
            function $g:\overline{A}\to Y$, then $g$ is uniquely determined by
            $f$.
            \begin{proof}
                Let $g:\overline{A}\to Y$ and $h:\overline{A}\to Y$ be
                continuous extensions of $f$. Assume there exists 
                $x\in \overline{A}$ such that $g(x)\neq h(x)$. Then since $Y$
                is Hausdorff and $g(x), h(x)\in Y$, then there exists
                neighborhoods $U_1$ and $U_2$ such that $g(x)\in U_1$, $h(x)\in
                U_2$, and $U_1\cap U_2=\varnothing$. We also note that $x\in
                g^{-1}(U_1)$ and $x\in h^{-1}(U_2)$. Hence, $x\in
                g^{-1}(U_1)\cap h^{-1}(U_2)$, which, by continuity, is open as
                it is the intersection of open sets. With $x\in \overline{A}$,
                then either $x\in A$ or $x\in A'$. If $x\in A$, then
                $f(x)=g(x)=h(x)$, which is a contradiction. If $x\in A'$, then
                there exists some $v\in A$ such that $x\neq v$ and $v\in
                g^{-1}(U_1)\cap h^{-1}(U_2)$. This implies that
                $f(v)=g(v)=h(v)$. Hence, $U_1\cap U_2\neq\varnothing$, which is
                a contradiction. Therefore, there does not exist
                $x\in\overline{A}$ such that $g(x)\neq h(x)$ and thus $g = h$.
            \end{proof}
        \item[20.1.] 
            \begin{enumerate}[label = (\alph*)]
                \item In $\mathbb{R}^n$, define 
                    \begin{equation*}
                        d'(\mathbf{x},
                        \mathbf{y})=\abs{x_1-y_1}+\cdots+\abs{x_n-y_n}.
                    \end{equation*}
                  Show that $d'$ is a metric that induces the usual topology on
                  $\mathbb{R}^n$. Sketch the basis elements under $d'$ when
                  $n=2$.
                  \begin{proof}
                      Let $\mathbf{x}, \mathbf{y}\in \mathbb{R}^n$. Then since
                      for each $1\leq i\leq n$, we have that $\abs{x_i-y_i}\geq
                      0$, then $\abs{\mathbf{x}-\mathbf{y}}\geq 0$. Similarly,
                      if $\abs{\mathbf{x}-\mathbf{y}}=0$, then
                      $\abs{x_i-y_i}=0$ for all $i$ and thus, $x_i=y_i$ for all
                      $i$, which implies that
                      $\mathbf{x}=\mathbf{y}$.\par\hspace{4mm} Since for each
                      $1\leq i\leq n$, $d(x_i, y_i)=d(y_i, x_i)$, i.e.,
                      $\abs{x_i-y_i}=\abs{y_i-x_i}$, then if $d'(\mathbf{x},
                      \mathbf{y})\neq d'(\mathbf{y}, \mathbf{x})$, then that
                      would imply that $\abs{x_i-y_i}\neq \abs{y_i-x_i}$, which
                      is a contradiction, since $d$ is a metric on
                      $\mathbb{R}$.\par\hspace{4mm} Finally, letting
                      $\mathbf{x}, \mathbf{y}, \mathbf{z}\in \mathbb{R}$. Then 
                      \begin{equation*}
                          \begin{split}
                              d(\mathbf{x}, \mathbf{z}) &=
                              \abs{x_1-z_1}+\cdots+\abs{x_n-z_n} 
                          \end{split}
                      \end{equation*}
                      and 
                      \begin{equation*}
                          d(\mathbf{x}, \mathbf{y})+d(\mathbf{y}, \mathbf{z})
                          = (\abs{x_1-y_1}+\abs{y_1-z_1})+\cdots+(\abs{x_n-y_n}+\abs{y_n-z_n})
                      \end{equation*}
                      Finally, we note that for each $i$,
                      $\abs{x_i-z_i}\leq\abs{x_i-y_i}+\abs{y_i-z_i}$ and thus
                      by induction it can be shown that
                      $\abs{x_1-z_1}+\cdots+\abs{x_n-z_n}\leq(\abs{x_1-y_1}+\abs{y_1-z_1})
                      +\cdots+(\abs{x_n-y_n}+\abs{y_n-z_n})$.\par\hspace{4mm}
                      To show that $d'$ induces the usual topology on
                      $\mathbb{R}^n$, we will show that the topology induced by
                      $d'$ is the same as the topology induced by $\rho$, which
                      by Theorem 20.3, is the same as the standard topology on
                      $\mathbb{R}^n$. To do this, we will let $\mathbf{x},
                      \mathbf{y}\in \mathbb{R}^n$. Then consider the two
                      metrics: $d'(\mathbf{x},
                      \mathbf{y})=\abs{x_1-y_1}+\cdots+\abs{x_n-y_n}$, and
                      $\rho(\mathbf{x}, \mathbf{y})=\max\{\abs{x_1-y_1},
                      \dots, \abs{x_n-y_n}\}=\abs{x_i-y_i}$, for some $1\leq
                      i\leq n$. This implies that
                      $\abs{x_1-y_1}+\cdots+\abs{x_n-y_n}\leq n\abs{x_i-y_i}$.
                      Hence, $d'(\mathbf{x}, \mathbf{y})\leq n\rho(\mathbf{x},
                      \mathbf{y})$. By Lemma 20.2, any basis element
                      $B_{d'}(\mathbf{x}, \varepsilon)$, of size $\varepsilon$,
                      contains a basis element $B_{\rho}(\mathbf{x},
                      \varepsilon/n)$. Similarly, for any basis element
                     $B_{\rho}(\mathbf{x}, \varepsilon)$ contains a basis
                     element $B_{d'}(\mathbf{x}, \varepsilon)$ since
                     $\rho(\mathbf{x}, \mathbf{y})\leq d'(\mathbf{x},
                     \mathbf{y})$, for all $\mathbf{x}, \mathbf{y}\in
                     \mathbb{R}^n$. Thus, the topologies induced by $d'$ and
                     $\rho$ are the same as the standard topology on
                     $\mathbb{R}^n$.\par\hspace{4mm} Finally, the basis
                     elements under $d'$ appear as follows:
                     \begin{figure}[htp!]
                        \centering
                        \incfig{basis}
                        \caption{Basis}
                        \label{fig:basis}
                     \end{figure}\hfill\par\newpage
                  \end{proof}
            \end{enumerate}
        \item[21.10] Using the closed set formulation of continuity (Theoreom
            18.1), show that the following are closed subsets of
               $\mathbb{R}^2$:
               \begin{align*}
                   &A = \{x\times y\mid xy=1\} \\
                   &S^1 = \{x\times y\mid x^2+y^2=1\} \\
                   &B^2 = \{x\times y\mid x^2+y^2\leq 1\}
               \end{align*}
               \begin{proof}
                   Consider the function $f:\mathbb{R}\to\mathbb{R}$ such that
                   $f(x)=x$. Then if $\varepsilon>0$, and $a\in \mathbb{R}$, if
                   we choose $\delta=\varepsilon$, it follows that
                   $\abs{x-a}<\delta$, then $\abs{f(x)-f(a)}<\varepsilon$.
                   Hence, $f$ is continuous. A similar argument can be used to
                   show that $g(y)=y$ is continuous. By Lemma 21.4, the
                   function $fg$ is a continuous function from
                   $\mathbb{R}\times \mathbb{R}$ into $\mathbb{R}$. Finally,
                   considering the singleton set $\{1\}$, which is closed, we
                   may then apply Theorem 18.1 which states that
                   $(fg)^{-1}(\{1\})=A$ is closed.\par\hspace{4mm} To show that
                   $S^1$ is closed, we first need to show that the function
                   $f(x)=x^2$ is continuous. Letting $\varepsilon>0$, and
                   $a\in\mathbb{R}$, then letting $K=\min\{1,
                   \frac{\varepsilon}{1+2\abs{a}}\}$. It follows that if
                   $\delta= K/2$, then $\abs{x-a}<1$ and so
                   $\abs{x}<1+\abs{a}$. Thus,
                   $\abs{x^2-a^2}=\abs{x+a}\abs{x-a}\leq\abs{x-a}
                   \abs{1+2\abs{a}}<\delta\abs{1+2\abs{a}}=\varepsilon$. Thus,
                   $f(x)=x^2$ is continuous. This implies that $g(y)=y^2$ is
                   continuous and by Lemma 21.4, $f+g$ is continuous. By
                   Theorem 18.1, $(f+g)^{-1}(\{1\})$ is closed.\par\hspace{4mm}
                   Lastly, we consider the set $[-1, 1]$ which is closed in
                   $\mathbb{R}$ and letting $f$ and $g$ be the same functions
                   as above, we get that $(f+g)$ is continuous, and that
                   $(f+g)^{-1}([-1, 1])=B^2$, which by Theorem 18.1, is closed.
               \end{proof}
    \end{enumerate}
\end{document}
