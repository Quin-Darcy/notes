\documentclass[12pt]{article}
\usepackage[margin=1in]{geometry}
\usepackage{graphicx}
\usepackage{amsmath}
\usepackage{amsthm}
\usepackage{amsfonts}
\usepackage{amssymb}
\usepackage{array}
\usepackage{enumerate}
\usepackage{slashed}
\usepackage{colonequals}
\usepackage{fancyhdr}

\pagestyle{fancy}
\fancyhf{}
\rhead{Darcy}
\lhead{MATH 220A}
\rfoot{\thepage}
\setlength{\headheight}{10pt}

\newtheorem{theorem}{Theorem}[section]
\newtheorem{corollary}{Corollary}[theorem]
\newtheorem{prop}{Proposition}[section]
\newtheorem{lemma}[theorem]{Lemma}
\theoremstyle{definition}
\newtheorem{definition}{Definition}[section]
%\theoremstyle{remark}
%\newtheorem*{remark}{Remark}

\newcommand{\abs}[1]{\lvert #1 \rvert}
\newcommand{\bigabs}[1]{\Bigl \lvert #1 \Bigr \rvert}
\newcommand{\bigbracket}[1]{\Bigl [ #1 \Bigr ]}
\newcommand{\bigparen}[1]{\Bigl ( #1 \Bigr )}
\newcommand{\ceil}[1]{\lceil #1 \rceil}
\newcommand{\bigceil}[1]{\Bigl \lceil #1 \Bigr \rceil}
\newcommand{\floor}[1]{\lfloor #1 \rfloor}
\newcommand{\bigfloor}[1]{\Bigl \lfloor #1 \Bigr \rfloor}
\newcommand{\norm}[1]{\| #1 \|}
\newcommand{\bignorm}[1]{\Bigl \| #1 \Bigr \| #1}
\newcommand{\inner}[1]{\langle #1 \rangle}
\newcommand{\set}[1]{{ #1 }}

\begin{document}
    \thispagestyle{empty}\hrule

    \begin{center}
        \vspace{.4cm} { \large MATH 220A}
    \end{center}
    {Name:\ Quin Darcy \hspace{\fill} Due Date: 2/19/21  \\
    { Instructor:}\ Dr.Martins \hspace{\fill} Assignment:
    Homework 2 \\ \hrule}

    \begin{enumerate}
        \item[8.]
            \begin{enumerate}
                \item Apply Lemma 13.2 to show that the countable collection
                    \begin{equation*}
                        \mathcal{B}=\{(a, b)\mid a<b, \text{ $a$ and $b$ rational }\} 
                    \end{equation*}
                is a basis that generates the standard topology on
                $\mathbb{R}$.
                    \begin{proof}
                        Let $\mathcal{T}_1$ be the topology generated by
                        $\mathcal{B}$ and let $\mathcal{T}_2$ be the standard
                        topology. Then we must show that for any
                        $U\in\mathcal{T}_2$ and $x\in U$, there exists
                        $B\in\mathcal{T}_2$ such that $x\in B\subset
                        U$. \par\hspace{4mm} Let $U\in\mathcal{T}_2$ and $x\in U$. Then by
                        Lemma 13.1, $U$ is the union of some collection of open intervals in
                        $\mathbb{R}$. With $x\in U$, then we have that for some
                        $a, b\in\mathbb{R}$, $x\in(a, b)$. If $a,
                        b\in\mathbb{Q}$, then we are done since in this case,
                        $(a, b)\in\mathcal{T}_2$ and $x\in(a, b)\subset U$. If
                        $a, b$ are not rational, then since $\mathbb{Q}$ is dense in
                        $\mathbb{R}$, we have that there exists
                        $a<a_2<x<b_2<b$, where $a_2, b_2\in\mathbb{Q}$. Thus
                        $(a_2, b_2)\in\mathcal{T}_2$ and $x\in(a_2, b_2)\subset
                        U$. Therefore, by Lemma 13.2, $\mathcal{T}_1$ is
                        a basis for $\mathcal{T}_2$.
                    \end{proof}
                \item Show that the collection 
                    \begin{equation*}
                        \mathcal{C}=\{[a, b)\mid a<b, \text{ $a$ and $b$
                        rational }\} 
                    \end{equation*}
                is a basis that generates a topology different from the lower
                limit topology on $\mathbb{R}$.
                    \begin{proof}
                        We want to show that $\mathcal{C}\neq\mathbb{R}_{l}$,
                        where $\mathbb{R}_l$ is the lower limit topology on
                        $\mathbb{R}$. To prove this we select the set
                        $[\sqrt{2}, 3)\in\mathbb{R}_l$ and let $x=\sqrt{2} $. From here we appeal to
                        the negation of Lemma 13.2. Namely, to show that there
                        does not exist a $B\in\mathcal{C}$ such that $x\in
                        B\subset[\sqrt{2}, 3 )$. This follows from
                        $x=\sqrt{2}$,
                        and so for all $[a, b)\subset[\sqrt{2}, 3)$ such that
                        $x\in[a, b)$ it implies $a=\sqrt{2}$ and thus $[a,
                        b)\not\subset\mathcal{C}$.
                    \end{proof}
            \end{enumerate}
        \item[3.] Consider the set $Y=[-1, 1]$ as a subspace of $\mathbb{R}$.
            Which of the following are open in $Y$? Which are open in
            $\mathbb{R}$?
                \begin{equation*}
                    \begin{split}
                        A &= \{x\mid \frac{1}{2}<\abs{x}<1\}, \\
                        B &= \{x\mid \frac{1}{2}<\abs{x}\leq 1\}, \\
                        C &= \{x\mid \frac{1}{2}\leq\abs{x}<1\}, \\
                        D &= \{x\mid\frac{1}{2}\leq\abs{x}\leq 1\}, \\
                        E &= \{x\mid 0<\abs{x}<1\text{ and
                        $1/x\not\in\mathbb{Z}_+$ }\}.  
                    \end{split}
                \end{equation*}
                \begin{proof}\par\hfill
                    \begin{enumerate}
                        \item Considering the standard topology on $\mathbb{R}$ and
                            letting a basis for this topology be denoted by
                            $\mathcal{B}$, then by Lemma 16.1, $\mathcal{B}_Y=\{B\cap
                            Y\mid B\in\mathcal{B}\} $ is a topology on $Y$. As such, we
                            have that $A=(-1, -\frac{1}{2})\cup(-\frac{1}{2}, 1)$ is open in $\mathbb{R}$ as it
                            is the union of two open sets in $X$, and $A\subset Y$.
                            Therefore, $A$ is open in $Y$ and open in
                            $\mathbb{R}$.
                        \item Considering the standard topology on
                            $\mathbb{R}$, we have that $B=[-1,
                            -\frac{1}{2})\cup(-\frac{1}{2},
                            1]$, which is not open in $\mathbb{R}$. However, we
                            can rewrite 
                                \begin{equation*}
                                    B=Y\cap\bigparen{(-\frac{3}{2},
                                    -\frac{1}{2})\cup(\frac{1}{2}, \frac{3}{2})}
                                \end{equation*}
                            which is the union of two basis elements of the
                            standard topology intersected with $Y$, and thus
                            $B$ is in the topology on $Y$. Hence, $B$ is open
                            in $Y$.
                        \item We can rewrite 
                                \begin{equation*}
                                    C = (-1, -\frac{1}{2}]\cup[\frac{1}{2}, 1)
                                \end{equation*}
                            which is not open in $\mathbb{R}$. Moreover, we
                            cannot do as we did above and express $C$ as the
                            intersection of open sets of $\mathbb{R}$ with $Y$
                            since for all open sets $U\in\mathbb{R}$ such that
                            $C\subset U$, we have $U\cap Y\neq C$. To see why,
                            suppose there is such an open set $U\in\mathbb{R}$.
                            Then since $\frac{1}{2}\in C\subset U$, it follows
                            that there exists an open set $U'\subset U$
                            such that $\frac{1}{2}\in U'$. However, this
                            implies that there exists an open set $U''\subset C$
                            such that $\frac{1}{2}\in U''$, which is not
                            possible.
                        \item By the same argument above, we can conclude that
                            $D$ is neither open in $\mathbb{R}$, nor is $D$
                            open in $Y$.
                        \item To show that $E$ is open in $Y$, we will show
                            that $E$ is open in $\mathbb{R}$ and appeal to
                            Lemma 16.1. We need to show that $E$ is an element
                            of the topology on $\mathbb{R}$. Thus, we need to
                            show that $E$ is either a basis element or the
                            union of basis elements.\par\hspace{4mm} Let $x\in
                            E$. Then if $0<x$, by the Archmedian principle,
                            there exists $n\in\mathbb{N}$ such that
                            $\frac{1}{n+1}<x<\frac{1}{n}$. Thus, for some
                            $n\in\mathbb{N}$, we have that
                            $x\in\bigparen{\frac{1}{n+1}, \frac{1}{n}}$. If
                            $x<0$, then immediately we see that
                            $\frac{1}{x}\not\in\mathbb{Z}_{+}$ and that
                            $x\in(-1, 0)$, which is open in $\mathbb{R}$.
                            Therefore, 
                                \begin{equation*}
                                    E\subseteq(-1,
                                    0)\cup\bigcup_{n\in\mathbb{N}}\bigparen{\frac{1}{n+1},
                                    \frac{1}{n}}
                                \end{equation*}
                            which is the union of basis elements of the
                            standard topology on $\mathbb{R}$. If $x$ is an
                            element of the set on the right, then either
                            $x\in(-1, 0)$ which implies $x\in E$, or for some
                            $n\in\mathbb{N}$,  $x\in(\frac{1}{n+1},
                            \frac{1}{n})$ which implies that
                            $1/x\not\in\mathbb{Z}_{+}$. In either case, we have
                            that $0<\abs{x}<1$. Therefore, $x\in E$ and 
                                \begin{equation*}
                                    E=(-1,
                                    0)\cup\bigcup_{n\in\mathbb{N}}\bigparen{\frac{1}{n+1},
                                    \frac{1}{n}}.
                                \end{equation*}
                            Thus $E$ is open in $\mathbb{R}$ and thus open in
                            $Y$.
                    \end{enumerate}
                \end{proof}
        \item Let $[a, b]\subset\mathbb{R}$ be a closed interval and consider
            the following set of functions:
                \begin{equation*}
                    C([a, b])=\{f:[a, b]\to\mathbb{R}\mid f(x)\text{ is
                    continuous on $[a, b]$ }\}. 
                \end{equation*}
            Using the properties of the Riemann integral, show that $d_{L^1}$
            is a metric on the space $C([a, b])$. 
                \begin{proof}
                    Let $f, g\in C([a, b])$. Then we want to show that
                    $\|f(x)-g(x)\|\geq 0$. If $f(x)\neq g(x)$, then define
                    $h(x)=f(x)-g(x)$. By properties of continuous functions,
                    $h(x)\in C([a, b])$. Furthermore, we have that
                        \begin{equation*}
                            -\abs{h(x)}\leq h(x)\leq \abs{h(x)}
                        \end{equation*}
                        and since $h(x)$ is continuous $[a, b]$, then $\abs{h(x)}$ is
                        continuous on $[a, b]$ and thus Riemann integrable on
                        $[a, b]$. Thus taking the integral of the above
                        inequality, we get that 
                        \begin{equation*}
                            0\leq\bigabs{\int_{a}^{b}h(x)dx}\leq\int_{a}^{b}\abs{h(x)}dx.
                        \end{equation*}
                    And if $f(x)=g(x)$, then $f(x)-g(x)=0$ which implies
                    that $\|f(x)-g(x)\|_{L^1}=0$. Therefore,
                    $\|f(x)-g(x)\|\geq 0$, with equality when
                    $f(x)=g(x)$.\par\hspace{4mm} Next we let $f(x), g(x)\in
                    C([a, b])$. We need to show that
                    $\|f(x)-g(x)\|=\|g(x)-f(x)\|$. This property follows
                    immediately from the commutativity of the real numbers.
                    Namely, that $f(x)-g(x)=g(x)-f(x)$ for all $x\in[a,
                    b]$.\par\hspace{4mm} Let $f(x), g(x), h(x)\in C([a, b])$.
                    We need to show that 
                        \begin{equation*}
                            \|f(x)-h(x)\|\leq\|f(x)-g(x)\|+\|g(x)-h(x)\|.
                        \end{equation*}
                    By properties of continuous functions, $f(x)-h(x)$,
                    $f(x)-g(x)$, and $g(x)-h(x)$ are all continuous on $[a,
                    b]$. Moreover, we have that 
                        \begin{equation*}
                            \abs{f(x)-h(x)}\leq\abs{f(x)-g(x)}+\abs{g(x)-h(x)}.
                        \end{equation*}
                    Lastly, we have that $\abs{f(x)-h(x)}$, $\abs{f(x)-g(x)}$,
                    and $\abs{g(x)-h(x)}$ are all continuous on $[a, b]$ and
                    thus Riemann integrable on $[a, b]$. Thus taking the
                    integral of the above inequality we get that
                        \begin{equation*}
                            \int_{a}^{b}\abs{f(x)-h(x)}dx\leq\int_{a}^{b}\abs{f(x)-g(x)}dx+\int_{a}^{b}\abs{g(x)-h(x)}dx
                        \end{equation*}
                    as desired.
                \end{proof}
    \end{enumerate}
\end{document}
