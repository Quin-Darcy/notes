\documentclass[12pt]{article}
\usepackage[margin=1in]{geometry}
\usepackage{graphicx}
\usepackage{amsmath}
\usepackage{amsthm}
\usepackage{amsfonts}
\usepackage{amssymb}
\usepackage{array}
\usepackage{enumerate}
\usepackage{slashed}
\usepackage{colonequals}
\usepackage{fancyhdr}

\pagestyle{fancy}
\fancyhf{}
\rhead{Darcy}
\lhead{CSC 35}
\rfoot{\thepage}
\setlength{\headheight}{10pt}

\newtheorem{theorem}{Theorem}[section]
\newtheorem{corollary}{Corollary}[theorem]
\newtheorem{prop}{Proposition}[section]
\newtheorem{lemma}[theorem]{Lemma}
\theoremstyle{definition}
\newtheorem{definition}{Definition}[section]
%\theoremstyle{remark}
%\newtheorem*{remark}{Remark}
\newenvironment{solution}
{\renewcommand\qedsymbol{$\blacksquare$}\begin{proof}[Solution]}
{\end{proof}}
\newenvironment{psmall}{\left(\begin{smallmatrix}}{\end{smallmatrix}\right)}

\newcommand{\abs}[1]{\lvert #1 \rvert}
\newcommand{\bigabs}[1]{\Bigl \lvert #1 \Bigr \rvert}
\newcommand{\bigbracket}[1]{\Bigl [ #1 \Bigr ]}
\newcommand{\bigparen}[1]{\Bigl ( #1 \Bigr )}
\newcommand{\ceil}[1]{\lceil #1 \rceil}
\newcommand{\bigceil}[1]{\Bigl \lceil #1 \Bigr \rceil}
\newcommand{\floor}[1]{\lfloor #1 \rfloor}
\newcommand{\bigfloor}[1]{\Bigl \lfloor #1 \Bigr \rfloor}
\newcommand{\norm}[1]{\| #1 \|}
\newcommand{\bignorm}[1]{\Bigl \| #1 \Bigr \| #1}
\newcommand{\inner}[1]{\langle #1 \rangle}
\newcommand{\set}[1]{{ #1 }}

\begin{document}
    \thispagestyle{empty}\hrule

    \begin{center}
        \vspace{.4cm} { \large CSC 35}
    \end{center}
    {Name:\ Quin Darcy \hspace{\fill} Due Date: 2/22/21  \\
    { Instructor:}\ Devin Cook \hspace{\fill} Assignment:
    Lab 1 \\ \hrule} 
    \vspace{4mm}

    Convert the following numbers into binary.
    \begin{enumerate}
        \item 42
            \begin{solution}
                We begin by breaking 42 down into a linear combination of
                powers of 2. Since this only requires one byte, then we start
                by determining the largest power of 2, $m_0$, such that
                $2^{m_0}\leq
                42$ and where $0\leq m_0\leq 7$. In this case, it $m_0=5$. 
                From there we do the same but with 42-$2^5=10$ instead of 42. 
                This gives $m_1=3$. Repeating this we get
                $m_2=1$. Hence,
                    \begin{equation*}
                        42 = 32 + 8 + 2=
                        (0)2^7+(0)2^6+(1)2^5+(0)2^4+(1)2^3+(0)2^2+(1)2^1+(0)2^0.
                    \end{equation*}
                Finally, collecting the coefficients on the powers of 2 we
                obtain
                    \begin{equation*}
                        42_2 = \text{\texttt{00101010}}.
                    \end{equation*}
            \end{solution}
        \item 451
            \begin{solution}
                We do the same as above and get that 
                    \begin{equation*}
                        \begin{split}
                            451 &= 256+195 \\
                            &= 256 +128+67 \\
                            &=256+128+64+3 \\
                            &=256+128+64+2+1 \\
                            &=\sum_{k=0}^{15}a_i2^{15-i},
                        \end{split}
                    \end{equation*}
                where $a_i=0$ for $0\leq i\leq 6$, $a_i=1$ for $7\leq
                i\leq 9$, $a_i=0$ for $10\leq i\leq 13$, $a_{14}=1$, and
                $a_{15}=0$. Thus 
                    \begin{equation*}
                        451_2=\text{\texttt{00000001 11000010}}.
                    \end{equation*}
            \end{solution}
    \end{enumerate}\newpage
    Convert the following strings into a series of bytes. Leave the result in
    hexadecimal.
    \begin{enumerate}
        \item Sacramento State
            \begin{solution}
                By referring to the ASCII chart, the conversion of the above
                string is as follows:
                    \begin{equation*}
                        \text{Sacramento State}=\text{\texttt{53 61 63 72 61 6D
                        65 6E 74 6F 20 53 74 61 74 65}}.
                    \end{equation*}
            \end{solution}
        \item My name is Quin.
            \begin{solution}
                By doing the same as we did above we get 
                    \begin{equation*}
                        \text{My name is Quin}=\text{\texttt{4D 79 20 6E 61 6D
                        65 20 69 73 20 51 75 69 6E}}.
                    \end{equation*}
            \end{solution}
    \end{enumerate}
\end{document}
