\documentclass[12pt]{article}
\usepackage[margin=1in]{geometry}
\usepackage{graphicx}
\usepackage{amsmath}
\usepackage{amsthm}
\usepackage{amsfonts}
\usepackage{amssymb}
\usepackage{array}
\usepackage{enumerate}
\usepackage{slashed}
\usepackage{colonequals}
\usepackage{fancyhdr}
\usepackage{enumitem}

\pagestyle{fancy}
\fancyhf{}
\rhead{Darcy}
\lhead{MATH 230B}
\rfoot{\thepage}
\setlength{\headheight}{10pt}

\newtheorem{theorem}{Theorem}[section]
\newtheorem{corollary}{Corollary}[theorem]
\newtheorem{prop}{Proposition}[section]
\newtheorem{lemma}[theorem]{Lemma}
\theoremstyle{definition}
\newtheorem{definition}{Definition}[section]
%\theoremstyle{remark}
%\newtheorem*{remark}{Remark}

\newcommand{\abs}[1]{\lvert #1 \rvert}
\newcommand{\bigabs}[1]{\Bigl \lvert #1 \Bigr \rvert}
\newcommand{\bigbracket}[1]{\Bigl [ #1 \Bigr ]}
\newcommand{\bigparen}[1]{\Bigl ( #1 \Bigr )}
\newcommand{\ceil}[1]{\lceil #1 \rceil}
\newcommand{\bigceil}[1]{\Bigl \lceil #1 \Bigr \rceil}
\newcommand{\floor}[1]{\lfloor #1 \rfloor}
\newcommand{\bigfloor}[1]{\Bigl \lfloor #1 \Bigr \rfloor}
\newcommand{\norm}[1]{\| #1 \|}
\newcommand{\bignorm}[1]{\Bigl \| #1 \Bigr \| #1}
\newcommand{\inner}[1]{\langle #1 \rangle}
\newcommand{\set}[1]{{ #1 }}
\newcommand{\R}{\mathbb{R}}

\begin{document}
    \thispagestyle{empty}\hrule

    \begin{center}
        \vspace{.4cm} { \large MATH 230B}
    \end{center}
    {Name:\ Quin Darcy \hspace{\fill} Due Date: 02/15/21   \\
    { Instructor:}\ Dr. Domokos \hspace{\fill} Assignment:
    Homework 1 \\ \hrule}

    \begin{enumerate}
        \item[4.1] Let $f:[a, b]\to\mathbb{R}$. Prove that if $f$ has a local
        maximum (or minimum) at $x_0=a$ or $x_0=b$ and $f$ is differentiable at
        $x_0$, then there exists $\delta>0$ such that
            \begin{equation*}
                f'(x_0)(x-x_0)\leq 0\quad(\text{or }f'(x_0)(x-x_0)\geq 0)
            \end{equation*}
        for all $x\in(x_0-\delta, x_0+\delta)\cap[a, b]$
            \begin{proof}
                Since $f$ has a local maximum at $x_0=a$, then there exists
                $\delta>0$ such that for all $x\in(x_0-\delta,
                x_0+\delta)\cap[a, b]$ we have that 
                    \begin{equation*}
                        f(x)\leq f(x_0)\Rightarrow f(x)-f(x_0)\leq 0.
                    \end{equation*}
                Moreover, we have that for all $\in(x_0-\delta,
                x_0+\delta)\cap[a, b]$, that $x-x_0\geq 0$. Hence, 
                    \begin{equation*}
                        \lim_{x\searrow x_0}\frac{f(x)-f(x_0)}{x-x_0}=f'(x_0)\leq 0
                    \end{equation*}
                since $f$ is differentiable at $x_0$. Thus, for any
                $x\in(x_0-\delta, x_0+\delta)\cap[a, b]$ it follows that
                $x-x_0\geq 0$ and so $f'(x_0)(x-x_0)\leq 0$
            \end{proof}
        \item[4.2] Let $f:\mathbb{R}\to\mathbb{R}$ and $x_0\in\mathbb{R}$. Show
            that if $f$ is continuous at $x_0$ and $\abs{f}$ is differentiable
            at $x_0$, then $f$ is differentiable at $x_0$.
            \begin{proof}
                We need to show that 
                    \begin{equation*}
                        \lim_{x\to x_ 0}\frac{f(x)-f(x_0)}{x-x_0}
                    \end{equation*}
                exists. Since $f$ is continuous at $x_0$, then for any
                $\varepsilon_1>0$, there exists $\delta_1>0$ such that
                $\abs{f(x)-f(x_0)}<\varepsilon_1$ when $\abs{x-x_0}<\delta_1$.
                We also have that $\abs{f}$ is differentiable at $x_0$. Hence,
                for any $\varepsilon_2>0$, there exists $\delta_2>0$ such that 
                    \begin{equation}
                        \bigabs{\frac{\abs{f(x)}-\abs{f(x_0)}}{x-x_0}-L_1}<\varepsilon_2
                    \end{equation}
                for some $L_2\in\mathbb{R}$ and whenever
                $\abs{x-x_0}<\delta_2$. Thus, we need to show that for all
                $\varepsilon>0$, there exists $\delta>0$ such that if
                $\abs{x-x_0}<\delta$, then
                    \begin{equation}
                        \bigabs{\frac{f(x)-f(x_0)}{x-x_0}-L_2}<\varepsilon,
                    \end{equation}
                for some $L_2\in\mathbb{R}$.\par\hspace{4mm} Let
                $\delta=\min\{\delta_1, \delta_2\}$, then for
                $\abs{x-x_0}<\delta$, it follows that
                $\abs{f(x)-f(x_0)}<\varepsilon_1$ holds and that (1) holds.
                Thus multiplying (1) by $\abs{x-x_0}$ we obtain
                    \begin{equation*}
                        \abs{\abs{f(x)}-\abs{f(x_0)}-L_1(x-x_0)}
                        <\varepsilon_2\abs{x-x_0}<\varepsilon_1\delta.
                    \end{equation*}
                By the Triangle Inequality we get that 
                    \begin{equation*}
                        \begin{split}
                            \abs{\abs{f(x)}-\abs{f(x_0)}-L_1(x-x_0)}
                            &\leq\abs{\abs{f(x)}-\abs{f(x_0)}}+\abs{L_1}\abs{x-x_0}
                            \\
                            &\leq\abs{f(x)-f(x_0)}+\abs{L_1}\delta \\
                            &<\varepsilon_1+\abs{L_1}\delta.
                        \end{split}
                    \end{equation*}
                Now if we select $\varepsilon_1, \varepsilon_2$ such that
                $\varepsilon_1+\abs{L_1}\delta\leq\varepsilon_2\delta$, then it
                follows that (2) holds iff
                $\abs{f(x)-f(x_0)-L_2(x-x_0)}<\varepsilon\delta$. And by the
                Triangle Inequality we get that
                $\abs{f(x)-f(x_0)-L_2(x-x_0)}<\varepsilon_1+\abs{L_2}\delta$.
                Finally, if $\varepsilon=\varepsilon_2+\abs{L_2}-\abs{L_1}$,
                then 
                    \begin{equation*}
                        \bigabs{\frac{f(x)-f(x_0)}{x-x_0}-L_2}<\varepsilon.
                    \end{equation*}
                Therefore the limit exists and $f$ is differentiable at $x_0$.
            \end{proof}
        \item[4.3] Suppose that $f$ is differentiable on $(a, b)$ and $f'$ is
            monotone increasing on $(a, b)$. Prove that $f'$ is continuous on
            $(a, b)$.
            \begin{proof}
                For contradiction, assume that $f'$ is discontinuous on $(a,
                b)$. Then by Corollary 3.19 and Theorem 3.20, $f'$ can only
                have at most countably many jump discontinuities. Let
                $x_0\in(a, b)$ such that $x_0$ is a discontinuity of the first
                kind. Then by Theorem 3.18 
                    \begin{equation*}
                        \lim_{x\nearrow x_0}f'(x)=L_1\leq f'(x_0)\leq L_2=
                        \lim_{x\searrow x_0}f'(x).
                    \end{equation*}
                Now let $\delta>0$ such that $[x_0-\delta,
                x_0+\delta]\subset(a, b)$. Then since $f'$ is monotone
                increasing, it follows that $f'$ satisfies the IVP on the above
                interval. Finally, letting
                $\varepsilon=\min\{\abs{f'(x_0)-L_1}, \abs{f(x_0)-L_2}\}$, it
                follows that for any $y$ such that
                $\abs{f'(x_0)-y}<\varepsilon$, we get that there does not exist
                $x\in (x_0-\delta, x_0+\delta)$ such that $f'(x)=y$. This
                contradicts the IVP. 
            \end{proof}
        \item[4.4] Let
                \begin{equation*}
                    f(x)=\begin{cases}x^p\sin\frac{1}{x}&\text{if}\quad x\neq
                    0 \\ 0&\text{if}\quad x=0.\end{cases}
                \end{equation*}
                \begin{enumerate}[label=(\alph*)]
                    \item For what values of $p$ is $f$ continuous on $\mathbb{R}$?
                        \begin{proof}
                            We saw from Example 4.6 that for $p=0$, $f$ is not
                            continuous at $x=0$. For $p>0$, it follows from the
                            fact that $\lim_{x\to 0}f(x)=0$, that $f$ is
                            continuous at 0 and thus continuous on
                            $\mathbb{R}$.
                        \end{proof}\newpage
                    \item For what values of $p$ is $f$ differentiable on $\mathbb{R}$?
                        \begin{proof}
                            Similar to the reasoning above, for $p=0$, $f$ is
                            not continuous at $x=0$ and thus not differentiable
                            at $x=0$. For $p=1$, we refer to Example 4.7 to
                            conclude that $f$ is not differentiable at $x=0$.
                            For $p>1$, we have that 
                                \begin{equation*}
                                    \frac{f(x)-f(0)}{x}=\frac{x^p\sin\frac{1}{x}-0}{x}=x^{p-1}\sin\frac{1}{x}
                                \end{equation*}
                                and since $p>1$, then $p-1>0$ and so
                                $\lim_{x\to 0}x^{p-1}\sin\frac{1}{x}=0$. Hence
                                $f$ is differentiable on $\mathbb{R}$ for
                                $p>1$.
                        \end{proof}
                    \item For what values of $p$ is $f'$ continuous on $\mathbb{R}$?
                        \begin{proof}
                            For $x\neq 0$, $f'$ is continuous as it is the
                            product, sum, and composition of continuous
                            functions. We have that 
                                \begin{equation*}
                                    f'(x)=px^{p-1}\sin\frac{1}{x}-x^{p-2}\cos\frac{1}{x}
                                \end{equation*}
                            and so for $0\leq p\leq 1$, $\lim_{x\to 0}f'(x)$
                            does not exist and so $f'$ is not continuous. For
                            $p\geq 2$, we have that $\lim_{x\to 0}f'(x)=0$ and
                            thus $f'$ is continuous for $p\geq 2$.
                        \end{proof}
                    \item For what values of $p$ is $f$ differentiable on $\mathbb{R}$?
                        \begin{proof}
                            Note that the exponents on the $x$ term in $f''$
                            are $p-2$ and $p-3$, respectively. Thus from the
                            difference quotient we obtain a power of $p-4$ on
                            some of the $x$ terms. Hence, for $p>4$, it follows
                            that $f$ is twice differentiable on $\mathbb{R}$.
                        \end{proof}
                \end{enumerate}
            \item[4.5] Let $f:(a, b)\to\mathbb{R}$ be differentiable on
                $(a, b)$ and assume that there exists $0\leq M<+\infty$ such
                that $\abs{f'(x)}\leq M$ for all $x\in(a, b)$.
                \begin{enumerate}[label=(\arabic*)]
                    \item Show that $f$ is uniformly continuous on $(a, b)$.
                        \begin{proof}
                            We begin by trying to show that $f$ is Lipschitz
                            continuous and then using Theorem 3.43 to prove that
                            $f$ is uniformly continuous. With this in mind, we
                            need to show that for all $x_1, x_2\in(a, b)$, we
                            have that
                                \begin{equation*}
                                    \abs{f(x_2)-f(x_1)}\leq M\abs{x_2-x_1}.
                                \end{equation*}
                            First, let $\varepsilon>0$ and define $a'=a+\varepsilon$ and
                            $b'=b-\varepsilon$ such that $a'<b$ and $a<b'$. Then
                            since $(a', b')\subset (a, b)$, it follows that $f$
                            is continuous on $[a', b']$ and differentiable on
                            $(a', b')$. Thus by Lagrange's Theorem, there
                            exists $x_0\in (a', b')$ such that 
                                \begin{equation*}
                                    f(b')-f(a')=f'(x_0)(b'-a')\leq M(b'-a').
                                \end{equation*}
                            Since $\varepsilon$ was arbitrary, then the above
                            holds for all $a', b'\in (a, b)$. Hence, $f$ is
                            Lipschitz continuous on $(a, b)$. By Theorem 3.43,
                            $f$ is uniformly continuous on $(a, b)$.
                        \end{proof}
                    \item Give an example of a function which has unbounded
                        derivative, but still is uniformly continuous.
                            \begin{proof}
                                Let $f(x)=\sqrt{x}$. Then $f$ is differentiable
                                on $(0, +\infty)$. Moreover, for any
                                $\varepsilon>0$, if $\delta=\varepsilon^2$,
                                then it follows that if $x, y\in (0, +\infty)$
                                such that $\abs{x-y}<\delta$, then
                                $\abs{\sqrt{x}-\sqrt{y}}<\varepsilon$. Hence,
                                $f$ is uniformly continuous.\par\hspace{4mm} However, we
                                have that $f'(x)=\frac{1}{2\sqrt{x}}$ and
                                $\lim_{x\to+\infty}\frac{1}{2\sqrt{x}}=+\infty$.
                                Thus, $f'(x)$ is not bounded.
                            \end{proof}
                \end{enumerate}
            \item[4.6] Let $f:[a, b]\to\R$ be continuous on $[a, b]$ and
                differentiable on $(a, b)$. Show that if $f'(x)\neq 0$ for all
                $x\in (a, b)$, then $f$ is strictly monotone increasing or
                decreasing on $[a, b]$.\par\hspace{4mm} Is it true that if $f$
                is strictly monotone increasing $f'(x)\neq 0$ for all $x\in (a,
                b)$?
                    \begin{proof}
                        Assume that $f'(x)\neq 0$ for all $x\in (a, b)$. Then
                        by the converse of Theorem 4.9, $f$ does not obtain
                        a local maximum or local minimum on $(a, b)$. Now
                        suppose that $f$ is not strictly monotonically
                        increasing and not monotonically decreasing. Then there
                        exists $x_1, x_2, x_3, x_4\in (a, b)$ such
                        that $x_1<x_2$ and $f(x_1)\geq f(x_2)$, and $x_3<x_4$
                        and $f(x_3)\leq f(x_4)$.\par\hspace{4mm}
                        Since $f$ is continuous on $[x_1, x_2]$, as well as on
                        $[x_3, x_4]$ and
                        differentiable on $(x_1, x_2)$ and $(x_3, x_4)$, then by Lagrange's
                        Theorem, there exists $y_1\in (x_1, x_2)$ and $y_2\in
                        (x_3, x_4)$ such that 
                            \begin{equation*}
                                f'(y_1)=\frac{f(x_2)-f(x_1)}{x_2-x_1}\leq
                                0\quad\text{and}\quad
                                f'(y_2)=\frac{f(x_4)-f(x_3)}{x_4-x_3}\geq 0.
                            \end{equation*}
                        By assumtion, $f'(x)\neq 0$, for all $x\in (a, b)$, and so it has to be
                        the case that $f'(y_1)<0$ and
                        $f'(y_2)>0$.\par\hspace{4mm} WLOG, assume
                        that $y_1<y_2$. Define the function $g$ to be the
                        restriction of $f$ to $[y_1, y_2]$. Then clearly $g$ is
                        differentiable on $[y_1, y_2]$, as $f$ is. Thus by
                        Theorem 4.11, $g'$ has the Intermidiate Value Property.
                        And since $g'(y_1)<0<g'(y_2)$, then there exists $y\in
                        (y_1, y_2)$ such that $g'(y)=f'(y)=0$. This is
                        a contradiction. Therefore, $f$ must either be
                        strictly monotonically increasing or decreasing.  
                    \end{proof}
            \item[4.7] Let $f:(0, +\infty)\to\R$ be differentiable on $(0,
                +\infty)$. Prove that if $\lim_{x\to+\infty}f(x)= M\in\R$, then
                for all $\varepsilon>0$ there exists $x(\varepsilon)>0$ such
                that $\abs{f'(x(\varepsilon))}<\varepsilon$.
                    \begin{proof}
                        Let $\varepsilon_0=\varepsilon/2>0$. Then by
                        assumption, there exists $N>0$ such that for all $x>N$,
                        $\abs{f(x)-M}<\varepsilon_0$. Now choose $x_1, x_2>M$
                        such that $x_2-x_1> 1$. Then
                        $\abs{f(x_1)-M}<\varepsilon_0$ and
                        $\abs{f(x_2)-M}<\varepsilon_0$ and so by the Triangle
                        Inequality, it follows that
                        $\abs{f(x_2)-f(x_1)}<2\varepsilon_0=\varepsilon$.\par\hspace{4mm}
                        Note that $f$ is continuous on $[x_1, x_2]$ and
                        differentiable on $(x_1, x_2)$. Thus by Lagrange's
                        Theorem, there exists $x_0\in(x_1, x_2)$ such that
                            \begin{equation*}
                                f'(x_0)=\frac{f(x_2)-f(x_1)}{x_2-x_1}<\frac{\varepsilon}{x_2-x_1}<\varepsilon.
                            \end{equation*}
                        Therefore, for any $\varepsilon>0$, there exists $x_0$
                        such that $\abs{f'(x_0)}<\varepsilon$.
                    \end{proof}
            \item[4.8] Let $f:[0, 1]\to[0, 1]$ be continuous on $[0, 1]$
                and differetiable on $(0, 1)$. Show that if $f'(x)\neq 1$
                for all $x\in(0, 1)$, then there exists a unique $x_0\in[0,
                1]$ such that $f(x_0)=x_0$.
                    \begin{proof}
                        For contradiction, assume that $f(x)\neq x$ for
                        all $x\in[0, 1]$. Then it follows that $f$ cannot
                        be a constant function since this would imply that
                        $f(x)=c\in[0, 1]$ and so for $x=c$, we get $f(c)=c$.
                        \par\hspace{4mm} Define
                            \begin{equation*}
                                S=\{\abs{f(x)-x}:x\in[0, 1]\}.
                            \end{equation*}
                        Then $S$ is closed and bounded above by 1 and below
                        by 0. Thus by the greastest lower bound property,
                        $\alpha=\inf(S)$ exists, and $\alpha\in S$. Hence,
                        there exists $x_0\in[0, 1]$ such that
                        $f(x_0)-x_0=\alpha$. If
                        $\alpha=0$, then $f(x_0)=x_0$ which is
                        a contradiction. Thus $\alpha>0$. \par\hspace{4mm} Now
                        define $g(x)=f(x)-x$. Note that $g$ is continuous on
                        $[0, 1]$ and differentiable on $(0, 1)$. By definition,
                        $\alpha$ is a local minimum of $g$. Note that if
                        $x_0\in(0, 1)$, then by Theorem 4.9, $g'(x_0)=0$,
                        which implies that $f'(x_0)=1$. Thus either
                        $g(0)=\alpha$ or $g(1)=\alpha$ and $g'(x)\neq 0$
                        for all $x\in(0, 1)$. By Exercise 4.6, this implies
                        that $g$ is either strictly monotone increasing or
                        strictly monotone
                        decreasing.\par\hspace{4mm} If $g(0)=\alpha$, then $g$ has to be
                        strictly monotone increasing otherwise, for some
                        $0<x$, $g(x)<\alpha$ which implies
                        $\alpha\neq\inf(S)$. It follows that $g(1)>\alpha$, which
                        implies that $f(1)>1+\alpha$ which is
                        impossible.\par\hspace{4mm} If $g(1)=\alpha$, then
                        we have that $f(1)=1+\alpha$ which is impossible.
                        Therefore, there exists $x\in[0, 1]$ such that
                        $f(x)=x$. \par\hspace{4mm} To prove uniqueness,
                        suppose there is $x_1, x_2\in[0, 1]$ such that
                        $f(x_1)=x_1$ and $f(x_2)=x_2$. Then
                        $g(x_1)=g(x_2)$. By Theorem 4.13, there exists
                        $x_0\in(x_1, x_2)$ such that $g'(x_0)=0$ which
                        implies that $f'(x_0)=1$ contradicting our
                        assumption.
                    \end{proof}
            \item[4.11] Suppose that $f$ is differentiable on $[a, b]$ and $f'$
                is continuous on $[a, b]$. Prove that $f$ is absolutley
                continuous on $[a, b]$.
                    \begin{proof}
                        Seeing as $f'$ is continuous on $[a, b]$, which is
                        closed and bounded, then by Theorem 3.26, $f'([a, b])$
                        is closed and bounded. Now select $L\in\mathbb{R}$ such
                        that $\abs{f'(x)}\leq L$ for all $x\in[a, b]$.
                        Moreover, since $f$ is continuous on $[a, b]$ and
                        differentiable on $[a, b]$, then for any $x, y\in [a,
                        b]$, there exists $x_0\in[a, b]$ such that 
                            \begin{equation*}
                                \begin{split}
                                    f'(x_0)&=\frac{\abs{f(y)-f(x)}}{\abs{y-x}}\leq
                                    L \\
                                    &\Rightarrow \abs{f(y)-f(x)}\leq L\abs{y-x}.
                                \end{split}
                            \end{equation*}
                        Hence, $f$ is Lipschitz continuous. By Theorem 3.43,
                        $f$ is absolutely continuous.
                    \end{proof}\newpage
            \item[4.14] Let $f, g:[a, +\infty)\to\mathbb{R}$ be continuous on
                $[a, \infty)$ and differentiable on $(a, \infty)$. Assume that
                $f(a)=g(a)$ and that $f'(x)\leq g'(x)$ for all $a<x$. Prove
                that
                    \begin{enumerate}[label=(\arabic*)]
                        \item $f(x)\leq g(x)$ for all $a\leq x$.
                            \begin{proof}
                                Assume for contradiction that there exists
                                $a\leq x$ such that $g(x)<f(x)$. Then since
                                both $f$ and $g$ are conintuous on $[a, x]$ and
                                differentiable on $(a, x)$, Theorem 4.15 yields
                                some $x_0\in(a, x)$ such that 
                                    \begin{equation*}
                                        g'(x_0)=f'(x_0)\frac{g(x)-g(x)}{f(x)-f(a)}
                                    \end{equation*}
                                Seeing as $f(a)=g(a)$ and $g(x)<f(x)$, then it
                                follows that 
                                    \begin{equation*}
                                        g'(x_0)=f'(x_0)\frac{g(x)-g(x)}{f(x)-f(a)}<f'(x_0).
                                    \end{equation*}
                                Thus, for some $a<x_0$, we have
                                $g'(x_0)<f'(x_0)$, which contradicts our
                                assumption. 
                            \end{proof}
                        \item $1+\ln x\leq x$ for all $1\leq x$.
                            \begin{proof}
                                Letting $f(x)=1+\ln x$ and $g(x)=x$, then we
                                have that $f'(x)=1/x$ and $g'(x)=1$. Thus for
                                any $1\leq x$, it follows that $f'(x)\leq
                                g'(x)$. Moreover, since $\ln 1=0$, then we also
                                have that $f(1)=g(x)$. Lastly, both $f$ and $g$
                                are continuous on $[1, +\infty)$ and
                                differentiable on $(1, +\infty)$. Therefore, by
                                part (1), we have that $f(x)=1+\ln x\leq
                                x=g(x)$ for all $1\leq x$.
                            \end{proof}
                        \item $0\leq \sin x\leq x$ for all $0\leq x\leq 1$.
                            \begin{proof}
                                We will show this in two parts. To show the
                                left inequality, we note that for all $0\leq
                                x\leq\pi$, we have $0\leq\sin $ and since
                                $1<\pi$, then this suffices to show the left
                                inequality. As for the right side, we note that
                                the second derivative of $\sin x$ is $-\sin x$ and that
                                $-\sin x\leq 0$ for all $0\leq x\leq \pi$. By
                                Theorem 4.16, this implies that $\cos x$ is
                                monotonically decreasing on $[0, 1]\subset[0,
                                \pi]$. Also note that $\abs{\cos x}\leq 1$ for
                                all $x\in\mathbb{R}$. Hence, $\cos x\leq 1$ for
                                all $0\leq x\leq 1$. Therefore, by part (a), we
                                have that $0\leq\sin x\leq x$ for all $0\leq
                                x\leq 1$.
                            \end{proof}
                        \item $1-\frac{x^2}{2}\leq\cos x\leq 1$ for all $0\leq
                            x\leq 1$.
                            \begin{proof}
                                From part (a) we can conclude that $\cos x\leq
                                1$ for all $0\leq x\leq 1$. Now we note that if
                                $f(x)=1-\frac{x^2}{2}$ and $g(x)=\cos x$, then
                                $f$ and $g$ are continuous on $[0,1]$ and
                                differentiable on $(0, 1)$. We also have that
                                $f(0)=1=\cos 0$. Finally, we have that 
                                $f'(x)=-x$ and $g'(x)=-\sin x$ and by part (3),
                                we know $\sin x\leq x$ for all $0\leq x\leq 1$.
                                Thus $-x\leq -\sin x$ for all $0\leq x\leq 1$.
                                Hence, by (1), the desired result follows.
                            \end{proof}
                    \end{enumerate}
    \end{enumerate}
\end{document}
