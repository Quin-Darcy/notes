\documentclass[12pt]{article}
\usepackage[margin=1in]{geometry}
\usepackage{graphicx}
\usepackage{amsmath}
\usepackage{amsthm}
\usepackage{amsfonts}
\usepackage{amssymb}
\usepackage{array}
\usepackage{enumerate}
\usepackage{slashed}
\usepackage{colonequals}
\usepackage{fancyhdr}
\usepackage{import}
\usepackage{xifthen}
\usepackage{pdfpages}
\usepackage{transparent}
\usepackage{enumitem}

\newcommand{\incfig}[1]{%
    \def\svgwidth{\columnwidth}
    \import{/home/arbegla/figures/}{#1.pdf_tex}
}

\pagestyle{fancy}
\fancyhf{}
\rhead{}
\lhead{}
\rfoot{\thepage}
\setlength{\headheight}{10pt}

\newtheorem{theorem}{Theorem}[section]
\newtheorem{corollary}{Corollary}[theorem]
\newtheorem{prop}{Proposition}[section]
\newtheorem{lemma}[theorem]{Lemma}
\theoremstyle{definition}
\newtheorem{definition}{Definition}[section]
\theoremstyle{definition}
\newtheorem{exmp}{Example}[section]

\newcommand{\abs}[1]{\lvert #1 \rvert}
\newcommand{\bigabs}[1]{\Bigl \lvert #1 \Bigr \rvert}
\newcommand{\bigbracket}[1]{\Bigl [ #1 \Bigr ]}
\newcommand{\bigparen}[1]{\Bigl ( #1 \Bigr )}
\newcommand{\ceil}[1]{\lceil #1 \rceil}
\newcommand{\bigceil}[1]{\Bigl \lceil #1 \Bigr \rceil}
\newcommand{\floor}[1]{\lfloor #1 \rfloor}
\newcommand{\bigfloor}[1]{\Bigl \lfloor #1 \Bigr \rfloor}
\newcommand{\norm}[1]{\| #1 \|}
\newcommand{\bignorm}[1]{\Bigl \| #1 \Bigr \| #1}
\newcommand{\inner}[1]{\langle #1 \rangle}
\newcommand{\set}[1]{{ #1 }}

    %\begin{figure}[htp!]
    %    \centering
    %      \incfig{S5}
    %      \caption{Group Action}
    %      \label{fig:action}
    %  \end{figure}\

\begin{document}
\title{Analysis Notes}
\author{Quin Darcy}
\date{Jan, 26 2021}
%\affil{\small{California State University, Sacramento}}
\maketitle
\section{Proofs}
    \begin{definition}
        Let $X, Y$ be metric spaces and $f:X\to Y$. 
            \begin{enumerate}[label=(\roman*)]
                \item We say that $f$ is \textbf{uniformly continuous} on $X$
                    if for all $\varepsilon>0$ there exists
                    $\delta=\delta(\varepsilon)$ such that $d(f(x_1),
                    f(x_1))<\varepsilon$ for all $x_1, x_2\in X$ with $d(x_1,
                    x_2)<\delta$.
                \item We say that $f$ is \textbf{Lipschitz continuous} on $X$
                    if there exists $L\geq 0$ such that 
                        \begin{equation*}
                            d(f(x_1), f(x_2))<Ld(x_1, x_2), \quad\forall x_1,
                            x_2\in X.
                        \end{equation*}
                    The number $L$ is called the \textit{Lipschitz constant} of
                    $f$.
                \item If $f$ is Lipschitz continuous and $0\leq L<1$, then we
                    say that $f$ is a \textbf{contraction}.
            \end{enumerate}
    \end{definition}
    \begin{theorem}
        Let $X, Y$ be metric spaces and $f:X\to Y$. Regarding the continuity
        of $f$ on $X$, the following implications hold:
            \begin{equation*}
                \text{Lipschitz continuous}\Rightarrow\text{uniformly
                continuous}\Rightarrow\text{continuous}.
            \end{equation*}
    \end{theorem}
        \begin{proof}
            If $f$ is Lipschitz continuous, then for any $\varepsilon>0$ we can
            choose $\delta=\delta(\varepsilon)=\frac{\varepsilon}{L+1}$. By
            doing this we get that for all $x, y\in X$
                \begin{equation*}
                    \abs{x-y}<\delta\Rightarrow\abs{x-y}<\frac{\varepsilon}{L+1}\Rightarrow\abs{x-y}(L+1)<\varepsilon
                \end{equation*}
            and so 
                \begin{equation*}
                    \abs{f(x)-f(y)}<L\abs{x-y}<(L+1)\abs{x-y}<\varepsilon.
                \end{equation*}
            Hence, $f$ is uniformly continuous. 
        \end{proof}
    \begin{theorem}
        Let $f:[a, b]\to\mathbb{R}$ and $x_0\in(a, b)$. If $f$ has a local
        maximum (or minimum) at $x_0$ and $f$ is differentiable at $x_0$, then
        $f'(x_0)=0$.
    \end{theorem}
        \begin{proof}
            
        \end{proof}
    \begin{theorem}[\textrm{Rolle's Theorem}]
        Let $f:[a, b]\to \mathbb{R}$ be continuous on $[a, b]$ and
        differentiable on $(a, b)$. If $f(a)=f(b)$, then there exists $x_0\in
        (a, b)$ such that $f'(x_0)=0$.
    \end{theorem}
        \begin{proof}
            If $f$ is constant on $[a, b]$, then $f'(x)=0$ for all $x\in (a,
            b)$.\par If $f$ is not constant on $[a, b]$, then it attains its
            maximum and minimum of $[a, b]$. From $f(a)=f(b)$ it follows taht
            one of them must occur inside $(a, b)$. To see why this is, assume
            the opposite; that $f$ attains both its minimum and maximum not in
            $(a, b)$. Then $f(a)$ is either a maximum or minimum (let's assume
            its a max) which means $f(b)$ is also the max. So where is the
            minimum?? That's right, it's in $(a, b)$. Anyways, there is some
            local minimum or maximum point $x_0\in (a, b)$ and by Theorem 1.2,
            $f'(x_0)=0$.
        \end{proof}
    \begin{theorem}[\textrm{Lagrange's Theorem}]
        Let $f:[a, b]\to\mathbb{R}$ be continuous on $[a, b]$ and
        differentiable on $(a, b)$. Then there exists $x_0\in (a, b)$ such
        that
            \begin{equation*}
                f(b)-f(a)=f'(x_0)(b-a).
            \end{equation*}
    \end{theorem}
        \begin{proof}
            Consider
                \begin{equation*}
                    h(x)=x\bigparen{f(b)-f(a)}-f(x)\bigparen{b-a}.
                \end{equation*}
            Seeing as $(f(b)-f(a))$ and $(b-a)$ are constants, and $x$ and
            $f(x)$ are both differentiable on $(a, b)$, then so is $h(x)$. 
        \end{proof}
    \section{The Riemann Integral}
    \begin{definition}
        Consider a closed and bounded interval $[a, b]$. A partition $P$ of
        $[a, b]$ is a set of points $P=\{x_0, x_1, \dots, x_n\}$ such that
        $a=x_0<x_1<\cdots<x_n=b$. If we have two partitions $P$ and $Q$ of the
        same interval $[a, b]$, we say that $Q$ is a refinement of $P$, if
        $P\subseteq Q$.
    \end{definition}
    \begin{definition}
        Let $f:[a, b]\to\mathbb{R}$ be a bounded function and $P=\{x_0, x_1,\dots, x_n\}$ 
        be a partition of $[a, b]$. We define the following
        quantities:
            \begin{equation*}
                \begin{split}
                    m_i(f)&=\inf \{f(x)\mid x_{i-1}\leq x\leq x_i\}, \\ 
                    M_i(f)&=\sup \{f(x)\mid x_{i-1}\leq x\leq x_i\} \\
                    \Delta x_i&= x_i-x_{i-1}.
                \end{split}
            \end{equation*}
        The norm of the partition $P$ is defined as 
            \begin{equation*}
                \|P\|=\max \{\Delta x_i\mid 1\leq i\leq n\} 
            \end{equation*}
        The lower Riemann sum of $f$ is associated to the partition $P$ is 
            \begin{equation*}
                \underline{S}(f, P)=\sum_{i=1}^{n}m_i(f)\Delta x_i.
            \end{equation*}
        The upper sum associated with the partition $P$ is 
            \begin{equation*}
                \overline{S}(f, P)=\sum_{i=1}^nM_i(f)\Delta x_i.
            \end{equation*}
        The lower Riemann sum of $f$ is 
            \begin{equation*}
                \underline{S}(f)=\sup \{\underline{S}(f, P)\mid\text{for all
                partitions $P$}\}.
            \end{equation*}
        The upper Riemann sum of $f$ is 
            \begin{equation*}
                \overline{S}(f)=\inf \{\overline{S}(f, P)\mid\text{for all
                partitions $P$}\}. 
            \end{equation*}
        We say that $f$ is Riemann integrable over $[a, b]$ if 
            \begin{equation*}
                \underline{S}(f)=\overline{S}(f).
            \end{equation*}
        If $f$ is Riemann integrable, then the common value of the upper
        and lower Riemann sum is denoted by
            \begin{equation*}
                \int_{a}^b f(x)dx.
            \end{equation*}
    \end{definition}
    \begin{lemma}
        Let $f:[a, b]\to\mathbb{R}$ be a bounded function. Then the following
        statements hold:
            \begin{enumerate}[label=(\arabic*)]
                \item If $P_1, P_2$ are partitions of $[a, b]$, $P_1\subset P_2$, 
                    then 
                        \begin{equation*}
                            \underline{S}(f, P_1)\leq\underline{S}(f,
                            P_2)\quad\text{and}\quad\overline{S}(f,
                            P_1)\geq\overline{S}(f, P_2).
                        \end{equation*}
                \item If $P$ and $Q$ are any two partitions of $[a, b]$, then 
                        \begin{equation*}
                            \underline{S}(f, P)\leq\overline{S}(f, Q).
                        \end{equation*}
                \item 
                        \begin{equation*}
                            \underline{S}(f)\leq \overline{S}(f)
                        \end{equation*}
                \item $\underline{S}(f)=\overline{S}(f)$ if and only if for all
                    $\varepsilon>0$ there exists a partition $P_{\varepsilon}$
                    such that
                        \begin{equation*}
                            \overline{S}(f, P_{\varepsilon})-\underline{S}(f,
                            P_{\varepsilon})<\varepsilon.
                        \end{equation*}
            \end{enumerate}
    \end{lemma}
    \begin{theorem}
        Let $f:[a, b]\to\mathbb{R}$ be a continuous function. Then $f$ is
        Riemann integrable on $[a, b]$.
    \end{theorem}
        \begin{proof}
            
        \end{proof}
    \begin{definition}
        Let $f:[a, b]\to\mathbb{R}$ and $x\in[a, b]$. For any $h>0$ we define
        the oscillation of $f$ on the interval $(x-h, x+h)$ as 
    \end{definition}
\end{document}
