\documentclass[12pt]{article}
\usepackage[margin=1in]{geometry}
\usepackage{graphicx}
\usepackage{amsmath}
\usepackage{amsthm}
\usepackage{amsfonts}
\usepackage{amssymb}
\usepackage{array}
\usepackage{enumerate}
\usepackage{slashed}
\usepackage{colonequals}
\usepackage{fancyhdr}
\usepackage{import}
\usepackage{xifthen}
\usepackage{pdfpages}
\usepackage{transparent}

\newcommand{\incfig}[1]{%
    \def\svgwidth{\columnwidth}
    \import{/home/arbegla/figures/}{#1.pdf_tex}
}

\pagestyle{fancy}
\fancyhf{}
\rhead{}
\lhead{}
\rfoot{\thepage}
\setlength{\headheight}{10pt}

\newtheorem{theorem}{Theorem}[section]
\newtheorem{corollary}{Corollary}[theorem]
\newtheorem{prop}{Proposition}[section]
\newtheorem{lemma}[theorem]{Lemma}
\theoremstyle{definition}
\newtheorem{definition}{Definition}[section]
\theoremstyle{definition}
\newtheorem{exmp}{Example}[section]

\newcommand{\abs}[1]{\lvert #1 \rvert}
\newcommand{\bigabs}[1]{\Bigl \lvert #1 \Bigr \rvert}
\newcommand{\bigbracket}[1]{\Bigl [ #1 \Bigr ]}
\newcommand{\bigparen}[1]{\Bigl ( #1 \Bigr )}
\newcommand{\ceil}[1]{\lceil #1 \rceil}
\newcommand{\bigceil}[1]{\Bigl \lceil #1 \Bigr \rceil}
\newcommand{\floor}[1]{\lfloor #1 \rfloor}
\newcommand{\bigfloor}[1]{\Bigl \lfloor #1 \Bigr \rfloor}
\newcommand{\norm}[1]{\| #1 \|}
\newcommand{\bignorm}[1]{\Bigl \| #1 \Bigr \| #1}
\newcommand{\inner}[1]{\langle #1 \rangle}
\newcommand{\set}[1]{{ #1 }}

    %\begin{figure}[htp!]
    %    \centering
    %      \incfig{S5}
    %      \caption{Group Action}
    %      \label{fig:action}
    %  \end{figure}\

\begin{document}
\title{Analysis Notes}
\author{Quin Darcy}
\date{Jan, 26 2021}
%\affil{\small{California State University, Sacramento}}
\maketitle
\section{Differetntiable Functions}
    \begin{definition}
        Let $f:[a, b]\to\mathbb{R}$ and $x\in[a, b]$. If $x\in(a, b)$ and
        $\lim_{h\to 0}\frac{f(x+h)-f(x)}{h}$ exists, then we say that $f$ is
        differentiable at $x$ and use the notation
            \begin{equation*}
                f'(x)=\lim_{h\to 0}\frac{f(x+h)-f(x)}{h}.
            \end{equation*}
        Note also that if $x=a$ and and $\lim_{h\searrow
        0}\frac{f(a+h)-f(a)}{h}$ exists, then $f$ is differentiable at $a$.
        Similarly for $x=b$.
    \end{definition}
    \begin{lemma}
        Let $f:[a, b]\to \mathbb{R}$ and $x\in[a, b]$. If $f$ is differentiable
        at $x$, then there exists a fuction $\phi(h)$ defined on a small
        neighborhood of 0 such that 
            \begin{equation*}
                f(x+h)-f(x)=\big(f'(x)+\phi(h))h,
            \end{equation*}
        and 
            \begin{equation*}
                \lim_{h\to 0}\phi(h)=0.
            \end{equation*}

    \end{lemma}
    \section{Integration}
    \begin{definition}
        Let $f:[a, b]\to\mathbb{R}$ and $x\in[a, b]$. For any $h>0$ we define
        the $\textbf{oscillation}$ of $f$ on the interval $(x-h, x+h)$ as 
            \begin{equation*}
                \text{osc}(f)(x-h, x+h)=\sup\{\abs{f(x_1)-f(x_2)}\mid x_1,
                x_2\in(x-h, x+h)\cap[a, b]\}.
            \end{equation*}
        We define the oscillation of $f$ at $x$ as 
            \begin{equation*}
                \text{osc}(f)(x)=\lim_{h\searrow 0}\text{osc}(f)(x-h, x+h).
            \end{equation*}
        Notice that if $0<h_1<h_2$, then 
            \begin{equation*}
                \text{osc}(f)(x-h_1, x+h_1)\leq\text{osc}(f)(x-h_2, x+h_2).
            \end{equation*}
    \end{definition}\newpage
    \begin{lemma}
        Let $f:[a, b]\to\mathbb{R}$ and $x\in[a, b]$. Then $f$ is coninuous at
        $x$ if and only if $\text{osc}(f)(x)=0$.
    \end{lemma}
        \begin{proof}
            First, assume that $f$ is continuous at $x$. Then we get that for
            any $\varepsilon>0$, there exists $\delta>0$, such that
            $\abs{f(x)-f(y)}<\varepsilon/2$, for all $y\in(x-\delta,
            x+delta)\cap[a, b]$. This implies that for any $x_1,
            x_2\in(x-\delta, x+\delta)\cap[a, b]$, we get that 
                \begin{equation*}
                    \abs{f(x_1)-f(x_2)}\leq\abs{f(x_1)-f(x)}+\abs{f(x)-f(x_2)}<\varepsilon.
                \end{equation*}
            Hence, 
                \begin{equation*}
                    \text{osc}(f)(x-\delta, x+\delta)\leq\varepsilon.
                \end{equation*}
            We also note that by one remark made in the definition, we have
            that for any $0<h<\delta$, we have
                \begin{equation*}
                    \text{osc}(f)(x-h, x+h)\leq\text{osc}(f)(x-\delta,
                    x+\delta).
                \end{equation*}
            In summary, we showed that for all $\varepsilon$, there exists
            $\delta$ such that for any $0<h<\delta$, we have
                \begin{equation*}
                    \text{osc}(f)(x-h, x+h)\leq\varepsilon
                \end{equation*}
            which implies that $\text{osc}(f)(x)=0$.\par Now assume
            that $\text{osc}(f)(x)=0$. Then for all $\varepsilon>0$, there
            exists some $H>0$ such that for any $0<h<H$ we have that 
                \begin{equation*}
                    \text{osc}(f)(x-h, x+h)\leq\varepsilon.
                \end{equation*}
            Now let $x_2=x$ and fix $0<h<H$, then by Definition 2.1, we get 
                \begin{equation*}
                    \abs{f(x_1)-f(x)}<\varepsilon, \quad\forall x_1\in(x-h,
                    x+h)\cap[a, b], 
                \end{equation*}
            which shows that $f$ is continuous at $x$.
        \end{proof}
    \begin{lemma}
        Let $f:[a, b]\to\mathbb{R}$be a bounded function and $\gamma>0$.
        Then the set 
            \begin{equation*}
                D_{\gamma}=\Big\{x\in[a,
                b]\mid\text{osc}(f)(x)\geq\gamma\Big\}
            \end{equation*}
        is compact.
    \end{lemma}
        \begin{proof}
            As $D_{\gamma}$ is a subset of a compact set, then all we need to
            show it that it is closed. That it, we want to show that its
            complement
                \begin{equation*}
                    D_{\gamma}^c=\{x\in[a, b]\mid \text{osc}(f)(x)<\gamma\} 
                \end{equation*}
            is open. With this in mind, consider $x\in D_{\gamma}^c$.
            Membership of this set means that 
                \begin{equation*}
                    \lim_{h\searrow 0}\text{osc}(f)(x-h,
                    x+h)<\gamma.
                \end{equation*}
            This means that there exists some $\varepsilon>0$, such that 
                \begin{equation*}
                    \text{osc}(f)(x)<\gamma-\varepsilon.
                \end{equation*}
            By the properties of infimums, there exists some $h_{\varepsilon}$
            such that 
                \begin{equation*}
                    \text{osc}(f)(x-h_{\varepsilon}, x+h_{\varepsilon})<\gamma.
                \end{equation*}
            Hence, for any $z\in(x-\frac{h_{\varepsilon}}{2},
            x+\frac{h_{\varepsilon}}{2})\cap[a, b]$ we have that 
                \begin{equation*}
                    (z-\frac{h_{\varepsilon}}{2},
                    z+\frac{h_{\varepsilon}}{2})\subseteq(x-h_{\varepsilon},
                    x+h_{\varepsilon})
                \end{equation*}
            and thus 
                \begin{equation*}
                    \text{osc}(f)(z-\frac{h_{\varepsilon}}{2},
                    z+\frac{h_{\varepsilon}}{2})<\gamma.
                \end{equation*}
            Therefore $\text{osc}(f)(z)<\gamma$ and $z\in D_{\gamma}^c$, which
            imples that $D_{\gamma}^c$ is open in $[a, b]$.
        \end{proof}
\end{document}
