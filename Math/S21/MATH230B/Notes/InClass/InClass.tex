\documentclass[12pt]{article}
\usepackage[margin=1in]{geometry}
\usepackage{graphicx}
\usepackage{amsmath}
\usepackage{amsthm}
\usepackage{amsfonts}
\usepackage{amssymb}
\usepackage{array}
\usepackage{enumerate}
\usepackage{slashed}
\usepackage{colonequals}
\usepackage{fancyhdr}
\usepackage{import}
\usepackage{xifthen}
\usepackage{pdfpages}
\usepackage{transparent}

\newcommand{\incfig}[1]{%
    \def\svgwidth{\columnwidth}
    \import{/home/arbegla/figures/}{#1.pdf_tex}
}

\pagestyle{fancy}
\fancyhf{}
\rhead{}
\lhead{}
\rfoot{\thepage}
\setlength{\headheight}{10pt}

\newtheorem{theorem}{Theorem}[section]
\newtheorem{corollary}{Corollary}[theorem]
\newtheorem{prop}{Proposition}[section]
\newtheorem{lemma}[theorem]{Lemma}
\theoremstyle{definition}
\newtheorem{definition}{Definition}[section]
\theoremstyle{definition}
\newtheorem{exmp}{Example}[section]

\newcommand{\abs}[1]{\lvert #1 \rvert}
\newcommand{\bigabs}[1]{\Bigl \lvert #1 \Bigr \rvert}
\newcommand{\bigbracket}[1]{\Bigl [ #1 \Bigr ]}
\newcommand{\bigparen}[1]{\Bigl ( #1 \Bigr )}
\newcommand{\ceil}[1]{\lceil #1 \rceil}
\newcommand{\bigceil}[1]{\Bigl \lceil #1 \Bigr \rceil}
\newcommand{\floor}[1]{\lfloor #1 \rfloor}
\newcommand{\bigfloor}[1]{\Bigl \lfloor #1 \Bigr \rfloor}
\newcommand{\norm}[1]{\| #1 \|}
\newcommand{\bignorm}[1]{\Bigl \| #1 \Bigr \| #1}
\newcommand{\inner}[1]{\langle #1 \rangle}
\newcommand{\set}[1]{{ #1 }}

    %\begin{figure}[htp!]
    %    \centering
    %      \incfig{S5}
    %      \caption{Group Action}
    %      \label{fig:action}
    %  \end{figure}\

\begin{document}
\title{Analysis Notes}
\author{Quin Darcy}
\maketitle
    \section{02/03/2021}
    \begin{theorem}[4.26]
            Let $f:[a, b]\to \mathbb{R}$ be a monotone increasing
            function. Then $f$ is differentiable almost everywhere.
        \end{theorem}
        \textbf{Note}: A monotone function can have at most countably many
        discontinuities ``almost everywhere"=except a set of measure zero.\par
        A monotone function is continuous almost everywhere. A monotone
        function can fail to be differentiable at uncountable many points.
        \section{2021-02-08}
        \textbf{Note}: Ex. 4.12 needs Gr\"onwrall inequality.\par\hspace{4mm}
        \subsection{The Riemann Integral}
        Let $f:[a, b]\to\mathbb{R}$ be a bounded function, and let $P=\{x_0,
        x_1,\dots, x_n\}$ be a partition of $[a, b]$. Then 
            \begin{equation*}
                \overline{S}(f, p)=?\quad shit
            \end{equation*}
        graph goes here.
        \begin{exmp}[5.3]
            $f:[a, b]\to\mathbb{R}$, $f(x)=c$, $\forall x\in[a, b]$. 
                \begin{equation*}
                    \begin{split}
                        \overline{S}(f, p)=\sum_{i=1}^{n} c\cdot\Delta
                        x_i=x\sum_{i=1}^n \Delta x_i=c(b-a)
                    \end{split}
                \end{equation*} 
        \end{exmp}\newpage
        \begin{lemma}
            Let $f:[a, b]\to\mathbb{R}$ be a bounded function. Then 
            \begin{enumerate}
                \item If $P_1, P_2$ are two partitions such that $P_1\subseteq
                    P_2$ then $\underline{S}(f, P_1)\leq\underline{S}(f, P_2)$
                    and $\overline{S}(f, P_1)\geq\overline{S}(f, P_2)$.
                \item If $P$ and $Q$ are any partitions, then 
                    \begin{equation*}
                        \underline{S}(f, P)\leq\overline{S}(f, Q).
                    \end{equation*}
                \item $\underline(f)\leq\overline{S}(f)$
                \item $\underline{S}(f)=\overline{S}(f)$ iff $\forall
                    \varepsilon>0$, there exists $P_{\varepsilon}$ partition
                    such that 
                    \begin{equation*}
                        \overline{S}(f, P_{\varepsilon})-\underline{S}(f,
                        P_{\varepsilon})<\varepsilon.
                    \end{equation*}
            \end{enumerate}
        \end{lemma}
        \begin{theorem}
            Let $f:[a, b]\to\mathbb{R}$ be a continuous function. Then $f$ is
            riemann integrable on $[a, b]$.
        \end{theorem}
            \begin{proof}
                $f$ is continuous on a compact set, so $f$ is uniformly
                continuous on $[a, b]$. Let $\varepsilon>0$, then there exists
                $\delta>0$ such that $\forall x, y\in[a, b]$ with
                $\abs{x-y}<\delta$, we have
                $f(x)-f(y)<\frac{\varepsilon}{2(b-a)}$. Then choose a partition
                $P_{\varepsilon}=\{x_0,\dots, x_n\} $ such that
                $\|P_{\varepsilon}\|<\delta$, then $\forall x, y\in[x_{i-1},
                x_i]$ we have
                    \begin{equation*}
                        \abs{f(x)-f(y)}<\frac{\varepsilon}{2(b-a)}
                    \end{equation*}
                so $M_i-m_i\leq\frac{\varepsilon}{2(b-a)}$, $\forall 1\leq
                i\leq m$
                    \begin{equation*}
                        \begin{split}
                            \overline{S}(f, P_{\varepsilon})-\underline{S}(f,
                            P_{\varepsilon})
                        \end{split}
                    \end{equation*}
            \end{proof}
        \begin{definition}
            Let $f:[a, b]\to\mathbb{R}$and $x\in[a, b]$ and $h>0$. Then 
                \begin{equation*}
                    \text{osc}(f)(x-h, x+h) = \sup \{\abs{f(x_1)-f(x_2)}\mid
                    x_1, x_2\in(x-h, x+h)\cap[a, b]\} 
                \end{equation*}
                If $0<h_1<h_2$ then $\text{osc}(f)(x-h_1,
                x+h_1)\leq\text{osc}(x-h_2, x+h_2)$
        \end{definition}
        \begin{theorem}
            Let $fL[a, b]\to\mathbb{R}$ and $x\in[a, b]$. Then $f$ is
            continuous at $x$ if and only if $\text{osc}(f)(x)=0.$
        \end{theorem}
            \begin{proof}
                Suppose thath $f$ is continuous at $x$. Let $\varepsilon>0$.
                Then there exists $\delta>0$ such that $\forall y\in(x-\delta,
                x+\delta)\cap[a, b]$ we have $\abs{f(x)-f(y)}<\varepsilon/2$.
                then $\forall x_1, x_2\in (x-\delta, x+\delta)\cap[a, b]$ we
                have $\abs{f(x_1)-f(x_2)}<\varepsilon$. Hence,
                $\text{osc}(f)(x-\delta, x+\delta)\leq\varepsilon$. Then
                $0<h<\delta$, $\text{osc}(f)(x-h, x+h)\leq\varepsilon$. So
                $\text{osc}(f)(x)\leq\varepsilon$, $\forall \varepsilon>0$.
                Therefore, $\text{osc}(f)(x)=0$.\par Suppose
                $\text{osc}(f)(x)=0$. Let $\varepsilon>0$. Then, $\exists H>0$
                such that 
                    \begin{equation*}
                        \text{osc}(f)(x-h, x+h)<\varepsilon, \quad\forall 0<h<H 
                    \end{equation*}
            \end{proof}
            Let $D$ be the set of discontinuities of $f$ on $[a, b]$. Then
            define
                \begin{equation*}
                    D_k=\{x\in[a, b]\mid \text{osc}(f)(x)\geq\frac{1}{k}\} 
                \end{equation*}
            Then $D=\bigcup_{k\in\mathbb{N}}D_k$. And Riemann integrability
            $\iff\mu(D_k)=0$ for all $k\in\mathbb{N}$
            $\mathbb{N}\mathbb{R}\mathbb{C}\mathbb{Z}matix$ 
\end{document}


