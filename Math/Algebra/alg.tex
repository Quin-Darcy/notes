\documentclass[12pt]{article}
\usepackage[margin=1in]{geometry}
\usepackage{graphicx}
\usepackage{amsmath}
\usepackage{amsthm}
\usepackage{amsfonts}
\usepackage{amssymb}
\usepackage{array}
\usepackage{enumerate}
\usepackage{slashed}
\usepackage{colonequals}
\usepackage{fancyhdr}
\usepackage{import}
\usepackage{xifthen}
\usepackage{pdfpages}
\usepackage{transparent}

\newcommand{\incfig}[1]{%
    \def\svgwidth{\columnwidth}
    \import{/home/arbegla/figures/}{#1.pdf_tex}
}

\pagestyle{fancy}
\fancyhf{}
\rhead{}
\lhead{}
\rfoot{\thepage}
\setlength{\headheight}{10pt}

\newtheorem{theorem}{Theorem}[section]
\newtheorem{corollary}{Corollary}[theorem]
\newtheorem{prop}{Proposition}[section]
\newtheorem{lemma}[theorem]{Lemma}
\theoremstyle{definition}
\newtheorem{definition}{Definition}[section]
\theoremstyle{definition}
\newtheorem{exmp}{Example}[section]

\newcommand{\abs}[1]{\lvert #1 \rvert}
\newcommand{\bigabs}[1]{\Bigl \lvert #1 \Bigr \rvert}
\newcommand{\bigbracket}[1]{\Bigl [ #1 \Bigr ]}
\newcommand{\bigparen}[1]{\Bigl ( #1 \Bigr )}
\newcommand{\ceil}[1]{\lceil #1 \rceil}
\newcommand{\bigceil}[1]{\Bigl \lceil #1 \Bigr \rceil}
\newcommand{\floor}[1]{\lfloor #1 \rfloor}
\newcommand{\bigfloor}[1]{\Bigl \lfloor #1 \Bigr \rfloor}
\newcommand{\norm}[1]{\| #1 \|}
\newcommand{\bignorm}[1]{\Bigl \| #1 \Bigr \| #1}
\newcommand{\inner}[1]{\langle #1 \rangle}
\newcommand{\set}[1]{{ #1 }}

\begin{document}
\section{MATH 210A Notes}
  \noindent\underline{A brief review of some asasorted resuls and definitions}
    \begin{enumerate}
      \item If $m$ and $n$ are integers, $m>0$, then there exists unique
        integers $q$ and $r$, $0\leq r<m$, such that $n=qm+r$.
      \item If $m$ and $n$ are integers, the greatest common factor of $m$ and
        $n$ (denoted $(m,n)$) is the largest positive integer, $g$, such that
        $g\mid m$ and $g\mid n$.\par\hspace{4mm} If $m$ and $n$ are positive
        integers, then there exists integers $x$ and $y$ such that
        $(m,n)=mx+ny$.\par\hspace{4mm} $m$ and $n$ are relatively prime iff
        $(m, n)=1$ iff there exists integers $x$ and $y$ such that $mx+ny=1$.
      \item Assume that $a=2^{t_1}\cdot3^{t_2}\cdots p_k^{t_k}$ and
        $b=2^{w_1}\cdot3^{w_2}\cdots p_k^{w_k}$ are the prime factorizations of
        $a$ and $b$. Then $a\mid b$ iff for all $i$, $t_i\leq w_i$. If
        $u_i=\min\{t_i, w_i\}$ and $v_i=\max\{t_i, w_i\}$, then $(a,
        b)=2^{u_1}\cdot 3^{u_2}\cdots p_k^{u_k}$ and lcm$(a, b)=[a,
        b]=2^{v_1}\cdot 3^{v_1}\cdots p_k^{v_k}$.
    \end{enumerate}
  \subsection{Group Theory}
    Assume that $(G, *)$ is a group, and that $(H, *)$ is a subgroup. Define $R$
    on $G$ by $aRb$ iff $ab^{-1}\in H$. Then $R$ is an equivalence relationon
    $G$, and if $a\in G$, then $Ha=\{ha:h\in H\}=[a]_R$. Since $R$ is an
    equivalnce relation, and since equivalence classes are either equal or
    dijoint, $aRb$ iff $bRa$ iff $ab^{-1}\in H$ iff $ba^{-1}\in H$ iff $a=hb$ for
    some $h\in H$ iff $b=ga$ for some $g\in H$ iff $a\in Hb$ iff $b\in aH$ iff
    $Ha=Hb$ iff $Ha\cap Hb\neq \varnothing$.\par\hspace{4mm} Recall that an
    equivalence reltion induces a partition on the set. The set of equivalnce
    classes of $G$ with respect to $R$ is denoted $G/H$ (the set of right cosets
    of $H$ in $G$). If the number of cosets is finite, then the number of cosets
    of $H$ is $G$ is called the \textbf{index of} $H$ in $G$, denoted
    $[G:H]$.\par\hspace{4mm} If $N$ is a subgroup of $G$, then $N$ is called
    \textbf{a normal subgroup} of $G$, denoted $N\triangleleft G$, iff for all
    $g\in G$ and for all $n\in N$, $gng^{-1}\in N$. If $*$ is an associative
    relation on $S$, and $R$ is an equivalence relation on $S$, then $R$ is
    called a \textbf{congruence relation} on $S$ with respect to $*$ iff for all
    $a, b, c, d\in S$, $aRb$ and $cRd$ implies $(a*b)R(c*d)$. For example,
    congruence mod $n$ is a congruence relation on $\mathbb{Z}$ with repect to
    $+$ and $*$.\par\hspace{4mm} If $N\triangleleft G$ and $R$ is an
    equivalence relation on $G$ defined by $aRb$ iff $ ab^{-1}\in N$, then $R$
    is a congruence relation on $G$ with respect to $*$. Moreover, the set of
    equivalence classes of $G$ with respect to $R$, $G/N$ is a group with
    respect to $\odot$.\par\hspace{4mm} A group $(G, *)$ \textbf{acts on} a set $S$ iff
    there exists $\varphi:G\times S \to S$ such that for all $g, h\in G$ and
    $s\in S$: $\varphi((g*h, s))=\varphi(g, \varphi((h, s)))$ and
    $\varphi((e, s))=s$.
      \begin{figure}[htp!]
        \centering
          \incfig{action}
          \caption{Group Action}
          \label{fig:action}
      \end{figure}\hfill\par\newpage
      \begin{exmp}
        Assume that $(K, *)$ is a group, and $H\subseteq_g K$. Define
        $\varphi:H\times K\to K$ by $\varphi((h, b))=h*b$. Then to show that
        $H$ acts on $K$, we let $a, b\in H$ and $c\in K$. By definition of
        $\varphi$ we have that 
          \begin{equation*}
            \begin{split}
              \varphi((a*b, c))&=(a*b)*c \\
              &=a*(b*c) \\
              &=a*\varphi((b, c)) \\
              &=\varphi((a, \varphi((b, c)))).
            \end{split}
          \end{equation*}
        Satisfying the first condition. Finally, note that $\varphi((e,
        c))=e*c=c$. Therefore, $K$ acts on $S$. Here $\varphi$ is called the
        action of \textbf{left translation}.
      \end{exmp}
      \begin{exmp}
        Assume that $(G, *)$ is a group, and $S=G$. Define $\varphi:G\times
        S\to S$ by $\varphi((g, s))=g*s*g^{-1}$. Then we want to show that $G$
        acts on $S$. Let $a, b\in G$ and $c\in S$. Then by definition of
        $\varphi$, it follows that 
          \begin{equation*}
            \begin{split}
              \varphi(a*b, c)&=(a*b)*c*(a*b)^{-1} \\
              &=(a*b)*c*(b^{-1}*a^{-1}) \\
              &=a*(b*c*b^{-1})*a^{-1} \\
              &=a*\varphi(b, c)*a^{-1} \\
              &=\varphi(a, \varphi(b, c)),
            \end{split}
          \end{equation*}
        as desired. Finally, we have that $\varphi(e, c)=e*c*e^{-1}=c$.
        Therefore, $G$ acts on $S$. Also, $\varphi$ is called the
        \textbf{action of conjugacy}.
      \end{exmp} 
    Assume that $s\in S$. Let $G_S=\{g\in G:\varphi(g, s)=s\}$, then
    $G_S\subseteq_g G$ and $G_S$ is called the \textbf{stabilizer} of $s$ in
    $G$, or the \textbf{isotropy group} of $s$.\par\hspace{4mm} Assume that
    $s\in S$. Let $G(s)=\{\varphi(g, s):g\in G\}$. Then $G(s)$ is called the
    \textbf{orbit} of $s$ in $S$, and $\{G(s):s\in S\}$ is a partition of $S$. 
      
      \begin{figure}[htp!]
        \centering
          \incfig{orb}
          \caption{Stabilizer and Orbit}
          \label{fig:orb}
      \end{figure}\hfill\par\newpage
      \begin{exmp}
        For left translation, let $a\in K$, then 
          \begin{equation*}
            H_a=\{g\in H:\varphi(g,
            a)=a\}=\{g\in H:g*a=a\}=\{e\}
          \end{equation*}
        is the isotropy group of $K$. And 
          \begin{equation*}
            H(a)=\{\varphi(g, a):g\in H\}=\{g*a:g\in H\}=Ha
          \end{equation*}
        is the orbit of $a$ in $K$. Hence, the set of orbits under the action
        of left translation is the set of left cosets.
      \end{exmp}
      \begin{exmp}
        For conjugacy, let $s\in S$. Then the isotropy group of $s$ is 
          \begin{equation*}
            G_s=\{g\in G:\varphi(g, s)=s\}=\{g\in G:g*s*g^{-1}=s\}=\{g\in
            G:g*s=s*g\}.
          \end{equation*}
        This set is also known as the \textbf{centralizer} or the
        \textbf{normalizer} of $s$, and is denoted as $N(s)$. Finally, the
        orbit of $s$ in $S$ is 
          \begin{equation*}
            G(s)=\{\varphi(g, s):g\in G\}=\{g*s*g^{-1}:g\in G\}.
          \end{equation*}
        This set is referred to as the set is referred to as the conjugates of
        $S$, or the \textbf{conjugacy class} of $S$, denoted $c(S)$.
      \end{exmp}
    Define a relation $R$ on $S$ by $aRb$ iff there exists $s\in S$ such that
    $a\in G(s)$ and $b\in G(s)$. Then $R$ is an equivalence relation on $S$,
    and $\{G(s):s\in S\}$ is the set of equivalence classes of $R$ on $S$.
      \begin{exmp}
        For left-translation, the above relation yields the set $\{H(s):s\in
        K\}=\{Hs:s\in K\}$. This set is the collection of left cosets of $H$ in
        $K$.
      \end{exmp}
      \begin{exmp}
        For conjugacy, we have that $aRb$ iff $\exists t\in S$ such that
        $a, b\in G(t)$ iff $\exists g, h\in G$ such that $\varphi(g, t)=a$ and
        $\varphi(h, t)=b$ iff $a=gtg^{-1}$ and $b=hth^{-1}$ iff
        $t=g^{-1}ag=h^{-1}bh$ iff $(hg^{-1})a(hg^{-1})^{-1}=b$ iff
        $b=\varphi((hg^{-1}), a)$. In this case, we would call $a$ and $b$
        conjugates.
      \end{exmp}
    \subsection{The Relationships Between $G_s$ and $G(s)$}
      Assume that $t\in G(s)$, say that $t=\varphi(g, s)$, then
      $G_t=gG_sg^{-1}$. 
        \begin{proof}
          Let $t\in G(s)$ with $t=\varphi(g, s)$ for some $g\in G$. We want to
          show that $gG_sg^{-1}\subseteq G_t$ and $G_t\subseteq gG_sg^{-1}$. To
          show the former, we let $x\in gG_s g^{-1}$. Then $x=ghg^{-1}$, where
          $\varphi(h, s)=s$. Then we have that 
            \begin{equation*}
              
            \end{equation*}
        \end{proof}
      Additionally, we have that 
        \begin{equation*}
          \abs{G(s)}=\frac{\abs{G}}{\abs{G_s}}.
        \end{equation*}
        \begin{proof}
          Define $\theta:G/G_s\to G(s)$ by $\theta(G_sa)=\varphi(a^{-1}, s)$.
          Then we first want to show that $\theta$ is a function. Before we do
          this, however, let us review the objects at play here and what is is
          we are trying to prove, as well as how exactly we are going to prove
          it. So what it is that we are showing is that for a given $s\in S$,
          the number of elements in the orbit of $S$ is the same as the number
          of elements of $G$ divided by the number of elements in the
          stabilizer.\par\hspace{4mm} Intuitively, the stabilizer of $s\in S$ is the
          subgroup of $G$ whose elements are those that `fix' the element $s$
          under the action $\varphi$. Whereas the orbit of $s\in S$ is the
          subset of $S$ whose elements are those which are mapped to under the
          action $\varphi$ given input $g\in G$ and fixed $s$. So what this
          result is saying, is that the cardinality of the stabilizer divides
          the cardinality of $G$ and the result of this division is a number
          denoting the size of the orbt. \par\hspace{4mm}
        \end{proof}
\end{document}
